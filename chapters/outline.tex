

\part{Derivational morphology}
\chapter{Parts of speech}
proper nouns. acn be inflected. dispute over whether can be made definite.
\chapter{Derivational suffixation}
can get lists frm Dum-Tragut, Cholakian grammar, variosu teaching grammars.
\section{Root attachment and derivation}
\section{Nominalizing suffixes}
\section{Adjectivalizing suffixes}
\section{Adverbalizing suffixes}
\section{Verbalizing suffixes}

intensive suffixes
\chapter{Derivational prefixation}
\section{Simple prefixes}
\section{Prefixoids  and compounding}
use the word interfix

ʃɑn-ɑ-X in kouyoumdjian with irregular dog
" 	\chapter{Prosodic morphology}"
\section{Reduplication}
discuss voicing assimilation for reduplicated verbs

look for words like թնդռչուկ
 թրթռալ
 կաթկթում
\section{Hypocoristics and truncation}


\chapter{Compounding}
olsen book
\section{Compound structure}
\section{Linking vowels in compounds}
\section{Morphological and semantic classifications}
\section{Bracketing paradox in compound inflection}
\section{Variation and fluctuation in compound inflection}
\chapter{Numerals and their derivatives}
tabita: ksan@yerrort
\part{Inflectional morphology}
suspended affixation for nouns and verbs

իմին մարդս 
\chapter{Nominal inflection and noun declension classes}
give each declension class
then a separate secion on all the flunctations, like how verbs jump across class
talk about irregular percolation and leveling of inflection in compounds
\section{Regular inflection}
talk about template for regular word 

{[tsi-jə]}

\section{Inflection classes}
\section{Leveling and fluctuations in irregular inflection}
semantic nuances, cholakian 2017:83
\section{Irregular inflection in compounds}
\section{Rules for definite marking}
proper noun and names. god. cholakian. polinsky. irregular cases.

ggenitive cholakian 112. maybe just postpone to syntax of noun-phrase section

\section{Uses of possessive suffix}
\section{Variation in plural possessives}
\section{Recursive inflection}

\chapter{Adjectival inflection in noun ellipsis} 
\chapter{Verbal inflection and verb conjugation classes}

xosvav
gervadz
\chapter{Lingua}

to be processed

2. Morphology

2.1. inflection
2.1.1. Noun-inflection
2.1.1.1. Which of the following means are used to express the syntactic and semantic functions of noun phrases?
2.1.1.1.1. bound affixes
2.1.1.1.2. morphophonemic alternations alone (internal change)
2.1.1.1.3. clitic particles
2.1.1.1.4. pre-/postpositions
2.1.1.1.5. word order
2.1.1.1.6. derivational processes (e.g. adjectivalization)
2.1.1.1.7. other means - specify
2.1.1.1.8. combinations of the above
2.1.1.2. How are the following syntactic functions expressed? Give full details regarding the various means utilized, their conditioning, and their productivity.
2.1.1.2.1. subject of intransitive verb
2.1.1.2.1.1. subject is agent (has control over situation)
2.1.1.2.1.2. subject is not agent
(e.g. Bats as wo?e ÔI fell down (sc. and it was my own fault)Õ, agentive; but so wo?e ÔI fell down (sc. through no fault of my own)Õ, nonagentive)
2.1.1.2.2. subject of transitive verb
2.1.1.2.2.1. subject is agent
2.1.1.2.2.2. subject is not agent
2.1.1.2.3. subject of copular construction
2.1.1.2.4. direct object
2.1.1.2.4.1. subject expressed as free element
2.1.1.2.4.2. subject expressed as bound (affixal) element only
2.1.1.2.4.3. subject unexpressed
(In some languages if there is no subject expressed the object will take a nominative case instead of an accusative case.)
2.1.1.2.5. indirect object (if there are several possibilities, describe any semantic differences)
2.1.1.2.6. object of comparison (e.g. Ôhe is bigger than meÕ)
2.1.1.2.7. object of equation (e.g. Ôhe is as big as meÕ)
2.1.1.2.8. other objects governed by verbs - list the possibilities and illustrate amply (we are concerned with the ÔobjectsÕ of verbs which are distinguished in some way, e.g. by case, from direct objects)
2.1.1.2.9. complement of copular construction
2.1.1.2.9.1. defining, e.g. Ôhe is a manÕ
2.1.1.2.9.2. identity, e.g. Ôhe is JohnÕ
2.1.1.2.9.3. role, e.g. Ôhe is a soldierÕ
2.1.1.2.9.4. other copular verbs, e.g. ÔbecomeÕ
2.1.1.2.10. subject-complement, e.g. ÔI was made kingÕ
2.1.1.2.11. object-complement, e.g. Ôwe made him kingÕ
2.1.1.2.12. objects governed by adjectives, e.g. Ôhe is like his fatherÕ, Ôhe is different from his brotherÕ. List the possibilities and illustrate them amply.
2.1.1.2.13. agent in passive/pseudopassive/impersonal constructions
2.1.1.2.14. topic (cf. section 1.12)
2.1.1.2.15. emphasized element (if specially marked) (cf. section 1.11 )
2.1.1.3. Apply the questions of 2.1.1.2 to all types of nonfinite or nominalized verb. Note any differences from the situation with finite verbs. The following types of nonfinite or nominalized verb may occur:
2.1.1.3.1. ÔabsoluteÕ construction, e.g. ÔJohn being a fool....Õ
2.1.1.3.2. infinitive, e.g. Ôfor me to go to heaven...Õ
2.1.1.3.3. gerund (verbal noun), e.g. ÔJohnÕs singing (of) two psalmsÕ
2.1.1.3.4. nominalization, e.g. ÔJohnÕs refusal (refusing) of the offerÕ
2.1.1.4. How are the following nonlocal semantic functions expressed?
2.1.1.4.1. benefactive e.g. ÔI did it for BillÕ
2.1.1.4.2. source ÔI heard it from BillÕ, Ôwool from a sheepÕ
2.1.1.4.3. instrumental ÔI hit him with a hammerÕ
2.1.1.4.3a. negative instrumental ÔI hit him without a hammerÕ
2.1.1.4.4. comitative ÔI went with BillÕ
2.1.1.4.4a. negative comitative ÔI went without BillÕ
2.1.1.4.5. circumstance Ôa man with dirty handsÕ
2.1.1.4.5a. negative circumstance Ôa man without dirty handsÕ
2.1.1.4.6. possessive
If different types of possessive occur, give full details of all the various types. The following types of distinction seem to occur:
2.1.1.4.6.1. alienable-inalienable (sometimes different types of inalienable possession)
2.1.1.4.6.2. temporary-permanent
2.1.1.4.6.3. present-past
2.1.1.4.7. possessed (may well have no special marking)
2.1.1.4.7.1. Is there a distinction between alienable and inalienable possessed¬ness?
2.1.1.4.8. quality
a man of/with humour
2.1.1.4.8a. negative quality
a man without humour
2.1.1.4.8b. reference quality
the honour of the man
2.1.1.4.9. quantity
a boat of a thousand tons
2.1.1.4.9a. reference quantity
a pound of sugar
2.1.1.4.10. material
a house of bricks
This house is built of bricks.
2.1.1.4.10a. negative material
This house was build without bricks.
2.1.1.4.11. manner
He kissed her with verve.
2.1.1.4.11a. negative manner
He kissed her without verve.
2.1.1.4.12. cause
exhausted by his wounds
2.1.1.4.13. purpose
I used it for my work.
2.1.1.4.14. function
I used the stick as a club.
2.1.1.4.15. reference
I told him about the incident.
2.1.1.4.16. essive
I was in Berlin as a soldier.
2.1.1.4.17. translative
We appointed him as general.
2.1.1.4.18. part-whole
the head of the dog, the top of the tree
2.1.1.4.19. partitive
2.1.1.4.19.1. partitive numeral
2.1.1.4.19.2. nonpartitive numeral Ôtwo boysÕ (if specially marked)
2.1.1.4.19.3. partitive quantifier Ôsome of the boysÕ
2.1.1.4.19.4. nonpartitive quantifier Ôsome boysÕ (if specially marked), Ôsome cheese Ô
2.1.1.4.19.5. partitive negative quantifier Ônone of the boysÕ
2.1.1.4.19.6. nonpartitive negative quantifier Ôno boysÕ (if specially marked), Ôno cheeseÕ
2.1.1.4.20. price ÔI bought it for two poundsÕ
2.1.1.4.21. value Ôa table worth five poundsÕ
2.1.1.4.22. distance ÔI chased him for a mileÕ
2.1.1.4.23. extent Ôa building a mile highÕ, Ôa car twelve feet longÕ
2.1.1.4.24. concessive Ôhe came despite the rainÕ
2.1.1.4.25. inclusion Ôeveryone including JohnÕ
2.1.1.4.26. exclusion Ôeveryone excluding/except JohnÕ
2.1.1.4.27. addition Ôthree people in addition to JohnÕ
2.1.1.4.28. vocative - is there a vocative particle, e.g. English O, and if so, is it obligatory?
2.1.1.4.29. citation form
2.1.1.4.30. label form (e.g. for shop-fronts, parcel-labels, etc.)



Top
2.1.1.5. How are the following local semantic functions expressed?

"Type of location 	at rest 	motion to 	motion from 	motion past"
"2.1.1.5.1. 	general 	at 	to 	from 	past"
"2.1.1.5.2. 	proximate 	near (to) 	near 	from near 	near"
"2.1.1.5.3. 	interior 	in(side) 	in(to) 	out of 	through"
"2.1.1.5.4. 	exterior 	outside 	up to 	away from 	past"
"2.1.1.5.5. 	anterior 	in front of 	in front of 	from in front of 	in front of"
"2.1.1.5.6. 	posterior 	behind 	behind 	from behind 	behind"
"2.1.1.5.7. 	superior 	above/over 	above 	from above 	over"
"2.1.1.5.8. 	superior-contact 	on 	on(to) 	off 	over"
"2.1.1.5.8a. 	surface 	on 	on(to) 	off 	over/across"
"2.1.1.5.9. 	inferior 	below/under 	below/under 	from under 	under"
"2.1.1.5.10. 	inferior-contact 	under 	under 	from under 	under"
"2.1.1.5.11. 	lateral 	beside 	beside 	from beside 	past"
"2.1.1.5.12. 	lateral-contact 	on 	on(to) 	off 	over, along"
"2.1.1.5.13. 	citerior 	on this side of 	to this side of 	from this side of 	on this side of"
"2.1.1.5.14. 	citerior-contact 	on this side of 	to this side of 	from this side of 	on this side of"
"2.1.1.5.15. 	ulterior 	beyond 	beyond 	from beyond 	beyond"
"2.1.1.5.16. 	ulterior-contact 	on the other side of/across 	across 	from across 	on the other side of"
"2.1.1.5.17. 	medial (2) 	between 	between 	from between 	between"
"2.1.1.5.18. 	medial (3+) 	among 	among 	from among 	through"
"2.1.1.5.19. 	circumferential 	- 	- 	- 	round"
"2.1.1.5.20. 	citerior-anterior 	opposite 	opposite 	from opposite 	on the other side"
"2.1.1.5.21. 	interior (long object) 	  	  	  	through/along"
"2.1.1.5.22. 	exterior (long object) 	  	  	  	past/along"
"2.1.1.5.23. 	superior 	  	  	  	along (above)"
"2.1.1.5.24. 	superior-contact (long object) 	  	  	  	along (on top of)"
"2.1.1.5.24a. 	surface (long object) 	  	  	  	along"
"2.1.1.5.25. 	inferior (long object) 	  	  	  	along (under)"
"2.1.1.5.26. 	inferior-contact (long object) 	  	  	  	along (under)"



Top

In types 21-26 we are concerned with motion past a long object in the direction of its length. As far as the first three columns are concerned there will normally be no difference from nonlong objects. The following questions concern motion past a long object in the direction at right angles to its length.
2.1.1.5.27. interior (long object) through/across
2.1.1.5.28. superior (long object) over
2.1.1.5.29. superior-contact (long object) over
2.1.1.5.29a. surface (long object) across
2.1.1.6. The following questions are concerned with location in time:
2.1.1.6.1. general
2.1.1.6.1.1. time of day Ôat 7 oÕclockÕ
2.1.1.6.1.2. period of day Ôin the afternoonÕ
2.1.1.6.1.3. day of the week Ôon MondayÕ
2.1.1.6.1.4. month of the year Ôin JanuaryÕ
2.1.1.6.1.5. year Ôin 1976Õ
2.1.1.6.1.6. festivals Ôat ChristmasÕ
2.1.1.6.1.7. seasons Ôin springÕ
2.1.1.6.2. frequentative: is there a means of indicating the frequentative expressions corresponding to 2.1.1.6.1.1-4 (e.g. Ôon MondaysÕ)?
2.1.1.6.3. punctual-future ÔIÕll be back in two hoursÕ
2.1.1.6.4. punctual-past ÔI was here two hours agoÕ
2.1.1.6.5. duration ÔI lived there for two yearsÕ, ÔIÕve arrived for a weekÕ
2.1.1.6.6. anterior-duration-past Ônothing happened until MondayÕ
2.1.1.6.7. anterior-duration-future Ônothing is going to happen until Monday Ô
2.1.1.6.8. posterior-duration-past Ônothing has happened since MondayÕ
2.1.1.6.9. posterior-duration-future ÔIÕll be here after MondayÕ, ÔIÕll be here from Monday (on)Õ
2.1.1.6.10. anterior-general ÔIÕve been here previous to MondayÕ
2.1.1.6.11. posterior-general ÔIÕll be here subsequent to MondayÕ
2.1.1.6.12. point in period-past ÔlieÕs been here within the last 2 hoursÕ, ÔIÕve had five calls in the last hourÕ, ÔIÕve had five calls in an hourÕ
2.1.1.6.13. point in period-future Ôbe back within (the next) two hoursÕ
Note any restrictions between the occurrence of these temporal constructions and the tense/aspect of the verb.
All the questions in sections Z.1.1.4-6 should be answered with the following syntactic positions in mind:
a) as modifying (adverbial) element in a normal clause/sentence
b) as the complement of a copular (not cleft) construction c) as attributive element in a noun phrase.
If any of the categories above cannot occur in any of these contexts, please note this.
2.1.1.7. Does the language display double case-marking?
In other words, do nouns standing in a particular attributive relationship to another (head) noun exhibit, in addition to their own case-marking, case agreement with the head noun? If this is so, describe:
2.1.1.7.1. under what circumstances it occurs.
2.1.1.7.2. which combinations of cases are possible.
2.1.1.7.3. whether it is optional or obligatory to have the second case-marking.
2.1.1.7.4. any phonological variation resulting from the juxtaposition of the two case-markers.
2.1.1.8.1. Does the language have a number-marking system in nouns? If so, qualify this as e.g.
2.1.1.8.1.1. singular-plural
2.1.1.8.1.2. singular-dual-plural
2.1.1.8.1.3. singular-dual-trial-plural
2.1.1.8.1.4. singular-dual-paucal-plural
2.1.1.8.1.5. other - specify
Do different classes of nouns behave differently in this respect (e.g. animate versus inanimate)?
2.1.1.8.2. Is the system of marking number obligatory or optional? In situations where number-marking is not obligatory, is there always some disambiguating factor present?
2.1.1.8.3. If the language has no system of number-marking in the noun, does it have other means of indicating number, such as the use of a word meaning ÔmanyÕ, etc.?
2.1.1.8.4. Is there a distinction between a collective and a distributive plural or dual, etc.? Give details.
2.1.1.8.5. If collective nouns occur, is it possible to form singulatives from these? Do these have noncollective plurals in addition?
2.1.1.8.6. Describe in detail how the number distinctions marked in the noun are realized, i.e. list the various morphs (if identifiable) or changes, describing any phonological or morphological conditioning involved, and indicating the degree of productivity of the various processes.
2.1.1.8.7. Do foreign words retain their native number marking or are they integrated? If they are integrated, how does this proceed?
2.1.1.9. Are nouns divided into classes or genders?
2.1.1.9.1. If this is so, list the classes together with their distinguishing affixes or markers. Give examples of members of the various classes. Describe any phonologically or morphologically conditioned variation in the class/gender markers and indicate the relative productivity of the various formations.
2.1.1.9.2. If the answer to 2.1.1.9 is yes, give a characterization, if possible, of the class-meaning of each class. If this is not possible, indicate if there are any semantically associated groups of nouns which belong overwhelmingly to individual classes.
2.1.1.9.3. If the various noun-classes are not marked on the noun itself but on other elements of the sentence (e.g. numerals, prepositions, verbs, etc.), describe the system in the same way as requested in 2.1.1.9.1.
2.1.1.9.4. Does the language have classifiers?
In other words does it possess a closed class of nouns which function as the heads of noun phrases when enumeration is involved? If so, list the various classifiers (supplying also their gloss as normal nouns) and describe the semantic classes of nouns associated with each, illustrating amply. Are the classifiers marked for any other semantic feature, e.g. politeness, size?
2.1.1.9.5. Are loan-words from other languages assigned to particular classes/genders on the basis of phonological or semantic criteria, or a mixture of both? Illustrate.
2.1.1.10. Is definiteness marked in noun phrases?
2.1.1.10.1. If so, describe how and where it is marked (e.g. separate word, affix on noun, affix on classifier, etc.), indicating if there is more than one method the conditions under which the various methods are used.
2.1.1.10.2. Is the marking of definiteness in the noun phrase optional or obligatory?
2.1.1.10.3. Does the form of the definiteness marker vary according to the spatial relationship between the entity concerned and participants in the speech act?
2.1.1.10.4. Is this optional or obligatory?
2.1.1.10.5. Is definiteness indicated with
2.1.1.10.5.1. proper names?
2.1.1.10.5.2. abstract nouns?
Distinguish here between normal use and cases where two contrasting examples of the same noun are involved with different qualifications.
2.1.1.10.6. If so, is this obligatory or optionalÕ?
2.1.1.11. Is indefiniteness marked in noun phrases?
2.1.1.11.1. If so, describe how it is marked, indicating if there is more than one method the conditions under which the various methods are used.
2.1.1.11.2. Is the marking of indefiniteness in the noun phrase optional or obligatory?
2.1.1.11.3. Does the form of the indefiniteness marker vary according to the spatial relationship between the entity concerned and participants in the speech act?
2.1.1.11.4. Is this optional or obligatory?
2.1.1.11.5. Is indefiniteness indicated with
2.1.1.11.5.1. nonsingular nouns?
2.1.1.11.5.2. mass nouns?
Distinguish here between the paradigm use of mass nouns and their use as count nouns indicating ÔsortsÕ.
2.1.1.11.6. If so, is this obligatory or optional?
2.1.1.12. Are referential and nonreferential indefiniteness distinguished, e.g. Persian Hasan yek kita:b-ra: xarid ÔHasan bought a specific bookÕ, Hasan yek kita:b xarid ÔHasan bought some book or otherÕ?
2.1.1.12.1. If so, describe how the distinction is marked, indicating if there is more than one method the conditions under which the various methods are used.
2.1.1.12.2. Is the marking of referentiality in the noun phrase optional or obligatory?
2.1.1.12.3. Is referentiality indicated with
2.1.1.12.3.1. nonsingular nouns?
2.1.1.12.3.2. mass nouns?
Distinguish here between the paradigm use of mass nouns and their use as count nouns indicating ÔsortsÕ.
2.1.1.12.4. If so, is this obligatory or optional?
2.1.1.13. Is genericness marked in noun phrases?
2.1.1.13.1. If so, describe how it is marked, indicating if there is more than one method the conditions under which the various methods are used.
2.1.1.13.2. Is the marking of genericness in the noun phrase optional or obligatory?
2.1.1.14. Are more important noun actors distinguished from less important (obviative) ones by any means? For example, affixation of either or both kinds of nouns? Is the distinction more than two-way?
2.1.1.14.1. If so, describe the affixation or other means involved, giving full details of any variation in the markers.
2.1.1.14.2. Does this only apply with animate (or pseudoanimate) nouns?
2.1.1.14.3. Does this system operate only when two possible actors are involved in the same sentence, or in some other unit, or are all animate nouns defined as more important and less important?
2.1.1.14.4. Is the system described optional or obligatory?



Top

2.1.2. Pronouns
2.1.2.1. Personal pronouns
2.1.2.1.1. Do free pronouns occur in the language? Answer questions in 2.1.2.1.1 with respect to: subject, direct object, indirect object, other positions.
2.1.2.1.1.1. Are free pronouns obligatory in all circumstances in
2.1.2.1.1.1.1. the first person?
2.1.2.1.1.1.2. the second person?
2.1.2.1.1.1.3. the third person?
2.1.2.1.1.2. Are free pronouns optional in all circumstances in
2.1.2.1.1.2.1. the first person?
2.1.2.1.1.2.2. the second person?
2.1.2.1.1.2.3. the third person?
2.1.2.1.1.3. Do free pronouns occur
2.1.2.1.1.3.1. in noncontrastive nonemphatic contexts in general?
2.1.2.1.1.3.2. in contexts where the referent(s) of the pronoun is/are emphasized?
2.1.2.1.1.3.3. in unemphatic contexts with imperative verbs?
2.1.2.1.1.3.4. in contexts with imperative verbs where the referent(s) of the pronoun is/are emphasized?
2.1.2.1.1.3.5. in answer to questions of the type Ôwho is that?Õ, i.e. Ô(it is) IÕ?
2.1.2.1.1.3.6. in cleft or pseudocleft constructions?
2.1.2.1.1.3.7. If the conditions of occurrence of free pronouns do not correspond with any of the above, state them in as much detail as possible.
2.1.2.1.1.4. If free pronouns occur in both emphatic and unemphatic contexts, is there a difference in either segmental (i.e. reduced versus unreduced) or suprasegmental (accented versus unaccented, tone variation, long vowel versus short vowel, etc.) structure?
2.1.2.1.1.5. Are reduced pronouns restricted to particular positions in the sentence structure? If so, give details.
2.1.2.1.2. What person distinctions are made in the pronouns?
2.1.2.1.2.1. 1st v. 2nd v. 3rd person
2.1.2.1.2.2. 1st v. nonfirst person
2.1.2.1.2.3. other - give details.
2.1.2.1.3. Does the language distinguish inclusion v. exclusion of the second person in the first person (Ôwe including youÕ v. Ôwe excluding youÕ), or inclusion v. exclusion of the third person in the first or second person (Ôwe including themÕ, Ôwe excluding themÕ, Ôyou including themÕ, Ôyou excluding themÕ)? If so, describe for each whether we have
2.1.2.1.3.1. inclusive v. exclusive
2.1.2.1.3.2. inclusive only
2.1.2.1.3.3. exclusive only
2.1.2.1.3.4. inclusive v. exclusive \
2.1.2.1.3.5. inclusive \
2.1.2.1.3.6. exclusive \
2.1.2.1.3.7. general only
2.1.2.1.4. Are pronouns marked for number?
2.1.2.1.4.1. Which of the following sets of distinctions occurs?
2.1.2.1.4.1.1. singular-plural
2.1.2.1.4.1.2. singular-dual-plural
2.1.2.1.4.1.3. singular-dual-trial-plural
2.1.2.1.4.1.4. singular-dual-trial-quadral-plural
2.1.2.1.4.1.5. singular-paucal-plural
2.1.2.1.4.1.6. singular-dual-paucal-plural
2.1.2.1.4.1.7. other - give details
2.1.2.1.4.2. Is the marking of any particular number distinction optional in any instance? If a particular distinction is not made which other subcategory takes over?
2.1.2.1.4.3. Is there overlapping reference between any of the subcategories? For example, in some languages with a dual and plural, the plural may also be used where two objects or persons are concerned, as well as the dual. In other languages with a dual and plural, the plural has only the meaning of Ôthree or moreÕ. Define in any case the reference of the plural subcategory in pronouns. Describe any features that determine the choice between the overlapping terms.
2.1.2.1.4.4. In some languages we have another kind of overlapping reference, where more than one term has unbounded reference, e.g. X = ÔoneÕ, Y = Ôtwo or moreÕ, Z = Ôthree or moreÕ. Does this exist? If so, describe any factors that determine the choice of the overlapping terms.
2.1.2.1.4.5. If the language has a paucal what is the exact range of this?
2.1.2.1.4.6. Can pronouns be associated in noun phrases with numerals, e.g. Ôwe two (men)Õ? Are such formations distinct from true duals, trials etc. in any way? Is there any obvious limit on the association of pronouns and numerals, e.g. Ôwe threeÕ, Ôwe hundred thousandÕ where the second might not be possible in some languages?
2.1.2.1.4.7. Is there a distinction between collective and distributive plurals, duals, etc.?
2.1.2.1.4.8. Is there a distinction between different types of nonsingular such that one indicates that the referents include all the possible referents (however defined), whereas the other indicates that only a subset of the possible referents is involved?
2.1.2.1.5.1. Does the language mark the different status of various 3rd person actors referred to by pronouns as more important versus less important (obviative/4th person)? Is this compulsory?
2.1.2.1.5.2. Do further degrees of obviation exist in pronouns (5th person, etc.)?
2.1.2.1.6. Are different degrees of proximity to the participants in the speech act marked in third person pronouns? If so, is this optional or obligatory?
2.1.2.1.6.1. Which distinctions are made?
2.1.2.1.7. Are there special anaphoric third person pronouns?
2.1.2.1.7.1. Do clashes between natural gender and grammatical gender arise with pronouns used anaphorically? How are these resolved?
2.1.2.1.8. Are there gender/class distinctions in pronouns? If so, describe
2.1.2.1.8.1. In contradistinction to the gender of the referent, is the sex of the speaker or hearer distinguished? If so, describe.
2.1.2.1.9. Are there special pronominal forms indicating the tribal, sectional, or family relationships of the referents? If so, describe.
2.1.2.1.9.1. Are there special pronominal forms indicating the tribal, sectional, or family relationships of the speaker or hearer to the referents? If so, describe.
2.1.2.1.10. List all the forms arising from the intersection of the above-mentioned categories - person, inclusion, number, obviation, proximity, anaphoricity, gender/class, kinship/tribal affiliation - in the unmarked case and most neutral status form (if case or status is marked in the pronouns of the language). If the gender/class distinctions are very numerous, and the various forms reflecting these distinctions are regularly formed by some process or other, it is not necessary to give all the gender/class variants provided their formation is explicated. Give details of any variant or reduced forms.
2.1.2.1.11. Does the pronoun agree with the verb in tense? Give full details of any tense-marking in the pronouns. Do the same for any other verbal category marked in the pronoun.
2.1.2.1.12. Does the language mark status distinctions in the pronoun, e.g. familiar, honorific, etc.? Describe all nonneutral forms here, giving an indication of the degree of status of the relevant participants (speaker, hearer, or third person), or of the circumstances of use involved.
2.1.2.1.12.1. If in certain circumstances the use of titles or other nouns is preferred, describe the conditions under which this is so, and
2.1.2.1.12.1.1. give a complete list of the forms if only a small closed class is involved
2.1.2.1.12.1.2. if a large number of terms is involved, state whether an open or closed class is involved, describe any subclasses of significance, and give examples
2.1.2.1.12.2. Indicate if forms from the neutral system acquire a different reference as status forms. List these cases.
2.1.2.1.12.3. If the system of person/inclusion/number/obviation/proximity/anaphoricity/gender/class/kinship/tribal affiliation distinctions made among the status forms is different from those made in the neutral system, describe the differences.
2.1.2.1.13.1. Are there special nonspecific indefinite pronouns (e.g. English one, French on)? If so, give the forms, explaining any differences in their usage.
2.1.2.1.13.2. Are any forms from the personal system used also as nonspecific indefinite pronouns (cf. English you)?
2.1.2.1.13.3. Do any nouns have the function of nonspecific indefinite pronouns? Which?
2.1.2.1.14. Describe any system of specific indefinite pronouns (e.g. English someone).
2.1.2.1.15. Are there special emphatic pronouns?
2.1.2.1.15.1. If so, are there any distinctions made in degrees of emphasis? Describe these.
2.1.2.1.15.2. If the various emphatic forms are derived according to some regular process, describe this, giving examples; otherwise list all the forms. If different distinctions are made among the emphatic pronouns from those made among the normal pronouns list all the forms in any case.
2.1.2.1.15.3. Are there selective emphatic pronouns? I.e. pronouns with the meanings Ôwe, but especially IÕ, Ôyou, but especially thouÕ, Ôthey, but especially he/sheÕ. If so, describe the system in detail.
2.1.2.1.16. Do complex pronouns occur giving a combination of different types of reference (e.g. both subject and object reference)? If so, give all the possible forms and their meanings.
2.1.2.1.17. Are constructions of the type pronoun-noun possible where both elements have the same reference, e.g. we firemen... . If so, is this possible with all pronouns or only with some. List those forms for which it is impossible.
2.1.2.1.18. Do constructions of the general type Ôwe (and) the priestÕ occur with the meaning ÔI and the priestÕ? If this phenomenon occurs in various numbers, dual, trial, etc., describe the meanings of the various types of combination.
2.1.2.1.18.1. Does this phenomenon also occur with pairs of pronouns, e.g. do constructions of the type Ôwe (and) thouÕ occur with the meaning ÔI and youÕ? If so, illustrate, describing any phonological changes occurring in the form of the pronouns. If there is a coordinating element present is this the normal coordinator for noun phrases?
2.1.2.1.19. Some languages have a sort of secondary pronoun system available, by which it is possible to specify in greater detail the precise composition of various nonsingular combinations of persons. In this system the different forms are constructed out of:
2.1.2.1.19.1. combinations of free pronouns (other than those produced by the ordinary means of coordination)
2.1.2.1.19.2. a free pronoun affixed with various pronominal affixes
2.1.2.1.19.3. other means.
If the language has such a system, describe the principles of its construction and list the various forms that occur, together with their meanings.
2.1.2.1.20. Is there a case system in pronouns?
2.1.2.1.20.1. If so, describe fully any deviations from that occurring with nouns, e.g.:
2.1.2.1.20.1.1. additional cases and their functions
2.1.2.1.20.1.2. absent cases
2.1.2.1.20.1.3. different uses of the cases
2.1.2.1.20.1.4. different markers for the cases
2.1.2.1.20.1.5. irregular forms of cases (give full details here, with paradigms if necessary)
2.1.2.2. Reflexive pronouns
2.1.2.2.1. Does the language have special reflexive pronouns, or common nouns used as reflexives?
2.1.2.2.2. If so, do these distinguish the following subcategories?
2.1.2.2.2.1. person
2.1.2.2.2.2. inclusion
2.1.2.2.2.3. number
2.1.2.2.2.4. obviation
2.1.2.2.2.5. proximity
2.1.2.2.2.6. anaphoricity
2.1.2.2.2.7. gender/class
2.1.2.2.2.8. kinship/tribal affiliation
2.1.2.2.2.9. status (for these see further 2.1.2.1)
2.1.2.2.3. Give all the forms resulting from the (intersection of the) above subcategories, unless the reflexive pronoun is derived from the personal pronoun by some regular process, in which case it is sufficient to describe the process and illustrate it amply. If there is just one reflexive, give it.
2.1.2.2.4. Are reflexive pronouns marked for case?
2.1.2.2.4.1. Is the means of expressing case in the reflexive the same as that used in the pronoun?
2.1.2.2.4.2. Describe any differences in detail.
2.1.2.2.5. If there is no reflexive pronoun, how is reflexivity expressed?
2.1.2.2.6. If there is a reflexive pronoun, are there also other ways of expressing reflexivity? Specify.
2.1.2.2.7. Does the reflexive pronoun have other uses? Specify.
2.1.2.3. Reciprocal pronouns
2.1.2.3.1. Does the language have special reciprocal pronouns, or common nouns used as reciprocals?
2.1.2.3.2. If so, do these distinguish the following subcategories?
2.1.2.3.2.1. person
2.1.2.3.2.2. inclusion
2.1.2.3.2.3. number
2.1.2.3.2.4. obviation
2.1.2.3.2.5. proximity
2.1.2.3.2.6. anaphoricity
2.1.2.3.2.7. gender/class
2.1.2.3.2.8. kinship/tribal affiliation
2.1.2.3.2.9. status (for these see further 2.1.2.1)
2.1.2.3.3. Give all the forms resulting from the (intersection of the) above subcategories, unless the reciprocal pronoun is derived from the personal pronoun by some regular process, in which case it is sufficient to describe the process and illustrate it amply. If there is just one reciprocal pronoun, give it.
2.1.2.3.4. Are reciprocal pronouns marked for case?
2.1.2.3.4.1. Is the means of expressing case in the reciprocal the same as in the noun?
2.1.2.3.4.2. Describe any differences in detail.
2.1.2.3.5. If there is no reciprocal pronoun, how is reciprocality expressed?
2.1.2.3.6. If there is a reciprocal pronoun, are there also other ways of expressing reciprocality?
2.1.2.3.7. Does the reciprocal pronoun have other uses? Specify.
2.1.2.4. Possessive pronouns
2.1.2.4.1. Does the language have special possessive pronouns?
2.1.2.4.2. Is there a distinction made between the following types of possession?
2.1.2.4.2.1. alienable/inalienable or subtypes of these
2.1.2.4.2.2. temporary/permanent
2.1.2.4.2.3. persons/animals/things
2.1.2.4.2.4. present/past
2.1.2.4.2.5. other
How are these distinctions marked?
2.1.2.4.3. Give a list of all the possessive pronouns of all the above types, unless they are derived by a regular process from the personal pronouns, in which case describe this process and illustrate amply.
2.1.2.4.4. Are possessive pronouns marked for case?
2.1.2.4.4.1. If so, is the means of expressing case in the possessive pronoun the same as in the noun?
2.1.2.4.4.2. Describe any differences in detail.
2.1.2.4.5. If there is no possessive pronoun, how is possession expressed with pronouns?
2.1.2.4.6. If there is a possessive pronoun, are there alternative ways of expressing possession with pronouns?
2.1.2.4.7. Does the language have reflexive possessive pronouns?
2.1.2.4.7.1. If so, describe them fully.
2.1.2.4.8. Does the language have reciprocal possessive pronouns?
2.1.2.4.8.1. If so, describe them fully.
2.1.2.4.9. Does the language have emphatic possessive pronouns?
2.1.2.4.9.1. If so, describe them fully.
2.1.2.4.10. Are there other types of possessive pronouns?
2.1.2.4.10.1. If so, describe them fully.
2.1.2.4.11. Can the above forms be used adjectivally (i.e. as a modifier of a nominal construction)?
2.1.2.4.11.1. Are there separate adjectival forms? If so, describe.
2.1.2.5. Demonstrative pronouns
2.1.2.5.1. Which of the following parameters are involved in the demonstrative pronouns of the language?
2.1.2.5.1.1. relative distance from the speaker; specify the number of degrees of distance, and their approximate reference (e.g. near, middle distance, far)
2.1.2.5.1.2. relative distance from the hearer; specify the number of degrees of distance, and their approximate reference
2.1.2.5.1.3. relative distance from speaker and hearer; specify the number of degrees of distance, and their approximate reference
2.1.2.5.1.4. equidistance from speaker and hearer
2.1.2.5.1.5. contact with the speaker
2.1.2.5.1.6. contact with the hearer
2.1.2.5.1.7. behind speaker
2.1.2.5.1.8. behind hearer
2.1.2.5.1.9. between speaker and hearer
2.1.2.5.1.10. on other side of hearer from speaker
2.1.2.5.1.11. on other side of speaker from hearer
2.1.2.5.1.12. equidistant from the speaker and some object
2.1.2.5.1.13. equidistant from the hearer and some object
2.1.2.5.1.14. on other side of some object from speaker
2.1.2.5.1.15. on other side of some object from hearer
2.1.2.5.1.16. inside some object
2.1.2.5.1.17. outside some object
2.1.2.5.1.18. near some object
2.1.2.5.1.19. vertical orientation with respect to the speaker; specify the number of degrees of height, and their approximate reference (e.g. higher, level, lower)
2.1.2.5.1.20. other spatial relationship with speaker, hearer, or other reference point; specify
2.1.2.5.1.21. visible/invisible
2.1.2.5.1.21.1. to the speaker
2.1.2.5.1.21.2. to the hearer
2.1.2.5.1.21.3. to both speaker and hearer
2.1.2.5.1.21.4. to some other person
2.1.2.5.1.22. known/unknown
2.1.2.5.1.22.1. to the speaker
2.1.2.5.1.22.2. to the hearer
2.1.2.5.1.22.3. to both speaker and hearer
2.1.2.5.1.22.4. to some other person
2.1.2.5.1.23. referred to in previous discourse
2.1.2.5.1.23.1. neutral
2.1.2.5.2.23.2. relative lapse of time, e.g. recently v. longer ago
2.1.2.5.1.24. time dimension; specify the number of degrees of relative time, and their approximate reference (e.g. future, present, past)
2.1.2.5.1.25. other parameters - specify
2.1.2.5.1.26. Is there a neutral demonstrative pronoun as distinct from a third person pronoun? If this is not the only demonstrative in the language, under what circumstances is it used?
2.1.2.5.2. Describe the various demonstrative pronouns resulting from the employment of the above parameters and their combinations, and give their meanings.
Since many demonstrative pronouns with basically spatial reference will have various derived meanings (e.g. the pronoun meaning Ôfar from the speakerÕ may in some languages receive the derived meaning Ôout of sightÕ, etc.), it is desirable to give as full detail as possible on such usages. Try and distinguish between primary and secondary usages, if possible. If it is the case that many combinations of the above parameters are possible and that these involve series of affixes corresponding with particular parameters affixed to a few basic stems, it is sufficient (assuming the processes involved are regular) to describe the means of formation of these complex pronouns, giving ample illustration, and stating which combinations of the various parameters are possible.
2.1.2.5.3. Are there demonstrative pronouns part of whose phonological structure may vary iconically in proportion to the degree of distance involved? Is this a continuum, or is there a fixed number of discrete distinctions involved?
2.1.2.5.4. Are demonstrative pronouns marked for number?
2.1.2.5.4.1. Is the means of expressing number in the demonstrative the same as that used with the noun?
2.1.2.5.4.2. Describe any differences in full detail.
2.1.2.5.5. Are demonstrative pronouns marked for class/gender?
2.1.2.5.5.1. Is the means of expressing class/gender in the demonstrative the same as that used in the noun?
2.1.2.5.5.2. Describe any differences in full detail.
2.1.2.5.6. Are demonstrative pronouns marked for case?
2.1.2.5.6.1. Is the means of expressing case in the demonstrative the same as that used in the noun?
2.1.2.5.6.2. Describe any differences in detail.
2.1.2.5.7. Are any other grammatical categories marked in the demonstrative?
2.1.2.5.7.1. If so, describe fully.
2.1.2.5.8. Can the above demonstrative pronouns all also be used adjectivally (attributively) or/and are there special adjectival forms?
2.1.2.5.8.1. If so, describe fully all differences between the pronominal and adjectival forms.
2.1.2.6. Interrogative pronouns and other question words
2.1.2.6.1. Does the language have interrogative pronouns?
2.1.2.6.1.1. If so, which of the following types does it have? List the forms.
2.1.2.6.1.1.1. general
2.1.2.6.1.1.2. selective (from a group)
2.1.2.6.1.1.3. other types - specify
Here follow
2.1.2.6.1.2 (parallel to 2.1.2.5.4) through 2.1.2.6.1.6.1 (parallel to 2.1.2.5.8.1 ).
2.1.2.6.2. List all other question words and their meanings. Specify whether different forms are used in direct and indirect questions.
2.1.2.7. Relative pronouns and other relative words
2.1.2.7.1. Does the language have special relative pronouns?
2.1.2.7.1.1. If so, which of the following types does it have? List the forms.
2.1.2.7.1.1.1. restrictive
2.1.2.7.1.1.2. nonrestrictive
2.1.2.7.1.1.3. other types - specify
Here follow
2.1.2.7.1.2 (parallel to 2.1.2.5.4) through 2.1.2.7.1.6.1 (parallel to 2.1.2.5.8.1).
2.1.2.7.2. List all other relative words and their meanings.
2.1.2.7.3. Do the words for ÔplaceÕ and ÔtimeÕ take relative pronouns, or relative words corresponding to ÔwhereÕ and ÔwhenÕ?



Top

2.1.3. Verb morphology
2.1.3.1. Voice
2.1.3.1.1. Passive
2.1.3.1.1.1. Personal passive: Which of the following passive constructions exist, and how are they formed (here and throughout section 2.1.3.1, indicate both changes in the morphology of the verb and in the syntactic expression of the noun phrase arguments of the verb):
2.1.3.1.1.1.1. The direct object of the active appears as subject of the passive.
2.1.3.1.1.1.2. The indirect object of the active appears as subject of the passive.
2.1.3.1.1.1.3. Some other constituent of the active appears as subject of the passive.
2.1.3.1.1.2. Impersonal passive: Are there passive constructions where no constituent appears in subject position? If so, can these be formed, and how, from verbs which in the active have
2.1.3.1.1.2.1. a direct object?
2.1.3.1.1.2.2. an indirect object?
2.1.3.1.1.2.3. some other object?
2.1.3.1.1.2.4. no object?
2.1.3.1.1.3. For each of the above types, indicate whether it is possible for the subject of the active to be expressed in the passive construction, and if so, how. Are there different forms depending on whether or not the subject of the active is agentive?
2.1.3.1.1.4.1. Does the passive have the same tenses and aspects as the active? Specify any differences fully.
2.1.3.1.1.4.2. Is there a distinction between dynamic and static passive (e.g. the house is being built versus the house is (already) built)? Specify.
2.1.3.1.2. Means of decreasing the valency (number of arguments) of a verb: Does the language have means, other than the passive, of decreasing the valency of a verb, and if so, how? (Note in particular any similarities to passives.)
2.1.3.1.2.1. formation of an intransitive verb from a transitive verb by not specifying the subject of the transitive (e.g. the water is boiling from John is boiling the water)
2.1.3.1.2.2. formation of an intransitive verb from a transitive verb by not specifying the direct object (e.g. John is eating from John is eating fish )
2.1.3.1.2.3. formation of a reciprocal intransitive verb by expressing both subject and direct object of the transitive as subject (e.g. John and Bill are fighting from John is fighting Bill)
2.1.3.1.2.4. other means of decreasing the valency of a verb
2.1.3.1.3. Means of increasing the valency of a verb: Does the language have means of increasing the valency of a verb, and if so, how? (One widespread pattern of valency-increasing is the relation between noncausative and causative.)
2.1.3.1.3.1.1. How is an intransitive verb made causative?
2.1.3.1.3.1.2. How is a transitive verb made causative?
2.1.3.1.3.1.3. How is a ditransitive verb (i.e. a verb with both direct and indirect object) made causative?
2.1.3.1.3.2. Is there any formal difference depending on the agentivity or otherwise of the causee?
2.1.3.1.3.3. Is it possible for the causee to be omitted? If so, can this lead to ambiguity (e.g. French j Ôai fait manger les cochons (i) ÔI have made the pigs eatÕ, (ii) ÔI have made someone eat the pigsÕ).
2.1.3.1.4. Are there special reflexive or reciprocal verb forms? Do these have any other uses? If so, describe.
2.1.3.2. Tense
Tense and aspect should be carefully distinguished in 2.1.3.2 and 2.1.3.3, though note should be made of forms that combine tense and aspect (e.g. the imperfect as combination of past tense and imperfective aspect), or that have both tense and aspect values (e.g. the pluperfect as past-in-the-past or perfect-iii-the-past); similarly, some forms may have both tense and mood values.
Distinguish absolute and relative tense. Absolute tenses involve a time specification relative to the present moment; relative tenses involve a time specification relative to some other specified point in time.
Indicate for each case whether tense specification is obligatory or optional.
2.1.3.2.1. Which of the following tenses are distinguished formally, and how?
2.1.3.2.1.1. universal (i.e. characteristic of all time, past, present, and future)
2.1.3.2.1.2. present
2.1.3.2.1.3. past
2.1.3.2.1.3.1. Is past further subdivided according to degree of remoteness?
2.1.3.2.1.3.2. Are there also relative tenses, relative to a point in the past, i.e. pluperfect (past-in-the-past), future-in-the-past?
2.1.3.2.1.4. future
2.1.3.2.1.4.1. Does this form also have modal and/or aspectual values?
2.1.3.2.1.4.2. Is future further subdivided according to degree of remoteness?
2.1.3.2.1.4.3. Are there also relative tenses, relative to a point in the future, i.e. future perfect (past-in-the-future), future-in-the-future?
2.1.3.2.2. Do the same tense distinctions obtain in all moods and nonfinite forms? If not, indicate the differences.
2.1.3.2.3. To what extent are the tenses absolute, and to what extent relative? Answer with respect to each of the following criteria:
2.1.3.2.3.1. mood
2.1.3.2.3.2. finiteness
2.1.3.2.3.3: main versus subordinate clause
2.1.3.3. Aspect
2.1.3.3.1. Perfect aspect
2.1.3.3.1.1. Is there a separate perfect aspect, i.e. distinct forms for indicating a past situation (event, process, state, act) that has present relevance? If so, how is it formed?
2.1.3.3.1.2. Which tenses does this form exist in (e.g. English present perfect I have seen, pluperfect (past perfect, perfect-in-the-past) I had seen, future perfect (perfect-in-the-future) I shall have seen )?
2.1.3.3.1.3. If there is a perfect aspect, which of the following can it indicate? Do any of the following have separate forms?
2.1.3.3.1.3.1. present result of a past situation
2.1.3.3.1.3.2. a situation that has held at least once in the period leading up to the present (e.g. have you ever been to London ?)
2.1.3.3.1.3.3. a situation that began in the past and is still continuing (e.g. I have been waiting for an hour already)
2.1.3.3.1.3.4. any others (e.g. a situation completed a short time ago, a situation that will shortly be completed, etc.)
2.1.3.3.1.4. Are there similarities between the expression of perfect aspect and recent past tense?
2.1.3.3.2. Aspect as different ways of viewing the duration of a situation
2.1.3.3.2.1. Which of the following, if any, are marked formally, either (a) regularly for all verbs where applicable, (b) only for certain lexical items?
2.1.3.3.2.1.1. perfective (aoristic) aspect (a situation viewed in its totality, without distinguishing beginning, middle, and end)
2.1.3.3.2.1.2. imperfective aspect (a situation viewed with respect to its internal constituency)
2.1.3.3.2.1.3. habitual aspect (a situation characteristic of a considerable stretch of time, e.g. English I used to play chess)
2.1.3.3.2.1.4. continuous aspect (nonhabitual imperfective aspect)
2.1.3.3.2.1.5. progressive aspect (continuous aspect of a nonstative (dynamic) verb)
2.1.3.3.2.1.6. ingressive aspect (beginning of a situation)
2.1.3.3.2.1.7. terminative aspect (end of a situation)
2.1.3.3.2.1.7.1. Is there a special form indicating the completion of another situation prior to the situation being described?
2.1.3.3.2.1.8. iterative aspect (repetition of a situation; note that habitual aspect (2.1.3.3.2.1.3) need not be iterative, e.g. the capital of Russia used to be St. Petersburg)
2.1.3.3.2.1.9. semelfactive aspect (a single occurrence of a situation)
2.1.3.3.2.1.10. punctual aspect (a situation that is viewed as not being able to be analyzed temporally, e.g. he coughed, referring to a single cough; contrast perfective aspect, where the situation is not analyzed, although there is no specification that it could not be)
2.1.3.3.2.1.11. durative aspect (a situation that is viewed as necessarily lasting in time)
2.1.3.3.2.1.12. simultaneous aspect (simultaneity with some other situation)
2.1.3.3.2.1.13. other aspects
2.1.3.3.2.1.14. Is there any way of indicating overtly a situation that leads to a logical conclusion (telic, accomplishment), as opposed to one that does not?
Thus English drink a gallon of water is telic (the action must come to an end when the gallon of water is consumed, and may not come to an end before then), and drink some water atelic (it can continue indefinitely, or stop at any time), although English makes no overt distinction here. Is there a way of indicating that the logical conclusion of a telic situation has been reached?
2.1.3.3.2.2.1. What possibilities are there for combining different aspectual values?
2.1.3.3.2.2.2. Are there any restrictions on the combination of different aspectual values with the various
2.1.3.3.2.2.2.1. voices?
2.1.3.3.2.2.2.2. tenses?
2.1.3.3.2.2.2.3. moods?
2.1.3.3.2.2.2.4. finite and nonfinite forms?
2.1.3.4. Mood
Which of the following exist as distinct morphological categories, and how are they marked?
2.1.3.4.1. indicative (this will exist as a separate form only in contrast to one or more other moods)
2.1.3.4.2. conditional
2.1.3.4.3. imperative
2.1.3.4.3.1. In which persons and numbers does the imperative have special forms?
2.1.3.4.4. optative (expression of a wish for something to come about)
2.1.3.4.5. intentional (intention to bring about some situation)
2.1.3.4.6. debitive (obligation to do something)
2.1.3.4.6.1. Is any distinction made between moral and physical obligation?
2.1.3.4.6.2. Is there any expression of different degrees of obligation?
2.1.3.4.7. potential (ability to do something)
2.1.3.4.7.1. Is there any distinction between physical ability and permission?
2.1.3.4.7.2. Is there a separate form for learned ability?
2.1.3.4.8. degree of certainty: are there ways in which the speaker can indicate the degree of certainty with which he makes an assertion (e.g. English he must be there, he is there, he may be there)?
2.1.3.4.9. authority for assertion: are there ways in which the speaker can indicate his authority for making an assertion, e.g. personal witnessing of situation, reliable secondhand information, unreliable secondhand information?
2.1.3.4.10. hortatory (encouraging)
2.1.3.4.11. monitory (warning)
2.1.3.4.12. narrative
2.1.3.4.13. consecutive (the situation being described follows on from some previously mentioned situation)
2.1.3.4.14. contingent (it is possible that...)
2.1.3.4.15. others - specify
2.1.3.5. Finite and nonfinite forms
Does the language distinguish finite and nonfinite verbal forms? List the forms and their uses, and indicate which forms have overt expression of the various:
2.1.3.5.1. voices
2.1.3.5.2. tenses
2.1.3.5.3. aspects
2.1.3.5.4. moods
2.1.3.6. Person/number/etc. (cf. 2.1.2.1 )
2.1.3.6.1. Which of the following (a) must be (b) may be coded in the verb?
2.1.3.6.1.1. subject
2.1.3.6.1.2. direct object
2.1.3.6.1.3. indirect object
2.1.3.6.1.4. benefactive
2.1.3.6.1.5. other - specify
2.1.3.6.2. For each of 2.1.3.6.1.1-5, how is agreement marked?
2.1.3.6.2.1. marker on verb - give these in detail
2.1.3.6.2.2. pronoun (a) clitic (b) nonclitic
2.1.3.6.2.3. other - describe processes involved
2.1.3.6.3. If only certain members of each class 2.1.3.6.1.1-5 are coded in the verb, what are the conditioning factors, and how do they operate?
2.1.3.6.3.1. word order
2.1.3.6.3.2. topic/comment structure
2.1.3.6.3.3. definiteness of noun phrase
2.1.3.6.3.4. animacy of noun phrase
2.1.3.6.3.5. deletion (nonoccurrence) of noun phrase
2.1.3.6.3.6. other - specify
2.1.3.6.4. What features of the noun phrase are coded in the verb? Refer to the list of features for pronouns (2.1.2.1.2 ff.) and nouns (2.1.1.10 ff.).
2.1.3.6.5. How is coding affected by
2.1.3.6.5.1. discrepancy between syntactic and semantic features?
2.1.3.6.5.2. coordination of noun phrases of different agreement classes?
2.1.3.6.6. Is agreement the same for all
2.1.3.6.6.1. voices?
2.1.3.6.6.2. tenses?
2.1.3.6.6.3. aspects?
2.1.3.6.6.4. moods?
2.1.3.6.6.5. finite and nonfinite forms?
Describe any differences.
2.1.3.6.7. Is identity or nonidentity between the subject of a verb and the subject of the following or preceding verb indicated? How?
2.1.3.6.7.1. With what degree of specificity (e.g. for person, number, etc.) is such agreement?
2.1.3.6.8. Are there special reflexive forms of the verb? If so, describe the formation of these forms in detail.
2.1.3.6.9. Are there special reciprocal forms of the verb? If so, describe the formation of these forms in detail.
2.1.3.6.10. Is there any distinction made between actions
2.1.3.6.10.1. towards the speaker?
2.1.3.6.10.2. away from the speaker?
2.1.3.6.10.3. towards the hearer?
2.1.3.6.10.4. away from the hearer?
2.1.3.6.10.5. towards a third person?
2.1.3.6.10.6. away from a third person?
2.1.3.6.10.7. other kinds of directionals?
2.1.3.6.11. Is a distinction made between different modes of body orientation, e.g. standing up, sitting down, with hands?
2.1.3.6.12.1. Does incorporation of the following elements take place? optionally/obligatorily
2.1.3.6.12.1.1. transitive noun subject
2.1.3.6.12.1.2. intransitive noun subject
2.1.3.6.12.1.3. noun direct object
2.1.3.6.12.1.4. noun indirect object
2.1.3.6.12.1.5. other nominal elements
2.1.3.6.12.1.6. adjectives; if so, referring to which argument?
2.1.3.6.12.1.7. adverbs; if so, which kinds?
2.1.3.6.12.1.8. pre-/postpositions
2.1.3.6.12.1.9. other elements - specify
2.1.3.6.12.2. For each of the above, describe the incorporation process, describing any changes that take place in the incorporated elements, any categories that are neutralized, and listing all irregular (suppletive or otherwise) incorporation forms.
2.1.3.7. If strings of verbs occur together in any construction, is there any change or loss of any features normally marked on the verb? Are any elements interposed?



Top

2.1.4. Adjectives
If the language has subclasses of adjectives that behave differently according to the various criteria below, please specify the subclasses, with any semantic or other correlates, and answer separately for each subclass; thus with verbal adjectives and nominal adjectives in Japanese. Similarly if there are individual adjectives or groups of adjectives that behave aberrantly.
2.1.4.1. Is any distinction made between predicative and attributive forms of adjectives? If so, specify.
2.1.4.2. Is there any distinction between absolute (permanent, normal) and contingent (temporary, abnormal) state? If so, how is this distinction expressed?
2.1.4.3.1. Do (a) attributive (b) predicative adjectives agree with nouns in terms of the following categories, and if so, how?
2.1.4.3.1.1. number
2.1.4.3.1.2. person
2.1.4.3.1.3. gender/class
2.1.4.3.1.4. case
2.1.4.3.1.5. definiteness/indefiniteness
2.1.4.3.1.6. other - specify
2.1.4.3.2. Does agreement depend on
2.1.4.3.2.1. relative position of noun and adjective?
2.1.4.3.2.2. whether or not the noun is overtly expressed?
2.1.4.3.3. How is agreement affected by
2.1.4.3.3.1. conflict between grammatical and semantic category values?
2.1.4.3.3.2. agreement with coordinated nouns some of which belong to different classes?
2.1.4.4. How are the various kinds of comparison expressed?
2.1.4.4.1. equality (e.g. as tall as John)
2.1.4.4.2. comparative (e.g. taller than John)
2.1.4.4.3. superlative
2.1.4.4.3.1. compared to other entities (e.g. this river is the widest, i.e. wider than any other river)
2.1.4.4.3.2. compared to itself at other points/times (e.g. this river is widest here, i.e. compared to the same river at other points)
2.1.4.4.3.4. others - specify
2.1.4.5. How are various degrees of a quality expressed?
2.1.4.5.1. in large measure (e.g. very tall)
2.1.4.5.2. in superabundance (e.g. too tall)
2.1.4.5.3. in small measure (e.g. rather tall)
2.1.4.5.4. others - specify
2.1.4.6. With predicative adjectives, are the categories that characterize the verbal morphology of the language
2.1.4.6.1. expressed in the adjective morphology? If so, how?
2.1.4.6.2. expressed by means of a copular verb?
2.1.5. Prepositions/postpositions
2.1.5.1. Give all pre-/postpositions or pre-/postpositional usages not mentioned in 2.1.1 and list these exhaustively together with their grammatical effects.
2.1.5.2. Do prepositions agree for any grammatical category with the nouns they govern?
2.1.5.2.1. If so, describe the system fully.
2.1.5.3. Do prepositions combine with the personal pronouns they govern to form series of personal forms? If so, describe all regularities and irregularities of the system.
2.1.5.4. Do prepositions combine with the articles of the noun phrases they govern to form prepositional articles? If so, describe all regularities and irregularities of the system.
2.1.6. Numerals/quantifiers
2.1.6.1. List the forms of the numerals used in counting, indicating the processes by which new numerals can be created for numbers between those expressed by separate forms.
2.1.6.2. Are distinct cardinal numeral forms used as attributes? If so, specify.
2.1.6.3. Are distinct numerals used for counting different kinds of objects? If so, specify.
2.1.6.4. How are ordinal numerals formed?
2.1.6.5. What other derivatives of numerals exist, and how are they formed?
2.1.6.6. List all the quantifiers, giving as precise a translation (or explanation) as possible. Quantifiers are words like some, each, all, every, no, either, neither, both, other.
2.1.6.6.1. List all quantifier compounds, e.g. English anyone, everybody, nothing, nowhere, whoever, Latin quivis, quicumque. Indicate all regularities and irregularities in form and meaning within the system.
2.1.6.6.2. Is quantification expressed by any other means, e.g. reduplication of a noun to mean Ôevery...Õ?
2.1.7. Adverbs
2.1.7.1. How are various kinds of comparison expressed?
2.1.7.1.1. equality (e.g. as quickly as)
2.1.7.1.2. comparative (e.g. more quickly than)
2.1.7.1.3. superlative (e.g. most quickly (of all))
2.1.7.1.4. others - specify
2.1.7.2. How are various degrees of a quality expressed?
2.1.7.2.1. in large measure (e.g. very quickly)
2.1.7.2.2. in superabundance (e.g. too quickly)
2.1.7.2.3. in small measure (e.g. rather quickly)
2.1.7.2.4. others - specify
2.1.8. Clitics
2.1.8.1. What kinds of clitic elements occur in the language?
2.1.8.1.1. personal pronouns
2.1.8.1.2. possessive pronouns
2.1.8.1.3. reflexive pronouns
2.1.8.1.4. reciprocal pronouns
2.1.8.1.5. auxiliary verbs
2.1.8.1.6. sentence (modal, interrogative, negative) particles
2.1.8.1.7. sentence connectives
2.1.8.1.8. anaphoric particles
2.1.8.1.9. others - specify
2.1.8.2. What positions do these clitics occupy?
2.1.8.2.1. preverbal
2.1.8.2.2. postverbal
2.1.8.2.3. sentence-final
2.1.8.2.4. sentence-initial
2.1.8.2.5. sentence-second position; if so, how is this defined (e.g. after first phonological word? after first phrasal constituent)?
2.1.8.2.6. other positions - specify
2.1.8.3. What is the relative order of clitics?
2.1.8.4. Are there any restrictions on possible combinations of clitics?
2.1.8.5. Are there means of expressing the meaning of the excluded combinations?



Top

2.2. Derivational morphology
What possibilities exist for deriving members of one category from those of the same or another category? For each pair of categories, indicate the formal means of derivation, and their semantic correlates. Are any of these processes iterative (e.g. double diminutive, causative of causative)? Indicate the degree of productivity of each process, and of its semantic regularity.
2.2.1.1. nouns from nouns
2.2.1.2. nouns from verbs
2.2.1.2.1. To what extent is the syntax of deverbal nouns similar to that of a sentence, and to what extent like that of a nonderived noun?
2.2.1.3. nouns from adjectives
2.2.1.3.1. To what extent is the syntax of deadjectival nouns similar to that of a sentence, and to what extent like that of a nonderived noun?
2.2.1.4. nouns from adverbs
2.2.1.5. nouns from any other category
2.2.2.1. verbs from nouns
2.2.2.2. verbs from verbs (see also the section on voice, 2.1.3.1 )
2.2.2.3. verbs from adjectives
2.2.2.4. verbs from adverbs
2.2.2.5. verbs from any other category
2.2.3.1. adjectives from nouns
2.2.3.2. adjectives from verbs
2.2.3.3. adjectives from adjectives
2.2.3.4. adjectives from adverbs
2.2.3.5. adjectives from any other category
2.2.4.1. adverbs from nouns
2.2.4.2. adverbs from verbs
2.2.4.3. adverbs from adjectives
2.2.4.4. adverbs from adverbs
2.2.4.5. adverbs from any other category
2.2.5. any other possibilities
2.2.6.1. Describe the possibilities for forming complex pre-/postpositions.
2.2.6.1.1. two prepositions (distinguish genuine compound prepositions of the type on to from sequences resulting from cases where a preposition has as its argument a prepositional phrase, e.g. from behind. In English these can for example be distinguished by means of the modification, e.g. from ten yards behind the car, *on ten yards to the table)
2.2.6.1.2. nominal formations, e.g. in front of
2.2.6.1.3. verbal formations, e.g. depending on
2.2.6.1.4. adjectival formations
2.2.6.1.5. other types
2.2.6.2. Are there simple derived prepositions?
2.2.6.2.1. denominal
2.2.6.2.2. deverbal, e.g. given
2.2.6.2.3. deadjectival, e.g. like
2.2.6.2.4. others
2.2.6.3. Compound morphology
What possibilities exist for compounding members of the same or different categories, and what semantic value(s) does each have? Answer for each of the combinations in 2.2.1-5. Indicate whether the components of a compound word may themselves be compound (e.g. English blackboard eraser).
\part{Function words}
clitics, particles, pronouns
make sure to cite clitic chapter in the phonology of stress section



There's an enclitic \textit{=ne} which indues a meaning of conditionality, among other subtle semantic effects  \cite[72ff]{Khanjian-2013-DissNegativeConcord}. It often goes with subjunctive verbs but can also go with indicative verbs. The presence vs absence of the indicative prefix gives a sense of "reality" vs "counterfactual/hypothetical" for the general sentence. Though it's really hard to pinpoint what feels different between the two sentences.

\begin{exe}
	\ex  \gll {ad-\'e-s=ne}, {g@-neG-v-\'i-m}\\
	hate-\textsc{th-2sgpres=ne}, \textsc{indc}-be.upset-\textsc{th-1sgpres}\\
	\trans `SUBJ: If you hate it, I will be upset.'
	\ex  \gll {g-ad-\'e-s=ne}, {g@-neG-v-\'i-m}\\
	\textsc{indc}-hate-\textsc{th-2sgpres=ne}, \textsc{indc}-be.upset-\textsc{th-1sgpres}\\
	\trans `INDC: If you hate it, then I am upset.'
	
\end{exe}

Lists of such unstressable clitics can be found in \citet[78]{Margaryan-1997-ArmenianPhonology} and \citet[72]{Khanjian-2013-DissNegativeConcord}.



Apparently Soukyasyan 2004:25 treats particles and pronouns as PWords

List of unstresed clitics (Margaryan 1997:78) though his data looks like postpositions too and ia gree



\part{Syntax of simple sentences}
\chapter{Sentence  structure and word order in declaratives}
\chapter{Simple transitives and intransitive sentences}
\chapter{Ditransitive sentences}
\part{Complex sentence of single sentences}
\chapter{Question formation and interrogatives}

clitic mə is always final, the stress section has data
\chapter{Effects of focus and backgrounding in word order}
\chapter{Negation}
\chapter{Reflexives and reciprocals}
\chapter{Special sentences}
\section{Weather terms}
\section{'Seems' }
\section{Imperatives and prohibitives}








\part{Syntax of the noun phrase}
\chapter{Structure of noun phrase}
\chapter{Genitive possession}
recursive. definite marking semantics. yeghiazaryan, cholakian
\chapter{Number-marking, numerals, and classifiers}
\section{Nouns that can't take plural}
abstract
mass
family անք
numeral+noun for dates or time (cholakian 2017:78)
\chapter{Differential object marking}
\chapter{Quantifiers}
\chapter{Pronouns, logophors, demonstratives, and substantives}
\part{Syntax of verbal valency and modalities}
\chapter{Transitivity alternations in verbal voice}
\chapter{Causativization}
\chapter{Passivization}
\chapter{Evidentiality}
\chapter{Particles}

\part{Structure of modifier phrases, adverbials, and adpositions}
\chapter{Adjective phrases}
\chapter{Adverbials}
\chapter{Postpositional phrases}
\chapter{Prepositional phrases}
\chapter{Comparative and superlative}
\part{Syntax of complex clauses}
\chapter{Coordination}
\chapter{Subordinate or subjunctive clauses}
\chapter{Relative clauses}
check out hodgson's RC papers to find any differences 
\part{Nominalized verbal clauses}
\chapter{Infinitival clause}
\chapter{Participial clause}

\part{Lexicon and sociolinguistics}
\chapter{Migrations, dialects, and general sociolinguistics}

joukoulian

kavassian
\chapter{Social phrases}
double check how to do rents and loan verbs, like is վարձել used for both
\section{Time}
\section{'Can'}
\section{Greetings and gratitude}
\chapter{Common terms}
\section{Kinship terms}
\section{Body part terms}
\section{Color}
\section{Lingua questionnaire}
5. Lexicon
5.1. Structured semantic fields
List the lexical items in the following semantic fields, with glosses or explanations, indicating the parameters that are relevant to the semantic distinctions made:
5.1.1. kinship terminology
5.1.1.1. by blood
5.1.1.2. by partial blood
5.1.1.3. by marriage
5.1.1.4. by adoption
5.1.1.4.1. permanent/temporary
5.1.1.4.2. religious/secular
5.1.1.5. by fostering
5.1.1.6. by affiliation
5.1.1.7. other parameters
5.1.2. colour terminology
Distinguish basic colour terms and other colour terms that indicate finer distinctions within the basic terms. Where possible, for each colour term indicate (a) its approximate range (b) its locus (the most typical value referred to by that colour term).
5.1.3. body parts
5.1.4. cooking terminology
5.1.5. any other structured semantic fields
5.2. Basic vocabulary
Give the normal equivalent, in the language concerned, of the following items:
5.2.1. all
5.2.2. and
5.2.3. animal
5.2.4. ashes
5.2.5. at
5.2.6. back
5.2.7. bad
5.2.8. bark
5.2.9. because
5.2.10. belly
5.2.11. big
5.2.12. bird
5.2.13. bite
5.2.14. black
5.2.15. blood
5.2.16. blow
5.2.17. bone
5.2.18. breast
5.2.19. breathe
5.2.20. burn
5.2.21. child
5.2.22. claw
5.2.23. cloud
5.2.24. cold
5.2.25. come
5.2.26. count
5.2.27. cut
5.2.28. day
5.2.29. die
5.2.20. dig
5.2.31. dirty
5.2.32. dog
5.2.33. drink
5.2.34. dry
5.2.35. dull
5.2.36. dust
5.2.37. ear
5.2.38. earth
5.2.39. eat
5.2.40. egg
5.2.41. eye
5.2.42. fall
5.2.43. far
5.2.44. fat/grease
5.2.45. father
5.2.46. fear
5.2.47. feather
5.2.48. few
5.2.49. fight
5.2.50. fire
5.2.51. fish
5.2.52: five
5.2.53. float
5.2.54. flow
5.2.55. flower
5.2.56. fly
5.2.57. fog
5.2.58. foot
5.2.59. four
5.2.60. freeze
5.2.61. fruit
5.2.62. full
5.2.63. give
5.2.64. good
5.2.65. grass
5.2.66. green
5.2.67. guts
5.2.68. hair
5.2.69. hand
5.2.70. he
5.2.71. head
5.2.72. hear
5.2.73. heart
5.2.74. heavy
5.2.75. here
5.2.76. hit
5.2.77. hold/take
5.2.78. horn
5.2.79. how
5.2.80. hunt
5.2.81. husband
5.2.82. I
5.2.83. ice
5.2.84. if
5.2.85. in
5.2.86. kill
5.2.87. knee
5.2.88. know
5.2.89. lake
5.2.90. laugh
5.2.91. leaf
5.2.92. leftside
5.2.93. leg
5.2.94. lie (i.e. be in lying position)
5.2.95. live
5.2.96. liver
5.2.97. long
5.2.98. louse
5.2.99. man/male
5.2.100. many
5.2.101. meat/flesh
5.2.102. moon
5.2.103. mother
5.2.104. mountain
5.2.105. mouth
5.2.106. name
5.2.107. narrow
5.2.108. near
5.2.109. neck
5.2.110. new
5.2.111. night
5.2.112. nose
5.2.113. not
5.2.114. old
5.2.115. one
5.2.116. other
5.2.117. person
5.2.118. play
5.2.119. pull
5.2.120. push
5.2.121. rain
5.2.122. red
5.2.123. right/correct
5.2.124. rightside
5.2.125. river
5.2.126. road
5.2.127. root
5.2.128. rope
5.2.129. rotten
5.2.130. round
5.2.131. rub
5.2.132. salt
5.2.133. sand
5.2.134. say
5.2.135. scratch
5.2.136. sea
5.2.137. see
5.2.138. seed
5.2.139. sew
5.2.140. sharp
5.2.141. short
5.2.142. sing
5.2.143. sit
5.2.144. skin
5.2.145. sky
5.2.146. sleep
5.2.147. small
5.2.148. smell
5.2.149. smoke
5.2.150. smooth
5.2.151. snake
5.2.152. snow
5.2.153. some
5.2.154. spit
5.2.155. split
5.2.156. squeeze
5.2.157. stab/pierce
5.2.158. stand
5.2.159. star
5.2.160. stick
5.2.161. stone
5.2.162. straight
5.2.163. suck
5.2.164. sun
5.2.165. swell
5.2.166. swim
5.2.167. tail
5.2.168. that
5.2.169. there
5.2.170. they
5.2.171. thick
5.2.172. thin
5.2.173. think
5.2.174. this
5.2.175. thou
5.2.176. three
5.2.177. throw
5.2.178. tie
5.2.179. tongue
5.2.180. tooth
5.2.181. tree
5.2.182. turn
5.2.183. two
5.2.184. vomit
5.2.185. walk
5.2.186. warm
5.2.187. wash
5.2.188. water
5.2.189. we
5.2.190. wet
5.2.191. what
5.2.192. when
5.2.193. where
5.2.194. white
5.2.195. who
5.2.196. wide
5.2.197. wife
5.2.198. wind
5.2.199. wing
5.2.200. wipe
5.2.201. with
5.2.202. woman
5.2.203. woods
5.2.204. worm
5.2.205. ye
5.2.206. year
5.2.207. yellow

