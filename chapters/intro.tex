\chapter{Introduction and overview}


\todo{stuff in the notes}

\subsection{Transcription and representations}


For every example in this grammar, we provide at most the following types of representations in the following order. Those with the asterisk * are present for every example.  

\begin{enumerate}
	\ex Representations used in this grammar:
	\begin{enumerate}
		\ex  *\textbf{Surface Pronunciation}: Every word or sentence is given a broad phonetic transcription in IPA. This transcription encodes noticeable types of allophony, such as voicing assimilation. This form is demarcated with either brackets [...] or nothing. 
		\ex  \textbf{Underlying Pronunciation}: When relevant, we provide the underlying pronunciation (underlying form, underlying representation) of words. This pronunciation is in IPA and it undoes allophony and other morpho-phonological alternations from the surface pronunciation. This representation is not based on the orthography or diachrony, but based on allophony and on    changes in a morpheme's pronunciation across its inflectional paradigm.  This form is demarcated with slashes /.../. 
		\ex  \textbf{Morphological Gloss}: For sentences and for some words, we provide a morpheme segmentation. In the phonology chapters, we tend to minimize morpheme segmentations if they're not relevant. Sentences get full segmentation. Paradigms have full segmentation. 
		\ex *\textbf{Translation}: We translate the word or sentence into English. In most but not all cases, the translation of a word can be found in online dictionaries. 
		\ex *\textbf{Orthographic Representation}: We write the sentence or word based on the traditional orthography of Western Armenian. In rare cases, if a word or sentence is from Eastern Armenian, we use t he reformed orthography. 
		\ex  \textbf{Transliteration}: When useful, we transliterate Western Armenian examples using our own transliteration scheme. Eastern Armenian data is transcribed with ISO 9985.\footnote{https://www.translitteration.com/transliteration/en/armenian-eastern-classical/iso-9985/} We generally provide transliterations only when we  are discussing the orthography of an example. This form is demarcated with arrows <>. 
		
	\end{enumerate}
\end{enumerate}

We illustrate with the following example (\ref{ex:intro:ex trans}).  

\begin{exe}
	\ex \glll meŋkʰ hos-teˈʁ-e-n ɡə-skəˈs-i-ŋkʰ \\
	/menkʰ hos-deʁ-e-n ɡ-skʰs-i-nkʰ/ \\
	we this-place-{\abl}-{\defgloss}  {\ind}-start-{\thgloss}-1{\pl}
	\\
	\trans `We start from here.' \label{ex:intro:ex trans}
	\\
	\armenian{Մենք հոստեղէն կը սկսինք։}
	\\
	<Menk' hosdeɣēn gə sgsink'.> 
\end{exe}


In some cases, we provide a hypothetical earlier pronunciation or version of a word. We use double slashes for this //...//. We also use double-slahes for intermediate forms of a word, i.e., a form where some but not all phonological rules have applied. 