\chapter{Syllable structure}\label{chapter:syllable}
This chapter discusses the syllable in Armenian. The first two sections give a basic overview of possible syllable types (\S\ref{section:syllable:overview}-\ref{section:syllable:ConsonantClusters}. The remaining sections go in depth on the range of possible complex codas (\S\ref{section:syllable:Final2C:FallingCommon}-\ref{section:syllable:OtherCodaRestrictions}, complex onsets (\S\ref{section:syllable:ComplexOnset},, and vowel hiatus repair or vowel-vowel sequences (\S\ref{section:syllable:VowelHiatus}). Throughout this chapter, we often give basic descriptive statistics on possible syllables from the \citeauthor{kouyoumdjian-1970-DictionaryArmenianEnglish} dictionary. By doing so, we give a stronger sense of what is a typical syllable vs. an atypical syllable.  We likewise give a stronger sense of the  range of attested or unattested syllables in Armenian. 

\section{Overview of syllable structure}\label{section:syllable:overview}
In terms of syllable structure, Armenian   in \textit{general} uses a maximal CVCC template. This means that a syllable can consist of a simple onset, or no onset. The syllable can have a coda, a complex coda, or no coda. Table \ref{tab:syll overview} illustrates the basic types of syllables. Throughout this overview section, we provide a count of such syllable types among monosyllabic words in \citet{kouyoumdjian-1970-DictionaryArmenianEnglish}'s dictionary.

\begin{table}[H]
	\centering
	\caption{Basic syllable types and their distribution in monosyllables}
	\label{tab:syll overview}
	\begin{tabular}{|l|l|lll|l| }
		\hline 
		Onset & CV & pʰu & `owl' & \armenian{բու}    & n=32\\ %  & 1.64\%   \\
		& CVC & pʰɑn & `thing' & \armenian{բան}  &  n=1099 \\%& 56.27\%  \\
		& CVCC & pɑɾtʰ & `complex' & \armenian{բարդ} &  n=695 \\% & 35.54\%  \\
		No Onset& V  & u & `and' & \armenian{ու}   &  n=7  \\%  & 0.31\%   \\
		& VC  & ɑʁ & `salt' & \armenian{աղ} &  n=53 \\%  & 2.71\%   \\
		& VCC  & ɑʁd & `stain' & \armenian{աղտ} &  n=67\\%   & 3.4\%   \\
		\hline 
		Total & &          & &      & 1953  %& 100.00\%
		\\ \hline
	\end{tabular}
	
\end{table}



Virtually any consonant can act as an onset or coda. To illustrate, the tables in Section \S\ref{section:segmentalPhono:cons} showed how each consonant can be found word-initially or word-finally. Virtually any vowel can be found in any type of syllable. For the core vowels /ɑ, e, i, o, u/, see Table \ref{tab:core vowel syll}; for the schwa see \ref{tab:schwa vowel syll types}.  For /ʏ/, it can be found with or without a coda, but it is very rare to find a /ʏ/ without an onset (Section \S\ref{section:segmentalPhono:vowel:frontRound}). 


% \subsection{Minimal and maximal word size}\label{section:syllable:MinimalMaximalWord}


In terms of word size,    there is no minimal syllable size for a word. That is, a word can be just a single syllable vowel \textit{V}, a single open syllable \textit{CV}. But as shown by the descriptive statistics  in Table\ref{tab:syll overview}, it is very rare to find words monosyllabic words which lack a coda (V, CV). Among monosyllables, it's more common to find words with codas.


There is no limit on how a big a word can be. See Section \S\ref{section:stress:regular:final} for examples of stress shift applying in very large words. Words can get larger whether by compounding, adding derivational suffixes, or by adding inflectional suffixes.  

% \subsection{Distribution of syllable types}\label{section:syllable:DistributionSyllableType}
Within polysyllabic words, virtually any type of syllable can be found in any position (Table \ref{tab:syll in word position}). That is, an open CV,  closed CVC, or closed CVCC syllable can be found word-initially, word-medially, or word-finally. 

\begin{table}[H]
	\centering
	\caption{Different syllable shapes in different word positions}
	\label{tab:syll in word position}
	\resizebox{\textwidth}{!}{%
		\begin{tabular}{|l|ll|ll|ll| }
			\hline 
			& \multicolumn{2}{l|}{Word-initial} & \multicolumn{2}{l|}{Word-medial} & \multicolumn{2}{l|}{Word-final} \\
			CV         & \textbf{ˈpʰɑ}.ˈɾi & `good' & hɑχ.\textbf{tɑ}.ˈɡɑn & `triumphal'  & t͡sə.ˈ\textbf{ɾi} & `free'
			\\
			& & \armenian{բարի} & &\armenian{յաղթական}  & & \armenian{ձրի}
			\\ \hline 
			CVC         & \textbf{ˈkʰul}.ˈbɑ & `sock' & d͡ʒəʃ.\textbf{mɑɾ}.ˈdel & `to verify' & hɑjd.ˈ\textbf{nel} & `to reveal'
			\\
			& & \armenian{գուլպայ} & & \armenian{ճշմարտել} & & \armenian{յայտնել}
			\\ \hline 
			CVCC         & \textbf{ˈkʰeχt͡ʃ}.kɑ.ˈjin & `boorish'  & əs.\textbf{kəsp}.nɑ.ˈɡɑn & `original' & mɑχ.ˈ\textbf{tɑŋkʰ} & `wish'
			\\
			& & \armenian{գեղջկային} & & \armenian{սկզբնական} & & \armenian{մաղթանք}
			\\ \hline 
			
		\end{tabular}
}\end{table}

There are some possible asymmetries in forming word-medial complex codas, discussed in Section \S\ref{section:syllable:OtherCodaRestrictions:AvoidMedial}. 

For onsetless syllables like V(C)(C), such syllables are generally restricted to the word-initial position (Table \ref{tab:syll in word position no onset}). Word-medially, such onsetless syllables are quite restricted. They can be found across a compound    a prefixoid boundary /ɑ/ in some words, and often in loanwords. But in this case,  there is a slight glottal stop before the V(C)(C) syllable. The glottal stop isn't marked in the traditional transcriptions in Armenian philology or dialectology. 


\begin{table}[H]
	\centering
	\caption{Onsetless syllables in different word positions}
	\label{tab:syll in word position no onset}
	\begin{tabular}{|l|ll|ll| }
		\hline 
		& \multicolumn{2}{l|}{Word-initial} & \multicolumn{2}{l|}{Not word-initial}\\
		V & \textbf{u}.ˈʒeʁ & `strong' & ɑ.me.n-ɑ-\textbf{ʔu}.ˈɾɑχ & `happiest'
		\\
		& & \armenian{ուժեղ} & & \armenian{ամենաուրախ}
		\\ \hline
		VC & \textbf{iʃ}.ˈχun & `rhubarb' & tʰɑ.tʰe.\textbf{ʔos} & `Thaddeus'
		\\
		& & \armenian{իշխուն} & & \armenian{Թադէոս}
		\\ \hline
		VCC &\textbf{əntʰ}.lɑj.ˈnel & `to enlarge'& i.mɑs.t-ɑ-\textbf{ʔiχt͡s} & `sensible'
		\\
		& & \armenian{ընդլայնել} & & \armenian{իմաստաիղձ}
		\\ \hline 
		
	\end{tabular}
\end{table}

Further description of these word-medial onset-less syllables is discussed in the section on vowel hiatus repair (\S\ref{section:syllable:VowelHiatus}), especially for loanword roots (\S\ref{section:syllable:VowelHiatus:Root}) and prefixoids (\S\ref{section:syllable:VowelHiatus:CompPrefix:Atypical}). 



\section{Consonant clusters in the syllable}\label{section:syllable:ConsonantClusters}
As said, the general template syllables is CVCC. Complex onsets are generally banned (\ref{section:syllable:ConsonantClusters:Onset}), while complex codas are generally at most two consonants with falling sonority (\S\ref{section:syllable:ConsonantClusters:Coda}). Flat sonority clusters can be created via extrasyllabic appendixes (\S\ref{section:syllable:ConsonantClusters:Appendix}). Final clusters of 3 consonants are exceedingly rare (\S\ref{section:syllable:ConsonantClusters:Coda3C}. 

The  survey in this section gives a very basic idea of the possible syllable in Armenian. For more in-depth coverage, Sections \S\ref{section:syllable:Final2C} and \S\ref{section:syllable:Final3C} catalog every types of word-final consonant cluster that we could find in the \citeauthor{kouyoumdjian-1970-DictionaryArmenianEnglish} dictionary. 

\subsection{Complex onsets are generally banned}\label{section:syllable:ConsonantClusters:Onset}
For complex onsets, they are virtually banned. The main exception is consonant-glide sequences (Table \ref{tab:mono cj}; \S\ref{section:syllable:ComplexOnset:glide}). Word-initially, these sequences are rather rare though and limited to [Cjɑ...] sequences (orthographically as C\armenian{եա} sequences). We only found 10 monosyllabic words with initial Cj sequences from the \citet{kouyoumdjian-1970-DictionaryArmenianEnglish} dictionary; and most of these were archaic words. 

\begin{table}[H]
	\centering
	\caption{Consonant-glide sequences as complex onsets in monosyllables}
	\label{tab:mono cj}
	\begin{tabular}{|llll| }
		\hline 
		<leart> & [ljɑɾtʰ] & `liver' & \armenian{լեարդ}
		\\
		<geank'> & [ɡjɑŋkʰ] & `life' & \armenian{կեանք} 
		\\\hline
	\end{tabular}
	
\end{table}

Some more cases are found for words which are prescriptively pronounced with a round vowel /ʏ/ like [kʰʏʁ] `village' \armenian{գիւղ}  (orthographic <Ciw>sequences ), but which can optionally be pronounced as [ju] in colloquial Western: [kʰjuʁ]. See Section \S\ref{section:segmentalPhono:vowel:frontRound}  for more data on these round vowels.  Eastern Armenian systematically pronounces such words with [ju] instead of [ʏ], thus having more cases of word-initial CjV sequences. 

For other restrictions on complex onsets, see Section \S\ref{section:syllable:ComplexOnset}. 
\subsection{Overview of complex codas}\label{section:syllable:ConsonantClusters:Coda}

Orthographically, Armenian has many words that are written with two final consonants (Table \ref{tab:final cc overview}). Their syllabification is surveyed in depth in Section \S\ref{section:syllable:Final2C}.  Some of these clusters are pronounced as just a complex coda. These clusters tend to have   falling-sonority: [bɑɾz] `simple'.  Obstruent complex codas are always homogeneous in voicing (\S\ref{section:segmentalPhono:allphonLaryng:assiimlation}.  Very rarely, we find words that end in two identical consonants, and this cluster is pronounced as a geminate or single long consonant: [dɑɾɾ] `element'.   But there are orthographic clusters which   falling-sonority clusters, but which usually take schwa epenthesis: [χɑɾən] `mixed'. As for clusters with flat or rising sonority, some take schwa epenthesis, while some can form consonant clusters without epenthesis. 


\begin{table}[H]
	\centering
	\caption{Pronunciation of final CC clusters in monosyllables}
	\label{tab:final cc overview}
	\resizebox{\textwidth}{!}{%
		\begin{tabular}{|ll|llll| l|}
			\hline 
			Sonority & Surface shape &&  & &&  \#   \\
			\hline 
			Falling  & CVCC  & <ba\textbf{rz}> & [bɑ\textbf{ɾz}] & `simple' & \armenian{պարզ}  & n=697 \\
			& CjVCC & <nea\textbf{rt}> & [njɑ\textbf{ɾtʰ} & `fiber' & \armenian{նեարդ} &  n=4   \\
			& VCC  & <a\textbf{zt}> & [ɑst] & `notice' & \armenian{ազդ} &  n=67  \\
			Falling  & CVCəC  & <xa\textbf{\.{r}n}> & [χɑ\textbf{ɾən}] & `mixed' & \armenian{խառն}  &  n=20  \\
			& VCəC  & <a\textbf{\.{r}n}> & [ɑ\textbf{ɾən}] & `wild sheep' & \armenian{առն}&  n=5   \\
			Geminate                 & CGCG   & <da\textbf{rr}> & [dɑ\textbf{ɾɾ}] & `element' & \armenian{տարր}   &  n=3   \\
			Flat     & CVCəC &  <e\textbf{ws}> & [je\textbf{vəs}] &  `morever' & \armenian{եւս} & n=6   \\
			& VCəC  &<i\textbf{nn}> & [i\textbf{nən}] & `nine' & \armenian{ինն} &  n=3   \\
			Rising   & CVCəC & <d͡za\textbf{nr}> & [d͡zɑ\textbf{nəɾ}] & `heavy' & \armenian{ծանր}    &   n=81 \\
			& VCəC  & <a\textbf{gn}> & [ɑ\textbf{ɡən}] & `eye' & \armenian{ակն}&  n=10  
			\\ \hline
		\end{tabular}
	}
\end{table}

Besides the above cat orgies of complex codas, there are some arbitrary restrictions on word-medial complex codas and some vowel-coda dependencies. These miscellaneous restrictions are covered in Section \S\ref{section:syllable:OtherCodaRestrictions}.  Schwa epenthesis is likewise a quite complicated morphophonological process, briefly overviewed in Section \S\ref{section:segmentalPhono:vowel:schwa} and  discussed in depth in \textcolor{red}{cite epenthesis chapter}. 
\subsection{Appendix or extrasyllabic consonants}\label{section:syllable:ConsonantClusters:Appendix}

For those flat or rising-sonority clusters which don't take schwa epenthesis, the final consonant is often analyzed as some type of extrasyllabic appendix (Table \ref{tab:final cc appendix}). The final segment is one of the following segments /kʰ, m, χ, s, ʃ/.  

\begin{table}[H]
	\centering
	\caption{Monosyllabic words with appendixes}
	\label{tab:final cc appendix} 
	\begin{tabular}{|ll|lll|l| }
		\hline 
		Appendix & Shape & & & & \# \\
		\hline 
		\textit{-kʰ} & CVCkʰ  & li\textbf{t͡skʰ} & `stuffing' & \armenian{լիցք} &   n=41 \\
		& VCkʰ  & ɑ\textbf{t͡ʃk}& `eye' & \armenian{աչք} &n=3  \\
		& CjVCkʰ & sjɑ\textbf{mkʰ} & `threshold' & \armenian{սեամք} & n=1  \\
		\textit{-m} & CVCm  &  ɡo\textbf{ʁm} & `side' & \armenian{կողմ}       & n=18 \\
		& VCm   & ɑ\textbf{ʃm} & `jade' & \armenian{աշմ} &  n=1  \\
		\textit{-χ} & CVCχ  &  vɑ\textbf{ʃχ}& `usury' & \armenian{վաշխ}   & n=11 \\
		& VCχ   & ɑ\textbf{χχ} & `baggage' & \armenian{աղխ} &  2=n  \\
		\textit{-s} & CVCs      & d͡zɑ\textbf{χs} & `expense' & \armenian{ծախս}   & n=13
		\\
		\textit{-ʃ} & CVCʃ     & d͡ʒɑ\textbf{fʃ} & `breast-plate' & \armenian{ճաւշ}   & n=1
		\\ \hline
	\end{tabular}
\end{table}

The appendix /m/ is quite common after fricatives (\S\ref{section:syllable:Final2C:FlatRising:FricNasalM}), while the   fricative appendixes are mostly found after other fricatives (\S\ref{section:syllable:Final2C:FlatRising:FricFric}). Some of these fricatives can also follow stops (\S\ref{section:syllable:Final2C:FlatRising:StopFric}) and complex codas (\S\ref{section:syllable:Final3C:Appendix}). 

Among these appendixes,   the nominalizer  suffix \textit{-kʰ} \armenian{ք} is special in how its consistently violates all syllable rules. It can follow any type of consonant or consonant cluster, including laterals (\S\ref{section:syllable:Final2C:FallingOther:LateralObs}), stops, (\S\ref{section:syllable:Final2C:FlatRising:StopStop}), affricates (\S\ref{section:syllable:Final2C:FlatRising:AffrStop}),  and  complex codas (\S\ref{section:syllable:Final3C:Appendix:K}). Diachronically, this suffix \textit{-kʰ} was a plural suffix in Classical Armenian, and thus it was freely added after words.  In the modern language, this inflectional suffix was reanalyzed as a derivational suffix, and it developed special behaviors in terms of syllabification. It can form a complex coda that are otherwise found in the language, such as \textit{ɾ-kʰ}, but it can also form consonant clusters that are otherwise absent.  Though there are some restrictions on word-medial appendixes (\ref{section:syllable:OtherCodaRestrictions:AvoidMedial:Ck}). 

Because of this special behavior, the suffix \textit{-kʰ} is often analyzed as not actually being part of the syllable. It is instead an extrasyllabic appendix, i.e., a segment that is added after any syllable (\citealt[83-4]{Vaux-1998-ArmenianPhono}; \citealt{VauxWolf-2009-Appendix}).. Representation \ref{rep:coda appendix strucur} illustrates. 

\begin{representation}
	Syllable structure of a complex coda vs. coda + appendix  \label{rep:coda appendix strucur}
	\centering
	\begin{tabular}{| l|l| }
		\hline        Complex coda in & 
		Coda + appendix in \\ 
		{}[bɑɾz] `simple' \armenian{պարզ} & 
		{}[lit͡sk] `stuffing' \armenian{լիցք} 
		\\
		\hline \begin{tikzpicture}
			
			\Tree  [.Word [.Syllable   [ [.Onset \textit{b}  ] ] [.Rime [.Nucleus \textit{ɑ} ] [.Coda   \textit{ɾ}   \textit{z} ] ] ] ]
		\end{tikzpicture}
		& 
		\begin{tikzpicture}
			
			\Tree [.Word   [.Syllable   [ [.Onset \textit{l}  ] ] [.Rime [.Nucleus \textit{i} ] [.Coda \textit{t͡s} ] ] ]  [ [ [  \textit{k} ] ] ] ] 
		\end{tikzpicture}
		\\ \hline 
	\end{tabular}
	
\end{representation}

Note that in Representation \ref{rep:coda appendix strucur}, we show the \textit{-kʰ} as attached directly to the word, and not to the Coda node. But it's possible that the appendix is actually added to the Syllable node instead. Crucially, the \textit{-kʰ} must be present somewhere within the phonological structure, so that it can trigger allophonic processes such as voicing assimilation (\S\ref{section:segmentalPhono:allphonLaryng:finalDeasp:derived}, \ref{section:segmentalPhono:allphonLaryng:assiimlation:regFinal}),   cf. voicing assimilation in Polish appendixes: \citealt{RubachBooij-1990-EdgeConstituentEffectPolish,Rubach-1996-NonsyllabicAnalysisVoiceAssiimlationPolish,Rubach-1997-ExtrasyllabicConsonantPolishDerivationalOptimalityTheory}). In fact, Dolatian () argues that this segment is attached to a prosodic constituent that's that is below the prosodic word, specifically the prosodic stem \citep{Downing-1999-ProsodicStem}.  \textcolor{red}{vaux and dolatian citation page}. 

\subsection{Maximality of complex codas}\label{section:syllable:ConsonantClusters:Coda3C}

For complex codas, these are  usually at most 2 consonants.  If the orthography has a final 3-consonant cluster (Table \ref{tab:final ccc mono}; \S\ref{section:syllable:Final3C}), this cluster is pronounced with either schwa epenthesis or with an appendix such as /{-kʰ, χ}/. \citet{kouyoumdjian-1970-DictionaryArmenianEnglish} lists only one word [veɾst] (a loanword from Russian) with a final 3-consonant cluster which a) we pronounce without epenthesis and which b) doesn't have an appendix. 

\begin{table}[H]
	\caption{Words with final 3-consonant clusters}
	\label{tab:final ccc mono} 
	\centering
	{%\resizebox{\textwidth}{!}{%
			\begin{tabular}{|ll|llll|l| }
				\hline 
				Sonority & Shape && &  & & \# 
				\\
				\hline 
				Falling & CVCəCC&  <xa\textbf{\.{r}nk'}>  & [χɑ\textbf{ɾəŋkʰ}] & `copulation' & \armenian{խառնք}  & n=1  \\
				Rising & 
				CVCCəC   & <pa\textbf{rt͡sr}> & [pʰɑ\textbf{ɾt͡səɾ} & `high' & \armenian{բարձր}      & n=34 \\
				& VCCəC & <a\textbf{jʒm}> & [ɑ\textbf{jʒəm}] & `now' & \armenian{այժմ} & n=25 \\
				Flat & CVCCəC  & <t͡ʃe\textbf{rmn}> & [t͡ʃe\textbf{ɾmən}] & `fever' & \armenian{ջերմն}  &n=3  \\
				\hline 
				
				Falling & CVCCkʰ  & <d͡za\textbf{jrk'}> & [d͡zɑ\textbf{jɾkʰ}] & `extremity' & \armenian{ծայրք}   & n=11 \\
				& VCCkʰ & <u\textbf{ɣxk'}> & [u\textbf{χχkʰ} & `torrent' & \armenian{ուղխք}            & n=2  \\
				Flat & CVCCk   & <gu\textbf{rd͡zk'}> & [ɡu\textbf{ɾt͡skʰ}] & `breast' & \armenian{կուրծք}             & n=19 \\
				& VCCkʰ & <a\textbf{nt͡s'k'}>&[ɑ\textbf{nt͡skʰ} & `passage' & \armenian{անցք}         & n=7  \\
				Rising & VCCχ   &  <a\textbf{sdɣ}> & [ɑ\textbf{stχ}]     & `star' & \armenian{աստղ}& n=1  \\
				\hline 
				Falling & CVCCC    & <ve\textbf{rsd}> & [ve\textbf{ɾst}] & `verst' & \armenian{վերստ}            & n=1 
				\\ \hline 
			\end{tabular}
		}
	\end{table}
	
	
	For those clusters that use schwa epenthesis, the final consonant is almost always a sonorant or fricative. The preceding cluster almost always have falling sonority.  These restrictions are because of diachrony. \textcolor{red}{cite vaux}  It has been postulated that in earlier stages of the language (Classical Armenian and Proto-Armenian), the ancestor of these <VCCC> [VCCəC] words would have an extra final syllable (perhaps <VCCCV> or <VCCVC>). Over time, the final syllable was lost, and the loss of a syllable required schwa epenthesis. 
	
	
	\section{Syllabification of final two-consonant clusters}\label{section:syllable:Final2C}
	This section goes through all attested and un-attested complex codas in Western Armenian. To find the attested clusters, we went through the \citeauthor{kouyoumdjian-1970-DictionaryArmenianEnglish} dictionary and kept track of all words that were written with two final consonants. We catalogued the consonants in terms of their sonority and pronunciation. 
	
	For sonority, we use the conventional sonority scale of  \textit{stop/affricate < fricative < nasal < liquid < glide < vowel}. 
	
	In Section \S\ref{section:syllable:Final2C:FallingCommon}, we go through  word-final consonant clusters that had falling sonority,  formed a complex coda in pronunciation,  and were very common in the dictionary, such as fricative-stop clusters like [ɑχp] `trash' \armenian{աղբ}.  In contrast, Section \S\ref{section:syllable:Final2C:FallingOther} goes through  clusters that had falling sonority but had some exceptional behavior. Such exceptional behavior is one of the following:
	\begin{itemize}[noitemsep,topsep=0pt]
		\item The orthographic cluster is   pronounced as a complex coda but is  very rare or restricted to loanwords like the lateral-fricative cluster in [vɑls]   `waltz'  \armenian{վալս}. This category includes clusters that are simply unattested in either \citeauthor{kouyoumdjian-1970-DictionaryArmenianEnglish}
		or other sources like Wiktionary. 
		
		\item The orthographic cluster requires an intervening schwa, either always    or optionally like rhotic-/n/ clusters in [χɑɾ(ə)n] `mixed' \armenian{խառն}. 
		\item The orthographic cluster is pronounced without a schwa but the second consonant was almost always a certain segment, suggesting that this segment is an extrasyllabic appendix, such as lateral-/kʰ/ clusters like [χelkʰ] `mind' \armenian{խելք}.
	\end{itemize}
	
	The dictionary likewise listed many words that end in a two-consonant cluster with either flat or rising sonority (\S\ref{section:syllable:Final2C:FlatRising}. Here, we find the same types of exceptional behavior: rarity vs. epenthesis vs. appendixes. Gemination was vanishingly rare as well (\S\ref{section:syllable:Final2C:Geminate}. 
	
	
	Because the data is quite complicated, we've had difficulty provided succinct summaries over the possible complex codas. Instead, each subsection has a list of what natural classes of clusters pattern together in terms of their syllabification. 
	
	
	
	\subsection{Falling-sonority and common complex codas}\label{section:syllable:Final2C:FallingCommon}
	The majority of common complex codas were falling sonority and belonged to one of the following groups based on the identity of the first and second consonant (C1, C2): 
	
	\begin{itemize}[noitemsep,topsep=0pt]
		\item Fricative /s,ʃ/ + stop (\S\ref{section:syllable:Final2C:FallingOther:FricFstop}
		\item Fricative /χ,ʁ/ + stop or affricate (\S\ref{section:syllable:Final2C:FallingCommon:FricXstop}
		\item Nasal /m/ + labial stop (\S\ref{section:syllable:Final2C:FallingCommon:NasalMstop}
		\item Nasal /n/ +   stop or affricate (\S\ref{section:syllable:Final2C:FallingCommon:NasalNstop}
		\item Rhotic /ɾ/ +   obstruent (\S\ref{section:syllable:Final2C:FallingCommon:RhoticObs}
		\item Glide /j/ +   consonant (\S\ref{section:syllable:Final2C:FallingCommon:GlideCons}
	\end{itemize}
	\subsubsection{Fricative /s,ʃ/ + stop}\label{section:syllable:Final2C:FallingCommon:FricSstop}
	The fricatives /s,ʃ/ can form complex codas with voiceless stops [p, t, k] with stop deaspiration. The most common stop is coronal [t].  The fricatives however cannot form complex codas with voiced obstruents, in order to avoid a voicing mismatch (\S\ref{section:ortho:mismatch:clusters}, \S\ref{section:segmentalPhono:allphonLaryng:assiimlation}). The fricative /s,ʃ/   also avoid combining with affricates (\S\ref{section:syllable:Final2C:FallingOther:FricSaff}). 
	
	The fricative /s/ is a pretty common segment. It can form complex codas with any type of voiceless stop: [sp, st, sk] (Table \ref{tab:compplex coda s stop}). Note the deaspiration on the stop. The [k] can be part of either the root (written as \armenian{գ,կ}) or part of the nominalizer suffix \textit{-kʰ} (written as \armenian{ք}). 
	
	\begin{table}[H]
		\centering
		\caption{Complex codas    where C1 is fricative /s/, and C2 is a voiceless stop}
		\label{tab:compplex coda s stop}
		\begin{tabular}{|l|lll|l| }
			\hline 
			{}[sp] & ˈvo\textbf{sp} & `lentil' & \armenian{ոսպ} & n=24 \\
			& bɑˈɾi\textbf{sp} & `fortress' & \armenian{պարիսպ} & \\ \hline 
			{}[st] &ˈpʰu\textbf{st} & `coral' & \armenian{բուստ} &  n=287 \\
			& nəˈbɑ\textbf{st} & `subsidy' & \armenian{նպաստ} & \\ \hline 
			{}[sk] & ˈɡɑ\textbf{sk} & `malt' & \armenian{կասկ} & n=26 \\ 
			& bəˈɾi\textbf{sk} & `drias plant' & \armenian{պրիսկ} & \\ \hline 
			{}[s-k] &ˈkʰe\textbf{s-k} & `head of hair' ($\sqrt{}$-{\nmlz}) & \armenian{գէսք}&  n=14 \\
			& ˈkʰes & `long hanging hair' & \armenian{գէս} & 
			\\ \hline 
		\end{tabular}
	\end{table}
	
	
	Similarly, the fricative /ʃ/ can form a complex coda with a voiceless stop [ʃp, ʃt, ʃk] (Table \ref{tab:compplex coda sh stop}). The [k] can be part of the root (written as \armenian{գ,կ}) or part of the nominalizer suffix \textit{-kʰ} (written as \armenian{ք}). 
	
	\begin{table}[H]
		\centering
		\caption{Complex codas    where C1 is fricative /ʃ/, and C2 is a voiceless stop}
		\label{tab:compplex coda sh stop}
		\begin{tabular}{|l|lll|l| }
			\hline 
			{}[ʃp] & ˈkʰu\textbf{ʃp} & `crevice' &  \armenian{գուշպ} &  n=1   \\  \hline 
			{}[ʃt] & ˈɡe\textbf{ʃt}  & `sect' & \armenian{կեշտ} & n=178 \\
			& pʰeˈhe\textbf{ʃt} & `paradise' & \armenian{բեհեշտ} & \\ \hline 
			{}[ʃk] & ˈmɑ\textbf{ʃk} & `cuticle' & \armenian{մաշկ} & n=47 \\
			& tʰəˈmi\textbf{ʃk} & `Damascus blade' & \armenian{դմիշկ} & \\ \hline 
			{}[ʃ-k] & ˈd᷂͡ʒo\textbf{ʃ-k} & `defamation' ($\sqrt{}$-{\nmlz}) & \armenian{ճօշք} & n=4 \\
			&cf.  d͡ʒoˈʃ-ɑ-l& `to defame' ($\sqrt{}$-{\thgloss}-{\infgloss})  &\armenian{ճօշալ}  & 
			\\ \hline 
		\end{tabular}
	\end{table}
	
	
	\subsubsection{Fricative /χ, ʁ/ + stop or affricate}\label{section:syllable:Final2C:FallingCommon:FricXstop}
	
	The fricative /χ/ can form a complex coda with any voiceless stop or affricate, with deaspiration on the stop: [χp, χt, χk, χt͡s, χt͡ʃ]  (Table \ref{tab:compplex coda x stop}). Though the most common complex coda involves [t]. The [k] can be part of either the root (written as \armenian{գ}) or part of the nominalizer suffix \textit{-kʰ} (written as \armenian{ք}). 
	
	
	\begin{table}[H]
		\centering
		\caption{Complex codas    where C1 is fricative /χ/, and C2 is a voiceless stop or affricate}
		\label{tab:compplex coda x stop}
		\begin{tabular}{|l|lll|l| }
			\hline 
			{}[χp] & ˈʃe\textbf{χp} & `blade' &   \armenian{շեղբ} & n=19   \\
			& ˈpʰɑ\textbf{χp} & `sheen' & \armenian{փաղփ} & \\ \hline 
			{}[χt] & ˈsɑ\textbf{χt} & `saddle' & \armenian{սախտ} & 2n=19 \\ 
			& tʰəˈɾɑ\textbf{χt} & `paradise' & \armenian{դրախտ} & \\ \hline 
			{}[χk] & ˈhe\textbf{χk} & `lazy' & \armenian{հեղգ} & n=1 \\\hline 
			{}[χ-k] & hɑˈd͡ʒɑ\textbf{χ-k} & `frequency' ($\sqrt{}$-{\nmlz})  &\armenian{յաճախք} &  n=45\\
			& cf. hɑd͡ʒɑˈχ-e-l & `to frequent' ($\sqrt{}$-{\thgloss}-{\infgloss}) && \\ \hline 
			{}[χt͡s] & ˈtʰe\textbf{χt͡s} & `peach' & \armenian{դեղձ} &n= 53 \\
			& ɑˈdɑ\textbf{χt͡s} & `timber'  & \armenian{ատաղձ} & \\ \hline 
			{}[χt͡ʃ] & ˈze\textbf{χt͡ʃ} & `discount' & \armenian{զեղչ}& n=26 \\
			& ˈkʰɑ\textbf{χt͡ʃ} & `lukewarm' & \armenian{գաղջ} & 
			\\ \hline 
		\end{tabular}
	\end{table}
	
	
	The fricative /ʁ/ is generally infrequent as a segment. But it can form complex codas with voiced stops and affricates (Table \ref{tab:compplex coda gh stop}). It cannot co-occur with voiceless obstruents. Any orthographic sequences of <ɣ> and a voiceless obstruent are pronounced as voiceless (\S\ref{section:ortho:mismatch:clusters}). 
	
	\begin{table}[H]
		\centering
		\caption{Complex codas    where C1 is fricative /ʁ/, and C2 is a voiced stop or affricate}
		\label{tab:compplex coda gh stop}
		\begin{tabular}{|l|lll|l| }
			\hline 
			{}[ʁb] & ˈd͡ʒɑ\textbf{ʁb} & `coffin' & \armenian{ջաղպ} & n=1 \\ \hline
			{}[ʁd] & ˈɡe\textbf{ʁd} & `stain' & \armenian{կեղտ} & n=21 \\
			& ˈu\textbf{ʁd} & `camel'  & \armenian{ուղտ} & \\ \hline 
			{}[ʁɡ] & ˈme\textbf{ʁɡ} & `soft' & \armenian{մեղկ} &n=21\\
			& məˈʒu\textbf{ʁɡ} & `gnat' & \armenian{մժուղկ} & \\ \hline 
			{}[ʁd͡z] & ˈme\textbf{ʁd͡z} & `soot' & \armenian{մեղծ} & n=37  \\
			& ˈze\textbf{ʁd͡z} & `dissolute' & \armenian{զեղծ} & \\ \hline 
			{}[ʁd͡ʒ] & ˈχi\textbf{ʁd͡ʒ} & `conscience' & \armenian{խիղճ} & n=4 \\
			& ˈχe\textbf{ʁd͡ʒ} & `wretched' & \armenian{խեղճ} & 
			\\ \hline 
		\end{tabular}
	\end{table}
	
	
	\subsubsection{Nasal /m/ + labial stop }\label{section:syllable:Final2C:FallingCommon:NasalMstop}
	The nasal /m/ can precede labial stops /pʰ, b/ (Table \ref{tab:compplex coda m p}). The cluster [mpʰ] is significantly more common than [mb]. However, /m/ seems to avoid forming a complex coda with other types of stop. A spurious exception is \textit{mkʰ} sequences which utilize an appendix. See Section \S\ref{section:syllable:Final2C:FallingOther:NasalMstop}. 
	
	\begin{table}[H]
		\centering
		\caption{Complex codas    where C1 is nasal /m/, and C2 is a labial stop}
		\label{tab:compplex coda m p}
		\begin{tabular}{|l|lll|l| }
			\hline 
			{}[mpʰ]& d͡zəˈd͡zu\textbf{mpʰ} & `sulfur' & \armenian{ծծումբ}& n=180
			\\
			& ɡɑˈʁɑ\textbf{mpʰ} & `cabbage' & \armenian{կաղամբ} & 
			\\
			{}[mb] & ˈɑ\textbf{mb} & `cloud' & \armenian{ամպ} & n=6 \\
			& ˈu\textbf{mb} & `gulp' & \armenian{ումպ} & 
			\\ \hline 
		\end{tabular}
	\end{table}
	
	The preponderance of [mpʰ] over [mb] is typologically surprising \citep{Pater-1999-AustronesianNasalSubstitutionOtherNCEffects} but it makes diachronic sense \citep{Begus-2019-PostNasalDevoicingBlurringProcess}. Most surface [mpʰ] clusters are written with final <mp> \armenian{մբ}. This sequence corresponds to Classical/Eastern Armenian [mb] clusters. In contrast, Western [mb] clusters are orthographically <mb> \armenian{մպ}, and they correspond to Classical/Eastern [mp]. The sound changes \textit{p$\rightarrow$b} and \textit{b$\rightarrow$pʰ} caused Western Armenian to end up having [mpʰ] be more common than [mb]. 
	
	\subsubsection{Nasal /n/ + stop or affricate}\label{section:syllable:Final2C:FallingCommon:NasalNstop}
	
	The nasal /n/ can form a complex coda with coronal /tʰ/ or /d/ (Table \ref{tab:compplex coda n t}). 
	
	
	\begin{table}[H]
		\centering
		\caption{Complex codas    where C1 is nasal /n/, and C2 is a coronal stop}
		\label{tab:compplex coda n t}
		\begin{tabular}{|l|lll|l| }
			\hline 
			{}[ntʰ]& dɑˈʁɑ\textbf{ntʰ} & `talent' & \armenian{տաղանդ} &  n=295
			\\
			& ɑtʰɑˈmɑ\textbf{ntʰ} & `diamond' & \armenian{ադամանդ} & 
			\\ \hline 
			{}[nd] & ˈχɑ\textbf{nd} &`lewd' & \armenian{խանտ} &  n=32 \\
			& ˈʒɑ\textbf{nd} & `pestilent' & \armenian{ժանտ}  & 
			\\ \hline 
		\end{tabular}
	\end{table}
	
	There are no examples of /n/ + a labial stop /pʰ, b/. It's unclear if this is an accidental gap, or if there's an active constraint against having /npʰ, nb/ complex codas. Such clusters can arise across difference syllables however (\S\ref{section:segmentalPhono:nasalPlace:other}). 
	
	
	The nasal /n/ can appear before velar /kʰ, ɡ/ (Table \ref{tab:compplex coda n k}). In this situation, the nasal becomes a velar [ŋ] (\S\ref{section:segmentalPhono:nasalPlace}). Note that the /kʰ/ can be part of   the same root as the nasal (written as \armenian{գ}). The nasal+stop can also be part of   common nominalizer suffixes /-ɑnkʰ, -unkʰ, -(ɑ)munkʰ/. The stop can also be part of a separate nominalizer suffix /-kʰ/. For all these latter cases,   the stop is written as \armenian{ք}. 
	
	
	\begin{table}[H]
		\centering
		\caption{Complex codas    where C1 is nasal /n/, and C2 is a velar stop}
		\label{tab:compplex coda n k}
		\begin{tabular}{|l|lll|l| }
			\hline 
			{}/nɡ/& ˈt͡sɑ\textbf{ŋɡ} & `list' & \armenian{ցանկ}&  n=203 \\
			& ɡəˈɾu\textbf{ŋɡ} & `heel' & \armenian{կրունկ} & \\ \hline 
			{}/nkʰ/ root& ˈɾu\textbf{ŋkʰ} & `nostril' & \armenian{ռունգ} &n=115
			\\
			& vəˈdɑ\textbf{ŋkʰ} & `danger' & \armenian{վտանգ}&  \\
			\hline
			{}/n-kʰ/  & ˈvɑ\textbf{ŋ-kʰ} & `convent' ($\sqrt{}$-{\nmlz}) & \armenian{վանք} &  n=759
			\\
			& cf. vɑn-ɑˈɡɑn &  `monastic'  ($\sqrt{}$-{\adjz}) & \armenian{վանական} &  
			\\
			% \hline 
			{}/-ɑnkʰ/ & ɑʃχɑˈd-ɑ\textbf{ŋkʰ} & `work'  ($\sqrt{}$-{\nmlz})& \armenian{աշխատանք} &  \\
			& cf. ɑʃχaˈd-i-l& `to work'  ($\sqrt{}$-{\thgloss}-{\infgloss}) & \armenian{աշխատիլ} &\\
			{}/-unkʰ/& həɾɑˈʃ-u\textbf{ŋkʰ} & `miracle'  ($\sqrt{}$-{\nmlz})& \armenian{հրաշունք} &  \\
			& cf. həˈɾɑʃ-k& `miracle'  ($\sqrt{}$-{\nmlz}) & \armenian{հրաշք}& \\
			% \hline 
			{}/-(ɑ)munkʰ/& bɑʃt-ɑmu\textbf{ŋkʰ} & `ceremony'  ($\sqrt{}$-{\nmlz})& \armenian{պաշտամունք} &  \\
			& cf. bɑʃˈt-e-l& `to worship'  ($\sqrt{}$-{\thgloss}-{\infgloss}) & \armenian{պաշտել}&
			
			\\ \hline 
		\end{tabular}
	\end{table}
	
	The nasal /n/ can precede any affricate /t͡s, d͡z, t͡ʃ, d͡z/ (Table \ref{tab:compplex coda n affr}).  
	
	
	\begin{table}[H]
		\centering
		\caption{Complex codas    where C1 is nasal /n/, and C2 is an affricate}
		\label{tab:compplex coda n affr}
		\begin{tabular}{|l|lll|l| }
			\hline 
			{}[nt͡s]& ˈdɑ\textbf{nt͡s} & `pear' & \armenian{տանձ} & n=134 \\
			& bəˈʁi\textbf{nt͡s} & `copper' &  \armenian{պղինձ} & \\
			\hline 
			{}[nd͡z]& ˈχɑ\textbf{nd͡z} & `bait' & \armenian{խանծ} &  n=25 \\ 
			& ɡoˈʁu\textbf{nd͡z} & `hard crust'  & \armenian{կողունծ}&   \\
			\hline 
			{}[nt͡ʃ] & ˈmɑ\textbf{nt͡ʃ} & `lad' & \armenian{մանչ} & n= 222 \\
			& ɑˈɡɑ\textbf{nt͡ʃ} & `ear' & \armenian{ականջ} & \\ 
			\hline 
			{}[nd͡ʒ] &ˈd͡ʒɑ\textbf{nd͡ʒ} & `fly' & \armenian{ճանճ} &  n=56 \\
			& jeˈʁi\textbf{nd͡ʒ} & `large nettle' & \armenian{եղինճ} & 
			\\ \hline 
		\end{tabular}
	\end{table}
	
	
	\subsubsection{Rhotic /ɾ/ + obstruent}\label{section:syllable:Final2C:FallingCommon:RhoticObs}
	The rhotic /ɾ/ can form a complex coda with a) any obstruent, and b) with /m/. The nasal /n/ however has complications in syllabification; postponed to Section \S\ref{section:syllable:Final2C:FallingOther:RhoticNasalN}. 
	
	The rhotic /ɾ/ can form a complex coda with any type of stop: /pʰ, b, tʰ, d, kʰ, ɡ/ (Table \ref{tab:compplex coda r stop}). For /kᵏ/, the stop can either be part of the root (written as \armenian{գ})  or arguably be the nominalizer suffix \textit{-kʰ} (written as \armenian{-ք}). 
	
	
	\begin{table}[H]
		\centering
		\caption{Complex codas    where C1 is rhotic /ɾ/, and C2 is a stop}
		\label{tab:compplex coda r stop}
		\begin{tabular}{|l|lll|l| }
			\hline 
			{}[ɾpʰ]  & ˈɑ\textbf{ɾpʰ} & `sunlight' & \armenian{արփ}& n=42 \\
			& ˈsu\textbf{ɾpʰ} & `holy' & \armenian{սուրբ} & \\ \hline
			{}[ɾb] &ˈd͡ʒɑ\textbf{ɾb} & `grease' & \armenian{ճարպ} &  n=216  \\
			& ˈd͡ze\textbf{ɾb} & `crevice' & \armenian{ծերպ} & \\ \hline 
			{}[ɾtʰ] & ˈvɑ\textbf{ɾtʰ} & `rose' & \armenian{վարդ} &  n=574 \\
			& zəˈvɑ\textbf{ɾtʰ} & `joyous' & \armenian{զուարթ} & \\ \hline 
			{}[ɾd] & ˈkʰo\textbf{ɾd} & `frog' & \armenian{գորտ} & n=298 \\
			& həˈbɑ\textbf{ɾd} & `proud' & \armenian{հպարտ} & \\ \hline 
			{}[ɾkʰ] & ˈhɑ\textbf{ɾkʰ} & `esteem' & \armenian{յարգ} & n=106 \\ 
			& ˈɡɑ\textbf{ɾkʰ} & `order' & \armenian{կարգ} & \\ 
			{}[ɾ-kʰ] &ˈpʰɑ\textbf{ɾ-kʰ} & `glory' ($\sqrt{}$-{\nmlz}) & \armenian{փառք} &   n=2756  \\ 
			& cf. pʰɑɾ-ɑ-zɑɾtʰ & `glorious' ($\sqrt{}$-{\lvgloss}-$\sqrt{}$) & \armenian{փառազարդ} & \\
			\hline 
			{}[ɾɡ] & ˈne\textbf{ɾɡ} & `paint' & \armenian{ներկ} &  n=187 \\ 
			&ɑˈd͡zɑ\textbf{ɾɡ} & `switch' & \armenian{ածարկ} & 
			\\ \hline 
		\end{tabular}
	\end{table}
	
	
	The rhotic /ɾ/ can be appear before any affricate /t͡s, d͡z, t͡ʃ, d͡ʒ/ (Table \ref{tab:compplex coda r affricate}). 
	
	
	\begin{table}[H]
		\centering
		\caption{Complex codas    where C1 is rhotic /ɾ/, and C2 is an affricate}
		\label{tab:compplex coda r affricate}
		\begin{tabular}{|l|lll|l| }
			\hline 
			{}[ɾt͡s] &ˈhɑ\textbf{ɾt͡s} & `issue' & \armenian{հարց} & n=187   \\
			& χoˈlo\textbf{ɾt͡s} & `orchid' & \armenian{խոլորձ} &  \\ \hline 
			{}[ɾd͡z]& ˈvo\textbf{ɾd͡z} & `belch' & \armenian{ործ}  & n=342 \\ 
			& ləˈbi\textbf{ɾd͡z} & `slippery' & \armenian{լպիրծ} & \\ \hline 
			{}[ɾt͡ʃ]  & ˈɑ\textbf{ɾt͡ʃ} & `bear' & \armenian{արջ}& n=42\\
			& hɑˈʁɑ\textbf{ɾt͡ʃ} & `currant' & \armenian{հաղարջ} & \\ \hline 
			{}[ɾd͡ʒ] &ˈɡo\textbf{ɾd͡ʒ} & `griffin' & \armenian{կորճ} &  n=37 \\
			& zəˈvɑ\textbf{ɾd͡ʒ} & `joyfully' & \armenian{զուարճ} & 
			\\ \hline 
		\end{tabular}
	\end{table}
	
	
	The rhotic can appear before the fricatives /s, z, ʃ, ʒ, χ/ (Table \ref{tab:compplex coda r fricative}). Complications arise for the other fricatives. 
	
	
	
	\begin{table}[H]
		\centering
		\caption{Complex codas    where C1 is rhotic /ɾ/, and C2 is a fricative}
		\label{tab:compplex coda r fricative}
		\begin{tabular}{|l|lll|l| }
			\hline 
			{}[ɾs] & ˈhɑ\textbf{ɾs} & `bride' & \armenian{հարս} & n=38 \\ 
			& ɑˈʁe\textbf{ɾs} & `supplication' & \armenian{աղերս} & \\ \hline 
			{}[ɾz] &ˈmɑ\textbf{ɾz} & `confine' & \armenian{մարզ} & n=14 \\
			& ˈbɑ\textbf{ɾz} & `simple' & \armenian{պարզ} & \\  \hline 
			{}[ɾʃ] & ˈkʰo\textbf{ɾʃ} & `gray' & \armenian{գորշ} & n=44 \\ 
			& ʃəˈʁɑ\textbf{ɾʃ} & `gauze' & \armenian{շղարշ} &  \\ \hline 
			{}[ɾʒ] &ˈvɑ\textbf{ɾʒ} & `accustomed' & \armenian{վարժ} & n=85 \\ 
			& ɑˈχo\textbf{ɾʒ}  & `pleasant' & \armenian{ախորժ} &  \\ \hline 
			{}[ɾχ] & ˈtʰɑ\textbf{ɾχ} & `sketch' &   \armenian{թարխ} & n=4 \\
			& ˈχo\textbf{ɾχ} & `hide' & \armenian{խորխ} &  \\ \hline 
		\end{tabular}
	\end{table}
	
	
	The \citeauthor{kouyoumdjian-1970-DictionaryArmenianEnglish} dictionary does have any words with final /ɾf, ɾv, ɾʁ/. For /ɾf, ɾv/, this is likely because these fricatives are quite rare in the first place. On Armenian Wiktionary, we've found a handful of words with final /ɾf, ɾv/. These all seem to be loanwords: [ɑlomoɾf]  `allomorph' \armenian{ալոմորֆ}, [neɾv] `nerve' \armenian{ներվ}. For /ɾʁ/, the handful of Wiktionary examples seem to be dialectal words that are absent from Western Armenian. 
	
	
	Orthographically, the rhotic can form a cluster with the fricative /h/, whether as \armenian{ռհ} or \armenian{րհ} (Table \ref{tab:compplex coda r h}). However, most words that have this final cluster don't pronounce the /h/, such that the [ɾh] pronunciation is archaic or obsolete. Only a subset of these words have the final /h/ still pronounced, thus creating a [ɾh] complex coda. See Section \S\ref{section:ortho:mismatch:hdeletion} for general data on this orthography-phonology mismatch. 
	
	
	\begin{table}[H]
		\centering
		\caption{Complex codas    where C1 is rhotic /ɾ/, and C2 is a fricative /h/}
		\label{tab:compplex coda r h}
		\begin{tabular}{|l|lll|l| }
			\hline 
			Silent <h> & ɑʃˈχɑ\textbf{ɾ} & `world' & \armenian{աշխարհ} &  n=38 \\ 
			& χoˈnɑ\textbf{ɾ} & `humble' & \armenian{խոնարհ} &  \\ \hline 
			Pronounced <h>  & ˈχo\textbf{ɾh} & `thought' & \armenian{խորհ} &   n=8 \\
			& ʒəˈbi\textbf{ɾh} & `insolent'  & \armenian{ժպիրհ} & \\ \hline 
		\end{tabular}
	\end{table}
	
	
	% րհ	ɾh
	% րխ	ɾχ
	
	% րմ	ɾm
	
	\subsubsection{Rhotic /ɾ/ + nasal /m/}\label{section:syllable:Final2C:FallingCommon:RhoticNasalM}
	Orthographically, there are many words that end in a rhotic  /ɾ/ + nasal /m/. These clusters are pronounced as [ɾm] without schwa epenthesis (Table \ref{tab:compplex coda r m}).  
	
	
	\begin{table}[H]
		\centering
		\caption{Complex codas    where C1 is rhotic /ɾ/, and C2 is a nasal /m/}
		\label{tab:compplex coda r m}
		\begin{tabular}{|l|lll|l| }
			\hline 
			{}[ɾm] & ˈzɑ\textbf{ɾm} & `tribe' & \armenian{զարմ}&  n=100 \\ 
			& 't͡ʃe\textbf{ɾm} & `warm' & \armenian{ջերմ} & \\
			& 'ɑ\textbf{ɾm} & `stamp' & \armenian{արմ} & \\
			& voˈʁo\textbf{ɾm} & `pity' & \armenian{ողորմ} & 
			\\ \hline 
		\end{tabular}
	\end{table}
	
	Word-medially however, the [ɾm] complex coda shows some idiosyncrasies (\S\ref{section:syllable:OtherCodaRestrictions:AvoidMedial:Cm}). 
	
	\subsubsection{Glide /j/ + consonant}\label{section:syllable:Final2C:FallingCommon:GlideCons}
	
	The glide /j/ can form a complex coda with   virtually any type of consonant. Though there are some accidental gaps in the \citeauthor{kouyoumdjian-1970-DictionaryArmenianEnglish} dictionary. 
	
	As a C1, the glide /j/ can precede virtually any type of stop, whether voiced or voiceless (Table \ref{tab:compplex coda j stop}). For final [jkʰ], the final /kʰ/ can either be part of the same morpheme as the glide (written as \armenian{յգ}), or part of a separate nominalizer suffix    \textit{-kʰ} (written as \armenian{յք}) . 
	
	\begin{table}[H]
		\centering
		\caption{Complex codas    where C1 is glide /j/, and C2 is a stop}
		\label{tab:compplex coda j stop}
		\begin{tabular}{|l|lll|l| }
			\hline 
			{}[jtʰ] & ˈχɑ\textbf{jtʰ} & `sting' & \armenian{խայթ} & n=95
			\\ 
			& məʃɑˈɡu\textbf{jtʰ} & `culture' & \armenian{մշակոյթ}& 
			\\ 
			\hline 
			{}[jd] &ˈɑ\textbf{jd} & `cheek' & \armenian{այտ} &   n=151
			\\ & bəˈdu\textbf{jd} & `tour' & \armenian{պտոյտ} & \\
			\hline 
			{}/jkʰ/   & ˈɑ\textbf{jkʰ} & `dawn' & \armenian{այգ}&  n=26
			\\ & ˈzu\textbf{jkʰ} & `twin' & \armenian{զոյգ} & \\\hline
			/j-kʰ/ &  ˈɡɑ\textbf{jkʰ} & `station' & \armenian{կայք} &  n=156
			\\ & cf.   ˈɡɑj  & `station' & \armenian{կայ} &   \\
			& həˈmɑ\textbf{jkʰ} & `charm' & \armenian{հմայք}  & 
			\\ & cf.  həmɑˈj-e-l & `to charm' & \armenian{հմայել} & \\ 
			\hline
			{}[jɡ] & ˈhɑ\textbf{jɡ} & masc. name & \armenian{Հայկ} &  n=20 \\
			& bɑˈɾu\textbf{jɡ} & question mark  & \armenian{պարոյկ}& \\ \hline 
		\end{tabular}
	\end{table}
	
	But as an accidental gap, the \citeauthor{kouyoumdjian-1970-DictionaryArmenianEnglish} dictionary doesn't list any final /jpʰ/ or /jb/ words. Such clusters however are not impossible, but they may be restricted to loanwords. For example, the name of the first Armenian letter is [ɑjpʰ] \armenian{այբ}, possibly a loanword of `alpha'. For /jb/, Armenian Wiktionary lists some such words (written with final \armenian{յպ})  but these seem to all be Russian loanwords. 
	
	
	The glide can precede an affricate (Table \ref{tab:compplex coda j aff}). The \citeauthor{kouyoumdjian-1970-DictionaryArmenianEnglish} dictionary lists word with a /j/ +  /t͡s, d͡z, d͡ʒ/. The dictionary lacks /jt͡ʃ/. This seems to be an accidental gap. Armenian Wiktionary likewise lacks words which would get pronounced with /jt͡ʃ/ in Western Armenian. 
	
	\begin{table}[H]
		\centering
		\caption{Complex codas    where C1 is glide /j/, and C2 is an affricate}
		\label{tab:compplex coda j aff}
		\begin{tabular}{|l|lll|l| }
			\hline 
			{}[jt͡s]& ˈɑ\textbf{jt͡s} & `visit' & \armenian{այց} &  n=221 \\
			& sɑˈɾu\textbf{jt͡s} &`frost' & \armenian{սառոյց} &  \\ \hline 
			{}[jd͡z] &ˈɑ\textbf{jd͡z} & `goat' & \armenian{այծ} &  n=71 \\
			& ɑɾˈɡɑ\textbf{jd͡z} & `wavering' & \armenian{առկայծ} & \\ \hline 
			{}[jd͡ʒ]  & bɑˈd͡ʒu\textbf{jd͡ʒ} & `adorned' & \armenian{պաճոյճ}& n= 22 \\ \hline
		\end{tabular}
	\end{table}
	
	
	
	
	The glide can also precede a voiced or voiceless fricative (Table \ref{tab:compplex coda j fric}). The \citeauthor{kouyoumdjian-1970-DictionaryArmenianEnglish} dictionary lists word with a /j/ +  /s, z, ʃ, ʒ/. 
	
	
	\begin{table}[H]
		\centering
		\caption{Complex codas    where C1 is glide /j/, and C2 is a fricative}
		\label{tab:compplex coda j fric}
		\begin{tabular}{|l|lll|l| }
			\hline 
			{}[js]& ˈhɑ\textbf{js} & `paste' &\armenian{հայս}  & n=220
			\\ 
			& ˈlu\textbf{js} & `light' & \armenian{լոյս} \\  
			\hline {}[jz] & ˈhu\textbf{jz} & `sap' & \armenian{հոյզ} &  n=84 \\
			& əŋˈɡu\textbf{jz} & `walnut' & \armenian{ընկոյզ} & \\
			\hline 
			{}[jʃ] & kʰəˈmu\textbf{jʃ} & `imagination' & \armenian{քմոյշ} & n=26
			\\ \hline 
			{}[jʒ] & ˈdu\textbf{jʒ} & `damage' & \armenian{տոյժ} & n=48 \\ 
			\hline 
			
		\end{tabular}
	\end{table}
	
	For final [ujʃ] and [ujʒ] sequences, such pronunciations are rather archaic for most roots. The modern language tends to turn such [ujʃ] sequences to [uʃ], such as archaic [ɑnujʃ] \armenian{անոյշ}  but modern [ɑnuʃ]  \armenian{անուշ}  `sweet'. Similarly, most words with  final [ujʒ] are pronounced with final  [uʒ], such as archaic [ujʒ] \armenian{ոյժ} vs. modern [uʒ] \armenian{ուժ}  `strength'. 
	
	The   fricatives /f, v, h/ are generally rare so their absence after /j/ is not surprising. Armenian Wiktionary listed a handful of words with final /jf, jv/; all of these are loanwords such as [sejv] `save' \armenian{սեյվ}  or [sejf]  `safe (n)' \armenian{սեյֆ}.  
	
	For the uvular /χ, ʁ/, these sounds aren't generally rare. The absence of /jχ/ or /jʁ/ may be an accidental gap. For example on Armenian Wiktionary, we found only two words that end in /jχ/, both of these are loanwords such as [ʃejχ] `sheikh' \armenian{շեյխ}.  For /jʁ/, Armenian Wiktionary only had one word [ʃujʁ] \armenian{շույղ} which was listed as a dialectal word, and thus wouldn't be in Western Armenian. 
	
	Finally, the glide can be precede  any other sonorant: /m, n, ɾ, l/ (Table \ref{tab:compplex coda j sonorant}). 
	
	
	\begin{table}[H]
		\centering
		\caption{Complex codas    where C1 is glide /j/, and C2 is a sonorant}
		\label{tab:compplex coda j sonorant}
		\begin{tabular}{|l|lll|l| }
			\hline 
			{}[jm] &ˈɡɑ\textbf{jm} & `mast' &  \armenian{կայմ} & n=10 \\ \hline
			{}[jn] &ˈt͡sɑ\textbf{jn} & `voice' & \armenian{ձայն} &  n=407 \\
			& sɑˈɡɑ\textbf{jn} & `but' & \armenian{սակայն} & \\ \hline 
			{}[jɾ] & ˈʒɑ\textbf{jɾ} & `rock' & \armenian{ժայռ} & n=282
			\\ & hɑmˈpʰu\textbf{jɾ} & `kiss' & \armenian{համբոյր} & \\ \hline 
			{}[jl] & ˈkʰɑ\textbf{jl} & `wolf' & \armenian{գայլ} &  n=138 \\
			& ʃəˈɾɑ\textbf{jl} & `prodigal' & \armenian{շռայլ} & \\ \hline 
		\end{tabular}
	\end{table}
	
	Note that for [jm], although this coda is possible, it seems very rare. For example, all of \citeauthor{kouyoumdjian-1970-DictionaryArmenianEnglish}'s examples were for compounds with the final root [ɡɑjm] `mast'. As we discuss elsewhere in Section \S\ref{section:syllable:Final2C:FlatRising:FricNasalM}), final [Cm] codas have quite complicated behaviors. Furthermore, word-medial [jm] codas seem even rarer (\S\ref{section:syllable:OtherCodaRestrictions:AvoidMedial:Cm}).  
	
	
	\subsection{Falling sonority but either rare,  extrasyllabic, or uses schwa epenthesis }\label{section:syllable:Final2C:FallingOther}
	The previous section looked final consonant clusters that were a) falling sonority, and b) were commonly syllabified as complex codas. This section goes through cases of falling sonority cluster that for some reason or another are either a) rare complex codas, b) potentially fake complex codas made up a coda and an appendix, or c) get an epenthetic schwa. Such clusters and their behavior are the following: 
	
	\begin{itemize}[noitemsep,topsep=0pt]
		\item Fricative /f,v/ + stop or affricate: rare, likely just accidental gaps (\S\ref{section:syllable:Final2C:FallingOther:FricFstop})
		\item Fricative /s,ʃ/ +     affricate: unattested, either   accidental gaps or banned (\S\ref{section:syllable:Final2C:FallingOther:FricFstop})
		\item Fricative /z,ʒ/ +     stop or affricate: unattested, likely  just  accidental  (\S\ref{section:syllable:Final2C:FallingOther:FricFstop})
		\item Fricative /h/ +     stop or affricate: rare, likely just coda + appendix  (\S\ref{section:syllable:Final2C:FallingOther:FricHstop})
		\item Nasal /m/ +    non-labial stop or affricate: rare, either generally banned or coda  + appendix  (\S\ref{section:syllable:Final2C:FallingOther:FricHstop})
		\item Nasal /m,n/ +    fricative: rare   word-finally; unclear if rarity is because of a ban or   just accidental gaps. Somewhat   avoided   word-medially  (\S\ref{section:syllable:Final2C:FallingOther:NasalFric})
		\item Rhotic /ɾ/ + nasal  /n/:  rare   word-finally, and often avoided with schwa epenthesis  (\S\ref{section:syllable:Final2C:FallingOther:RhoticNasalN})
		\item Lateral /l/ + obstruent: rare and most are analyzable as coda + appendix  (\S\ref{section:syllable:Final2C:FallingOther:LateralObs})
		\item Lateral /l/: nasal /m/: unattested outside of loanwords  (\S\ref{section:syllable:Final2C:FallingOther:LateralNasalM})
		\item Lateral /l/ + nasal /n/:  triggers    schwa epenthesis (\S\ref{section:syllable:Final2C:FallingOther:LateralNasalN})
	\end{itemize}
	
	\subsubsection{Fricative /f,v/ + stop or affricate }\label{section:syllable:Final2C:FallingOther:FricFstop}
	
	In general, the  fricatives /f, v/ seem to avoid being the first consonant in a complex coda. In the \citet{kouyoumdjian-1970-DictionaryArmenianEnglish} dictionary, we found very few words with such clusters. The words which existed are also low-frequency words. 
	
	When C1 is a fricative /f/, the C2 can be a stop /p, t, k/ (Table \ref{tab:compplex coda fC}).   /fp/ and /ft/  seem restricted to loanwords, especially Biblical loanwords as in the table below or other Semitic loanwords like [nɑft] `oil' \armenian{նավթ}.  /fk/  seems restricted to cases where the \textit{k} is the nominalizer suffix \textit{-kʰ} \armenian{-ք}. Thus the /fk/ cluster could arguably be treated as being a false complex coda, where \textit{f} is a coda but \textit{k} is an appendix. 
	
	\begin{table}[H]
		\centering
		\caption{Final CC clusters  where C1 is fricative /f/}
		\label{tab:compplex coda fC}
		\begin{tabular}{|l|lll|l| }
			\hline 
			{}[fp]  & ˈho\textbf{fp} & `Job' & \armenian{Յովբ} & n=1 \\
			\hline        {}[ft] & ˈnɑ\textbf{ft} & `naphta' & \armenian{նաւթ} & n=12 \\
			& behˈmo\textbf{ft} & `behemoth' & \armenian{բեհմովթ} & \\
			\hline {}[fk] & χoˈɾo\textbf{fk} &`roasting'  & \armenian{խորովք} &  n=29
			\\ & cf. χoɾoˈv-e-l & `to roast' & \armenian{խորովել} & \\
			\hline
		\end{tabular}
	\end{table}
	
	When C1 is /v/, C2 can be /b, d, ɡ/ (Table \ref{tab:compplex coda vC}). Again, these words are few and rare. Of the words in \citep{kouyoumdjian-1970-DictionaryArmenianEnglish}, a lot of these words are names of flora and fauna; these may possibly be old loanwords. 
	
	
	\begin{table}[H]
		\centering
		\caption{Final CC clusters    where C1 is fricative /v/}
		\label{tab:compplex coda vC}
		\begin{tabular}{|l|lll|l| }
			\hline 
			{}[vb]  & ʒiˈlɑ\textbf{vb} & `white broom (plant)' & \armenian{ժիլաւպ} & n=1 \\
			\hline       {}[vd] & ɑɾɑˈbo\textbf{vd} & `dried fig' & \armenian{արաբովտ} & n=5 \\
			& heɾesiˈjo\textbf{vd} & `heretic' & \armenian{հերեսիովտ} & \\
			\hline {}[vɡ] & mɑniˈjo\textbf{vɡ}&`madioc plant'  & \armenian{մանիովկ} &  n=2
			\\ & kʰəˈʁɑ\textbf{vɡ} & `small river fish ' & \armenian{գղաւկ} & \\
			\hline
		\end{tabular}
	\end{table}
	
	We found no cases of Western words with a labial fricative and then an affricate. One possible case is a non-Western dialectal word \armenian{թովջ} on Wiktionary, which seems to have something to do with taxes. We can at best guess that it's pronounced as [tʰoft͡ʃ]. The rarity of such cases suggests that Western Armenian just doesn't have such clusters. 
	\subsubsection{Fricative /s, ʃ/ +  affricate}\label{section:syllable:Final2C:FallingOther:FricSaff}
	It seems that /s,ʃ/ cannot form a complex coda with affricates. We found no final /s,ʃ/ + affricate examples in either the \citeauthor{kouyoumdjian-1970-DictionaryArmenianEnglish} dictionary or Armenian Wiktionary.  This may just be an accidental gap. But this could be also due to some constraint against having an /s,ʃ/ + affricate cluster because both the fricatives   and the affricate would have their own type of frication. To illustrate, the voiceless affricates are /t͡s, t͡ʃ/, and they both have end in a fricative-like element. Thus, it is possible that Armenian bans words like \textit{*ɑst͡s} in order to avoid a complex coda that both starts and ends with an \textit{s}-like element.\footnote{Wiktionary did have some Russian loanwords with a final /ʃt͡ʃ/ clusters, but  these don't exist in Western Armenian. And even if they did exist, loanwords often violate a language's phonological constraints. }
	
	
	\subsubsection{Fricative /z, ʒ/ + stop or affricate}\label{section:syllable:Final2C:FallingOther:FricZstop}
	There seems to be an accidental gap such that there are no words that end in a [zC] or [ʒC] cluster. 
	
	
	
	For [zC], the absence of final [zC] words may just be an accidental gap since a) Eastern Armenian can end in such sequences like [skizb] `beginning' <skizp> \armenian{սկիզբ}, and b) other fricatives like /s/ don't have such gaps. It is possible that this accidental gap arose via diachrony. Orthographically, a Western cluster [zb] would be written as \armenian{զպ} <zb>. But such an orthographic cluster would have to be pronounced as [zp] in Classical Armenian and Eastern Armenian; such a cluster is unattested for Eastern Armenian (= 0 hits on Wiktionary). In contrast, a [zb] cluster in Classical/Eastern would correspond to a [sp] cluster in Western: [əskisp] `beginning'. Thus, in the development of Classical to modern Western Armenian, [zb] sequences switched to being [sp], but no such original cluster could have changed to [zb]. 
	
	Note that although there are many words that end in an orthographic cluster of <z> plus a voiceless sound, such clusters are pronounced as voiceless: <azk> [ɑsk] `people' \armenian{ազգ}. See Section \S\ref{section:ortho:mismatch:clusters}. 
	
	Similar for [ʒC], it is unclear if Western Armenian either a) bans [ʒC] complex codas as a language-general rule, or if b) the absence of such clusters is an accidental gap. It is possible that the absence of such clusters is an accidental gap that's caused by diachrony, for the same reasons as for the absence of [zC] clusters.  
	
	
	
	
	
	\subsubsection{Fricative /h/ + stop or affricate}\label{section:syllable:Final2C:FallingOther:FricHstop}
	The fricative /h/ seems to avoid being in a complex coda (Table \ref{tab:compplex coda  hC}). In the \citep{kouyoumdjian-1970-DictionaryArmenianEnglish} dictionary, we found only 5 words that end in a falling-sonority /hC/ cluster. For C2, all these words involved the suffix \textit{-kʰ} \armenian{-ք}, suggesting that these words may instead be parsed as ending in a coda+appendix rather than a complex coda. 
	
	\begin{table}[H]
		\centering
		\caption{Final CC clusters    where C1 is fricative /h/}
		\label{tab:compplex coda  hC}
		\resizebox{\textwidth}{!}{%
			\begin{tabular}{|lll|lll| }
				\hline 
				ˈχɑ\textbf{h-k} & `kitchen' & \armenian{խահք}  & cf.  ˈχɑh & `dish' & \armenian{խահ} \\
				\hline 
				ˈbɑ\textbf{h-k} & `fasting' & \armenian{պահք} & cf.    ˈbɑh & `preservation'  & \armenian{պահ} 
				\\
				pʰeɾtʰ-ɑ-bɑ \textbf{h-k} & `garrison' & \armenian{բերդապահք} & cf.    ˈpʰeɾtʰ & `fortress'  & \armenian{բերդ} 
				\\
				ʃɑpʰɑtʰ-ɑ-bɑ \textbf{h-k} & `Sabbatarians' & \armenian{շաբաթապահք} & cf.    ʃɑpʰɑtʰ & `Saturaday'  & \armenian{շաբաթ} 
				\\
				\hline 
				
			\end{tabular}
		}
	\end{table}
	
	On Armenian Wiktionary, we found a handful more words with a final [hC] cluster, but these were all limited to specific non-standard dialects like Karabagh Armenian, thus their pronunciations cannot be extended to Western Armenian. 
	\subsubsection{Nasal /m/ + non-labial stop or affricate}\label{section:syllable:Final2C:FallingOther:NasalMstop}
	The nasal /m/ seems to   start clusters only with labial stops, and it avoids all other consonants.  
	
	For example, the \citeauthor{kouyoumdjian-1970-DictionaryArmenianEnglish} dictionary lists zero final /mtʰ, md/ clusters. It has only one final /mɡ/ word: [dɑmɡ] `damp' \armenian{տամկ}. The dictionary states that is word is derived from a synonymous [dɑmuɡ] \armenian{տամուկ}. Thus suggests that this word [dɑmɡ] is just a grammaticalized weak form of the larger [dɑmuɡ] word. Armenian Wiktionary provides a handful of examples of final orthographic <mt, mt', md>,   but these are all either loanwords or obscure dialectal words that aren't found in   Western Armenian. 
	
	
	Before velar /kʰ/, the nasal /m/ is found (Table \ref{tab:compplex coda  m k appendix}). \citeauthor{kouyoumdjian-1970-DictionaryArmenianEnglish} lists 18 final /mkʰ/ words. But, all of these examples involve the nominalizer suffix \textit{-kʰ} \armenian{-ք}. So these examples could arguably be syllabified as a coda + appendix \textit{-kʰ}. 
	
	
	\begin{table}[H]
		\centering
		\caption{Final CC clusters    where C1 is nasal /m/ and C2 is nominalizer /kʰ/}
		\label{tab:compplex coda  m k appendix}
		\begin{tabular}{|lll|lll| }
			\hline 
			ˈtʰe\textbf{m-kʰ} & `face' & \armenian{դէմք} & cf. ˈtʰem & `facing' & \armenian{դէմ} \\ 
			ɡɑ\textbf{m-kʰ} & `will' & \armenian{կամք} & cf. ɡɑˈm-i-l & `to will' ($\sqrt{}$-{\thgloss}-{\infgloss}) & \armenian{կամիլ}
			\\
			χəˈnɑ\textbf{m-kʰ} & `care' & \armenian{խնամք} & cf. χəˈnɑm & `care' & \armenian{խնամ}
			\\
			ʃoˈʁo\textbf{m-kʰ} & `flattery' & \armenian{շողոմք} & cf. ʃoˈʁom & `flattery' & \armenian{շողոմ}
			\\
			hɑˈme\textbf{m-kʰ} & `aromatics' & \armenian{համեմք} & cf. hɑˈmem & `aroma' & \armenian{համեմ}
			\\
			\hline 
			
		\end{tabular}
	\end{table}
	
	
	Neither \citeauthor{kouyoumdjian-1970-DictionaryArmenianEnglish} nor Wiktionary provided any examples of final /m/ + affricate clusters. 
	
	\subsubsection{Nasal /m,n/ + fricative}\label{section:syllable:Final2C:FallingOther:NasalFric}
	Nasals can form complex codas with stops and affricates. But it seems that nasals rarely  form complex codas with fricatives. In \citeauthor{kouyoumdjian-1970-DictionaryArmenianEnglish} dictionary, we  
	we found only 19 words that end in a <VNC>  cluster a) where N is a nasal, b) C is a fricative, and c) the fricative is part of the root (Table \ref{tab:compplex coda  m s appendix}).\footnote{This last condition is important because there are some words like <ims> [iməs] `mine' \armenian{իմս}, which although they end in an orthographic <ms> cluster, the <s> is actually     the possessive suffix  is \textit{-s}. This suffix always triggers schwa epenthesis after consonants (\textcolor{red}{cite chapter possessive schwa}.}   All 19 of these examples had the cluster /ms/. The nasal-fricative cluster is pronounced. 
	
	
	\begin{table}[H]
		\centering
		\caption{Final CC clusters    where C1 is nasal /m/ and C2 is fricative /s/}
		\label{tab:compplex coda  m s appendix}
		\begin{tabular}{|lll| }
			\hline 
			ˈko\textbf{ms} & `count' & \armenian{կոմս}  
			\\
			ˈdo\textbf{ms} & `ticket' & \armenian{տոմս}  
			\\
			tʰəˈɑm + ˈdo\textbf{ms} & `money + ticket' & \armenian{դրամ, տոմս}  
			\\
			$\rightarrow$   tʰəɾɑm-ɑ-ˈdo\textbf{ms} & `banknote'&   \armenian{դրամատոմս}  
			\\
			\hline 
			
		\end{tabular}
	\end{table}
	
	There are two generalization. First, it seems that the nasal /m/ can form a cluster with only the fricative /s/, and with no other fricative. Based on this, we argue that the /s/ in these words is actually an extrasyllabic appendix /-s/, not part of a complex coda. See Section \S\ref{section:syllable:Final2C:FlatRising} for more data on this appendix.  As counter-examples, we only found a few cases of orthographic <m>+fricative clusters on Armenian Wiktionary. All these were either  obvious loanwords like [tʰəɾijumf] `triumph' \armenian{տրիումֆ}, dialectal words, or words that can exist in Eastern Armenian but not Western. 
	
	Second, the nasal /n/ seems to avoid forming word-final complex codas with fricatives. We found zero such examples in \citeauthor{kouyoumdjian-1970-DictionaryArmenianEnglish}. On Armenian Wiktionary, the attested examples look primarily as either loanwords like [ɑlijɑns] `alliance' \armenian{ալիանս} or [oɾɑnʒ] `orange' \armenian{օրանժ}, or obscure dialectal words from non-Western Armenian. 
	
	However, it seems easier to find   /n/+fricative complex codas inside words than at the end of words (\S\ref{section:syllable:OtherCodaRestrictions:AvoidMedial:NasalFric}). Consider /nʃ/.  The native lexicon doesn't have any word-final [nʃ] complex codas. For example, Wiktionary lists a handful of such words and they're obvious loanwords: [ɾomɑnʃ] `Romansch' \armenian{ռոմանշ}. But it seems possible to create such clusters word-medially. Consider the word `pressure' [d͡ʒənʃ-um] \armenian{ճնշման}. The suffix /-um/ is a special nominalizer suffix. For words with this nominalizer suffix, the standard genitive form is created by replacing [-um] with [-m-ɑn]: [dʒənʃ-m-ɑn]. However in casual speech, HD observes that his speech almost always turns this [nʃ] sequence into [nt͡ʃ]. It's an open question if such  behavior  means that [nʃ] complex codas are truly absent from Armenian, or if the affrication is a type of low-level phonetic change. 
	
	Similarly, although we couldn't find a word with a final [nχ] cluster, we did find a few cases with word-medial [nχ]. For the latter category, the passive suffix /v/ can follow complex codas in Eastern Armenian:  [kɑnχ.vel] `to be anticipated' \armenian{կանխվել}, but not in Western Armenian [ɡɑn.χə.vil] \armenian{կանխուիլ}. This suggests that word-medial [nχ] are possible in principle, just rare, and they are subject to dialectal variation. 
	
	
	
	
	\subsubsection{Rhotic /ɾ/ + nasal /n/}\label{section:syllable:Final2C:FallingOther:RhoticNasalN}
	The rhotic /ɾ/ can form a complex coda with any obstruent and with the nasal /m/. But with the nasal /n/, we find the following complications: 
	\begin{itemize}[noitemsep, topsep=0pt]
		\item with monosyllabic roots, the /ɾn/ cluster preferably gets epenthesis [ɾən], but schwa-less [ɾn] is possible in casual  speech for some words
		\item with polysyllabic roots, the /ɾn/ cluster avoids epenthesis for some roots to get [ɾn]
		\item with compounds that have a monosyllabic final root, the /ɾn/ again prefers epenthesis [ɾən]
	\end{itemize}
	
	More restrictions are found when the /ɾn/ sequence is word-medially, discussed in Section \S\ref{section:syllable:OtherCodaRestrictions:AvoidMedial:rN}. 
	
	For the first group of `monosyllabic roots' (Table \ref{tab:compplex mono rm}), the root orthographically has one vowel followed by a cluster  <rn>  \armenian{րն} or <\.rn> \armenian{ռն}: <xa\.rn> `mixed'. Prescriptively, the final rhotic-nasal sequence is pronounced with an intervening epenthetic schwa: [χɑɾən]. This is the `prescriptive rule' to pronounce these words. In casual speech, the norm is to also use the schwa, but it is possible to delete the schwa: [χɑɾn]. \citeauthor{kouyoumdjian-1970-DictionaryArmenianEnglish}  lists 24 such roots. 
	
	
	\begin{table}[H]
		\centering
		\caption{Final CC clusters    where C1 is rhotic /ɾ/ and C2 is nasal  /n/ -- `monosyllabic' root}
		\label{tab:compplex mono rm}
		\begin{tabular}{|llll| }
			\hline 
			<xa\textbf{\.rn}> & [ˈχɑ\textbf{ɾən}] & `mixed' & \armenian{խառն} 
			\\
			<t͡se\textbf{\.rn}> & [ˈt͡se\textbf{ɾən}] & `hand' & \armenian{ձեռն} 
			\\
			<ta\textbf{\.rn}> & [ˈtʰɑ\textbf{ɾən}] & `bitter' & \armenian{դառն} 
			\\
			<sa\textbf{\.rn}> & [ˈsɑ\textbf{ɾən}] & `mountain' & \armenian{սառն} 
			\\
			<pu\textbf{\.rn}> & [ˈpʰu\textbf{ɾən}] & `fist' & \armenian{բուռն} 
			\\
			<t'o\textbf{\.rn}> & [ˈtʰo\textbf{ɾən}] & `grandson' & \armenian{թոռն} 
			\\
			\hline  
		\end{tabular}
	\end{table}
	
	
	It is an open question on how often the schwa-less forms are used in natural speech, and it's unknown what linguistic or extra-linguistic factors would condition the optional use of the schwa-less form.  
	
	Alongside the above orthographically monosyllabic roots, there are also orthographically polysyllabic roots (Table \ref{tab:compplex poly rm}). These roots contain two orthographic vowels, and end in a rhotic-nasal sequence: <eɣe\.rn> `crime'. However for pronunciation, the norm is to \textbf{not} add a schwa: [jeʁeɾn].  We found 23 such roots in \citeauthor{kouyoumdjian-1970-DictionaryArmenianEnglish}.  For some of these words though, the use of schwa is possible: [tʰitʰeɾ(ə)n] `butterfly'. 
	
	
	\begin{table}[H]
		\centering
		\caption{Final CC clusters    where C1 is rhotic /ɾ/ and C2 is nasal  /n/ -- `polysyllabic' root}
		\label{tab:compplex poly rm}
		\begin{tabular}{|llll| }
			\hline 
			<eɣe\.rn> & [jeˈʁe\textbf{ɾn}] & `crime' & \armenian{եղեռն}
			\\ 
			<t'it'e\.rn> & [tʰiˈtʰe\textbf{ɾn}] & `butterfly' & \armenian{թիթեռն}
			\\
			<aga\.rn> & [ɑˈɡɑ\textbf{ɾn}] & `citadel' & \armenian{ակառն}
			\\
			<gawa\.rn> & [ɡɑˈvɑ\textbf{ɾn}] & `trench' & \armenian{կաւառն}
			\\
			<ʃowa\.rn> & [ʃəˈvɑ\textbf{ɾn}] & `lance' & \armenian{շուառն}
			\\
			<lise\.rn> & [liˈse\textbf{ɾn}] & `axle-tree' & \armenian{լիսեռն}
			\\
			
			\hline  
		\end{tabular}
	\end{table}
	
	In terms of frequency, although there's equal numbers of both roots in \citeauthor{kouyoumdjian-1970-DictionaryArmenianEnglish}'s dictionary, the two groups have impressionistically difference usages. The monosyllabic roots are very common words, while the polysyllabic roots are all very low-frequency.  It is possible that the difference in prosodic behavior is tied with this frequency difference. 
	
	
	It is possible that the over-arching generalization is that word-final [ɾn] complex codas prefer being in minimally bisyllabic words, thus triggering epenthesis in a word like [χʰɑɾən] `mixed'. However, when compounds are formed from these /ɾn/-final words (Table \ref{tab:compplex mono rm compound}), we find that the compound inherits the schwa behavior of its component stems: [ɑvɑz-ɑ-ˈχɑɾən] `sand mixed'. \citeauthor{kouyoumdjian-1970-DictionaryArmenianEnglish} lists 148 such compounds with a final `monosyllabic' root. 
	
	
	
	
	\begin{table}[H]
		\centering
		\caption{Final CC clusters    where C1 is rhotic /ɾ/ and C2 is nasal  /n/ -- derived compounds}
		\label{tab:compplex mono rm compound}
		\begin{tabular}{|lll| }
			\hline 
			ɑˈvɑz + ˈχɑ\textbf{ɾən} &`sand + mixed' & \armenian{աւազ, խառն}
			\\
			$\rightarrow$ ɑvɑz-ɑ-ˈχɑ\textbf{ɾən}  & `sand mixed' & \armenian{ազատախառն} 
			\\
			sɑˈɡɑv + ˈχɑ\textbf{ɾən} &`few + hand' & \armenian{սակաւ, ձեռն}
			\\
			$\rightarrow$ sɑɡɑv-ɑ-ˈt͡se\textbf{ɾən}  & `sand mixed' & \armenian{սակաւաձեռն} 
			\\
			tʰeˈtʰev + ˈpʰe\textbf{ɾən} &`light + load' & \armenian{թեթեւ, բեռն}
			\\
			$\rightarrow$ tʰetʰev-ɑ-ˈpʰe\textbf{ɾən}  & `sand mixed' & \armenian{թեթեւաբեռն} 
			\\
			hɑˈɾʏɾ + ˈtʰu\textbf{ɾən} &`hundred + door' & \armenian{հարիւր, դուռն}
			\\
			$\rightarrow$ hɑɾʏɾ-ɑ-ˈtʰu\textbf{ɾən}  & `having a hundred doors' & \armenian{հարիւրադուռն} 
			\\
			\\
			\hline  
		\end{tabular}
	\end{table}
	
	Polysyllabic roots like [tʰitʰeɾ(ə)n]  butterfly also percolate their lack of a  schwa to compounds. But such compounds are even rarer than their component root. \citeauthor{kouyoumdjian-1970-DictionaryArmenianEnglish} lists  only 14 such compounds: [ɑt͡ʃ-ɑ-tʰiˈtʰe\textbf{ɾən}] `pavonian butterfly'. 
	
	
	\subsubsection{Lateral /l/ + obstruent}\label{section:syllable:Final2C:FallingOther:LateralObs}
	For the lateral /l/, this sound seems to avoid starting a complex coda in native Armenian words. 
	
	When C1 is /l/ and C2 is anything but the nominalizer /kʰ/ (Table \ref{tab:compplex coda l not k}), the \citeauthor{kouyoumdjian-1970-DictionaryArmenianEnglish} dictionary lists only 16 words. 13 look like obvious loanwords. The other 3 have an unclear origin. 
	
	\begin{table}[H]
		\centering
		\caption{Final CC clusters    where C1 is lateral /l/ and C2 is not nominalizer /kʰ/}
		\label{tab:compplex coda l not k}
		\begin{tabular}{|lll|lll| }
			\hline 
			ˈɑ\textbf{lpʰ} & `alpha' & \armenian{ալփ} & ɡe\textbf{lb} & `kelp' & \armenian{կելպ} 
			\\
			ˈvo\textbf{ltʰ} & `volt' & \armenian{վոլթ}&       ɑsˈpɑ\textbf{ld} & `asphalt' & \armenian{ասփալտ}
			\\
			ɡoˈpⁿɑ\textbf{ltʰ} & `cobalt' & \armenian{կոբալտ}   &əsˈpɑ\textbf{ld} & `spalt' & \armenian{սպալտ}   
			\\
			bɑˈsɑ\textbf{ld} & `basalt' & \armenian{պասալտ}&       bɑˈzɑ\textbf{ld}  & `basalt' & \armenian{պազալտ}
			\\
			ˈtʰɑ\textbf{lɡ} & `talc' & \armenian{թալկ}&     ˈtʰɑ\textbf{lkʰ} & `talc' & \armenian{թալք}
			\\
			
			ɡoˈbɑ\textbf{lɡ}& `copalche' & \armenian{կոպալկ} &      ˈvɑ\textbf{ls} & `waltz' & \armenian{վալս}
			\\
			ˈfi\textbf{lm} & `film' & \armenian{ֆիլմ}& &  &
			\\
			\hline 
			ˈkʰɑ\textbf{lχ} & `horned cumin' & \armenian{քալխ} & ˈʁu\textbf{lb} & `common gromwell' & \armenian{ղուլպ}
			\\
			sɑˈpɑ\textbf{ld} & `raw fruit' & \armenian{սաբալտ} & & & 
			\\
			\hline  
		\end{tabular}
	\end{table}
	
	Many such clusters are found on Armenian Wiktionary. Again, it seems that many of Wiktionary's examples are loanwords, obscure flora/fauna, or obscure dialectal words. Whether or not there are some Armenian dialects that allow /lC/ doesn't say anything about the avoidance of such clusters in Western Armenian. 
	
	The above data show that /lC/ clusters are generally restricted to non-native words (Table \ref{tab:compplex coda lk}).  Spurious counter-examples are words with a final /lkʰ/ sequence (32 words in \citeauthor{kouyoumdjian-1970-DictionaryArmenianEnglish}). But here, the /kʰ/ is part of the nominalizer suffix \textit{-kʰ} \armenian{-ք}; so all these words could be syllabified as ending in a coda /l/ + appendix /kʰ/. 
	
	
	\begin{table}[H]
		\centering
		\caption{Final CC clusters    where C1 is lateral /l/ and C2 is nominalizer /kʰ/}
		\label{tab:compplex coda lk}
		\begin{tabular}{|lll|lll| }
			\hline 
			ˈje\textbf{l-kʰ} & `ascent' & \armenian{ելք} & cf. jeˈl-ɑd͡z &`risen' ($\sqrt{}$-{\rptcp})  &\armenian{ելած}
			\\
			ˈχe\textbf{l-kʰ} & `mind' & \armenian{խելք} & cf. χel-ɑˈt͡si  &`smart' ($\sqrt{}$-{\adjz})     &\armenian{խելացի}
			\\
			ˈt͡so\textbf{l-kʰ} & `flash' & \armenian{ցոլք} & cf. t͡soˈl-ɑ-l    &`to flash' ($\sqrt{}$-{\thgloss}-{\infgloss})     &\armenian{ցոլալ}
			\\
			ɑɾˈkʰe\textbf{l-kʰ} & `obstacle' & \armenian{արգելք} & cf. ɑɾkʰeˈl-it͡ʃ     &  `preventative'   ($\sqrt{}$-{\adjz})     &\armenian{արգելիչ}\\
			deˈsi\textbf{l-kʰ} & `apparition' & \armenian{տեսիլք} & cf. deˈsil       &  `sight'  &\armenian{տեսիլ}
			
			\\
			\hline  
		\end{tabular}
	\end{table}
	
	\textcolor{red}{check macak to see whats wrong with the liquid}
	
	
	\subsubsection{Lateral /l/ + nasal /m/}\label{section:syllable:Final2C:FallingOther:LateralNasalM}
	
	We could not find any native words with a final <lm> in \citeauthor{kouyoumdjian-1970-DictionaryArmenianEnglish}. The only case we found was the obvious loanword [film] `film' \armenian{ֆիլմ}. Wiktionary likewise only had loanwords. 
	\subsubsection{Lateral /l/ + nasal /n/}\label{section:syllable:Final2C:FallingOther:LateralNasalN}
	The lateral /l/ generally avoids forming a complex coda with obstruents, with such clusters largely restricted to loanwords (\S\ref{section:syllable:Final2C:FallingOther:LateralObs}). When a word ends in an orthographic <ln> sequence (Table \ref{tab:compplex epen   ln}), we find obligatory schwa epenthesis. \citeauthor{kouyoumdjian-1970-DictionaryArmenianEnglish} lists only 13 such words. 
	
	
	\begin{table}[H]
		\centering
		\caption{Schwa epenthesis in final CC clusters    where C1 is lateral /l/ and C2 is nasal  /n/  }
		\label{tab:compplex epen   ln}
		\begin{tabular}{|llll| }
			\hline 
			<ow\textbf{ln}> & [ˈu\textbf{lən}] & `neck' & \armenian{ուլն}
			\\ 
			<anow\textbf{ln}> & [ɑˈnu\textbf{lən}] & `Spanish spider' & \armenian{անուլն}
			\\ 
			<part͡sea\textbf{ln}> & [pɑɾtˈɾjɑ\textbf{lən}] & `the Most High (God)' & \armenian{բարձրեալն}
			\\ 
			\hline  
		\end{tabular}
		
	\end{table}
	
	\subsection{Flat or rising sonority and either rare, extrasyllabic, or uses schwa epenthesis}\label{section:syllable:Final2C:FlatRising}
	Armenian orthography has many words that end in a consonant cluster that has flat or rising sonority. Some of these form rare complex codas, some are likely a coda + appendix, and some undergo schwa epenthesis. 
	
	\begin{enumerate}[noitemsep,topsep=0pt]
		\item When C1 is a stop:
		\begin{itemize}
			\item  + stop or affricate: rare and most are either loanwords or coda + appendix /-kʰ/ (\S\ref{section:syllable:Final2C:FlatRising:StopStop})
			\item  + fricative: rare, some are likely a coda + appendix, and some take schwa epenthesis  (\S\ref{section:syllable:Final2C:FlatRising:StopFric})
			\item  + nasal: either loanword or takes schwa epenthesis  (\S\ref{section:syllable:Final2C:FlatRising:StopNasal})
			\item  + rhotic /ɾ/: usually schwa epenthesis, but schwa-less forms are possible (\S\ref{section:syllable:Final2C:FlatRising:StopRhotic})
			\item  + lateral /l/: loanwords and takes schwa epenthesis (\S\ref{section:syllable:Final2C:FlatRising:StopLateral})
		\end{itemize}
		\item When C1 is an affricate:
		\begin{itemize}
			\item   + stop: rare and most are either loanwords or coda + appendix /-kʰ/ (\S\ref{section:syllable:Final2C:FlatRising:AffrStop})
			\item  + fricative: rare and most   undergo schwa epenthesis  (\S\ref{section:syllable:Final2C:FlatRising:AffrFric})
			\item  + sonorant: rare and undergoes schwa epenthesis  (\S\ref{section:syllable:Final2C:FlatRising:AffrSono})
		\end{itemize}
		\item When C1 is a fricative:
		\begin{itemize}
			\item Fricative + fricative: most use a fricative appendix, and some have  schwa epenthesis  (\S\ref{section:syllable:Final2C:FlatRising:FricFric})
			\item  + nasal /m/: relatively common and likely just coda + appendix (\S\ref{section:syllable:Final2C:FlatRising:FricNasalM})
			\item  + nasal /n/: relatively common and undergoes schwa epenthesis (\S\ref{section:syllable:Final2C:FlatRising:FricNasalN})
			\item  + rhotic /ɾ/: rare  and  undergoes schwa epenthesis, with possible schwa elision (\S\ref{section:syllable:Final2C:FlatRising:FricRhotic})
			\item  + lateral /l/: rare  and  undergoes schwa epenthesis (\S\ref{section:syllable:Final2C:FlatRising:FricLateral})
		\end{itemize}
		\item When C1 is a nasal:
		\begin{itemize}
			\item  + nasal /m/: rare  and  undergoes schwa epenthesis (\S\ref{section:syllable:Final2C:FlatRising:NasalNasalM})
			\item  + nasal /n/: common  and  undergoes schwa epenthesis (\S\ref{section:syllable:Final2C:FlatRising:NasalNasalN})
			\item  + rhotic /ɾ/: rare  and  undergoes schwa epenthesis (\S\ref{section:syllable:Final2C:FlatRising:NasalRhotic})
			\item  + lateral /l/: rare  and  undergoes schwa epenthesis (\S\ref{section:syllable:Final2C:FlatRising:NasalLateral})
		\end{itemize}
	\end{enumerate}
	\subsubsection{Stop + stop or affricate}\label{section:syllable:Final2C:FlatRising:StopStop}
	Orthographically, final   stop + affricate clusters are exceedingly rare. We found no cases in \citeauthor{kouyoumdjian-1970-DictionaryArmenianEnglish}. As for Armenian Wiktionary, they seem restricted to loanwords like [əskott͡ʃ$\sim$əskot͡ʃ] `scotch tape' \armenian{սքոթչ}.
	
	We did find  cases of final stop-stop clusters. Such final stop-stop clusters  are limited to three categories: some roots, some loanwords, and an open class of words where the nominalizer \textit{-kʰ} is added after root-final stop. 
	
	For the first category, we found some native roots on Wiktionary, such as     [t͡sɑdɡ] `jump' \armenian{ցատկ} and [dɑɡd] `musical bar' \armenian{տակտ} on Wiktionary. \citeauthor{kouyoumdjian-1970-DictionaryArmenianEnglish} did not have any such roots.  Word-medially, a handful more cases are found like [d͡z\textbf{əb}d.jɑl] `incognito' \armenian{ծպտեալ}, but HD fees that the avoidance of this complex coda feels more common: [d͡zə\textbf{b.dj}ɑl]. 
	
	
	For the second  category,   the \citeauthor{kouyoumdjian-1970-DictionaryArmenianEnglish} dictionary listed only two words which had  stop-stop coda (Table \ref{tab:flat complex stop stop }). These words seem to be loanwords judging by how their Armenian form was very similar to their translation. 
	
	
	\begin{table}[H]
		\centering
		\caption{Final CC clusters    where C1    and C2 are stops   }
		\label{tab:flat complex stop stop }
		\begin{tabular}{|lll| }
			\hline 
			%  ˈd͡zɑ\textbf{d͡zɡ} & `covered' & \armenian{սալածածկ} \\ 
			t͡ʃɑlˈʁu\textbf{bd} & `true jalap' & \armenian{ջալղուպտ} \\ 
			kʰɑˈɾi\textbf{pt} & `charpybdis' & \armenian{քարիբդ} \\ 
			\hline  
		\end{tabular}
		
	\end{table}
	
	More loanwords are found on Wiktionary, such as [fɑɡd] `fact' \armenian{ֆակտ}. 
	As for the third category, 
	The segment \textit{kʰ} is special in how it can follow any type of consonant or consonant-cluster. This is because Armenian has a nominalizer suffix \textit{-kʰ} \armenian{-ք} that is used to form many nouns. This suffix can be added after any consonant even if it creates a flat-sonority cluster. When this suffix is added, it  triggers the devoicing of preceding obstruents (\S\ref{section:segmentalPhono:allphonLaryng:assiimlation:regFinal}). 
	
	
	In \citeauthor{kouyoumdjian-1970-DictionaryArmenianEnglish}'s dictionary, the nominalizer \textit{-kʰ} is quite common after any stop: [d͡zu\textbf{p-k}] `fluctuation' (Table \ref{tab: appendix stop k }). It triggers devoicing of preceding stops: [jeɾɑχʃe\textbf{p-k}] `torment'. It can also follow a velar stop /kʰ, ɡ/ to create a long (geminate) version of itself: [kʰɑ\textbf{k-kʰ}] `separation'. 
	
	\begin{table}[H]
		\centering
		\caption{Final CC clusters    where C1 is a stop   and C2 is an appendix  /kʰ/  }
		\label{tab: appendix stop k }
		\begin{tabular}{|l|lll|l|  }
			\hline 
			/p-kʰ/$\rightarrow$[p-k] & ˈd͡zu\textbf{p} & `fluctuation' ($\sqrt{}$) & \armenian{ծուփ}  & n=3
			\\ 
			&  ˈd͡zu\textbf{p-k} & `fluctuation' ($\sqrt{}$-{\nmlz}) & \armenian{ծուփք} &
			\\ 
			\hline   
			/b-kʰ/$\rightarrow$[p-k] & jeɾɑχˈʃe\textbf{b} & `scar' ($\sqrt{}$) & \armenian{երաշխէպ}  & n=7
			\\ 
			&  jeɾɑχˈʃe\textbf{b-k} & `torment' ($\sqrt{}$-{\nmlz}) & \armenian{երաշխէպք} &
			\\ 
			\hline  
			/t-kʰ/$\rightarrow$[t-k] & ɑʁoˈ\textbf{tʰ}-e-l & `to pray' ($\sqrt{}$-{\thgloss}-{\infgloss}) & \armenian{աղօթել}  & n=15
			\\ 
			&  ɑˈʁo\textbf{t-k} & `prayer' ($\sqrt{}$-{\nmlz}) & \armenian{աղօթք} &
			\\ 
			\hline  
			/d-kʰ/$\rightarrow$[t-k] & ɑˈʁe\textbf{d} & `misfortune' ($\sqrt{}$) & \armenian{աղէտ}  & n=42
			\\ 
			&  ɑˈʁe\textbf{p-k} & `calamity' ($\sqrt{}$-{\nmlz}) & \armenian{աղէտք} &
			\\ 
			\hline  
			/k-kʰ/$\rightarrow$[k-kʰ] & ˈho\textbf{kʰ} & `concern' ($\sqrt{}$) & \armenian{հոգ}  & n=5
			\\ 
			&  ˈho\textbf{k-kʰ} & `hindrance' ($\sqrt{}$-{\nmlz}) & \armenian{հոգք} &
			\\ 
			\hline  
			/k-kʰ/$\rightarrow$[k-kʰ] & ˈho\textbf{kʰ} & `concern' ($\sqrt{}$) & \armenian{հոգ}  & n=39
			\\ 
			&  ˈho\textbf{k-kʰ} & `hindrance' ($\sqrt{}$-{\nmlz}) & \armenian{հոգք} &
			\\ 
			\hline  
			/ɡ-kʰ/$\rightarrow$[k-kʰ] & ˈkʰɑ\textbf{kʰ} & `to untie' ($\sqrt{}$-{\thgloss}-{\infgloss}) & \armenian{քակել}  & n=39
			\\ 
			&  ˈkʰɑ\textbf{k-kʰ} & `separation' ($\sqrt{}$-{\nmlz}) & \armenian{քակք} &
			\\ 
			
			\hline  
		\end{tabular}
		
	\end{table}
	
	
	
	
	
	
	\subsubsection{Stop + fricative }\label{section:syllable:Final2C:FlatRising:StopFric}
	It is relatively rare to find word-final orthographic clusters that end in a stop + fricative. In the \citeauthor{kouyoumdjian-1970-DictionaryArmenianEnglish} dictionary, we found the following categories:   stop + /s/, stop +  /χ/, and stop + /ʁ/. For /Cs/ and /Cχ/,   the stop-fricative cluster is pronounced together without schwa epenthesis. The final fricative acts as an appendix. But for /Cʁ/, we find schwa epenthesis
	
	
	For stop + /s/, we found many examples (Table \ref{tab: appendix stop s }). The stop can be a voiceless [p, t, k]. 
	
	\begin{table}[H]
		\centering
		\caption{Final CC clusters    where C1 is a stop   and C2 is a fricative /s/ }
		\label{tab: appendix stop s }
		\begin{tabular}{|l|lll|l|  }
			\hline 
			{}[ps] & ˈtʰi\textbf{ps} & `sirup' & \armenian{տիբս} & n=4 \\
			& jeˈɾe\textbf{ps} & `priest' & \armenian{երեփս} & \\ \hline 
			{}[ts] & hɑnˈbe\textbf{ds} & `vainly' & \armenian{յանպէտս} & n=4 \\
			& umˈbe\textbf{ds} & `uselessly' & \armenian{ումպէտս} & \\\hline
			{}[ks] & meˈdɑ\textbf{ks} & `silk' & \armenian{մետաքս} & n=19 \\
			& əstɑˈmo\textbf{ks} & `stomach'  & \armenian{ստամոքս} & \\
			\hline  
		\end{tabular}
		
	\end{table}
	
	For the stop + /s/ clusters, it's difficult to tell how many of these are purely native words vs. old nativized borrowings, such as the word for [əstɑmoks] `stomach' which has an old Greek origin \citep[269]{Adjarian-1979-Etymology}, or the obvious loanword [ɡeɡɾops] `Cecrops' \armenian{Կեկրոպս}. 
	
	For some words, the final /s/ is potentially ambiguous with a possessive suffix /-s/. This suffix is pronounced with a schwa after a consonant: /pʰɑɾ-s/ [pʰɑɾ-əs] `my finger' \armenian{բառս}.  For an archaic low frequency word like  <yeds> \armenian{յետս} `behind', it wouldn't be surprising if some speakers parsed this word as one root  [hets], or as a complex word with a root /hed/ `with' and possessive suffix /s/: [hedəs]. On Armenian Wiktionary, we've found variation in how   orthographic stop+/s/ clusters are transcribed, suggesting this same ambiguity from morphology. 
	
	For stop + /χ/, we found only one example in \citeauthor{kouyoumdjian-1970-DictionaryArmenianEnglish}: [dipχ] `common mistletoe' \armenian{տիպխ}. 
	
	For stop + /ʁ/, these are relatively few (Table \ref{tab: schwa stop ʁ }).  They take schwa epenthesis. 
	
	
	\begin{table}[H]
		\centering
		\caption{Schwa epenthesis in final CC clusters    where C1 is a stop and C2 is a fricative /ʁ/}
		\label{tab: schwa stop ʁ }
		\begin{tabular}{|l|llll|l|  }
			\hline 
			<bɣ>$\rightarrow$[bəʁ] & <gow\textbf{bɣ}>  & ˈɡu\textbf{bəʁ} & `padlock'  & \armenian{կուպղ} & n=2
			\\  \hline 
			<dɣ>$\rightarrow$[dəʁ] & <si\textbf{dɣ}>  & ˈsi\textbf{dəʁ} & `waterpot'  & \armenian{սիտղ} & n=8 
			\\  \hline 
			<kɣ>$\rightarrow$[kʰəʁ] & <ʃi\textbf{kɣ}>  & ˈʃi\textbf{kʰəʁ} & `hock'  & \armenian{շիգղ} & n=3
			\\  \hline 
			<gɣ>$\rightarrow$[ɡəʁ] & <si\textbf{gɣ}>  & ˈsi\textbf{ɡəʁ} & `shekel'  & \armenian{սիկղ} & n=1
			\\  \hline 
			
		\end{tabular}
		
	\end{table}
	
	Based on comparing the \citeauthor{kouyoumdjian-1970-DictionaryArmenianEnglish} dictionary against Wiktionary, it seems that all other possible stop + fricative clusters are either unattested, found in non-Western dialects, or are loanwords: [vakf] `waqf' \armenian{վակֆ}. 
	\subsubsection{Stop + nasal}\label{section:syllable:Final2C:FlatRising:StopNasal}
	There are words which end in an orthographic stop + nasal cluster. In native words,   the nasal is always /n/. We could not find  native words   with /m/. 
	
	For final stop + /m/ clusters, \citeauthor{kouyoumdjian-1970-DictionaryArmenianEnglish} had no such examples. On Wiktionary, we found some examples but they all seem like loanwords: [ɾitʰm$\sim$ɾitʰəm] `rhythm' \armenian{ռիթմ} with unclear syllabification. 
	
	As for stop + /n/,  we always find schwa epenthesis in pronunciation (Table \ref{tab: schwa stop n }). Many of these words though have an archaic connotation, or are restricted to high-level formal registers.  
	
	
	\begin{table}[H]
		\centering
		\caption{Schwa epenthesis in final CC clusters    where C1 is a stop and C2 is a nasal /n/}
		\label{tab: schwa stop n }
		\resizebox{\textwidth}{!}{%
			\begin{tabular}{|l|llll|l|  }
				\hline 
				<p(')n>$\rightarrow$[pʰən] & <a\textbf{p'n}>  & ˈɑ\textbf{pʰən} & `shore (archaic)' & \armenian{ափն} & n=8
				\\  \hline 
				<t(')n>$\rightarrow$[tʰən] & <ka\textbf{t'n}>  & ˈɡɑ\textbf{tʰən} & `milk (archaic)' & \armenian{կաթն} & n=3
				\\  \hline 
				<dn>$\rightarrow$[dən] & <o\textbf{dn}> & voˈ\textbf{dən} & `foot  (archaic)' & \armenian{ոտն}  & n=103
				\\ 
				& <ma\textbf{dn}> & mɑˈ\textbf{dən} & `finger (archaic)' & \armenian{մատն}  & 
				\\  \hline 
				<k(')n>$\rightarrow$[kʰən] & <yantow\textbf{kn}> & hɑnˈtʰu\textbf{kʰən} & `audacious' & \armenian{յանդուգն} &   n=18 \\  
				& <yo\textbf{kn}> & ˈho\textbf{kʰən} & `numerous'& \armenian{յոքն} &  
				\\ \hline
				<g(')n>$\rightarrow$[ɡən] &<t͡s'a\textbf{gn}> & ˈt͡sɑ\textbf{ɡən} & `misery' & \armenian{ցակն} &   n=76
				\\
				& <areka\textbf{gn}> & ɑɾeˈkʰɑ\textbf{ɡən} & `sun' & 
				\\
				\hline  
			\end{tabular}
		}
	\end{table}
	
	We could not find final <bn>$\rightarrow$[bən] clusters on either \citeauthor{kouyoumdjian-1970-DictionaryArmenianEnglish} or Wiktionary,  but this is likely an accidental gap. 
	
	For final <dn>-[dən], \citeauthor{kouyoumdjian-1970-DictionaryArmenianEnglish} listed 103 such words. The vast majority of them (n=99) were derived from the words   `foot' or finger':  [vɑɾtʰ-ɑ-mɑdən] `rose-fingered' \armenian{վարդամատն}. 
	
	The reason why many of these words have an archaic connotation is because of word-final nasal deletion in final <Cn> cluster. Such a nasal deletion happened in the development of Classical Armenian to Modern Armenian. Thus the Classical word for `finger' is traditionally pronounced as [mɑdən] \armenian{մատն}, while the modern common word is just [mɑd] \armenian{մատ}. The nasal is retained in archaic-sounding words and their derivatives. See \textcolor{red}{cite nasal liaison} for more discussion on such nasals. 
	\subsubsection{Stop + rhotic /ɾ/}\label{section:syllable:Final2C:FlatRising:StopRhotic}
	Word-finally, stop + /ɾ/ usually undergo schwa epenthesis (Table \ref{tab: schwa stop rhotic  }). This is the prescriptive norm, but there is a degree of variation.  The rhotic /ɾ/ can be orthographically either <r> \armenian{ր} or <ṙ> \armenian{ռ}. 
	
	
	\begin{table}[H]
		\centering
		\caption{Schwa epenthesis in final CC clusters    where C1 is a stop and C2 is a rhotic /ɾ/}
		\label{tab: schwa stop rhotic  }
		\begin{tabular}{|l|llll|l|  }
			\hline 
			<pr>$\rightarrow$[pʰəɾ] & <i\textbf{pr}> & ˈi\textbf{pʰəɾ} & `as' & \armenian{իբր} & n=1 
			\\ \hline 
			<br,bṙ>$\rightarrow$[bəɾ] & goi\textbf{br}> & ˈɡu\textbf{bəɾ} & `tar' & \armenian{կուպր} & n=4
			\\ \hline 
			<t(')ṙ>$\rightarrow$[tʰəɾ] & <k'ara\textbf{tr}> & kʰɑˈɾɑ\textbf{tʰəɾ} & `plover'  & \armenian{քարադր} &  n=14 
			\\ 
			& <mē\textbf{t'r}> & ˈme\textbf{tʰəɾ} & `meter'& \armenian{մէթր} & 
			\\ \hline 
			<dr>$\rightarrow$[dəɾ] & <nō\textbf{dr}> & ˈno\textbf{dəɾ} & `cursive'  & \armenian{նօտր} &  n=33
			\\ 
			& <a\textbf{dr}> & ˈɑ\textbf{dəɾ} & `flame'& \armenian{ատր} & 
			\\ \hline 
			<k'r>$\rightarrow$[kʰəɾ] & <da\textbf{kr}> & ˈdɑ\textbf{kʰəɾ} & `brother-in-law' & \armenian{տագր} & n=6 
			\\ \hline 
			<gr,gṙ>$\rightarrow$[ɡəɾ] & <sa\textbf{gr}> & ˈsɑ\textbf{ɡəɾ} & `axe' & \armenian{սակր} & n=3
			\\ \hline 
		\end{tabular}
		
	\end{table}
	
	Prescriptively, the norm is to have a schwa in a word-final stop-rhotic cluster: <p'ok'r> $\rightarrow$ [pʰokʰəɾ] `small' \armenian{փոքր}. But in casual speech, some of these words can be pronounced without a schwa: [pʰokʰɾ]. The absence of this schwa is possibly tied with schwa elision (\S\ref{section:segmentalPhono:sandhi:schwaElision}). The absence of this schwa does however vary across the types of stops. For example,   [kʰəɾ] sequences can easily simplify to [kʰɾ] in speech: [vɑkʰəɾ, vɑkʰɾ] `tiger' \armenian{վագր}. As can [tʰəɾ] sequences and [dəɾ] sequences: [metʰəɾ, metʰɾ] `meter' \armenian{մէթր}, [dedəɾ, dedɾ] `tract' \armenian{տետր}. But [pʰəɾ] clusters don't: [ipʰəɾ] `as' \armenian{իբր}. 
	
	\subsubsection{Stop + lateral /l/}\label{section:syllable:Final2C:FlatRising:StopLateral}
	Falling or rising sonority clusters with a final /l/ are quite rare. In the \citeauthor{kouyoumdjian-1970-DictionaryArmenianEnglish} dictionary, we found only 2 words that ended in an orthographic stop + <l> cluster. One is an obvious borrowing and gets schwa epenthesis:  $<$kowntsdabl$>\rightarrow$[kʰuntʰəstɑbəl] `constaple'  \armenian{գունդստապլ}. The other is a possible native word. We're unfamiliar with this word but we think it should be pronounced with a schwa as well: $<$mowgl$>\rightarrow$[muɡəl] `false myrrh' \armenian{մուկլ}. 
	
	Outside of dictionary, we have come across other loanwords with final orthographic <Cl> cluster. Such clusters get epenthesis again: <t͡s'igl>$\rightarrow$[t͡siɡəl] `cycle' \armenian{ցիկլ}. 
	\subsubsection{Affricate + stop }\label{section:syllable:Final2C:FlatRising:AffrStop}
	Affricates and stops both have low sonority. They usually cannot form a complex coda together. There are two classes of exceptions: a handful of  roots , a handful of loanwords, and a large set of appendix-final words. This third category   consists of words that end in the nominalizer \textit{-kʰ}. 
	
	For the first   class, the \citeauthor{kouyoumdjian-1970-DictionaryArmenianEnglish} dictionary listed only one word  which has an affricate-stop coda. That word is a compound [sɑl-ɑ-d͡zɑd͡zɡ] `paved with stones' that used the root [d͡zɑd͡zɡ] `cover' \armenian{ծածկ}. 
	
	
	% \begin{table}[H]
		%     \centering
		%     \caption{Final CC clusters    where C1 is  an affricate   and C2 is a stop   }
		%     \label{tab:flat complex affr stop }
		%       \begin{tabular}{|lll| }
			%      \hline 
			%      ˈd͡zɑ\textbf{d͡zɡ} & `covered' & \armenian{սալածածկ} \\ 
			%     %  t͡ʃɑlˈʁu\textbf{bd} & `true jalap' & \armenian{ջալղուպտ} \\ 
			%     %  kʰɑˈɾi\textbf{pt} & `charpybdis' & \armenian{քարիբդ} \\ 
			
			% \hline  
			%     \end{tabular}
		
		% \end{table}
	
	For the second category of loanwords,  Armenian Wiktionary list some examples such as   [tonet͡sk] `Donetsk' \armenian{Դոնեցկ}.
	
	The third category are words with the nominalizer   \textit{-kʰ}, which violates all the syllable structure rules of Armenian (Table \ref{tab: appendix affr k }). This suffix can follow any consonant, including affricates, and it triggers devoicing. This suffix is analyzed as being an extrasyllabic segment simply because it consistently acts in a bizarre fashion.\footnote{Armenian Wiktionary also lists stop-stop clusters that are  were dialectal non-Western words which seemed to use a cognate of the nominalizer \textit{-kʰ}. For example, we speculate that the entry <t'inamet͡ʃ'g>  \armenian{թինամեչկ} is one such entry, and Wiktionary just defines it as a possible cognate to some word \armenian{թիկնամեջք} <t'iknamet͡ʃ'k'> which ends in the nominalizer \textit{-kʰ}.    }
	
	\begin{table}[H]
		\centering
		\caption{Final CC clusters    where C1 is an affricate   and C2 is an appendix  /kʰ/  }
		\label{tab: appendix affr k }
		\begin{tabular}{|l|lll|l|  }
			\hline 
			/t͡s-kʰ/$\rightarrow$[t͡s-k] & t͡ʃɑˈχɑ\textbf{t͡s} & `mill'     ($\sqrt{}$)  & \armenian{ջաղաց}&n=55 \\
			& t͡ʃɑˈχɑ\textbf{t͡s-k} & `mill' ($\sqrt{}$-{\nmlz})  & \armenian{ջաղացք}&  
			\\ 
			\hline  
			/d͡z-kʰ/$\rightarrow$[t͡s-k] & dɑɾɑˈ\textbf{d᷂͡z}-e-l & `to spread' ($\sqrt{}$-{\thgloss}-{\infgloss})  & \armenian{տարածել}&n=71 \\
			& dɑɾɑˈ\textbf{t͡s-k} & `spread' ($\sqrt{}$-{\nmlz})  & \armenian{տարածք}&  
			\\ 
			\hline  
			/t͡ʃ-k/$\rightarrow$[t͡ʃ-k] & χɑχɑˈ\textbf{t͡ʃ}-e-l & `to gargle' ($\sqrt{}$-{\thgloss}-{\infgloss})  & \armenian{խախաջել}&n=23 \\
			& χɑˈχɑ\textbf{t͡ʃ-k} & `gargle' ($\sqrt{}$-{\nmlz})  & \armenian{խախաջք}&  
			\\ 
			\hline  
			/d͡ʒ-kʰ/$\rightarrow$[t͡ʃ-k]   &  ˈd͡ʒo\textbf{d͡ʒ} &  `oscillation' ($\sqrt{}$)  & \armenian{ճօճ}&n=4 \\
			& ˈd͡ʒo\textbf{t͡ʃ-k} & `swing' ($\sqrt{}$-{\nmlz})  & \armenian{ճօճք}& 
			\\ 
			\hline  
		\end{tabular}
		
	\end{table}
	
	Note that the sequence [t͡sk] is quite common because the nominalizer \textit{-kʰ} can be added after the resultative participle  suffix \textit{-ɑd͡z}, and cause devoicing: [ɑjɾ-ɑ\textbf{d͡z}] `burnt' ($\sqrt{}$-{\rptcp} \armenian{այրած} and [ɑjr-ɑ\textbf{t͡s-k}]  `scald' ($\sqrt{}$-{\rptcp}-{\nmlz}) \armenian{այրածք}. 
	
	\subsubsection{Affricate + fricative }\label{section:syllable:Final2C:FlatRising:AffrFric}
	It is relatively rare to find final orthographic clusters of an affricate + fricative. In the \citeauthor{kouyoumdjian-1970-DictionaryArmenianEnglish} dictionary, we found only one word [pʰit͡ʃχ] `common ivy' \armenian{փիչխ} where the affricate-fricative cluster is pronounced together without a schwa. The final /χ/ is one of the possible fricative appendixes that can be added at the end of syllables. 
	
	Besides this word, we found clusters which undergo schwa epenthesis: affricate + /s/, affricate + /ʁ/. Epenthesis is the norm (Table \ref{tab:flat affr fric schwa}). 
	
	\begin{table}[H]
		\centering
		\caption{Schwa epenthesis in final CC clusters    where C1   is an affricate and  C2 is a fricative}
		\label{tab:flat affr fric schwa}
		\begin{tabular}{|l|llll|l|  }
			\hline 
			<d͡ʒɣ>$\rightarrow$[dʒəʁ] & <go\textbf{d͡ʒɣ}> & ˈɡo\textbf{d͡ʒəʁ} & `stump' \armenian{կոճղ} &  n=2
			\\ \hline  
			<t͡s'ɣ>$\rightarrow$[t͡səʁ] & <p'o\textbf{t͡s'ɣ}> & ˈpʰo\textbf{t͡səʁ} & `rake' \armenian{փոցղ} &  n=1
			\\ \hline  
			<t͡s's>$\rightarrow$[t͡səs] & <t͡ʃ'ori\textbf{t͡s's}> & t͡ʃoˈɾi\textbf{t͡səs} & `four times' \armenian{չորիցս} &  n=15
			\\ \hline  
			
		\end{tabular}
		
	\end{table}
	
	For affricate  + /ʁ/, some of these words allow optional deletion of the schwa: [ɡod͡ʒəʁ, ɡod͡ʒʁ] `stump'. This is part of schwa elision (\S\ref{section:segmentalPhono:sandhi:schwaElision}). 
	
	For <t͡s's>$\rightarrow$[t͡səs], this sequence seems to be restricted to numeral-related words, e.g., <vet͡sit͡ss>$\rightarrow$[vet͡sit͡səs] `six times' \armenian{վեցիցս}. 
	
	\subsubsection{Affricate + sonorant }\label{section:syllable:Final2C:FlatRising:AffrSono}
	It is extremely rare to find any words which end in an orthographic affricate+sonorant cluster, such as hypothetical /t͡s+n/, /d͡ʒ+ɾ/, and so on. We found zero such cases in the \citeauthor{kouyoumdjian-1970-DictionaryArmenianEnglish} dictionary. Armenian Wiktionary also has few clear cases of such clusters. However, the data that we find shows that affricate + sonorant clusters pattern like stop+sonorant clusters in undergoing schwa epenthesis.  
	
	
	Although affricate + nasal clusters are un-attested,  HD's intuition is that their syllabification is the same as stop+sonorant cluster. That is, because word-final /dn/ clusters undergo schwa epenthesis, then so does word-final /t͡sn/. For example, a nonce word \armenian{մացն} <mat͡s'n> is syllabified as [mɑt͡sən]. But of course, because these words don't exist, we can't say much about them. We suspect that their absence is more of an accidental gap. 
	
	For affricate + rhotic clusters, the \citeauthor{kouyoumdjian-1970-DictionaryArmenianEnglish} dictionary doesn't list any words. But Wiktionary lists a handful. Here we find schwa epenthesis: <t͡s'ad͡zr>$\rightarrow$[t͡sɑd͡zəɾ] `low' \armenian{ցածր}. 
	
	
	\subsubsection{Fricative + fricative }\label{section:syllable:Final2C:FlatRising:FricFric}
	Fricatives can take any stop to form a complex coda. Root-final fricative-fricative clusters do exist. But these are severely limited to a handful of fricative combinations, and to an apparent finite number of roots. 
	
	There are largely two categories of root-final fricative-fricative clusters. The first category are orthographic clusters that are pronounced as they are spelled, creating a consonant cluster in pronunciation: <d͡zaxs> [d͡zɑχs] `cost' \armenian{ծախս}. We analyze this category of clusters as containing an extrasyllabic final fricative. The second group is words that are spelled with a fricative-fricative cluster, but this cluster undergoes schwa epenthesis in speech: <ews>  [jevəs] `moreover' \armenian{եւս}.
	
	We discuss the epenthetic group first because it is an extremely small class (Table \ref{tab:flat fric fric schwa}). In \citeauthor{kouyoumdjian-1970-DictionaryArmenianEnglish}'s dictionary, we only found 3 words that end in an orthographic fricative-fricative cluster, and that get schwa epenthesis. These 3 words all use /vs/   and they are function words. 
	
	
	
	\begin{table}[H]
		\centering
		\caption{Schwa epenthesis in final CC clusters    where C1   and C2 are fricatives}
		\label{tab:flat fric fric schwa}
		\begin{tabular}{|l|lll|l|l|  }
			\hline 
			<e\textbf{ws}> &  /(j)e\textbf{vs}/ & [ˈje\textbf{vəs}] & `moreover' &  \armenian{եւս} \\
			<ajle\textbf{ws}> &  /ɑjle\textbf{vs}/ & [ɑjˈle\textbf{vəs}] & `anymore' &  \armenian{այլեւս} \\
			<t'ere\textbf{ws}> &  /tʰeɾe\textbf{vs}/ & [tʰeˈɾe\textbf{vəs}] & `perhaps' &  \armenian{թերեւս} \\
			\hline  
		\end{tabular}
		
	\end{table}
	
	For the  category of pronounced clusters, there are a handful of roots which end in a fricative + /χ,ʃ,s/ (Table \ref{tab:flat fric fric   x}). The most frequent C2 is /χ/.  Each type of /Cχ/ cluster is found in a handful roots: [χɑɾisχ] `anchor' \armenian{խարիսխ}. Such roots can be productively used to form new words via compounding and prefixation: [sɑɾ-ɑ-χɑɾisχ] `ice anchor' \armenian{սառախարիսխ}  
	
	
	\begin{table}[H]
		\centering
		\caption{Final CC clusters    where C1 is fricative and C2 is /χ/}
		\label{tab:flat fric fric   x}
		\begin{tabular}{|l|lll|l|l|  }
			\hline 
			&   & & & \# of roots & \# of derivatives 
			\\
			/sχ/ & χɑˈɾi\textbf{sχ} & `anchor' & \armenian{խարիսխ} & n=2 & n=11 \\ 
			& χoˈɾi\textbf{sχ} & `honey-comb' & \armenian{խորիսխ} &   &   \\ 
			\hline 
			/ʃχ/ &ˈvɑ\textbf{ʃχ}  & `usury' & \armenian{վաշխ} & n=7 & n=44  \\
			&jeˈɾɑ\textbf{ʃχ}  & `surety' & \armenian{երաշխ} &   &   \\
			\hline 
			/χχ/ &ˈze\textbf{χχ}  & `lewd' & \armenian{զեղխ} & n=8 & n=18  \\
			& ˈse\textbf{χχ}  & `melon' & \armenian{սեղխ} &   &   \\
			\hline  
		\end{tabular}
		
	\end{table}
	
	
	For the /χχ/ clusters, the two segments are spelled differently <ɣx> \armenian{ղխ}, suggesting that they diachronically arose from two separate segments that eventually became an identical long (geminate) segment. In fact, one of the examples from \citeauthor{kouyoumdjian-1970-DictionaryArmenianEnglish} is the word `melon' [seχχ] \armenian{սեղխ}. The more common rendition of this word is however   [seχ] \armenian{սեխ} where there is no gemination. 
	
	The other pronounced fricative-fricative clusters are /fs, fʃ, χs/ (Table \ref{tab:flat fric fric  sh}) which are all restricted to a handful of roots in \citeauthor{kouyoumdjian-1970-DictionaryArmenianEnglish}'s dictionary. From this set, only word [d͡zɑχs] `cost' and its derivatives are frequent. The word `star' can vary in pronunciation: <asdɣ> [ɑstχ, ɑsχ] `star' \armenian{աստղ} (\S\ref{section:syllable:Final3C:Appendix:X}). 
	
	\begin{table}[H]
		\centering
		\caption{Final CC clusters    where C1 is fricative and C2 is /s,ʃ/}
		\label{tab:flat fric fric  sh}
		\begin{tabular}{|l|lll|l|l|  }
			\hline 
			&   & & & \# of roots & \# of derivatives 
			\\\hline
			/fʃ/ & ˈd͡ʒɑ\textbf{fʃ} & `breast-plate' & \armenian{ճաւշ}   &n=1 &    n=0 \\ 
			\hline 
			/fs/ & ˈze\textbf{fs} & `Zeus' & \armenian{Զեւս}   &n=1 &    n=0 \\ 
			\hline 
			/χs/ &ˈd͡zɑ\textbf{χs}  & `cost' & \armenian{ծախս} & n=2 & n=4   \\
			& ˈtʰu\textbf{χs} & `incubation' & \armenian{թուխս} &   &   \\
			\hline  
		\end{tabular}
		
	\end{table}
	
	On Armenian Wiktionary,   we also found a handful of instances     in non-dialectal words:   [nɑ\textbf{ʃχ}] `pattern' \armenian{նաշխ}, [kʰedən-so\textbf{χs}] `earth-crawling' \armenian{գետնսողս}. 
	
	Based on the above data, it is obvious that fricative-fricative clusters are highly restricted in terms of a) what combinations are attested, and b) how many roots have these clusters. Because both of these factors are small, these post-fricative /ʃ, χ, s/ fricatives have been analyzed as being extrasyllabic appendixes \textcolor{red}{cite vaux}. Meaning that a word like [d͡zɑχs] `cost' doesn't end in a genuine complex coda [χs], but that the syllable is [d͡zɑχ] and the /s/ is added outside the syllable. 
	\subsubsection{Fricative + nasal /m/ }\label{section:syllable:Final2C:FlatRising:FricNasalM}
	The nasal /m/ has quite bizarre syllabification sometimes. As part of a complex coda, /m/+ fricative clusters are generally rare (\S\ref{section:syllable:Final2C:FallingOther:NasalFric}). They are rare in that a) few words exist with an /m/+fricative cluster, and b) few of these existing words are high-frequency. In contrast, it is quite common to find words with a fricative + /m/ cluster (Table \ref{tab:flat fric nasal m}). And many of these words are high-frequency. 
	
	
	\begin{table}[H]
		\centering
		\caption{Final CC clusters    where C1 is fricative and C2 is a nasal  /m/}
		\label{tab:flat fric nasal m}
		\begin{tabular}{|l|lll|l|l|  }
			\hline 
			&& &&  \# of roots & \# of derivatives
			\\ \hline
			{}[zm] & ˈɡɑ\textbf{zm} & `construction' & \armenian{կազմ} & n=6 & n = 39
			\\
			& ˈχɑ\textbf{zm} & `quarrel' & \armenian{խազմ} & & 
			\\
			{}[hm] & ˈdo\textbf{hm} & `family' & \armenian{տոհմ}& n=1 & n=15 
			\\\hline 
			{}[ʁm] &ˈho\textbf{ʁm}& `wind' & \armenian{հողմ} & n=12 & n=33 \\
			&ˈɡo\textbf{ʁm}& `side' & \armenian{կողմ} & & \\ 
			&ˈme\textbf{ʁm}& `soft' & \armenian{մեղմ} & & \\ 
			\hline 
			{}[ʃm] & tʰəˈɾo\textbf{ʃm} & `stamp' & \armenian{դրոշմ} & n=5 & n=11 \\ 
			& [ɑ\textbf{ʃm}] & `jade' & \armenian{աշմ} & & 
			\\
			\hline  
		\end{tabular}
		
	\end{table}
	
	For /zm/, the \citeauthor{kouyoumdjian-1970-DictionaryArmenianEnglish} dictionary lists 6 roots that end in [zm]. Of these roots, the root [ɡɑzm] `construction' was used to derive other words, specifically 39 compounds and prefixed words like [noɾ-ɑ-ɡɑzm] `newly-formed' \armenian{նորակազմ} (literally `new+form').  Then there is a handful (n=5) of borrowings with the foreign suffix \textit{-izm} like [fɑʃi\textbf{zm}] `fascism' \armenian{ֆաշիզմ}. 
	
	Note that for /zm/ clusters, some sources report schwa epenthesis \textcolor{red}{cite vaux}. For example, the word [ɡɑzm] `construction', the typical pronunciation is to not use a schwa. But Vaux reports a schwa form [ɡɑzəm]. It is possible that such a schwa is a transient vowel that's created by going from [z] to [m]. 
	
	
	For /hm/, only one root in the dictionary had this cluster:  [dohm] `family. All other words were compounds from this root like [ɑzɑd-ɑ-dohm] `noble' \armenian{ազատակողմ}, literally `free+family'. 
	
	For /ʁm/, we found 12 roots. Of that 12, [hoʁm] and [ɡoʁm] form all the derived words that have the /ʁm/ cluster: [vet͡s-ɑ-ɡoʁm] `hexagonal' \armenian{վեցակողմ}, literally `six+side'. 
	
	For /ʃm/, we found 5 roots. Of these roots, [tʰəɾoʃm] formed all the derived forms with this cluster: [nɑmɑɡ-ɑ-tʰəɾoʃm] `postage-stamp' \armenian{նամակադրոշմ}, literally `letter+stamp'. 
	
	For /Vʒm/ words, we found no such cases. The closest was a <ayʒm> [ɑjʒəm] `now' \armenian{այժմ} where the  <ʒm> cluster is after a consonant. The fact that this word is pronounced as [ɑjʒəm] and not [ɑjəʒm] suggests either that a) [ʒm] can't be a complex coda or coda+appendix cluster, or b) schwa epenthesis avoids creating [jə] sequences if possible. Data is obviously too limited to know. 
	
	
	Besides the above words, we found only  one case of orthographic final <sm> in the \citeauthor{kouyoumdjian-1970-DictionaryArmenianEnglish} dictionary: <t͡ʃasm> `chimera' \armenian{ջասմ}. This word is   likely an old loanword. We're not if a schwa is needed here: [tʃɑs(ə?)m]. Similarly we found one case of <vm>: <hrovm> `Rome' \armenian{հրովմ}. We think the pronunciation is likely without a schwa before the nasal: [həɾovm]. Part of the ambiguity is that these words seem like obvious loanwords and HD never heard of such clusters before before. And finally, one case of <xm> is an obvious loanword without a schwa: <traxm> [tʰəɾɑχm] `Greek currency'  \armenian{դրախմ}. 
	
	In sum,    fricative + /m/ clusters are pretty common. But we're not sure why. The fact that only /m/ can form these clusters but not /n/ suggests that there is something special about the nasal /m/. \textcolor{red}{mention vaux}. We treat this /m/ as an extrasyllabic appendix in these words, though we're not sure how Armenian came to develop this strange behavior.  Furthermore, such fricative-nasal clusters show more idiosyncrasies word-medially (\S\ref{section:syllable:OtherCodaRestrictions:AvoidMedial:Cm}). 
	
	\subsubsection{Fricative + nasal /n/}\label{section:syllable:Final2C:FlatRising:FricNasalN}
	Although word-final fricative + /m/ clusters are pronounced without an intervening schwa, word-final clusters of fricative + /n/ regularly undergo epenthesis (Table \ref{tab:epen fric nasal n}).
	
	\begin{table}[H]
		\centering
		\caption{Schwa epenthesis in final CC clusters    where C1 is a fricative   and C2 is a  nasal /n/}
		\label{tab:epen fric nasal n}
		\resizebox{\textwidth}{!}{%
			\begin{tabular}{|l|llll|l|  }
				\hline 
				<vn>$\rightarrow$[vən] & <goralio\textbf{vn}> & ɡoɾɑliˈjo\textbf{vən} & `coral' & \armenian{կորալիովն} & n=3
				\\
				\hline  
				<sn>$\rightarrow$[sən] & <how\textbf{sn}> & ˈhu\textbf{sən} & `brink' & \armenian{հուսն} & n=27
				\\ 
				& <orbē\textbf{sn}> & voɾˈbe\textbf{sən} & `the wherefore' & \armenian{որպէսն} & n=26
				\\
				\hline  
				<zn>$\rightarrow$[zən] & <a\textbf{zn}> & ˈɑ\textbf{zən} & `nation' & \armenian{ազն} & n=36
				\\ 
				& <t'aka\textbf{zn}> & tʰɑˈkʰɑ\textbf{zən} & `prince' & \armenian{թագազն} & 
				\\ 
				\hline 
				<ʃn>$\rightarrow$[ʃən] & <ta\textbf{ʃn}> & ˈtɑ\textbf{ʃən} & `contract' & \armenian{դաշն} & n=7
				\\ 
				& <yankow\textbf{ʃn}> & hɑŋˈkʰu\textbf{ʃən} & `closely' & \armenian{յանգուշն} & 
				\\
				\hline  
				<ʒn>$\rightarrow$[ʒən] & <a\textbf{ʒn}> & ˈɑ\textbf{ʒən} & `crack' & \armenian{աժն} & n=1
				\\  \hline 
				<xn>$\rightarrow$[χən] & <to\textbf{xn}> & ˈtʰo\textbf{χən} & `funnel' & \armenian{դոխն} & n=1
				\\ 
				\hline 
				<ɣn>$\rightarrow$[ʁən] & <sde\textbf{ɣn}>  & əsˈte\textbf{ʁən} & `dactyl' & \armenian{ստեղն} & n=17 \\
				& <e\textbf{ɣn}>& je\textbf{ʁən} & `hind' &\armenian{եղն} & 
				\\ \hline 
			\end{tabular}
		}
	\end{table}
	
	As with stop + /n/ clusters (\S\ref{section:syllable:Final2C:FlatRising:StopNasal}), a lot of these fricative + /n/ words have an archaic connotation. 
	
	For <vn>$\rightarrow$[vən] clusters, these seem to be restricted to loanwords. While <fn>$\rightarrow$[fən] and <hn>$\rightarrow$[hən] clusters seem to be   accidental gaps. 
	
	\subsubsection{Fricative +  rhotic /ɾ/}\label{section:syllable:Final2C:FlatRising:FricRhotic}
	Word-finally, an orthographic cluster of a fricative + rhotic /ɾ/ undergoes schwa epenthesis (Table \ref{tab:epen fric rhotic }). 
	
	
	\begin{table}[H]
		\centering
		\caption{Schwa epenthesis in final CC clusters    where C1 is   is a fricative and C2 is a rhotic /ɾ/}
		\label{tab:epen fric rhotic }
		\begin{tabular}{|l|llll|l|  }
			\hline 
			<wr>$\rightarrow$[vəɾ] & <a\textbf{wr}>$\rightarrow$ˈɑ\textbf{vəɾ} & `bile' & \armenian{աւր} &    n=1
			\\
			\hline  
			<sr>$\rightarrow$[səɾ] & <nō\textbf{sr}>$\rightarrow$ˈno\textbf{səɾ} & `coarse' & \armenian{նօսր} &    n=6
			\\\hline 
			<zr>$\rightarrow$[zəɾ] & <e\textbf{zr}>$\rightarrow$ˈje\textbf{zəɾ} & `shore' & \armenian{եզր} &   n=11
			\\
			\hline  
			<ʒ>$\rightarrow$[ʒəɾ] & <ʒa\textbf{hr}>$\rightarrow$ˈʒɑ\textbf{həɾ} & `virus' & \armenian{ժահր} &    n=4
			\\
			\hline  
			
			<xr>$\rightarrow$[χəɾ] & <d͡ʒa\textbf{xr}>$\rightarrow$ˈd͡ʒɑ\textbf{χəɾ} & `flight' & \armenian{ճախր} &    n=3
			\\
			\hline  
			<ɣr>$\rightarrow$[ʁəɾ] & <d͡za\textbf{ɣr}>$\rightarrow$ˈd͡zɑ\textbf{ʁəɾ} & `mocking' & \armenian{ծաղր} &    n=9
			\\
			\hline  
			
		\end{tabular}
		
	\end{table}
	
	
	For final <zr>$\rightarrow$[zəɾ] sequences, all of \citeauthor{kouyoumdjian-1970-DictionaryArmenianEnglish}'s examples are either [jezəɾ] `shore' or its derivatives: [kʰed-ezəɾ] `river-bank'  \armenian{գետեզր} where [kʰəd] is `river'. 
	
	For final  <wr,vr>$\rightarrow$[vəɾ], more cases come from   loanwords on Wiktionary: <manewr> $\rightarrow$ [mɑˈnevəɾ] `maneuver' \armenian{մանևր}. 
	
	These fricative-rhotic clusters are generally few. Thus  in \citeauthor{kouyoumdjian-1970-DictionaryArmenianEnglish}, there are accidental gaps for the fricatives /f, ʃ/. On Wiktionary, we did find a loanword with [fəɾ]: <ʃifr>$\rightarrow$[ʃifəɾ] `cipher' \armenian{շիֆր}. 
	
	Prescriptively, all fricative-rhotic clusters are pronounced with a schwa: <meɣr>$\rightarrow$[meʁəɾ] `honey' \armenian{մեղր}. But in casual speech, some of these words allow the deletion of the schwa: [meʁɾ]. This is a case of schwa elision. In our experience, [ʁəɾ] sequences often reduce to just [ʁəɾ] in natural speech, while it is less common to see such reduction or schwa elision after the other fricatives. 
	\subsubsection{Fricative + lateral /l/}\label{section:syllable:Final2C:FlatRising:FricLateral}
	The \citeauthor{kouyoumdjian-1970-DictionaryArmenianEnglish} dictionary lists only one word that end in an orthographic cluster of a fricative + /l/. This word was <p'ahl>  `stallion' \armenian{փահլ}. This word is pronounced with a schwa: [pʰɑhəl]. 
	
	On Wiktionary, we found some other cases from loanwords. These again take schwa epenthesis: <p'azl>  [pʰɑzəl] `puzzle' \armenian{փազլ}
	
	\subsubsection{Nasal + nasal /m/}\label{section:syllable:Final2C:FlatRising:NasalNasalM}
	We did not find such final clusters in \citeauthor{kouyoumdjian-1970-DictionaryArmenianEnglish}. Nor did we find any such words on Wiktionary. However,  there is a colloquial word  that has this orthographic cluster and get a schwa: <d͡ʒanm> $\rightarrow$ [d͡ʒɑnəm] `my dear' \armenian{ճանմ}. This word is borrowed from Turkish `canim' [d͡ʒanim  `my dear'. 
	\subsubsection{Nasal +  nasal /n/}\label{section:syllable:Final2C:FlatRising:NasalNasalN}
	There are many words which end in an orthographic nasal-nasal sequence: <mn> and <nn>. In pronunciation, a schwa is added between the nasals (Table \ref{tab:epen schwa schwa}).  
	
	
	\begin{table}[H]
		\centering
		\caption{Schwa epenthesis in final CC clusters    where C1 is   and C2 are nasals}
		\label{tab:epen schwa schwa}
		\begin{tabular}{|ll|ll|l|  }
			\hline 
			\multicolumn{2}{| l|}{Archaic suffix \textit{-umən}} & \multicolumn{2}{l| }{ Other   \textit{-mən}}  & \textit{-nən} \\
			<xajt'ow\textbf{mn}>  & <derewow\textbf{mn}>  & <ada\textbf{mn}>  & <owre\textbf{mn}>  & <i\textbf{nn}> 
			\\
			/χɑjtʰ-u\textbf{mn}/ & /deɾev-u\textbf{mn}/ & /ɑdɑ\textbf{mn}/ & /uɾu\textbf{mn}/  & /inn/
			\\
			{}[χɑjˈtʰu\textbf{mən}] & {}[deɾeˈvu\textbf{mən}] & [ɑˈdɑ\textbf{mən}]  & [uˈɾe\textbf{mən}] & [i\textbf{nən}]
			\\
			\armenian{խայթումն} & \armenian{տերեւումն} & \armenian{ատամն} & \armenian{ուրեմն} & \armenian{ինն}
			\\
			`pricking' & `foliation' & `tooth' & `thus' & `nine' 
			\\ 
			\hline 
			n=325 & & n=36 & & n=13 \\ 
			\hline  
		\end{tabular}
		
	\end{table}
	
	Words with a final <mn> [mən] cluster are generally archaic. One common situation are words that end in the nominalizer suffix [-um]. This suffix is  pronounced  [-um] in the modern language, with an orthographic final <m> \armenian{ում}. But in more archaic stages of Western Armenian, the form of this suffix [-umən] with an orthographic nasal cluster <mn> \armenian{-ումն}. \citeauthor{kouyoumdjian-1970-DictionaryArmenianEnglish}'s dictionary uses these archaic forms, thus he has a lot of words that would be pronounced with a final [-umən]. 
	
	
	\subsubsection{Nasal + rhotic /ɾ/}\label{section:syllable:Final2C:FlatRising:NasalRhotic}
	Word-finally, nasal + /ɾ/ clusters   undergo schwa epenthesis. The nasal can be /m/ or /n/ (Table \ref{tab: nasal stop rhotic  }).       
	
	
	\begin{table}[H]
		\centering
		\caption{Schwa epenthesis in final CC clusters    where C1 is a nasal /m,n/ and C2 is a rhotic /ɾ/}
		\label{tab: nasal stop rhotic  }
		\begin{tabular}{|l|llll|l|  }
			\hline 
			<mr>$\rightarrow$[məɾ] & <ha\textbf{mr}> & ˈhɑ\textbf{məɾ} & `speechless' &  \armenian{համր} &  n=3
			\\ \hline 
			<nr>$\rightarrow$[nəɾ] & <ma\textbf{nr}> & ˈmɑ\textbf{nəɾ} & `small' &  \armenian{մանր} &  n=5
			\\ \hline 
		\end{tabular}
		
	\end{table}
	
	I am not aware of cases where, in spoken speech, the [nəɾ] cluster is reduced to *[nəɾ]: [mɑnəɾ] `small', not [*mɑnɾ]. 
	
	\subsubsection{Nasal + lateral /l/}\label{section:syllable:Final2C:FlatRising:NasalLateral}
	Nasals cannot start a complex coda with laterals. We   found no orthographic cases of final <ml> or <nl> clusters in \citeauthor{kouyoumdjian-1970-DictionaryArmenianEnglish}. On Wiktionary, the few examples that we found were all loanwords. And these get schwa epenthesis: <pṙanl> [pʰəɾɑnəl] \armenian{բռանլ} from French `branle', and <greml> [ɡəɾeməl] `Kremlin' \armenian{Կրեմլ} from Russian. 
	
	\subsection{Geminate codas}\label{section:syllable:Final2C:Geminate}
	There are few roots end in a geminate, i.e. roots where the final consonant is repeated. In \citeauthor{kouyoumdjian-1970-DictionaryArmenianEnglish}'s dictionary, we have found the following types of word-final repeated segments: [kkʰ], [χχ],  and [ɾɾ]. Medial geminates are variably avoided (\S\ref{section:syllable:OtherCodaRestrictions:AvoidMedial:Gemin}). 
	
	
	For [kk], this gemination is derived from adding the nominalizer /-kʰ/ after a velar stop: [hok-kʰ] `hindrance' \armenian{հոգք}. This cluster is discussed in  Section \S\ref{section:syllable:Final2C:FlatRising:StopStop}. This cluster is arguably syllabified as a coda + appendix, such as [.hok.\{kʰ\}]. Similar geminates are created in final [VCk-k] clusters via this appendix (\S\ref{section:syllable:Final3C:Appendix:K}).  
	
	For [χχ], this cluster is limited to a handful of roots, some of which sound archaic: [seχχ] `melon' \armenian{սեղխ}. This cluster is part of a pattern where a final [χ] can be found in different fricative-fricative clusters, again as a type of extrasyllabic consonant. See Section \S\ref{section:syllable:Final2C:FlatRising:FricFric}. 
	
	Finally, the [ɾɾ] pattern is found in a handful of roots. This is orthographically represented with a final rhotic getting repeated: <rr> \armenian{րր}. The [ɾɾ] sequence is pronounced as a single long rhotic. In HD's impression, this orthographic <rr> sequence is pronounced as [ɾɾ] more often in Eastern Armenian than in Western Armenian. In his Western judgments, it feels more typical to degeminate this final cluster into just a singleton [ɾ] (\S\ref{section:segmentalPhono:sandhi:degemination}). 
	
	In \citeauthor{kouyoumdjian-1970-DictionaryArmenianEnglish}'s dictionary, we found 23 words which end in an <rr> [ɾɾ] sequence (Table \ref{tab:rr}). Of these 23 words, 4 are roots. The other  words were all derived from the root [dɑɾɾ] `element' (18 words) or [ɑntʰoɾɾ] `peaceful' (1 word). 
	
	
	\begin{table}[H]
		\centering
		\caption{Final CC clusters    where C1-C2 are geminate [ɾɾ] }
		\label{tab:rr}
		\begin{tabular}{|lll|  }
			\hline 
			ˈdɑ\textbf{ɾɾ} & `element' & \armenian{տարր} 
			\\
			ɑnˈtʰo\textbf{ɾɾ} & `peaceful' & \armenian{անդորր} 
			\\
			ˈje\textbf{ɾɾ} & `mustiness' & \armenian{երր} 
			\\
			ˈdo\textbf{ɾɾ} & `turion' & \armenian{տորր} 
			\\
			\hline  
			ˈmed͡z + ˈdɑɾɾ & `big + element' &\armenian{մեծ տարր} \\
			$\rightarrow$ med͡z-ɑ-dɑ\textbf{ɾɾ} & `spacious' & \armenian{մեծատարր} 
			\\
			\hline  
		\end{tabular}
		
	\end{table}
	
	Besides the above sequences of identical sounds, we've come across [zz] sequences in onomatopoeic words like [bəzz] `buzz' \armenian{պըզզ}, [uff] `wow' \armenian{ուֆֆ}. We've also found  loanwords like [miss] `Miss' \armenian{միսս} and [finn] `Finn' \armenian{ֆինն}. 
	
	
	\section{Syllabification of final three-consonant clusters}\label{section:syllable:Final3C}
	Armenian allows at most two consonants in a complex coda. Exceptions are rare and seem limited to non-nativized loanwords (\S\ref{section:syllable:Final3C:NoNative}). Among native words, if the orthography ends in a sequence of 3 consonants, then we see one of two outcomes:
	\begin{itemize}[topsep=0pt, noitemsep]
		\item The cluster is a (complex) coda plus one or two appendixes (\S\ref{section:syllable:Final3C:Appendix}). 
		\item The cluster is pronounced with  schwa epenthesis (\S\ref{section:syllable:Final3C:Schwa}). 
	\end{itemize}
	
	\subsection{No native complex codas with three consonants}\label{section:syllable:Final3C:NoNative}
	It seems that Armenian generally avoids words that have a complex coda larger than two consonants. The only example that we found in \citeauthor{kouyoumdjian-1970-DictionaryArmenianEnglish}  was a loanword [veɾst] `Verst' \armenian{վերստ}, as the name for a Russian unit of length. 
	
	Armenian Wiktionary provides much more examples of words with final 3-consonant complex codas. Again, these are loanwords, especially   proper nouns like [lisit͡ʃɑnsk]  `Lysychansk'  \armenian{Լիսիչանսկ} or[tʰɑjms] `The Times' \armenian{թայմս} . Some Wiktionary loanwords even have flat-sonority: [ɑtʰjunɡd] `adjunct' \armenian{ադյունկտ}.  
	
	\subsection{Codas with  an appendix}\label{section:syllable:Final3C:Appendix}
	A consonant is an appendix if it can be pronounced after a consonant cluster, despite having flat sonority (\S\ref{section:syllable:ConsonantClusters:Appendix}). Such appendixes are common in final two-consonant sequences (\S\ref{section:syllable:Final2C:FlatRising}), but can also occur in final 3-consonant sequences.  The set of possible final appendixes that we found in in 3-consonant clusters is     /-kʰ, χ, s/. Very rarely, we find a cluster with two final appendixes /χ,kʰ/ (\S\ref{section:syllable:Final3C:Appendix:TwoAppendix}). 
	\subsubsection{Complex coda + appendix /-kʰ/ }\label{section:syllable:Final3C:Appendix:K}
	
	The nominalizer suffix \textit{-kʰ} can be added after virtually any attested complex coda (Table \ref{tab:complex coda + k falling}). In some cases, the sequence of three consonants has continuous falling sonority. This occurs when the second consonant is a rhotic, nasal, or fricative. 
	
	\begin{table}[H]
		\centering
		\caption{Complex codas + appendix /-kʰ/ with continuous falling sonority}
		\label{tab:complex coda + k falling}
		\begin{tabular}{|l|lll|l|  }
			\hline 
			{}[VCl-kʰ] & ʃɑˈɾɑ\textbf{jl}& `ray' & \armenian{շառայլ} & n=4 \\
			& ʃɑˈɾɑ\textbf{jl-kʰ}& `glimmer' & \armenian{շառայլք} &   \\ \hline 
			{}[VCɾ-kʰ] & həˈɾɑ\textbf{jɾ}& `burned' & \armenian{հրայր} & n=12 \\
			& həˈɾɑ\textbf{jɾ-kʰ}& `conflagration' & \armenian{հրայրք} &   \\ \hline 
			{}[VCŋ-kʰ] & pʰənɑˈt͡sɑ\textbf{jn}& `onomatopoeia' & \armenian{բնաձայն} & n=4 \\
			& pʰənɑˈt͡sɑ\textbf{jŋ-kʰ}& `onomatopoeia' & \armenian{բնաձայնք} & \\ \hline 
			{}[VCs-k] & ˈbɑ\textbf{ɾs}& `Persian' & \armenian{պարս} & n=6  \\
			&  ˈbɑ\textbf{ɾs-k}& `Persia' & \armenian{Պարսք} & \\ \hline 
			{}[VCʃ-k] & ˈdu\textbf{jʒ}& `damage' & \armenian{տոյժ} & n=2 \\
			&  ˈdu\textbf{jʃ-k}& `damage' & \armenian{տռյժք} & \\ \hline 
		\end{tabular}
		
	\end{table}
	
	The suffix devoices preceding fricatives: see the [VCʃ-k] row above (\S\ref{section:segmentalPhono:allphonLaryng:assiimlation:regFinal}). The suffix tends to deaspirate after obstruents (\S\ref{section:segmentalPhono:allphonLaryng:finalDeasp:derived}). 
	
	Among falling-sonority clusters, the suffix can likewise follow the lateral /l/: see row [VCl-kʰ] in Table \ref{tab:complex coda + k falling}. As explained in Section \S\ref{section:syllable:Final2C:FallingOther:LateralObs}, the lateral generally resists preceding a consonant in the same syllable, except for the suffix \textit{-kʰ}. 
	
	Although the appendix can create falling-sonority clusters, it is much more common to find flat-sonority clusters (Table \ref{tab:complex coda + k flat}). The suffix \textit{-kʰ} can be easily added after a complex coda that ends in a stop or affricate of any place of articulation. 
	
	\begin{table}[H]
		\centering
		\caption{Complex codas + appendix /-kʰ/ with flat  sonority}
		\label{tab:complex coda + k flat}
		\begin{tabular}{|l|lll|l|  }
			\hline 
			{}[VCp-k] & ot͡sɑˈɡe\textbf{ɾb} & `snake-like' & \armenian{օձակերպ} & n=9 \\
			& ot͡sɑˈɡe\textbf{ɾp-k} & `Ophidia'    & \armenian{օձակերպք} & 
			\\ \hline 
			{}[VCt-k] &ˈɡi\textbf{ɾtʰ} & `well-educated' & \armenian{կիրթ} &  n=39 \\
			&ˈɡi\textbf{ɾt-k} & `instruction' & \armenian{կիրթք} &      \\ \hline 
			{}[VCk-k] &ˈje\textbf{ɾɡ} & `work' & \armenian{երկ} &  n=6 \\
			&ˈje\textbf{ɾk-k} & `work of art' & \armenian{երկք} &      \\ \hline 
			{}[VCt͡s-k]&ˈhu\textbf{nt͡s} & `harvest' & \armenian{հունձ}& n=34
			\\
			&ˈhu\textbf{nt͡s-k} & `harvest' & \armenian{հունձք}& 
			\\ \hline 
			{}[VCt͡ʃ-k]  & ɑˈnu\textbf{ɾt͡ʃ} & `dream' & \armenian{անուրջ}& n=12 \\
			& ɑˈnu\textbf{ɾt͡ʃ-k} & `dream' & \armenian{անուրջք} & 
			\\ \hline 
		\end{tabular}
		
	\end{table}
	
	Among these final three-consonant clusters, there are cases where the suffix \textit{-kʰ} follows another velar stop: see row [VCk-k] in Table \ref{tab:complex coda + k flat}. The two are pronounced as one long geminate. 
	
	
	On a last note, in \citeauthor{kouyoumdjian-1970-DictionaryArmenianEnglish}'s dictionary, we only found one case of a /VCCkʰ/ word that's pronounced with schwa epenthesis: <xa\textbf{ṙnk'}> [χɑ\textbf{ɾənkʰ}] `copulation' \armenian{խառնք}. Here, the schwa is necessary because [ɾn] is difficult to pronounce as a complex coda in general. See the other sections for   variations on how [ɾn] clusters are produced, both word-finally (\S\ref{section:syllable:Final2C:FallingOther:RhoticNasalN}) and word-medially (\S\ref{section:syllable:OtherCodaRestrictions:AvoidMedial:rN}). 
	
	
	\subsubsection{Complex coda + appendix /χ/}\label{section:syllable:Final3C:Appendix:X}
	Among appendixes, the suffix \textit{-kʰ} is the most common. Another common appendix is /χ/. This fricative can follow obstruents such as stops (\S\ref{section:syllable:Final2C:FlatRising:StopFric}) and fricatives (\S\ref{section:syllable:Final2C:FlatRising:FricFric}). It can likewise follow complex codas. 
	
	The only relevant example we came across in \citeauthor{kouyoumdjian-1970-DictionaryArmenianEnglish} is the word for `star'. Orthographically, this word ends in a consonant cluster: <asdɣ> \armenian{աստղ}. In Eastern Armenian, this cluster undergoes schwa epenthesis: [ɑstəʁ]. In some sub-dialects of Western Armenian such as in Istanbul, Istanbuli speakers tell us that this word is also pronounced with epenthesis:  [ɑstəʁ]. But in the Lebanese sub-dialect of Western Armenian, there is no schwa epenthesis. The cluster is pronounced either as [ɑstχ] or with stop deletion [ɑsχ]. 
	
	For the [ɑstχ] case, we analyze the [st] as forming a complex coda. The [χ] is then an appendix. 
	
	The word for `star' has many derivatives (8 in \citeauthor{kouyoumdjian-1970-DictionaryArmenianEnglish}). Compounds that end with this root show the same patterns of pronunciation: [d͡zoˈv-ɑstχ] `starfish' \armenian{ծովաստղ}, literally `sea-star'. 
	
	\subsubsection{Complex coda +   appendix /s/}\label{section:syllable:Final3C:Appendix:S}
	Besides /kʰ,χ/, another common appendix is /s/. This segment can follow stops (\S\ref{section:syllable:Final2C:FlatRising:StopFric}). It can likewise follow a complex coda that ends in a stop. 
	
	In \citeauthor{kouyoumdjian-1970-DictionaryArmenianEnglish}'s dictionary, we found three such cases. Two of them are alternate spellings of the loanword [əspiŋks] `sphinx' \armenian{սպինքս, սփինքս}. The other word is the name of some fish species [ɡɑɾɑŋks] `Cavalla' \armenian{կարանքս}. 
	
	Because we only found the above three examples, it's possible that cases of complex codas + /s/  without schwa epenthesis are limited to either a) loanwords or b) words where the /s/ follows a velar stop [k]. Other clusters of consonants + /s/ undergo schwa epenthesis (\S\ref{section:syllable:Final2C:FlatRising:StopFric},\ref{section:syllable:Final2C:FlatRising:AffrFric}). 
	
	\subsubsection{Coda + two appendixes (χ, kʰ)}\label{section:syllable:Final3C:Appendix:TwoAppendix}
	
	Interestingly, there are some words which have a sequence of a coda \textit{χ}, plus an appendix \textit{χ}, and then an appendix \textit{kʰ}. Orthographically, the two \textit{χ} sounds are spelled differently, suggesting their different origins (Table \ref{tab: x x k flat}). 
	
	
	\begin{table}[H]
		\centering
		\caption{Sequence of appendixes in final /χχkʰ/ clusters}
		\label{tab: x x k flat}
		\begin{tabular}{|llll|  }
			\hline 
			<aɣx> & ˈɑχχ & `baggage' & \armenian{աղխ}
			\\
			<aɣxk'> & ˈɑχχ-k & `closing' & \armenian{աղխք}
			\\ \hline 
			<owɣx> & ˈuχχ & `torrent' & \armenian{ուղխ}
			\\
			<owɣxk'> & ˈuχχ-k & `torrent' & \armenian{ուղխք}
			\\ \hline 
		\end{tabular}
		
	\end{table}
	
	We only found two such words in \citeauthor{kouyoumdjian-1970-DictionaryArmenianEnglish} that had this final cluster. 
	
	
	
	\subsection{Schwa epenthesis in other word-final three-consonant clusters}\label{section:syllable:Final3C:Schwa}
	Schwa epenthesis     occurs for final clusters of two consonants that cannot be pronounced together (\S\ref{section:syllable:Final2C:FallingOther}\ref{section:syllable:Final2C:FlatRising}). We likewise see schwa epenthesis in certain final clusters of three consonants /VCCC\#/ Such clusters are almost always syllabified with an epenthetic before the last consonant [VCCəC] and rarely with a schwa after the first consonant [VCəCC]. The final consonant is almost always a fricative or sonorant, and the preceding two-consonant cluster often has falling sonority. The reason for this is discussed in Section \S\ref{section:syllable:ConsonantClusters:Coda3C}. 
	
	\begin{itemize}[noitemsep, topsep=0pt]
		\item /VCCs/ $\rightarrow$ [VCCəs] (\S\ref{section:syllable:Final3C:Schwa:s}
		\item /VCCʁ/ $\rightarrow$ [VCCəʁ] (\S\ref{section:syllable:Final3C:Schwa:ʁ}
		\item /VCCm/ $\rightarrow$ [VCCəm] (\S\ref{section:syllable:Final3C:Schwa:m}
		\item /VCCn/ $\rightarrow$ [VCCən], very rarely [VCəCn] (\S\ref{section:syllable:Final3C:Schwa:n}
		\item /VCCɾ/ $\rightarrow$ [VCCəɾ] (\S\ref{section:syllable:Final3C:Schwa:R}
		\item /VCCl/ $\rightarrow$ [VCCəl] (\S\ref{section:syllable:Final3C:Schwa:L}
	\end{itemize}
	\subsubsection{Epenthesis before final /s/ in /VCCs/ clusters}\label{section:syllable:Final3C:Schwa:s}
	There are very few words that end in an orthographic cluster of two consonants + /s/. Of the few words we found, these clusters undergo schwa epenthesis:  <VCCs>$\rightarrow$[VCCəs] (Table \ref{tab: cc s schwa epenthesis}). 
	
	
	
	\begin{table}[H]
		\centering
		\caption{Schwa epenthesis in final 3-consonant clusters with final /s/}
		\label{tab: cc s schwa epenthesis}
		\begin{tabular}{| l|ll|l|   }
			\hline
			<yns>$\rightarrow$[jns] &<hzorako\textbf{yns}> & həzoɾɑˈku\textbf{jnəs} &        n=12
			\\
			& \armenian{հզօրագոյնս}&`powerfully'  &
			\\ \hline 
			<rk's>$\rightarrow$[ɾkʰəs] &<ne\textbf{rk's}> & ˈne\textbf{ɾkʰəs} &     n=1
			\\
			& `inside'& \armenian{ներքս}&  
			\\ \hline 
			
		\end{tabular}
		
	\end{table}
	
	In the \citeauthor{kouyoumdjian-1970-DictionaryArmenianEnglish} dictionary,  all the cases for the <yns>$\rightarrow$[jns] cluster  involved the final root  [-kʰujnəs] \armenian{-գոյնս} which is used to create a type of adverbial  superlative meaning. 
	
	Many more cases are found when the final /s/ is the 1{\sg}  possessive suffix. This suffixes always triggers a schwa after a consonant: <mart> [mɑɾtʰ] `man' \armenian{մարդ} vs. <marts> [mɑɾtʰ-əs] `my man' \armenian{մարդս}. This is discussed in \textcolor{red}{posessive schwa} 
	
	
	\subsubsection{Epenthesis before final /ʁ/ in /VCCʁ/ cluster}\label{section:syllable:Final3C:Schwa:ʁ}
	There are very few words that end in an orthographic cluster of two consonants + /ʁ/. For most words, this cluster is pronounced with schwa epenthesis before the final sound: <VCCɣ>$\rightarrow$[VCCəʁ]. The main exception is the word `star' [ɑstχ] \armenian{աստղ} discussed in  Section \S\ref{section:syllable:Final3C:Appendix:X}. 
	
	In the \citeauthor{kouyoumdjian-1970-DictionaryArmenianEnglish} dictionary, we found relatively few cases of such clusters (Table \ref{tab: cc gh schwa epenthesis}). Most of them display schwa epenthesis. 
	
	\begin{table}[H]
		\centering
		\caption{Schwa epenthesis in final 3-consonant clusters with final /ʁ/}
		\label{tab: cc gh schwa epenthesis}
		\begin{tabular}{| l|llll|l|   }
			\hline
			<sdɣ>$\rightarrow$[stəʁ] &<o\textbf{sdɣ}> & ˈvo\textbf{stəʁ} &  `lime twig' & \armenian{ոստղ}&      n=1
			\\ \hline 
			<nkɣ>$\rightarrow$[ŋkʰəʁ] &<ʃi\textbf{nkɣ}> & ˈʃi\textbf{ŋkʰəʁ} &  `snake cucumber'& \armenian{շինգղ}&      n=4
			\\ \hline 
			<ngɣ>$\rightarrow$[ŋɡəʁ] &<a\textbf{ngɣ}> & ˈɑ\textbf{ŋɡəʁ} &  `angle' & \armenian{անկղ}&      n=1
			\\ \hline 
			<rgɣ>$\rightarrow$[ɾɡəʁ] &<a\textbf{rgɣ}> & ˈɑ\textbf{ɾɡəʁ} &  `box' & \armenian{արկղ}&      n=7
			\\ \hline 
			
		\end{tabular}
		
	\end{table}
	
	Besides the above clusters, Armenian Wiktionary lists some non-dialectal words like <gownt͡sɣ> [ɡunt͡səʁ] `clod' \armenian{կունձղ}. 
	
	
	\subsubsection{Epenthesis before final /m/ in /VCCm/ clusters}\label{section:syllable:Final3C:Schwa:m}
	Word-final orthographic clusters of two consonants +  /m/ are rare. In \citeauthor{kouyoumdjian-1970-DictionaryArmenianEnglish}, we found only 3 examples (Table \ref{tab: cc m schwa epenthesis}). These three examples all look morphologically or diachronically related to each other. Here we find schwa epenthesis.
	
	
	\begin{table}[H]
		\centering
		\caption{Schwa epenthesis in final 3-consonant clusters with final /m/}
		\label{tab: cc m schwa epenthesis}
		\begin{tabular}{|llll|  }
			\hline 
			<ayʒm> & ˈɑjʒəm & `now' & \armenian{այժմ}
			\\
			<aṙayʒm> & ɑˈɾɑjʒəm & `presently' & \armenian{առայժմ}
			\\
			<t͡s'ayʒm> & ˈt͡sɑjʒəm  & `till now' & \armenian{ցայժմ}
			\\ \hline 
		\end{tabular}
		
	\end{table}
	
	Based on the above limited data, it seems that final /ʒm/ clusters just generally prefer epenthesis (cf. Section \S\ref{section:syllable:Final2C:FlatRising:FricNasalM}).
	
	\subsubsection{Epenthesis before final /n/ in /VCCn/ clusters}\label{section:syllable:Final3C:Schwa:n}
	
	There are many words that end in an orthographic cluster of two consonants plus the nasal /n/. In such words, schwa epenthesis is needed. In the most typical case, the schwa is added immediately before the nasal: <VCCn>$\rightarrow$[VCCən]. For the two-consonant cluster, this cluster is almost always a cluster would have formed a complex coda, if only the nasal /n/ was absent. It is quite rare to find cases of epenthesis as <VCCn>$\rightarrow$[VCəCn]
	
	To illustrate, Table \ref{tab: cc n schwa epenthesis} organizes the set of final /CCn/ clusters that we found in \citeauthor{kouyoumdjian-1970-DictionaryArmenianEnglish}, based on the natural class of the CC cluster.  
	
	
	\begin{table}[H]
		\centering
		\caption{Schwa epenthesis in final 3-consonant clusters with final /n/ such that the preceding consonants could have formed a complex coda}
		\label{tab: cc n schwa epenthesis}
		\begin{tabular}{|l|llll|l|  }
			\hline 
			fric. + affr. + /n/ & <ta\textbf{ɣt͡sn}> & ˈtʰɑ\textbf{χt͡sən}  & `horsemint' &  \armenian{դաղձն}  & n=2 
			\\ \hline 
			fric. + stop  + /n/ & <a\textbf{sgn}> & ˈɑ\textbf{skən}  & `garnet' &  \armenian{ասկն}  & n=4 
			\\ \hline 
			nasal  + stop      + /n/ & <pa\textbf{ngn}> & ˈpʰɑ\textbf{ŋɡən} & `tale' &  \armenian{բանկն}  & n=14
			\\ \hline 
			nasal  + affr.     + /n/ & <gow\textbf{nt͡sn}> & ˈɡu\textbf{nt͡sən}  & `clod' &  \armenian{կունձն}  & n=13
			\\ \hline 
			rhotic  + obstr.          + /n/ & <a\textbf{rtn}> & ˈɑ\textbf{ɾtʰən} & `lance' &  \armenian{արդն}  & n=38
			\\ \hline 
			rhotic  + /m/           + /n/ & <se\textbf{rmn}> & ˈse\textbf{ɾmən} & `seed' &  \armenian{սերմն}  & n=6
			\\ \hline 
			glide  + cons.     + /n/ & <to\textbf{yzn}> & ˈtʰu\textbf{jzən}  & `frivolous' &  \armenian{դոյզն}  & n=5 
			\\ \hline 
		\end{tabular}
	\end{table}
	
	For the words above, the pre-nasal consonant cluster would have formed a complex coda if the nasal was absent. For example, <towrkn> [tʰuɾkʰən] `potter's wheel' \armenian{դուրգն} vs. <t'owrk'> [tʰuɾkʰ] `Turk' \armenian{թուրք}. 
	
	Besides the above words, there are also words where the pre-nasal cluster is flat or rising-sonority (Table \ref{tab: cc n schwa epenthesis after m}). But in these clusters, the middle consonant would have formed an appendix if the nasal was absent. For example, word-final [ʁm] can form a syllabifiable cluster, whether we analyze the [m] as a coda or appendix. Adding a nasal /n/ after this cluster leads to schwa epenthesis. 
	
	
	\begin{table}[H]
		\centering
		\caption{Schwa epenthesis in final 3-consonant clusters with final /n/ such that the preceding consonants could have formed a coda+appendix}
		\label{tab: cc n schwa epenthesis after m}
		\begin{tabular}{|l|llll|l|  }
			\hline 
			<ɣmn>$\rightarrow$[ʁmən] & <sa\textbf{ɣmn}> & ˈsɑ\textbf{ʁmən} & `embryo' & \armenian{սաղմն}&   n=9
			\\
			& cf. <go\textbf{ɣm}> & ˈɡo\textbf{ʁm} & `side' & \armenian{կողմ} & 
			\\ \hline 
			<ʃxn>$\rightarrow$[ʃχən]& <sda\textbf{ʃxn}> & əsˈtɑ\textbf{ʃχən} & `styrax tree' & \armenian{ստաշխն}&   n=9
			\\
			&cf.  <pte\textbf{ʃx}> & pʰəˈtʰe\textbf{ʃχ} & `mayor' & \armenian{բդեշխ} & 
			
			\\ \hline 
		\end{tabular}
	\end{table}
	
	It is much more common for the pre-nasal cluster to have  a syllabifiable cluster like <Vrtn> (cf. <Vrt>), then to have this cluster be some unsyllabifiable    <Vktn>  (cf. <Vkt>). I found virtually no such clusters in \citeauthor{kouyoumdjian-1970-DictionaryArmenianEnglish}. The closest example I found was the word <go\textbf{zrn}> [ˈɡo\textbf{zəɾn}] `young came' \armenian{կոզռն} where the <VCCn> cluster is syllabified with a schwa after the first consonant [VCəCn] instead of before the nasal [VCCən]. The earlier location of the schwa is because [ɾn] is a licit complex coda, and schwa epenthesis tries to maximize the size of complex codas \textcolor{red}{cite schwa epenthesis}. 
	
	The relatively high number of falling-sonority clusters in before final /n/ is likely because of the diachronic origins of these clusters (\S\ref{section:syllable:ConsonantClusters:Coda3C}). 
	\subsubsection{Epenthesis before final /ɾ/ in /VCCɾ/ clusters}\label{section:syllable:Final3C:Schwa:R}
	There are many words which end in an orthographic cluster of two consonants plus the rhotic /ɾ/. Here, the rule is that these clusters are syllabified with schwa epenthesis before the rhotic: <VCCr>$\rightarrow$[VCCəɾ]. There are some attested cases of variable schwa elision in these clusters. 
	
	In \citeauthor{kouyoumdjian-1970-DictionaryArmenianEnglish}'s dictionary, there were many cases of <VCCr> words (Table \ref{tab: cc r schwa epenthesis}). Here, the preceding consonants almost always have falling sonority; if the rhotic was absent, then the cluster would have been a complex coda. But because the rhotic is present, we have schwa epenthesis.  
	
	
	\begin{table}[H]
		\centering
		\caption{Schwa epenthesis in final 3-consonant clusters with final /ɾ/ such that the preceding consonants could have formed a complex coda}
		\label{tab: cc r schwa epenthesis}
		\begin{tabular}{|l|llll|l|  }
			\hline 
			fric. + stop   + /ɾ/ & <ko\textbf{ɣdr}> & ˈkʰo\textbf{ʁdəɾ} & `tender' & \armenian{գողտր} & n=60
			\\ \hline 
			fric. + affr. + /ɾ/ & <k'a\textbf{xt͡s'r}> & ˈkʰɑ\textbf{χt͡səɾ} &`sweet'  &  \armenian{քաղցր} & n=2
			\\ \hline 
			nasal  + stop  + /ɾ/ & <sa\textbf{ndr}> & ˈsɑ\textbf{ndəɾ} & `comb'& \armenian{սանտր}   & n=23
			\\ \hline 
			nasal  + affr.   + /ɾ/ & <t'a\textbf{nt͡sr}> & ˈtʰɑ\textbf{nt͡səɾ} & `thick'  & \armenian{թանձր} &  n=2
			\\ \hline 
			rhotic   + obstr.   + /ɾ/ & <t'a\textbf{nt͡sr}> & ˈtʰɑ\textbf{nt͡səɾ} & `tough' & \armenian{կարծր} & n=10
			\\ \hline 
			glide   + cons.   + /ɾ/ & <ga\textbf{ysr}> & ˈɡɑ\textbf{jsəɾ} & `emperor' & \armenian{կարծր} & n=2
			\\ \hline 
			
		\end{tabular}
	\end{table}
	
	As with <VCCn> clusters, the final rhotic is almost always found after a consonant cluster that could have been a complex coda. I have found only one exception in the \citeauthor{kouyoumdjian-1970-DictionaryArmenianEnglish} dictionary: an obvious loanword <magdr> [mɑɡdəɾ] `mactra' \armenian{մակտր}. 
	
	
	
	
	\subsubsection{Epenthesis before final /l/ in /VCCl/ clusters}\label{section:syllable:Final3C:Schwa:L}
	There are vanishingly few words that end in an orthographic cluster of three consonants such that the final consonant is /l/: <VCCl>.
	
	In \citeauthor{kouyoumdjian-1970-DictionaryArmenianEnglish}, we only found one example, and this example is a loanword with schwa epenthesis: <mankl> [mɑŋkʰəl] `mangle' \armenian{մանգլ}.
	
	On Armenian Wiktionary, we found more cases of this final consonant, and they behaved the same in terms of schwa epenthesis. Many of them were loanwords like <ansampl> [ɑnsɑmpəl] `ensemble' \armenian{անսամբլ}. But some seemed like native words: <pangl> [pʰɑŋɡəl] `riddle' \armenian{բանկլ}.  
	\subsubsection{Epenthesis in other types of VCCC clusters}\label{section:syllable:Final3C:Schwa:Other}
	
	In word-final 3-consonant clusters,   the final consonant is almost always either an appendix /kʰ, s, χ/ or a consonant that triggers schwa epenthesis. For this second group of schwa-inducing consonants, these consonants come from a restricted set  /s, ʁ, m, n, ɾ, l/. The reason is diachronic, as explained before in Section \S\ref{section:syllable:ConsonantClusters:Coda3C}. 
	
	The \citeauthor{kouyoumdjian-1970-DictionaryArmenianEnglish} dictionary did not have any other final consonants that triggered schwa epenthesis in 3-consonant clusters. On Armenian Wiktionary, we found a handle of other possible final consonants that trigger epenthesis: <kovnt> [kʰovəntʰ] `type of circle-dance' \armenian{գովնդ}, and <eɣrt> [jeʁəɾtʰ] `type of willow' \armenian{եղրդ}. For these words, the schwa creates a complex coda because schwa epenthesis generally prefers creating complex codas over simple codas *\textit{jeʁɾətʰ}. 
	
	
	
	Some are words from other dialects like `still' \armenian{աշտխ}. We exclude them because they're not Western Armenian, so we don't know how they `should' be pronounced anyway.  
	\section{Other restrictions on complex codas}\label{section:syllable:OtherCodaRestrictions}
	Besides the restrictions on falling sonority and appendixes, there are some minor restrictions on possible complex codas. These restrictions are the following:
	\begin{enumerate}[noitemsep, topsep= 0pt]
		\item Complex codas are generally infrequent word-medially, but still attested (\S\ref{section:syllable:OtherCodaRestrictions:Infrequent}). 
		\item Some complex codas are relatively easy to form word-finally, but difficult to form word-medially, or variably avoided word-medially   (\S\ref{section:syllable:OtherCodaRestrictions:AvoidMedial}). 
		\item Some vowel-consonant combinations are generally avoided within the same syllable, specifically when the vowel is a high vowel or schwa  (\S\ref{section:syllable:OtherCodaRestrictions:VowelCons}). 
		\item Some consonant clusters receive schwa epenthesis  in  careful speech, but they can be pronounced as clusters without schwas casual speech because of schwa elision  (\S\ref{section:syllable:OtherCodaRestrictions:VowelCons:Schwa}). 
		
	\end{enumerate}
	\subsection{Infrequency and asymmetry of word-internal complex codas}\label{section:syllable:OtherCodaRestrictions:Infrequent}
	In general, Western Armenian allows word-internal complex codas. Thus, if some complex coda is attested word-finally, then it is in principle also possible word-initially and word-medially. For example, the plural suffix \textit{-neɾ} can be added after any polysyllabic word, creating a complex coda:  [mɑχ.tʰɑ\textbf{ŋkʰ}] `wish' and [mɑχ.tʰɑ\textbf{ŋkʰ}-neɾ] `wishes' \armenian{մաղթանքներ}. And for monosyllabic roots, we can add the sequence [n=e] `{\defgloss}=is' on them: [{mɑ\textbf{ɾtʰ}}] `man' and [mɑ\textbf{ɾtʰ}-n=e] `is the man' \armenian{մարդն է}. 
	
	
	Furthermore, if some complex coda is banned or atypical word-finally then it also banned or atypical  word-medially as well. For example, stop-stop complex codas like [ɡd] are generally avoided word-finally. Thus also word-medially, the complex coda  [ɡd] is avoided in native words. Loanwords unsurprisingly provide  cases of [ɡd]: [e.le\textbf{ɡd}.ɾon] `electron' \armenian{էլեկտրոն} which can be resyllabified as [e.leɡ.dɾon], thus creating a complex onset and  further signaling the non-native status of this word. 
	
	However, once we look at only uninflected words, it can be difficult to find such word-internal complex codas simply because of the following issues:
	\begin{enumerate}[noitemsep,topsep=0pt]
		\item Word-medial complex codas are relatively rare in the lexicon. 
		\item Complex codas in roots tend to be root-final instead of root-initial. 
		\item Compounds tend to insert a vowel between the roots. 
		\item C-initial   suffixes and CC-final prefixes  are few. 
	\end{enumerate}
	
	For the first point on the general rarity of word-internal complex codas,   Armenian Wiktionary (Jan 2021) has around 184,862 lexemes that have a listed syllabification.   But of these 184K words, only    8,095  words started with a (C)VCC syllable,        2,834 had a   listed word-medial CVCC,  and 296 were reported with a medial VCC syllable.\footnote{These numbers can't be fully trusted though because Armenian Wiktionary is rampant with syllabification errors.  }  These low numbers don't mean that word-internal complex codas are impossible, just relatively rarer. 
	
	For the second point on roots, Although complex codas are allowed in the language, there are relatively few roots which contain a word-medial complex coda. Thus, it is difficult to find a root example for every possible complex coda. Some of the few examples   are     [hɑ\textbf{jd}.ni] `clear' \armenian{յայտնի},     [tʰɑ\textbf{ɾkʰ}.mɑn] `translator' \armenian{թարգման},  and  [bɑ\textbf{ʃt}.pɑn] `protector' \armenian{պաշտպան}. 
	
	
	For the third point on compounds, when new words are created by adding derivational suffixes or by compounding, word-medial complex codas can be created: the compound [ɑjs-t͡ʃɑp] `this much' \armenian{այսչափ} made from the roots [ɑjs, t͡ʃɑpʰ] `this, much.' But it is much more common to include a linking vowel /-ɑ-/: [mɑ\textbf{ɾtʰ}, seɾ] `man, love' \armenian{մարդ,սէր} create a compound [mɑ\textbf{ɾ}.tʰ-ɑ-seɾ] `philanthropist' \armenian{մարդասէր} without a medial complex-coda.
	
	For the fourth point on derivational morphology, there  are a handful of    derivational prefixoids   that have complex codas: the prefixoid \textit{ɑntʰɾ-} \armenian{անդր-} is analogous to the English prefix \textit{trans-}, such as [ɑ\textbf{ntʰ}.ɾ-ɑtʰ.lɑnd.jɑn] `trans-Atlantic'  \armenian{անդրադլանտեան}. Another common example is the prefixoid [jeɾɡ-] `bi-,di-' \armenian{երկ-} such as [je\textbf{ɾɡ-}d͡ʒeχk] `bifurcation' \armenian{երկճեղք} derived from the word [d͡ʒəχk] `crack' \armenian{ճեղք}. 
	
	However, even when we use derivational suffixes,  it is relatively common for the complex coda to precede one of the following consonants:  a rhotic /ɾ/, a nasal /n/, glide /j/, a velar stop /ɡ/ (often devoiced [k]). Each type of consonant often correlates with some morphophonological idiosyncrasy. 
	
	For the rhotic, consider [vo\textbf{sk}.ɾ-ɑjin] `osseous' \armenian{ոսկրային}, derived from the root [voskoɾ] `bone' \armenian{ոսկոր} with syncope of /o/ (see \textcolor{red}{syncope chapter}.  
	
	For the nasal, consider the word  [əs.kə\textbf{sp}.n-ɑ.ˈɡɑn] `original'  \armenian{սկզբնական}, derived from [əski.\textbf{sp}] `beginning' \armenian{սկիզբ}. The nasal is not part of the suffix /-ɑɡɑn/, but is a morphologically-epenthetic nasal due to a historic relic of nasal weakening (see \textcolor{red}{nasal liaison}). 
	
	For the glide /j/, there are three common derivational suffixes that start with a glide /j/: /jɑn, jɑɡ, jɑl, jɑ/ \armenian{եան, եակ, եալ, եայ}. For all  these suffixes,  we can optionally syllabify the glide with the preceding consonant: [zəm.ɾu\textbf{χt}.jɑ, [zəm.ɾu\textbf{χ.tj}ɑ] `made of emeralds' \armenian{զմրուխտեայ}  from    [zəm.ɾu\textbf{χt}] `emerald' \armenian{զմրուխտ}.  See Section \S\ref{section:syllable:ComplexOnset:medialGlide}.  
	
	And finally for the stop /ɡ/, a common derivational suffixal is \textit{-iɡ}: [tʰɑ.pʰɑ\textbf{n.t͡s}-iɡ] `transparent' \armenian{թափանցիկ}. This suffix can undergo further derivation and undergo vowel reduction, creating a complex coda: [tʰɑ.pʰɑ\textbf{nt͡s}.k-u.tʰjun] `transparency' \armenian{թափանցկութիւն}. This suffix \textit{-iɡ} is also used to form an irregular plural suffix for the word [mɑɾtʰ] `person' \armenian{մարդ} as [mɑ\textbf{ɾtʰ}-iɡ] `people' \armenian{մարդիկ}. This irregular plural can undergo derivational morphology,   creating a complex coda:  [mɑ\textbf{ɾt}.-k-ɑ.ɡɑn] `humane'  \armenian{մարդկական}. 
	
	
	To demonstrate these tendencies, we took the 2,834  Armenian Wiktionary which were listed as having a word-medial CVCC syllable.   From this set, the majority (n=848, 29.92\%) preceded the fricative /v/. This fricative was the passive suffix  /v/ (spelled \armenian{ու} in the traditional orthography, \armenian{վ} in reformed). This suffix can follow complex codas in Eastern Armenian [ʃərɑjl-v-e-l] `to be wasted' \armenian{շռայլվել}, but it triggers a schwa after CC clusters in Western Armenian [ʃəɾɑjlə-v-i-l]. Thus this group is inadmissible in Western Armenian. After /v/, the other most common   post-coda sounds were  /ɾ/ \armenian{ր} (n=520, 18.35\%), 
	/n/ \armenian{ն} (n=341, 12.03\%),   /j/ \armenian{յ} (n=254, 	8.96\%), and /ɡ/ \armenian{կ} (n=131, 4.62\%). Besides these frequent consonants, we also found  <b> \armenian{պ}  (n=77, 2.72\%) but these were mostly due to the root  [\textbf{bɑʃt}.pɑn] `protector' \armenian{պաշտպան}.  
	
	Thus for medial CVCC syllables   on Armenian Wiktionary, the above consonants accounted for over 75\% of reported consonants that follow a complex coda in Eastern Armenian. The other 25\% seemed like an arbitrary distribution of other consonants without any clear generalizations. 
	
	In sum, Armenian allows complex codas anywhere in the word. But there are some accidental asymmetries due to the structure of the Armenian lexicon and morphology. 
	
	\subsection{Avoidance of certain word-medial  complex codas}\label{section:syllable:OtherCodaRestrictions:AvoidMedial}
	The previous section explained that essentially every possible word-final complex coda is also possible word-medially. But there are certain consonant clusters that are permitted word-finally, but seem to be avoided word-medially. These avoided clusters are  grouped as the following. We go through each case below. 
	\textcolor{red}{cite tese sections into the previous sections}
	\begin{itemize}[noitemsep, topsep=0pt]
		\item /Cm/:  /jm, ɾm, ʁm, zm, ʃm, hm/ (\S\ref{section:syllable:OtherCodaRestrictions:AvoidMedial:Cm})
		\item /ɾn/  (\S\ref{section:syllable:OtherCodaRestrictions:AvoidMedial:rN})
		\item nasal-fricative: /ns, ms/  (\S\ref{section:syllable:OtherCodaRestrictions:AvoidMedial:NasalFric})
		\item fricative-fricative:   /χs, χʃ, ʃχ/ (\S\ref{section:syllable:OtherCodaRestrictions:AvoidMedial:FricFric})
		\item appendix /C-kʰ/ (\S\ref{section:syllable:OtherCodaRestrictions:AvoidMedial:Ck})
		\item geminates (\S\ref{section:syllable:OtherCodaRestrictions:AvoidMedial:Gemin})
	\end{itemize}
	
	
	\subsubsection{Avoiding word-medial /Cm/ complex codas}\label{section:syllable:OtherCodaRestrictions:AvoidMedial:Cm}
	Word-finally, /jm/ is a very rare cluster: [ɡɑ\textbf{jm}] `mast' (\S\ref{section:syllable:Final2C:FallingCommon:GlideCons}). Inflection with the definite suffix /-n/ can create a word-medial complex coda: [ɡɑ\textbf{jm}-n=e] `must-{\defgloss}=is' \armenian{կայմն է}. But we could not find any other cases word-medial [jm] complex coda. 
	
	Word-finally, /ɾm/ can form a complex coda as in [t͡ʃe\textbf{ɾm}]  `warm' \armenian{ջերմ} (\S\ref{section:syllable:Final2C:FallingCommon:RhoticNasalM}). But word-medially, it seems that this [ɾm] is almost always before  a nasal /n/ or glide /j/: [t͡ʃe\textbf{ɾm}.n-ɑ.ɡɑn] `feverish' \armenian{ջերմնական}. Here the nasal is epenthetic because of an obscure morphological rule (see \textcolor{red}{nasal liaison}).\footnote{In very few cases, we found root-final [ɾm] precede a glide, with possible resyllabification: [χɑ\textbf{ɾm}.jɑ, χɑ\textbf{ɾ}.mjɑ] `fake jewel' \armenian{խարմեայ} where /-ja/ is a derivational suffix.}  Exceptions are few, such as [ɑ\textbf{ɾm}.dikʰ] `cereal' \armenian{արմտիք}. 
	
	
	Further evidence for the above  restriction comes from vowel reduction. There are a few roots which can create a [ɾm] complex coda when undergoing reduction: from [ɡɑɾ.miɾ] `red' \armenian{կարմիր} to [ɡɑ\textbf{ɾm}.ɾil] `to grow red' \armenian{կարմրիլ}. But, for this root, it is impressionistically more common to have a schwa between the two consonants: [ɡɑɾ.mə.ɾil]. Similar reduction patterns are found for the root [mɑɾmin] `body' \armenian{մարմին} but [mɑ\textbf{ɾm}.n-ɑɡɑn, mɑɾ.mə.n-ɑɡɑn] `corporeal' \armenian{մարմնական}. In HD's impression, the schwa form is more common, while the schwaless form sounds higher register. 
	
	
	Word-finally, fricative + /m/ clusters are attested (\S\ref{section:syllable:Final2C:FlatRising:FricNasalM}), but each displays some idiosyncrasies word-medially. 
	
	The final cluster /ʁm/ can form a complex coda even though it has rising sonority: [ɡo\textbf{ʁm}] `side' \armenian{կողմ}.  It is possible to analyze this cluster as comprised of a coda and appendix. Word-medially, this coda can be created, but it seems to almost always precede either a nasal /n/ or glide /j/: [ɡo\textbf{ʁm}.n-ɑɡɑn] `lateral' \armenian{կողմնական} due to an epenthetic nasal (), and [kʰɑɾ-ɑ-ɡo\textbf{ʁm}-jɑn] \armenian{quadrilateral} \armenian{քառակողմեան} with possible resyllabification  [kʰɑɾ-ɑ-ɡo\textbf{ʁ.mj}ɑn]. 
	
	As before,   this suggests that the medial complex coda [ʁm] has to before /n/ or /j/. Further evidence comes from vowel reduction. The root [ɑʁmuɡ] `noise' \armenian{աղմուկ} reduces to [ɑʁ\textbf{məɡ}-e-l] `to disturb' \armenian{աղմկել}. The high vowel is replaced by a schwa, and we can't just delete the schwa to form *\textit{ɑʁm.ɡel}. The avoidance of a schwa here is evidence that [ʁm] is an undesired complex coda before an obstruent like /ɡ/. 
	
	For /ʃm/, this rising-sonority cluster is attested word-finally [tʰə.ɾo\textbf{ʃm}] `stamp' \armenian{դրոշմ}. However, word-medially, the only cases we could find involved adding /n/-initial inflection: [tʰə.ɾɒ\textbf{ʃm}-ner] `stamps' \armenian{դրոշմեր}. We couldn't find clear cases of such clusters before derivational suffixes. Further, \citeauthor{kouyoumdjian-1970-DictionaryArmenianEnglish} only listed one word with an intervocalic /VʃmCV/ cluster, and this cluster gets schwa epenthesis in HD's judgments: <ga\textbf{ʃm}powṙn>  [ɡɑ\textbf{ʃəm}pʰuɾən] `robust' \armenian{կաշմբուռն}, instead of *\textit{ɡɑ\textbf{ʃm}.pʰuɾən}. 
	
	
	For /zm/, this rising-sonority cluster can form a word-final complex coda: [bɑ.de.ɾɑ\textbf{zm}] `war' \armenian{պատերազմ}. It can become word-medial by adding the  plural suffix \textit{-neɾ}: [bɑ.de.ɾɑ\textbf{zm}-neɾ] `wars' \armenian{պատերազմներ}. Derivational suffixation again shows a tendency for a subsequent /j/: [bɑ.de.ɾɑ\textbf{zm}.jɑn] `of war' \armenian{պատերազմեան}, with possible resyllabification  [bɑ.de.ɾɑ\textbf{z}.mjɑn] (\S\ref{section:syllable:ComplexOnset:medialGlide}). In contrast, vowel reduction avoids creating this cluster: [χɑz.\textbf{muz}] `must (n)' \armenian{խազմուզ} derives [χɑz.\textbf{mə.z}-ɑt͡ʃɑpʰ] `gleucometer' \armenian{խազմզաչափ} instead of  *\textit{χɑ\textbf{zm}.zɑ.t͡ʃɑp}. 
	
	The other fricative-/m/ clusters also seem absent word-medially except before /n/-initial inflection. For example,  medial /hm/ can  be formed in [do\textbf{hm}-n=e] `family-{\defgloss}=is' \armenian{տոհմն է}. But we could not find other word-medial case. And /sm/ is too rare even word-finally. 
	
	\subsubsection{Avoiding word-medial /ɾn/ complex codas}\label{section:syllable:OtherCodaRestrictions:AvoidMedial:rN}
	
	For /ɾn/, these cluster can be pronounced word-finally as a complex coda [t͡se\textbf{ɾn}] `hand' \armenian{ձեռն}. But schwa epenthesis is much more typical [t͡se\textbf{ɾən}] (\S\ref{section:syllable:Final2C:FallingOther:RhoticNasalN}. Word-medially before a consonant, epenthesis is the norm and (in HD's judgment) the only possible pronunciation, such as in the derived word [t͡se\textbf{ɾən}-pʰeɡ] `one-handed' \armenian{ձեռնբեկ}. Schwa epenthesis in bound roots likewise avoids creating [ɾn] complex codas: <v\textbf{ɾ̇n}del>$\rightarrow$[və\textbf{ɾən}del] `to expel' \armenian{վռնտել} instead of *\textit{və\textbf{ɾn}.del}. 
	
	\subsubsection{Avoiding word-medial nasal-fricative  complex codas }\label{section:syllable:OtherCodaRestrictions:AvoidMedial:NasalFric}
	
	
	For /ns/, this cluster is extremely rare word-finally. As surveyed in Section \S\ref{section:syllable:Final2C:FallingOther:NasalFric}, final [ns] complex codas seem to be primarily loanwords like [finɑ\textbf{ns}] `finance' \armenian{ֆինանս}. Word-medially, possible /VnsCV/ clusters arise via making compounds of the bound root /oɾens-/. Before C-initial suffixes or roots, this /ns/ cluster be pronounced with or without a schwa: [oɾe\textbf{ns}-tiɾ, soɾe\textbf{nəs}-tiɾ] `legislator' \armenian{օրէնսդիր}. In HD's judgment, the schwa form is much more typical in Western Armenian. VP informs us that the schwa-less form is however more common in Eastern Armenian.  Thus, it seems that for Western Armenian at least, medial [ns] complex codas are dispreferred. 
	
	For /ms/, this cluster is relatively rare word-finally and it can form complex codas: [do\textbf{ms}] `ticket' \armenian{տոմս}. Inflection can also create a medial case:  [do\textbf{ms}-n=e] `ticket-{\defgloss}=is' \armenian{տոմսն է}. This cluster is found     word-medially for words that are derived from irregular root   [ɑmis] `month' \armenian{ամիս}. Such as the genitive    [ɑ\textbf{ms}-vɑ] `month-{\gen}' \armenian{ամսուայ} with an optional schwa  [ɑ\textbf{məs}-vɑ].  Other cases seem too always involve adding a glide /j/ after the [ms] cluster: [ɑmen-ɑ\textbf{ms}-jɑ] `monthly' \armenian{ամենամսեայ}, and the /VmsjV/ cluster can syllabify as either [Vms.jɑ] or [Vm.sjɑ]. 
	\subsubsection{Avoiding word-medial fricative-fricative clusters   }\label{section:syllable:OtherCodaRestrictions:AvoidMedial:FricFric}
	
	For the fricative-fricative clusters  like /ʃχ, χʃ, χs/.   These clusters are rare word-finally, thus even rarer word-medially (\S\ref{section:syllable:Final2C:FlatRising:FricFric}). So it's difficult to know whether their word-medial rarity is due to statistical change or something more. Some of the examples that we found are  [əmpʰo\textbf{ʃχ}nel] `to savor' \armenian{ըմբոշխնել}, [bɑ\textbf{χʃ}kil] `to be refreshed' \armenian{պաղշկիլ}, [tʰə\textbf{χs}.mɑjɾ] `brooding hen' \armenian{թխսմայր}. 
	
	\subsubsection{Avoiding word-medial /C-kʰ/ appendixes }\label{section:syllable:OtherCodaRestrictions:AvoidMedial:Ck}
	
	
	For /C-kʰ/, as said many times, the nominalizer suffix \textit{-kʰ} can be added after any consonant cluster. 
	
	For example, consider a polysyllabic word like    the verb [dɑɾɑd͡z-el] `to spread' \armenian{տարածել}. We can derive the noun [dɑɾɑt͡s-k] `spread' \armenian{տարածք}. The appendix \textit{k} can be made word-medial by adding an inflectional suffix: [dɑɾɑt͡s-k-neɾ] `spread-{\pl}' \armenian{տարածքներ}. 
	
	\textcolor{red}{cite vaux dolation}
	
	However, the morphology generally avoids placing this suffix between a consonant-final and a consonant-initial derivational suffix morphology. For example, the root [mud] `entry (archaic') \armenian{մուտ} has a more common form as [mut-k] `entry' \armenian{մուտք} with the nominalizer.  It can be preceded by C-initial inflection like [mut-k-n=e] `entry-{\defgloss}-is' \armenian{մուտքն է}.   The root `entry' [mut-k] can also be used in compounds like [ɑɾev-mut-k] `West' \armenian{արեւմուտք}  where [ɑɾev] means `sun'. 
	
	The adjective `Western' however is formed by deleting the \textit{kʰ}, adding \textit{-jɑn}, and reducing the root vowel: [ɑɾev-məd-jɑn] `Western' \armenian{արեւմտեան}.  Because of this apparent avoidance stem-medial \textit{-kʰ}, most analyses of the appendix \textit{-kʰ} argue that this suffix must be at the end of the word before inflection. I have only found a handful of counter-examples: [ɑɾev-mut-k-t͡si]  `Westerner' \armenian{արեւմուտքցի}. 
	
	\subsubsection{Avoiding word-medial geminate complex codas}\label{section:syllable:OtherCodaRestrictions:AvoidMedial:Gemin}
	
	For geminates, because geminate codas are extremely rare word-finally, it's not surprising that they are rarer word-medially. The main example that we found is in the derivatives of the word [məɾ.ɾiɡ] `tempest' \armenian{մրրիկ} which form geminate codas: [mə\textbf{ɾɾ}.ɡɑ.jin] `turbulent' \armenian{մրրկային}. However, HD reports that it's also common to remove this medial geminate by either degemination  or by using a schwa: [mə\textbf{ɾ}.ɡɑ.jin, mə.\textbf{ɾəɾ}.ɡɑ.jin]. This suggests that word-medial geminates are somewhat avoided word-medially. 
	
	\subsection{Vowel-based or consonant-based restrictions on codas and complex codas}\label{section:syllable:OtherCodaRestrictions:VowelCons}
	For virtually any pair of vowels and consonants, we can create a coda or complex coda. But, there are some restrictions for glides and  /i,u/ (\S\ref{section:syllable:OtherCodaRestrictions:VowelCons:High}). There are more restrictions for the schwa /ə/ (\S\ref{section:syllable:OtherCodaRestrictions:VowelCons:Schwa}). And the front vowel /ʏ/ simply always takes a coda when in the final syllable (\S\ref{section:segmentalPhono:vowel:frontRound}). 
	
	\subsubsection{Restrictions on /j/ and /v/ codas}  \label{section:syllable:OtherCodaRestrictions:VowelCons:High}
	
	
	The consonants  /j/ and /v/  are   relatively common codas in Armenian. But they have various restrictions on what types of vowels they can follow. 
	
	First, in word-final complex codas like /VjC/, the glide only follows /ɑ/ or /u/ in native words. For example, in \citet{kouyoumdjian-1970-DictionaryArmenianEnglish}'s dictionary, we found 703 words with final /ɑjC/ like [lɑjn] `wide' \armenian{լայն}, and 1272 words with final /ujC/ like d [kʰujn] `color' \armenian{գոյն}; see more examples in Section \S\ref{section:syllable:Final2C:FallingCommon:GlideCons}. 
	
	
	The only other vowel that we found in \citet{kouyoumdjian-1970-DictionaryArmenianEnglish} was /e/ in a single loanword: [kʰempʰejn] `campaign' \armenian{քէմփէյն}. Wiktionary had more cases of loanwords with /ej/.  Wiktionary also had some examples of final /əj,oj/, but these were all non-Western words or loanwords like [kɑzojl] `gas oil' \armenian{գազոյլ}. 
	
	Second,   word-final simplex codas like /Vj/ are generally rare (Table \ref{tab:vj list}). In the \citet{kouyoumdjian-1970-DictionaryArmenianEnglish} dictionary,   there are many words (n=958) that are spelled with a final glide <y> \armenian{յ}; but  most such cases involve a silent unpronounced glide like <d͡zaṙay> [d͡zɑɾɑ] `servant' \armenian{ծառայ}. After factoring out silent glides, we found only 62 words with a final glide coda. These have either /ɑ, e, o/. 
	
	\begin{table}[H]
		\centering
		\caption{Restriction on word-final /j/ codas}
		\label{tab:vj list}
		\begin{tabular}{|l|lll|l|}
			\hline 
			/ɑj/ & ˈh\textbf{ɑj} & `Armenian' & \armenian{հայ} & n=41
			\\
			& ɾusɑˈh\textbf{ɑj} & `Russian-Armenian' & \armenian{ռուսահայ} &  
			\\
			& ˈpʰ\textbf{ɑj} & `verb'  '  & \armenian{բայ} &  
			\\
			\hline 
			/ej/ & ˈtʰ\textbf{ej} & `tea' & \armenian{թէյ} & n=2 
			\\
			& ˈb\textbf{ej} & `bey' & \armenian{պէյ} & 
			
			\\ \hline 
			/oj/ &ˈkʰ\textbf{oj} & `existent'& \armenian{գոյ} &  n=19
			\\
			&ˈχ\textbf{oj} & `ram' & \armenian{խոյ} &  
			\\
			& iŋkⁿnɑˈkʰ\textbf{oj}& `self-existent' & \armenian{ինքնագոյ} & 
			\\
			\hline 
			
		\end{tabular}
	\end{table}
	
	For /uj/, this vowel+coda sequence is unattested word-finally in \citet{kouyoumdjian-1970-DictionaryArmenianEnglish}. On Wiktionary, we found a handful of /uj/-final words in loanwords and interjetions, such  [uj] `ouch' \armenian{ույ}.	Word-medially, /uj/ can be easily derived via   resyllabification. For example, /u/ can be used with /j/ as in [kʰ\textbf{ujn}] `color' \armenian{գոյն} but [kʰ\textbf{uj}.neɾ] `colors' \armenian{գոյներ}. 
	
	Third,   /ij/ seems to be rare even as a word-medial syllable.   For example in the \citeauthor{kouyoumdjian-1970-DictionaryArmenianEnglish} dictionary, there is only one such word that has a syllable ending in /ij/:  [\textbf{ij}.nɑl] `to fall' \armenian{իյնալ}. This is a high-frequency irregular verb.  However, in HD's judgment, it's also common to pronounce this word as just [i.nɑl] without a glide.  
	
	Wiktionary listed only a handful of other examples of [(C)ij] syllables, but these were all either obscure dialectal words, or loanwords like [novoɾos\textbf{ij}sk] `Novorossiysk' \armenian{Նովոռոսիյսկ}. 
	
	Fourth, alongside /ij/, a similar restriction seems to be that  /uv/ codas are rare or banned (Table \ref{tab:v consonant v list}). We couldn't find any  such cases in \citeauthor{kouyoumdjian-1970-DictionaryArmenianEnglish}. On Wiktionary, the handful of examples we found were either obscure dialectal words, or loanwords like  [ve.z\textbf{uv}] `Vesuvius' \armenian{վեզուվ}. In contrast, \citet{kouyoumdjian-1970-DictionaryArmenianEnglish} had many       word-final /v/ codas for other vowels. 
	
	\begin{table}[H]
		\centering
		\caption{Restriction on word-final    /v/  codas}
		\label{tab:v consonant v list}
		\begin{tabular}{|l|lll|l|}
			\hline 
			/ɑv/ & ˈh\textbf{ɑv} & `chicken' & \armenian{հաւ} & n= 410 
			\\ & pʰəˈn\textbf{ɑv} & `never' & \armenian{բնաւ} & 
			\\
			\hline 
			/ev/ & t͡s\textbf{ev} & `form' & \armenian{ձեւ} & n=426
			\\ &  ɑˈɾ\textbf{ev} & `sun' & \armenian{արեւ} & 
			\\
			\hline 
			/ov/ &ˈd͡z\textbf{ov} & `sea' & \armenian{ծով} & n=144
			\\ & ʒoˈʁ\textbf{ov} & `meeting' & \armenian{ժողով} & 
			\\
			\hline 
			/iv/ & ˈtʰ\textbf{iv} & `number' & \armenian{պատիւ} & n=155
			\\&  bɑˈd\textbf{iv} & `honor' & \armenian{պատիւ} & 
			\\
			\hline 
			
		\end{tabular}
	\end{table}
	
	It is likely that the reason why /ij, uv/ codas are generally banned is because the vowel and coda would be too similar. The glide /j/ is phonologically a consonant-form of /i/. As for /v/, Armenian does not have a productive /w/ phoneme, so /v/ is the most similar consonant form of /u/.  Such similarities for /i/-/j/ and  /u/-/v/ are also found in vowel hiatus repair (\S\ref{section:syllable:VowelHiatus}). 
	
	Although [ij] and [uv] codas are generally banned, it is possible to have [i.jV] and [u.vV] sequences. That is, the sequence /ij/ can be formed as long as the glide is part of the next syllable: [kʰini-ji] `wine-{\gen}' \armenian{գինիի}. Such a sequence is easily created because the regular genitive suffix is \textit{-i}. 
	
	Similarly for /uv/, such sequences are attested when the /v/ is part of the next syllable: [ve.ɾ\textbf{u.v}ɑɾ] `up' \armenian{վերուվեր}. Though it seems that even [u.v] sequences are rather infrequent. 
	
	And although /ij/ and /uv/ are generally avoided as codas, it is quite easy to form /ji/ and /vu/ syllables. For /ji/,  this syllable is easily created when the genitive suffix /i/ triggers  glide epenthesis after a vowel: [d͡zɑɾɑ-ji] `servant-{\gen}' \armenian{ծառայի}. The /vu/ sequence is easily made when the productive nominalizer /-um/ is added after a /v/-final root: [χoɾov-um] `roasting' \armenian{խորովում}. Furthermore, the sequences /iv/ and /uj/ are also easily found codas: [tʰ\textbf{iv}] `number' \armenian{թիւ}, [z\textbf{uj}kʰ] `twin' \armenian{զոյգ}. 
	
	
	
	
	
	
	\subsubsection{Restrictions on codas for  schwas} \label{section:syllable:OtherCodaRestrictions:VowelCons:Schwa}
	
	The schwa is a common vowel in Armenian and it can have virtually any type of coda or complex coda. But there are some restrictions involving  a) glide codas (\S\ref{section:syllable:OtherCodaRestrictions:VowelCons:Schwa:Glide}, b) rhotic-fricative complex codas (\S\ref{section:syllable:OtherCodaRestrictions:VowelCons:Schwa:RhoticFric}), and c) rhotic-nasal complex codas (\S\ref{section:syllable:OtherCodaRestrictions:VowelCons:Schwa:RhoticNasal}).   
	
	
	
	\subsubsubsection{Schwas avoid glide codas} \label{section:syllable:OtherCodaRestrictions:VowelCons:Schwa:Glide}
	
	First, it seems that the schwa cannot have a glide /j/ coda. We could not find any examples /(C)əj/ syllables in \citeauthor{kouyoumdjian-1970-DictionaryArmenianEnglish}. As for Wiktionary, we did find a handful of examples but these were all obscure dialectal words that don't exist in Western Armenian. 
	
	Furthermore, it seems that even the sequence /əj/ is banned regardless of syllable structure. That is, it is very rare to find a word where a schwa precedes a glide /j/. The closest examples we could find were atypical colloquial pronunciations of word-initial [CʏC] (\S\ref{section:segmentalPhono:vowel:frontRound}).   Very rarely, a  [CʏC] word like like [kʰʏʁ] `village' \armenian{գիւղ}  can be pronounced as [CəjuC] or [CəjʏC]: [kʰə.juʁ, kʰə.jʏʁ]. 
	
	Note that although /əj/ is a rare sequence, the sequence /jə/ is quite common. This is because the definite suffix is \textit{-ə} after glide-final roots: [pʰɑj-ə] `verb-{\defgloss}' \armenian{բայը}. 
	
	\subsubsubsection{Schwas avoid rhotic-fricative codas codas} \label{section:syllable:OtherCodaRestrictions:VowelCons:Schwa:RhoticFric}
	
	Second, it seems that the schwa avoids having a rhotic-fricative complex coda like [əɾs], [əɾʃ], and so on. The evidence comes mainly from schwa epenthesis.  Data is summarized in Table \ref{tab:ərs good bad}. 
	
	\begin{table}[H]
		\centering
		\caption{Avoidance of schwa-rhotic-fricative syllables }
		\label{tab:ərs good bad}
		\begin{tabular}{|l|lllll|}
			\hline 
			<CrsC>&<b\textbf{ṙs}del> & bə.\textbf{ɾəs}.tel & *b\textbf{əɾs}.tel & `to wrinkle'&  \armenian{պռստել}  \\
			$\rightarrow$[Cə.ɾəs.C]         &<t'\textbf{ṙs}nel> & tʰə.\textbf{ɾəs}.ˈnel & *tʰ\textbf{əɾs}.nel & `to soften'  &  \armenian{թռսկել}
			\\
			\hline 
			<CrʃC> &<p'\textbf{ṙʃ}dal> & pʰə.\textbf{ɾəʃ}.tɑl & *pʰ\textbf{əɾʃ}.tɑl & `to sneeze'&  \armenian{փռշտալ}  \\
			$\rightarrow$[Cə.ɾəʃ.C]       &<t'\textbf{ṙʃ}nel> & tʰə.\textbf{ɾəʃ}.ˈnel & *tʰ\textbf{əɾʃ}.nel & `to wither'  &  \armenian{թռշնել}
			\\
			\hline 
			<CrχC> &<t͡ʃ'\textbf{rχ}gal> &t͡ʃə.\textbf{ɾəχ}.ˈkɑl & *t͡ʃ\textbf{əɾχ}.kɑl & `to clatter'  &  \armenian{չրխկալ}
			\\
			$\rightarrow$[Cə.ɾəχ.C]        &<t'\textbf{rχ}gal> &tʰə.\textbf{ɾəχ}.ˈkɑl & *tʰ\textbf{əɾχ}.kɑl & `to rattle'  &  \armenian{թրխկալ}
			\\
			\hline 
			
		\end{tabular}
	\end{table}
	
	For [əɾs], the orthography provides few words where we have a /CɾsC/ cluster that would need schwa epenthesis. In these few cases, the schwa is added between the /ɾs/ cluster: <s\textbf{rs}gel> [sə.\textbf{ɾəs}.kel] `to sprinkle' \armenian{սրսկել}. If an [əɾs] complex coda was easily allowed, then we would incorrectly expect *\textit{s\textbf{əɾs}.kel}, cf.  [h\textbf{ɑɾs}.nikʰ] `wedding' \armenian{հարսնիք}. 
	
	Similar behavior is found for [əɾʃ] : <p'\textbf{ṙʃ}ni> [pʰə.\textbf{ɾəʃ}.ni] `type of tree', cf. [ɡ\textbf{ɑɾʃ}.neʁ] `sinewy' \armenian{կարշնեղ}. And for [əɾχ]: <t\textbf{rχ}got͡s'> [tʰə.\textbf{ɾəχ}.kot͡s] `gunshot' \armenian{դրխկոց},    cf. [m\textbf{ɑɾχ}] `resinous pine' \armenian{մարխ}.
	
	
	For the other  orthographic combinations of consonant +  rhotic + fricative combinations, these clusters were either absent or astonishingly rare in dictionaries. We could find suitable examples to see how they would be pronounced in Western Armenian.   
	
	
	Note that although the general behavior is for a schwa /ə/ to avoid getting a rhotic-fricative complex codas, such syllables do   occur in special morphologically-induced circumstances. Consider the word [tʰu\textbf{ɾs}] `outside' \armenian{դուրս}. Before a vowel-initial derivational suffix, the root's high vowel is reduced to a schwa: [tʰə\textbf{ɾ.s}-e.t͡si] `foreigner' \armenian{դրսեցի}. In very rare cases,     a nasal is inserted after this root: [tʰə\textbf{ɾs}.n-ɑɡɑn] `exterior' \armenian{դրսնական}. In such derivatives, the morphology allows a [əɾs] syllable to maintain similarity with the root [tʰuɾs], and to avoid a non-similar output like *\textit{tʰə.\textbf{ɾəs}.nɑ.ɡɑn}.\footnote{This need for similarity of roots  does show some variation.  For example [m\textbf{əɾ.s}-i-l] `to feel cold' \armenian{մրսիլ} and its derivative [m\textbf{əɾs}.-kod, mə.\textbf{ɾəs}.-kod]   `chilly' \armenian{մրսկոտ}. } 
	
	If we look beyond Western Armenian and into Eastern Armenian, we find similar analogical effects happen in passivization \textcolor{red}{passive chapter}. In both dialects, the passive stem tends to be phonologically identical to the active stem. In Western Armenian, the passive suffix /v/ cannot follow any complex coda, so it triggers a schwa whether after any rhotic-fricative cluster: [tʰ\textbf{əɾ.ʒ}-e-l] `to infringe' \armenian{դրժել} but passive [tʰ\textbf{əɾ.ʒ}ə-v-i-l] `to be infringed' \armenian{դրժուիլ}, cf. [m\textbf{ɑɾz}-e-l] `to exercise' \armenian{մարզել} and passive [m\textbf{ɑɾ.z}ə-v-i-l] `to be exercised' \armenian{մարզուիլ}.
	
	But in Eastern Armenian, the passive /v/ can follow complex codas: [m\textbf{ɑɾz}.-v-e-l] \armenian{մարզվել}, and it  can follow [əɾF] syllables:   [d\textbf{əɾʒ}.-v-e-l] \armenian{դրժվել}.  The passive data thus shows that schwas generally avoid having a rhotic-fricative complex coda, unless there is a morphological pressure to maintain identity of roots. 
	
	\subsubsubsection{Schwas avoid word-medial rhotic-nasal codas} \label{section:syllable:OtherCodaRestrictions:VowelCons:Schwa:RhoticNasal}
	Because of how the morphology of Armenian works, the orthography creates consonant + rhotic + fricative clusters word-medially, not word-finally. Thus, although word-medial [əɾs] complex codas are banned, it is unknown whether these clusters are fine word-finally. This ambiguity is absent for word-medial rhotic-nasal complex codas, summarized in Table \ref{tab:ərnasl good bad}.   
	
	
	
	\begin{table}[H]
		\centering
		\caption{Avoidance of word-medial schwa-rhotic-nasal syllables  }
		\label{tab:ərnasl good bad}
		\begin{tabular}{|l|lllll|}
			\hline 
			<CrmC> &<k\textbf{rm}p'al> & kʰə.\textbf{ɾəm}.ˈpʰɑl &  *kʰ\textbf{əɾm}.pʰɑl & `to thump'&  \armenian{գրմփալ}  \\
			$\rightarrow$[Cə.ɾəm.C]      &<x\textbf{ṙm}p'et͡s'> & χə.\textbf{ɾəm}.ˈpʰot͡s  & *χ\textbf{əɾm}.pʰot͡s & `snore'  &  \armenian{խռմփոց}
			\\
			\hline 
			<CrnC> &<z\textbf{rn}kal> & zə.\textbf{ɾəŋ}.kʰɑl & *z\textbf{əɾŋ}.kʰɑl & `to tinkle'&  \armenian{զրնգալ}  \\
			$\rightarrow$[Cə.ɾən.C]       &<h\textbf{ṙn}t͡ʃ'el> & hə.\textbf{ɾən}.ˈt͡ʃel & *h\textbf{əɾn}.t͡ʃel & `to snort'  &  \armenian{հռնչել}
			\\
			\hline 
			
		\end{tabular}
	\end{table}
	
	
	Interestingly, although the schwa can't have a rhotic-fricative complex coda, it can have a  word-final /ɾn/ complex coda : [ɡo.z\textbf{əɾn}] `young camel' \armenian{կոզռն}. But word-medially, such complex codas are generally avoided for all types of vowels: <a\textbf{ṙn}t͡ʃ'agan> [ɑ.\textbf{ɾən}.t͡ʃɑ.ˈɡɑn] `relative' \armenian{առնչական}  (\S\ref{section:syllable:OtherCodaRestrictions:AvoidMedial:rN}). Schwas also avoid this complex coda word-medially:  <p\textbf{ṙn}gil> [pʰə.\textbf{ɾəŋ}.ɡil] `to be inflamed' \armenian{բռնկիլ}. 
	
	For /ɾm/, this cluster can be formed word-medially for non-schwa vowels, but with various restrictive tendencies (\S\ref{section:syllable:OtherCodaRestrictions:AvoidMedial:Cm}). For the schwa, it seems that it cannot have a /ɾm/ complex: <x\textbf{ṙm}p'al> [χə.\textbf{ɾəm}.pʰɑl] `to snore' \armenian{խռմփալ}. 
	
	
	
	\subsection{Complex codas created by schwa elision}\label{section:syllable:OtherCodaRestrictions:Elision}
	\textcolor{red}{write after schwa elision}
	abstamp
	\section{Complex onset restrictions}\label{section:syllable:ComplexOnset}
	The typical syllable in Armenian can be described as CVC or CVCC without a complex onset (\S\ref{section:syllable:ConsonantClusters:Onset}). However, in principle, Armenian does creating complex onsets. But the possible types of complex onsets are significantly restricted and they can categorized as one of the following:
	
	\begin{enumerate}[noitemsep,topsep=0pt]
		\item consonant + glide sequences that are formed word-initially  (\S\ref{section:syllable:ComplexOnset:glide})
		\item consonant + glide sequences that are formed word-medially but with  variable resyllabification (\S\ref{section:syllable:ComplexOnset:medialGlide})
		\item consonant-consonant clusters that typically get a schwa [CəCV] but can lose the schwa in casual speech from schwa elision [CCV] (\S\ref{section:syllable:ComplexOnset:Elision})
	\end{enumerate}
	
	Another category are  consonant + sonorant sequences from non-nativized loanwords. For example, the name <k'lara> [kʰlɑɾɑ] `Clara' \armenian{Քլարա} or the word <gram> [ɡəɾɑm] `gram' \armenian{կրամ}. However, such loanwords can be nativized by adding a schwa: [kʰəlɑɾɑ, ɡəɾɑm]. We don't discuss such loanwords further because they are quite limited. 
	
	
	\subsection{Complex onsets of consonant + glide}\label{section:syllable:ComplexOnset:glide}
	
	Grammars often report that the only acceptable `normal' complex onset is a consonant + glide /j/ combination, such as [ɡjɑŋkʰ] `life' \armenian{կեանք}. However, although such complex codas exist, they are significantly restricted in their distribution. These restrictions involve the following asymmetries:
	\begin{enumerate}[noitemsep,topsep=0pt]
		\item Few word-initial /Cj/ sequences. 
		\item   /Cj/ is restricted to being almost always before /ɑ// 
		\item  Variation in how word-medial /Cj/ clusters are syllabified. 
	\end{enumerate}
	
	First, word-initially, such [Cj] clusters are limited to a handful of roots (Table \ref{tab: cj word imitial}). The \citeauthor{kouyoumdjian-1970-DictionaryArmenianEnglish} dictionary listed 12 roots that start with a [CjV] sequence. We list these roots below. 
	
	
	\begin{table}[H]
		\centering
		\caption{Word-initial [Cj] complex onsets }
		\label{tab: cj word imitial}
		\begin{tabular}{|lll|lll|  }
			\hline 
			ˈɡjɑl & `to exist' & \armenian{կեալ}
			& 
			ˈʁjɑɡ & `rudder' & \armenian{ղեակ}
			\\
			ˈsjɑv & `black' & \armenian{սեաւ}
			& 
			ˈkʰjɑɾ & `necklace' & \armenian{քեառ}
			\\
			ˈljɑɾtʰ & `liver' & \armenian{լեարդ}
			& 
			ˈɡjɑŋkʰ & `life' & \armenian{կեանք}
			\\
			ˈnjɑɾtʰ & `fibre' & \armenian{նեարդ}
			& 
			ˈdjɑɾkʰ & `gentlemen' & \armenian{տեարք}
			\\
			ˈsjɑmkʰ & `threshold' & \armenian{սեամք}
			& 
			ˈsjɑmkʰ & `threshold' & \armenian{սեամք}
			\\
			ˈljɑɾən & `mountain' & \armenian{լեառն}
			& 
			ˈdjɑɾən & `of the Lord' & \armenian{տեառն}
			\\
			% \hline 
			ˈzjɑn & `pain' & \armenian{զեան} & & & 
			\\
			\hline 
			
		\end{tabular}
	\end{table}
	
	5 of these roots are then used to form 39 derivatives, such as [njɑɾtʰ-ɑvoɾ] `fibrilous' \armenian{նեարդաւոր}. The total number of [CjV] words ends up as just 49 words (12 roots and 39 derivatives). 
	
	Second, as is visible above, these word-initial [CjV] clusters are limited to cases where the vowel is /ɑ/. For the other vowels, there are some variable cases of [Cju] clusters. Some roots like [kʰʏʁ] `village' \armenian{գիւղ} have a front-round vowel [ʏ] that can optionally be pronounced as [ju], as in [kʰjuʁ]. This is discussed in Section \S\ref{section:segmentalPhono:vowel:frontRound}.\footnote{However, Armenian dialects do vary in how often they have [Cj] complex onsets.  The Western Armenian [ʏ] vowel corresponds to a [ju] sequence in Eastern Armenian. Does a Western word like [ɑχpʏɾ] `fountain' \armenian{աղբիւր} is pronounced as [ɑχpjuɾ] in Eastern Armenian. Eastern Armenian thus has  significantly more cases of [Cj] complex onsets than Western Armenian. 
	}
	
	
	Another exceptional case is the archaic word \armenian{մեօքակերպ} <meōk'agerb> `similar to us'. As an orthographic rule, the sequence \armenian{եօ} <ōk> is supposed to be pronounced as [jo]. For this word, the default pronunciation is thus [mjokʰɑɡeɾb] with a [Cj] complex onset. But HD feels that because this word is low-frequency (and previously unknown to him), then this word can also be pronounced as [mejokʰɑɡeɾb] with glide epenthesis. 
	
	
	In fact, it seems that in Western Armenian, the [Cj] complex onset must always precede either a /ɑ/ (in the usual case) or a /u, o/ (in rare cases). We examined Armenian Wiktionary, and we confirmed this generalization. Armenian Wiktionary did report a handful of words that  with [Cj] and other vowels.   But these counter-examples  are    either   obvious loanwords like [ɡjoɾliŋkʰ] (WA),  [kjoɾliŋɡ]  (EA) `curling' \armenian{կյորլինգ}, or        dialectal words that don't genuinely  exist in Western Armenian. 
	
	Third, even though [Cj] complex onsets are permitted in Western Armenian, these clusters are dispreferred word-medially (Table \ref{tab:cj suffix}). Specifically, if a /Cj/ cluster is intervocalic, then the /Cj/ cluster can be pronounced either as a complex onset [V.CjV] or into separate syllables [VC.jV]. For example,  most word-medial cases of [Cj] are due to adding one of the following derivational suffixes: /jɑn, jɑɡ, jɑl, jɑ/. Each of these suffixes allow resyllabification. This variation is discussed in more depth in Section \S\ref{section:syllable:ComplexOnset:medialGlide}. 
	
	\begin{table}[H]
		\centering
		\caption{Variation in re-syllabification for /j/-initial suffixes}
		\label{tab:cj suffix}
		\begin{tabular}{|l|ll|ll|}
			\hline 
			/jɑn/     &   ɑ.tʰɑ.mɑn.ˈ\textbf{tʰj}ɑn  &  `diamond-encrusted' & cf. ɑ.tʰɑ.mɑntʰ & `diamond'
			\\
			& ɑ.tʰɑ.mɑn\textbf{tʰ.j}ɑn  & \armenian{ադամանդեան}  & & \armenian{ադամանդ}
			\\ \hline 
			/jɑl/     &  ʃeʃ.ˈ\textbf{tʰj}ɑl  &  `accented' & cf. ˈʃeʃt & `stress'
			\\
			& ʃeʃ\textbf{t.ˈj}ɑl & \armenian{շեշտեալ}  & & \armenian{արբած}
			\\\hline 
			/jɑ/     &  pʰɑj.ˈ\textbf{dj}ɑ   &  `wooden' & cf. ˈpɑjd & `wood'
			\\
			& pʰɑj\textbf{d.ˈj}ɑ & \armenian{փայտեայ}  & & \armenian{փայտ}
			\\\hline 
			/jɑɡ/     &  de.ˈ\textbf{ʁj}ɑɡ   &  `well-informed' & cf. ˈdeʁ & `place'
			\\
			& de\textbf{ʁ.ˈj}ɑɡ & \armenian{տեղեակ}  & & \armenian{տեղ}
			\\\hline 
			
		\end{tabular}
	\end{table}
	
	
	
	
	In sum, even though [Cj] complex onsets are allowed, they seem to be either a marginal  or restricted part of the language, such that the language has strategies to avoid pronouncing the [Cj] onsets. 
	
	\subsection{Variation in syllabifying word-medial consonant-glide clusters}\label{section:syllable:ComplexOnset:medialGlide}
	
	Armenian generally avoids all complex onsets except for /Cj/ sequences. However, there is significant free variation in how easily or how often [Cj] complex onsets are formed. 
	
	Word-initially, a [Cj] complex onset generally cannot alternate with other forms. Thus a word like [\textbf{ɡj}ɑŋkʰ] `life' \armenian{կեանք} is pronounced only with a [Cj] complex onset. But there a few words which show free variation such that [Cj] is replaced by [Cij]: [\textbf{lj}ɑɾt $\sim$ \textbf{lij}ɑɾtʰ] `liver' \armenian{լեարդ}. 
	
	Word-medially, we find much more free variation. This variation is correlated with the following parameters:
	\begin{itemize}[noitemsep, topsep = 0pt]
		\item Whether the /Cj/ follows a vowel /VCjV/ or consonant /VCCjV/. 
		\item If the pre-/j/ consonants in /VCCjV/   form a falling-sonority complex coda. 
		\item If the pre-/j/ consonants is two consonants or more /V(C)CCjV/. 
		\item Orthographic rules for pronouncing [Cj] for different words. 
		\item Preferences for or against [Cj] by certain suffixes. 
	\end{itemize}
	
	We go through the various parameters, summarized in Table \ref{tab: cj variation}.
	
	\begin{table}[H]
		\centering
		\caption{Variation in syllabifying word-medially /Cj/ clusters}\label{tab: cj variation}
		\resizebox{\textwidth}{!}{%
			\begin{tabular}{|l| llll| }
				\hline 
				{}[V.CjV$\sim$VC.jV]&  ɡə.ˈ\textbf{ɾj}ɑ  &  ɡə\textbf{ɾ.ˈj}ɑ& `turtle'    & \armenian{կրիայ}
				\\
				& he.ˈ\textbf{kʰj}ɑtʰ  & he\textbf{kʰ.ˈj}ɑtʰ&`fable' 
				&  \armenian{հէքեաթ}
				\\ \hline 
				{}[VC.CjV$\sim$VCC.jV]& pʰəɾ.ˈ\textbf{tʰj}ɑ & pʰə\textbf{ɾtʰ.ˈj}ɑ& `woolen'    & \armenian{բրդեայ}
				\\
				& uχ.ˈ\textbf{tj}ɑl  & uχ\textbf{t.ˈj}ɑl&`votive' 
				&  \armenian{ուխտեալ}
				\\ \hline 
				{}[VC.CjV, *VCC.jV]& əs.toɾ.ɑkʰ.ˈ\textbf{ɾj}ɑl & *əs.toɾ.ɑkʰ\textbf{ɾ.ˈj}ɑl & `undersigned'    & \armenian{ստորագրեալ}
				\\
				& jotʰ.ˈ\textbf{nj}ɑɡ  & *jo\textbf{tʰn.j}ɑɡ& `septet' 
				&  \armenian{եօթնեակ}
				\\ \hline 
				{}[VCC.CjV, *VCCC.jV]& vosp.ˈ\textbf{nj}ɑ  & *vosp\textbf{n.j}ɑ & `freckled'    & \armenian{ոսպնեայ}
				\\
				& jeɾɡ.ɡoʁm.ˈ\textbf{nj}ɑ   & *jeɾɡ.ɡoʁm\textbf{n.j}ɑ & `bilateral' 
				&  \armenian{երկկողմնեայ}
				\\ \hline 
				{}[V(C)C.ɾjV, *V(C)Cɾ.jV]& sɑnd.ˈ\textbf{ɾj}ɑ  & *sɑnd\textbf{ɾ.j}ɑ & `pectinal'    & \armenian{սանտրեայ}
				\\
				or [V(C).Cəɾ.jV]& sɑn.də\textbf{ɾ.ˈj}ɑ   &  & &
				\\
				& vosk.ˈ\textbf{ɾj}ɑ  & *vosk\textbf{ɾ.j}ɑ & `bony'    & \armenian{ոսկրեայ}
				\\
				& vos.kə\textbf{ɾ.ˈj}ɑ   &  & &
				\\  
				& meʁ.ˈ\textbf{ɾj}ɑl  & *meʁ\textbf{ɾ.j}ɑ & `honeyed'    & \armenian{մեղրեալ}
				\\
				& me.ʁə\textbf{ɾ.ˈj}ɑ   &  & &
				\\ \hline 
				
			\end{tabular}
		}
	\end{table}
	
	
	First, consider the intervocalic parameter. In /VCjV/ sequences, the /Cj/ can form either a complex onset or be in different syllables: [se\textbf{n.jɑ}ɡ] or [se.\textbf{nj}ɑɡ] `room' \armenian{սենեակ}. The choice of syllabification is partially just free variation, but there is some correlation with dialect.  In our experience, Eastern Armenian speakers are more likely to use [V.CjV] than [VC.jV], while Western Armenian speakers are more likely to use [V.CjV]. 
	
	Second, if the /Cj/ precedes a consonant, then resyllabification is possible if the preceding consonant cluster can form a complex coda: [bɑn.\textbf{dj}ɑl $\sim$ bɑn\textbf{d.j}ɑ] `prisoner' \armenian{բանտեալ}. But if the preceding consonant cluster can't form complex codas in the language, then we don't have resyllabification: [lus.\textbf{nj}ɑɡ] but not *\textit{lus\textbf{n.j}ɑɡ} `lunatic' \armenian{լուսնեակ}. 
	
	Note that the preceding cluster can even be complex coda that   arguably includes an  extrasyllabic appendix (\S\ref{section:syllable:ConsonantClusters:Appendix}). For example, word-final [ks] clusters are attested and they're arguably comprised of an appendix \textit{-s}. These structures can be formed via /Cj/ resyllabification: [me.dɑk.\textbf{sj}ɑ $\sim$ me.dɑk\textbf{s.j}ɑ] `silken' \armenian{մետաքսեայ}. 
	
	Third, when the /j/ follows three consonants, then the default pronunciation is to create the [Cj] complex onset: [hɑɾs.\textbf{nj}ɑɡ] `chrysalis' \armenian{հարսնեակ}. The three consonants cannot form a complex coda because 3-consonant codas are generally banned in Armenian: *\textit{hɑɾs\textbf{n.j}ɑɡ} (\S\ref{section:syllable:ConsonantClusters:Coda3C}). 
	
	
	An interesting type of variation occurs when the glide /j/ follows a /(C)Cɾ/ cluster. Here, the prescriptive pronunciation is to create a complex onset: [ɑ.ʁek.sɑnd.\textbf{ɾj}ɑn] `Alexandrine' \armenian{աղէքսանդրեան}. Resyllabification with a 3-consonant coda is at best odd to hear: *\textit{ɑ.ʁek.sɑnd\textbf{ɾ.j}ɑ}. However, in casual speech, HD has observed that some speakers (including himself) can epenthesize a schwa before the /ɾ/ for at least some of these: [ɑ.ʁek.sɑn.də\textbf{ɾ.ˈj}ɑn]. 
	
	
	
	This sub-pattern of pre-rhotic schwas can interact  with vowel reduction. Consider the root [əndiɾ] `chosen' \armenian{ընտիր}. Adding a suffix /-jɑl/ causes the vowel to reduce to zero in the prescriptive form:    [ənd.\textbf{ɾj}ɑl] `elected' \armenian{ընտրեալ}. Resyllabification is of course impossible: *\textit{ənd\textbf{ɾ.j}ɑl}. However in colloquial speech, a schwa can occur before the /ɾ/: [ən.də\textbf{ɾj}ɑl]. Other examples include  deriving [t͡ʃɑ.mit͡ʃ] `raisin' \armenian{չամիչ} to [t͡ʃɑm.\textbf{t͡ʃj}ɑ, t͡ʃɑ.mə\textbf{t͡ʃ.j}ɑ] `plum cake' \armenian{չամչեայ}  but not *\textit{t͡ʃɑm\textbf{t͡ʃ.j}ɑ} It's unclear if these schwas here are  epenthetic or a reduced form of the root high vowel.   
	
	
	Another sub-variation is that in colloquial speech, it is possible to sometimes change a /CCjV/ form into [CCijV]. For example, [dɑs.njɑɡ] `decade' \armenian{տասնեակ} can be pronounced as [dɑs.ni.jɑɡ]. Data is too limited to make any concrete generalizations on such variation. But HD suspects that it's relatively common when the /j/ is part of a suffix that follows a rising-sonority consonant sequence.
	
	Fourth, the traditional orthography for Armenian has rather opaque rules on how to spell [Cj] clusters.  The orthographic cluster \armenian{իա} <ia> is found inside various roots and suffixes. When this vowel sequence is after the word-initial consonant, the sequence is pronounced with glide epenthesis: <d\textbf{ia}r> [d\textbf{ijɑ}ɾ] `mister' \armenian{տիար}. But after a word-medial consonant, some cases of <ia> are pronounced strictly as [jɑ] like <mariam> [mɑɾ.jɑm] `Mariam' \armenian{Մարիամ},\footnote{Eastern Armenian seems to allow the form [mɑ.\textbf{ɾj}ɑm] but this sounds odd to HD's Western ears.}   some can show variation between [jɑ] and [ijɑ]: <tasdiarag> [tʰɑst\textbf{ijɑ}ɾɑɡ, tʰɑst\textbf{jɑ}ɾɑɡ]  `educator' \armenian{դաստիարակ}, and  some always take [ija]:  <badriark> [bɑdɾ\textbf{ijɑ}ɾkʰ] \armenian{պատրիարք}.  Thus, certain roots seem to idiosyncratically ban or allow [Cj]. 
	
	Fifth, some suffixes seem to have individual preferences for whether they allow [Cj] complex onsets or not.   Consider the nominalizer suffix /{-utʰʏn/ \armenian{-ութիւն}. This suffix has many possible pronunciations, such with replacing /ʏ/ with a glide-vowel sequence: [-utʰjun] (\S\ref{section:segmentalPhono:alloEastern:palatalization}). One pronunciation is to create a [Cj] sequence as a complex onset: [uɾɑχ-u.ˈtʰjun] `happiness' \armenian{ուրախութիւն}. But it's also possible to make the /Cj/ sequence be in separate syllables:  [uɾɑχ-utʰ.ˈjun]. In HD's judgment, the complex onset [Cj] form feels more natural than [C.j]. 
	
	This morpheme-specific behavior can interact with the other parameters for /j/-resyllabification. Consider the country-naming suffix   <ia> \armenian{իա}. After a single consonant, this suffix is often pronounced as just [-jɑ]: [i.dɑ\textbf{l.j}ɑ] `Italy' \armenian{Իտալիա}. Resyllabification is possible [i.dɑ\textbf{.lj}ɑ], but HD feels this is substantially less common.  After two consonants that can be a complex coda, HD reports possible resyllabification: [tʰu\textbf{ɾk.j}ɑ, tʰu\textbf{ɾ.kʰj}ɑ] `Turkey' \armenian{Թուրքիա}. But if the consonant cluster    can't form a complex coda, then the [ijɑ] form feels more preferred than [jɑ]: [ɑŋkʰ.li.jɑ] instead of [ɑ\textbf{ŋkʰ.lj}ɑ], and not *\textit{ɑ\textbf{ŋkʰl.j}ɑ} `England' \armenian{Անգլիա}. 
	
	Another case of morpheme-specific variation is the family-name suffix (patronymic suffix) [-jɑn] \armenian{-եան}. In principle, this suffix can form [Cj] complex onsets: [dov.le.\textbf{tʰj}ɑn $\sim$ dov.le\textbf{tʰ.j}ɑn] `Deovletian' \armenian{Տէօվլէթեան}. However, in   HD's judgment, it is much more typical to break the /Cj/ sequence into separate syllables. Unfortunately, dictionary data is too restricted to determine the exact rates or preferences for this suffix.  
	
	
	Thus, although [Cj] complex onsets are possible, it seems that the language tries to remove such complex onsets if possible. Based on how /V(C)(C)CjV/ clusters are syllabified, it seems that creating [Cj]  acts as a last resort. 
	
	
	
	
	
	
	
	
	\subsection{Complex onsets created from schwa elision}\label{section:syllable:ComplexOnset:Elision}
	
	\textcolor{red}{write}
	
	\todo{for some reason i dont see ծնծղայ as deriving ծնծղահար in my dictionary even though its on nayiri. did i forget to find cases of այ with glide deletion?}
	
	\section{Vowel-vowel sequences or vowel hiatus repair}\label{section:syllable:VowelHiatus}
	Vowel hiatus is when the morphology brings together two vowels, creating a hiatus or break from one syllable to another. A sequence of vowels must be pronounced in some way, meaning that the vowel hiatus must be repaired. In Armenian, possible repairs include  inserting a glide /j/ (glide epenthesis), inserting a glottal stop /ʔ/, changing one of the vowels to a consonant, deleting one of the vowels, or merging the two vowels into one vowel.\footnote{Unfortunately, there are no words that end in a vowel /ʏ/ (\S\ref{section:segmentalPhono:vowel:frontRound}), and there are virtually no words that end in /ə/ and that can take derivational suffixes.  
	}
	
	This section goes over how vowel hiatus is repaired in diverse morphological contexts. The choice of rule varies by the type of first vowel and by the morphological identity of the second vowel. The following subsections go through these morphological categories and then phonological subcategories. Furthermore, some vowel-vowel sequences can undergo diverse repair rules without a clear preference, while some sequences can undergo mainly one rule with few exceptions.  
	
	\begin{itemize}[noitemsep,topsep=0pt]
		\item In roots, vowel hiatus is largely diachronic or limited to loanwords (\S\ref{section:syllable:VowelHiatus:Root}).
		\item  Before derivational suffixes, various rules are possible, and the choice varies by vowel and sometimes by the word (\S\ref{section:syllable:VowelHiatus:Derived}).
		\item In compounds and after prefixoids, some words use various repair rules (similar to derivational suffixation), while some use just glottal stop epenthesis (\S\ref{section:syllable:VowelHiatus:CompPrefix}). 
		\item  Before regular inflectional suffixes, glide epenthesis is the main strategy  (\S\ref{section:syllable:VowelHiatus:Inf}).
		\item  Before irregular inflectional suffixes, different strategies are possible  (\S\ref{section:syllable:VowelHiatus:Irregular}).
		\item  Before clitics, glide epenthesis is the main strategy  (\S\ref{section:syllable:VowelHiatus:Clitic}).
	\end{itemize}
	
	Although vowel hiatus repair rules are naturally limited to only vowel-vowel sequences, there are some consonant-initial suffixes  that   use rules as a type of paradigmatically-induced overapplication (\S\ref{section:syllable:VowelHiatus:Over}). There are also some minor problems in diachrony (\S\ref{section:syllable:VowelHiatus:Diachrony}). 
	
	
	
	
	
	\subsection{Vowel hiatus in underived words and roots}\label{section:syllable:VowelHiatus:Root}
	In this chapter, we almost exclusively focus on how vowel hiatus or vowel-vowel sequences are handled in derived forms, not underived forms. This is because underived forms don't give solid evidence on whether what we see is actually a vowel-vowel sequence, or something else. 
	
	
	To illustrate, consider the word <gr\textbf{ia}y> `turtle' \armenian{կրիայ}. Although the orthography shows a vowel sequence <ia>, this word is pronounced as [ɡəɾ\textbf{jɑ}]. The mismatch between the orthographic <ia> and the pronounced [jɑ] is because of diachrony. In Classical Armenian, such orthographic vowel sequences likely reflected some type of diphthong. Over time, this diphthong turned into a Modern [jɑ] sequence. Synchronically however, even though the [jɑ] is spelled as two vowels <ia>, there is  no synchronic evidence that the pronounced [j] is derived from an underlying /i/.  Assuming the schwa is epenthetic,   the Armenian child has no reason to think that [ɡəɾjɑ] is underlyingly /ɡɾiɑ/ instead of just /ɡɾjɑ/.  Similar orthography-phonology mismatches include words like <\textbf{eō}tə>  that is pronounced as [\textbf{jo}tʰə]  `seven' \armenian{եօթը}. See Section \S\ref{section:segmentalPhono:cons:glide} for discussion on the phonemic status of /j/. 
	
	A large category of such mismatches involves the digraph <ow> \armenian{ու} that is read as [v] before vowels, but [u] elsewhere. In a root before a vowel, the digraph is read as [v]: <aɣ\textbf{owē}s> [ɑʁ\textbf{ve}s] `fox' \armenian{աղուէս}.  If the [v] is part of an unsyllabiable consonant  cluster, then we get an extra schwa: <n\textbf{owa}z> [n\textbf{əvɑ}z] `less' \armenian{նուազ}.  The reason why such mismatches exist is again because of Classical Armenian where  clusters like <owē, owa> were   likely read as some diphthong [ʊ͜e,ʊ͜ɑ]. But in the modern language, this diphthong was replaced by a [vɑ] sequence. Because such roots don't show any alternations, the child has no reason to treat a word like   [ɑʁ\textbf{ve}s] as derived from an underlying /ɑʁues/ instead of just /ɑʁves/. 
	
	For such underived words, we only found two corners of the grammar where it is likely that such an orthographic cluster does reflect an underlying vowel-vowel sequence. Both corners   involve free variation. One such corner is for the word <hr\textbf{ea}y> `Jewish' \armenian{հրեայ} where the cluster is variably pronounced as [həɾ\textbf{ejɑ}] (more archaic) or [həɾ\textbf{jɑ}] (more modern). However, such examples likely reflect a change in the underlying representation from /hɾejɑ/ (or /hɾeɑ/) to /hɾjɑ/. 
	
	The other corner involves non-nativized loanwords. Consider a word like <k'\textbf{ao}s> [kʰ\textbf{ɑʔo}s] `chaos' \armenian{քաոս} or  <bant'\textbf{eo}n> [bɑntʰeʔon] `pantheon'  \armenian{պանթէոն}. Such words are obviously loanwords, and their vowel sequence is borrowed in tact, and their hiatus is resolved via a glottal stop. When such words get more frequent or common, their pronunciation is simplified. For example, consider <madt'\textbf{eo}s> `Matthew' \armenian{Մատթէոս}, which prescriptively can be pronounced as [mɑtʰ\textbf{eʔo}s], but is more commonly pronounced as [mɑtʰ\textbf{jo}s]. 
	
	Such cases of free variation indicate that vowel sequences must be repaired in some way. But they don't tell us what are productive means of handling vowel-vowel sequences that are created from the morphology, i.e., derived forms. At best, such underived forms tell us what are possible rules for reading orthographic clusters (orthography-phonology mismatches: Section \S\ref{section:ortho:mismatch}) and how a word's pronunciation can get simplified   over time as it gets nativized or more frequent. 
	
	\subsection{Vowel hiatus repair between roots and derivational suffixes}\label{section:syllable:VowelHiatus:Derived}
	
	When studying vowel hiatus in Armenian, the most common contexts in derivational morphology are /V+ɑ/ and /V+u/.  For derivational morphology,  the majority of derivational suffixes start with /ɑ/. Compounds are also typically formed by combining two stems with the linker /ɑ/. For the other vowels, a common /u/-initial derivational suffix is the nominalizer /-utʰjʏn/ \armenian{-ութիւն}. There are some derivational suffixes for /e/, /o/, and other /u/ suffixes. 
	
	
	
	It is rather difficult to study how some phonological process works in derivational morphology  for various reasons. The first is that different dictionaries have different lists of words. Thus a word that we found in one major dictionary like \citeauthor{kouyoumdjian-1970-DictionaryArmenianEnglish} might not exist in another major dictionary. Second, creating new words is a very creative process. So we can't say that some vowel-vowel sequence is always repaired in some specific way, simply because some dictionary out there may list such a sequence with an alternative repair. The third reason is variability. For a given root, we can find some derivatives that utilize one rule, and other set of derivatives that utilize another rule. 
	
	The above factors ultimately mean that any study on vowel hiatus repair that is using a dictionary has limitations. However, because the dictionary we use is rather large, we hope that the following descriptive patterns that we provide can reflect the tendencies in the language.
	
	\begin{itemize}[noitemsep,topsep=0pt]
		\item /ɑ/ + vowel: repaired via glide epenthesis for most words, but /ɑ/ deletion for some words (\S\ref{section:syllable:VowelHiatus:Derived:A}).
		\item /e/ + vowel: repaired via glide epenthesis for most words, but /e/ deletion for some words,).
		\item /i/ + vowel: repaired via glide epenthesis, deletion, or coalescence/merger (/i-ɑ/$\rightarrow$[e]), depending on the word  (\S\ref{section:syllable:VowelHiatus:Derived:I}).
		\item /ə/ + vowel: unattested. 
		\item /o/ + vowel: repaired via glide epenthesis  (\S\ref{section:syllable:VowelHiatus:Derived:O}).
		\item /u/ + vowel: repaired via de-vocalization for most words  (/u-ɑ/$\rightarrow$[vɑ], /u-i/$\rightarrow$[vi], etc.), and glide epenthesis for some    (\S\ref{section:syllable:VowelHiatus:Derived:U}).
		
	\end{itemize}
	
	Note that in derivational morphology,   glide epenthesis can often be replaced with a glottal stop epenthesis, with unclear sociolinguistic effects or factors. 
	
	
	\subsubsection{Stem final /ɑ/}\label{section:syllable:VowelHiatus:Derived:A}
	When a /ɑ/-final stem precedes   the compound linker /ɑ/ or a vowel-initial derivational suffix, there are two attested repair rules: deleting  the stem /ɑ/ or glide epenthesis. The choice between the two rules correlates with spelling. If  the stem /ɑ/ is spelled as with a final <a> \armenian{ա}, then deletion is the default rule. But if the stem /ɑ/ is spelled as <ay> \armenian{այ}, then glide epenthesis is the default rule. However, there are a handful of stems which either display both rules or  display the opposite rule. 
	
	
	First let us consider stems with a final <ay>. Glide epenthesis is the default rule before a derivational suffix that starts with an /ɑ/ (/-ɑX/), /u/ (in /-utʰjʏn/), /u/ in some other suffix (/-uX/), /i/ (-/iX/, /e/ (/-eX/), and /o/ (/-oX/). Glide epenthesis is also the rule before the compound linker /-ɑ-/. 
	
	
	
	In Table \ref{tab:vowel vowel a deriv aj normal} and onward,  we  list the number of roots that we found  in \citeauthor{kouyoumdjian-1970-DictionaryArmenianEnglish} such that a) this root had this specific phonology-orthographic structure, b) this root was listed as an entry in the dictionary, and c) the dictionary listed derivatives for this root, and d) the derivatives displayed only this vowel hiatus rule in this vowel-vowel sequence.  For illustration, we also show the underlying form of the suffix. We show a simplified segmentation for verbal suffixes, meaning /-il/ instead of /-i-l/ ({\thgloss}-{\infgloss}). 
	
	
	\begin{table}[H]
		\centering
		\caption{Glide epenthesis in derivation for  stems with final [ɑ]  <ay>}
		\label{tab:vowel vowel a deriv aj normal}
		\resizebox{\textwidth}{!}{%
			\begin{tabular}{|l|llll|l| }
				\hline 
				+/-ɑX/& <hsg\textbf{ay}> & həsˈk\textbf{ɑ} & `giant' & \armenian{հսկայ} & n=45
				\\
				+/-ɑɡɑn/ &<hsg\textbf{aya}gan> & həsk\textbf{ɑj-ɑ}ˈɡɑn & `giant-like' & \armenian{հսկայական}  & 
				\\ \hline 
				+/-ɑ-/  & <kowrb\textbf{ay}> & kʰuɾˈb\textbf{ɑ} & `hose' & \armenian{գուրպայ} & n=35
				\\
				&<kowrb\textbf{aya}kord͡z> & kʰuɾb\textbf{ɑj-ɑ-}ˈkʰoɾd͡z & `hosier' & \armenian{գուրպայագործ}  &  
				\\
				& cf <kord͡z> & kʰoɾd͡z& `work' & \armenian{գործ} & 
				\\     \hline 
				+/-utʰjʏn/& <xapep\textbf{ay}> & χɑpʰeˈpʰ\textbf{ɑ} & `deceitful' & \armenian{խաբեբայ} & n=59
				\\
				&<xapep\textbf{ayow}t'iwn> & χɑpʰepʰ\textbf{ɑj-u}ˈtʰjʏn & `deception'  & \armenian{խաբեբայութիւն}  & 
				\\
				\hline 
				+/-uX/& <k'ahan\textbf{ay}> & kʰɑhɑˈn\textbf{ɑ} & `priest' & \armenian{քահանայ} & n=7
				\\
				+/-uhi/ &<k'ahan\textbf{ayow}hi> & kʰɑhɑn\textbf{ɑj-u}ˈhi & `priestess'  & \armenian{քահանայուհի}  & 
				\\
				\hline 
				+/-iX/& <ʃrt͡ʃak\textbf{ay}> & ʃəɾt͡ʃɑˈkʰ\textbf{ɑ} & `roamer' & \armenian{շրջագայ} & n=7
				\\
				+/-il/ &<k'ʃrt͡ʃak\textbf{ayi}l> & ʃəɾt͡ʃɑkʰ\textbf{ɑˈj-i}l  & `to stroll'  & \armenian{շրջագայիլ}  & 
				\\
				\hline 
				+/-eX/& <p'ilisop'\textbf{ay}> & pʰilisopʰ\textbf{ɑ} & `philosopher' & \armenian{փիլիսոփայ} & n=17
				\\
				+/-el/ &<p'ilisop'\textbf{aye}l> & pʰilisopʰ\textbf{ɑˈj-e}l  & `to philosophize'  & \armenian{փիլիսոփայել}  & 
				\\
				\hline 
				+/-oX/& <xn\textbf{ay}> & χən\textbf{ɑ} & `caution' & \armenian{խնայ} & n=11
				\\
				+/-oʁ/ &<xn\textbf{ayo}ɣ> & χən\textbf{ɑˈj-o}ʁ  & `thrifty'  & \armenian{խնայող}  & 
				\\
				\hline 
				
			\end{tabular}
		}
	\end{table}
	
	
	In contrast, for stems that end in <a>, the default behavior is deleting the stem <a> (Table \ref{tab:vowel vowel a deriv a normal}). 
	
	\begin{table}[H]
		\centering
		\caption{Stem-vowel deletion   in derivation for  stems with final [ɑ]  <a>}
		\label{tab:vowel vowel a deriv a normal}
		\resizebox{\textwidth}{!}{%
			\begin{tabular}{|l|llll|l| }
				\hline 
				+/-ɑX/& <asi\textbf{a}> & ɑsˈj\textbf{ɑ} & `Asia' & \armenian{Ասիա} & n=18
				\\
				+/-ɑɡɑn/ & <asi\textbf{a}gan> & ɑsj-\textbf{ɑ}ˈɡɑn & `Asian' &\armenian{ասիական} & 
				\\\hline 
				+/-ɑ-/& <fizik'\textbf{a}> & fiziˈkʰ\textbf{ɑ} & `physics' & \armenian{ֆիզիքա} & n=10
				\\
				& <fizik'\textbf{a}kēd> & fizikʰ-\textbf{ɑ}-ˈkʰed & `physicist' & \armenian{ֆիզիքագէտ} & 
				\\
				& cf. <kēd> &    ˈkʰed & `learned (archaic)' & \armenian{գէտ} & 
				\\\hline 
				+/-utʰjʏn/& <k'aɣte\textbf{a}> & kʰɑχtej\textbf{ɑ} & `Chaldea' & \armenian{Քաղդէա} & n=1
				\\
				& <k'aɣte\textbf{ow}t'iwn> & kʰɑχtej-\textbf{u}ˈtʰjʏn & `astrology' &\armenian{քաղդէութիւն} & 
				\\\hline 
				+/-uX/& <frans\textbf{a}> & fəɾɑns\textbf{ɑ} & `France' & \armenian{Ֆրանսա} & n=1
				\\
				+/-uhi/ & <frans\textbf{ow}hi> & fəɾɑns-\textbf{u}ˈhi & `French woman' &\armenian{ֆրանսուհի} & 
				\\\hline 
				+/-eX/& <kaɣɣi\textbf{a}> & kʰɑʁʁij\textbf{ɑ} & `Gaul' & \armenian{Գաղղիա} & n=1
				\\
				+/-eɾen/ & <kaɣɣi\textbf{e}rēn> & kʰɑʁʁij-\textbf{e}ˈɾen & `Gallic language' &\armenian{գաղղիերէն} & 
				\\\hline 
				
			\end{tabular}
	}\end{table}
	
	Based on the data so far, it is clear that glide epenthesis is the norm for stems spelled with <ay>, while deletion is the norm for words spelled with <a>. But, there is some degree of variation (Table \ref{tab:vowel vowel a deriv a not normal glide}). In the \citeauthor{kouyoumdjian-1970-DictionaryArmenianEnglish} dictionary, we found some stems that are spelled with <ay> but display deletion before some derivational suffixes. 
	
	
	
	\begin{table}[H]
		\centering
		\caption{Glide epenthesis in some   stems with final [ɑ]  <a>}
		\label{tab:vowel vowel a deriv a not normal glide}
		\resizebox{\textwidth}{!}{%
			\begin{tabular}{|l|llll|}
				\hline 
				{}[ɑ]+/-eni/& <nowm\textbf{a}> & nuˈm\textbf{ɑ} & `Mandarin' & \armenian{նումա} \\
				& <nowm\textbf{aye}ni> & num\textbf{ɑj-e}nˈi & `Mandarin orange' & \armenian{նումայենի}
				\\ \hline
				{}[ɑ]+/-ɑ-/& <delt\textbf{a}> & delˈtʰ\textbf{ɑ} & `delta' & \armenian{տէլդա} \\
				&<delt\textbf{aya}gerb> & deltʰ\textbf{ɑj-ɑ-}ˈɡeɾb & `deltoid' & \armenian{տէլդայակերպ}
				\\
				& cf. <gerb> & ˈɡeɾb & `manner' & \armenian{կերպ}
				\\ \hline
			\end{tabular}
	}\end{table}
	
	Besides the above exceptions, we found a handful more words that are spelled with <ay> but display vowel deletion in their attested derivatives (Table \ref{tab:vowel vowel a deriv ay not normal delete}).  For some of these words like [ɑɡɾɑ-pʰeɾɑd] `toothless', it is possible that these words underlyingly never had the linking vowel /-ɑ-/ in the first place /ɑɡɾɑ + pʰeɾɑd/, thus there is no vowel hiatus to repair.  See Section \S\ref{section:syllable:VowelHiatus:CompPrefix:Atypical} for such compounds. 
	
	
	\begin{table}[H]
		\centering
		\caption{Vowel deletion    in some   stems with final [ɑ]  <ay>}
		\label{tab:vowel vowel a deriv ay not normal delete}
		\resizebox{\textwidth}{!}{%
			\begin{tabular}{|l|llll|}
				\hline 
				{}[ɑ]+/-ɑnɑl/& <momi\textbf{ay}> & momˈj\textbf{ɑ} & `mummy' & \armenian{մոմիայ}
				\\
				& <momi\textbf{a}nal> & momj-\textbf{ɑˈn}ɑl & `to get mummified' & \armenian{մոմիանալ}
				\\ \hline
				{}[ɑ]+/-ɑ-/& <amir\textbf{ay}> & ɑmiˈɾ\textbf{ɑ} & `lord' & \armenian{ամիրայ}
				\\
				& <amir\textbf{a}bed> & ɑmiɾ-\textbf{ɑ}-ˈbed & `caliph' & \armenian{ամիրապէտ}
				\\
				&cf.  <bed> &   ˈbed & `leader' & \armenian{պէտ}
				\\ \hline
				{}[ɑ]+/-ɑ-/& <agṙ\textbf{ay}> & ɑɡɾ\textbf{ɑ} & `tooth' & \armenian{ակռայ}
				\\
				& <agṙ\textbf{a}p'eṙad> & ɑɡɾ-\textbf{ɑ}-pʰeˈɾɑd & `gap-toothed' & \armenian{ակռափեռատ}
				\\
				&cf.  <p'eṙad> &    pʰeˈɾɑd  & `toothless' & \armenian{փեռատ}
				\\ \hline
			\end{tabular}
	}\end{table}
	
	What is more revealing is that some words are spelled with <ay>, display glide epenthesis in some derivatives, but vowel deletion in other derivatives (Table \ref{tab:vowel vowel a deriv ay not normal both}). 
	
	
	
	\begin{table}[H]
		\centering
		\caption{Vowel deletion or glide deletion   in some   stems with final [ɑ]  <ay>}
		\label{tab:vowel vowel a deriv ay not normal both}
		\resizebox{\textwidth}{!}{%
			\begin{tabular}{|l|llll|}
				\hline 
				& <sadan\textbf{ay}> & sɑdɑˈn\textbf{ɑ} & `devil' & \armenian{սատանայ} 
				\\
				+/-utʰjʏn/ & <sadan\textbf{ayow}t'iwn> & sɑdɑn\textbf{ɑj-u}ˈtʰjʏn & `devilry' & \armenian{սատանայութիւն} 
				\\
				+/-iɡ/ & <sadan\textbf{ayi}g> & sɑdɑˈn-\textbf{i}ɡ & `devilet' & \armenian{սատանիկ} 
				\\ \hline
				& <aṙarg\textbf{ay}> & ɑɾɑɾˈɡ\textbf{ɑ} & `object' & \armenian{առարկայ} 
				\\
				+/-ɑɡɑn/& <aṙarg\textbf{aya}gan> & ɑɾɑɾɡ\textbf{ɑj-ɑ}ˈɡɑn& `objective' & \armenian{առարկայական} 
				\\
				+/-el/& <aṙarg\textbf{e}l> & ɑɾɑɾˈɡ-\textbf{e}l  & `to object' & \armenian{առարկել} 
				\\ \hline 
				& <mek'en\textbf{ay}> & mekʰen\textbf{ɑ} & `machine' & \armenian{մեքենայ} 
				\\
				+/-ɑpʰɑɾ/& <mek'en\textbf{aya}par> & mekʰeˈn\textbf{ɑj-ɑ}pʰɑɾ& `mechanically' & \armenian{մեքենայաբար} 
				\\
				+/-el/& <mek'en\textbf{a}par> & meˈkʰen-\textbf{ɑ}pʰɑɾ  & `mechanically' & \armenian{մեքենաբար} 
				\\\hline 
			\end{tabular}
		}
	\end{table}
	
	Based on the above main patterns and exceptions, we conclude that glide epenthesis is the norm for words spelled with <ay>, while vowel deletion is the norm for words spelled with <a>. Exceptions exist for both classes of words. 
	
	
	However, we do not think that the mental grammar of Armenian directly uses these orthographic rules to know when to do glide epenthesis. Such orthographic rules are instead diachronic accidents, discussed more in Section \S\ref{section:syllable:VowelHiatus:Diachrony}. 
	
	Instead, the correlation with orthography is an indirect  correlation with morphological structure. The <a>-spelled words tend to end with a country-naming suffix,  such as /-ɑ/ in [fəɾɑns-ɑ] `France' \armenian{Ֆրանսա},  /-jɑ/ in [ɑs-jɑ] `Asia' \armenian{Ասիա}. Other cases are obvious loanwords like [fizikɑ] `physics' \armenian{ֆիզիքա}. In contrast, the  <ay>  spelling are mostly simple native words.  There are of course a few exceptions for this native-loanword generalizations, for example the word  [lɑmɑ] `Lama' is an obvious loanword but it is spelled with <ay> \armenian{լամայ}, and it gets glide epenthesis: [lɑmɑj-ɑ-bed] `chief Lama' \armenian{լամայապէտ}. 
	
	Further, the <ay> spelling is drastically more common than <a>, simply because <ay> is used for native words. For example, in the \citeauthor{kouyoumdjian-1970-DictionaryArmenianEnglish} dictionary, we found at least 100 words with a final <ay> and that have vocalic derivatives, compared with only 24 words with <a>.  
	
	Thus we argue that the actual rule for vowel hiatus repair is that native words with final [ɑ] get glide epenthesis, while loanwords or country-names get vowel deletion. Exceptions are limited to the items discussed above.  
	
	
	
	\subsubsection{Stem final /e/}\label{section:syllable:VowelHiatus:Derived:E}
	It is rather rare to find a word that ends with /e/. When a stem-final /e/ precedes a vowel-initial derivational suffix, the most common repair rule is glide epenthesis (Table \ref{tab:vowel vowel e deriv glide normal}).  But there are some cases where either the /e/ or the following vowel is deleted. We first show cases with glide epenthesis. 
	
	
	\begin{table}[H]
		\centering
		\caption{Glide epenthesis in derivation for  stems with final [e]}
		\label{tab:vowel vowel e deriv glide normal}
		\resizebox{\textwidth}{!}{%
			\begin{tabular}{|l|llll|l| }
				\hline 
				+/-ɑX/ & <hiwl\textbf{ē}> & hʏˈl\textbf{e} & `atom' &\armenian{հիւլէ} &   n=19
				\\ 
				& <hiwl\textbf{ēa}gan> & hʏl\textbf{ej-ɑ}ˈɡɑn& `atomic' & \armenian{հիւլէական}  & 
				\\ \hline
				+/-ɑ-/ & <osdr\textbf{ē}> & vostˈɾ\textbf{e} & `oyster' & \armenian{ոստրէ} & n=10 \\
				& <osdr\textbf{ēa}vad͡ʒaṙ> & vostɾ\textbf{ej-ɑ-}vɑˈd͡ʒɑɾ & `oyster-selleter' & \armenian{ոստրէավաճառ} &    \\
				& cf. <vad͡ʒaṙ>  & vɑˈd͡ʒɑɾ &`sale' &  \armenian{վաճառ} & 
				\\ \hline  
				+/-utʰjʏn/ & <gat'oɣig\textbf{ē}> & ɡatʰoʁiˈɡ\textbf{e} & `cathedral' & \armenian{կաթողիկէ} & n=6
				\\
				& <gat'oɣig\textbf{ēow}t'iwn>  & ɡatʰoʁiɡ\textbf{ej-u}ˈtʰjʏn & `Catholicism' & \armenian{կաթողիկէութիւն} & 
				\\ \hline 
				+/-uX/ & <markar\textbf{ē}> & mɑɾkʰɑˈɾ\textbf{e} & `prophet' & \armenian{մարգարէ} & n=1
				\\
				+/-uhi/ & <markar\textbf{ēow}hi>  & mɑɾkʰɑɾ\textbf{ej-u}ˈhi & `prophetess' & \armenian{մարգարէուհի} & 
				\\ \hline 
				+/-iX/ & <baxr\textbf{ē}> & bɑχˈɾ\textbf{e} & `money' & \armenian{պախրէ} & n=1
				\\
				+/-iɡ/ & <baxr\textbf{ēi}g>  & bɑχɾ\textbf{eˈj-i}ɡ & `small money' & \armenian{պախրէիկ} & 
				\\ \hline 
				
			\end{tabular}
		}
	\end{table}
	
	There are of course exceptions (Table \ref{tab:vowel vowel e deriv other weird}). We found words  which can a) delete the stem /e/, b) merge the the stem and suffix /e/ into one vowel, c) delete the vowel of the suffix or linker, or d) use glide epenthesis. Most of these exceptional roots would use one strategy in one derivative, but another strategy in another
	
	
	\begin{table}[H]
		\centering
		\caption{Glide epenthesis vowel deletion in derivation for some stems with final [e]}
		\label{tab:vowel vowel e deriv other weird}
		\resizebox{\textwidth}{!}{%
			\begin{tabular}{|lllll| }
				\hline 
				& <ap'rodit\textbf{ē}> 
				& ɑpʰɾodiˈtʰ\textbf{e} & `Aphrodite' & \armenian{Ափրոդիտէ} \\
				+/-ɑɡɑn/&    <ap'rodit\textbf{a}gan> & ɑpʰɾodiˈtʰ-\textbf{ɑ}ˈɡɑn & `venereal' & \armenian{ափրոդիտական} 
				\\ \hline 
				&   <xahow\textbf{ē}> & χɑhˈv\textbf{e} & `coffee' & \armenian{խահուէ} \\
				+/-eni/ &  <xahow\textbf{ē}ni> & χɑhv-\textbf{e}ˈni & `coffee-tree' & \armenian{խահուենի} \\
				+/-ɑɾɑɾ/& < xahow\textbf{ea}rar> & χɑhv\textbf{ej-ɑ}ˈɾɑɾ &   `coffeehouse keeper' & \armenian{խահուէարար}
				\\ \hline 
				&<paz\textbf{ē}> & pʰɑˈz\textbf{e} & `falcon' & \armenian{բազէ} \\
				+/-ɑnot͡s/ & <paz\textbf{ē}not͡s'> & pʰɑz\textbf{e}-ˈnot͡s  & `hawking-pouch' & \armenian{բազէնոց} \\
				+/-ɑ-ɡiɾtʰ/ & <paz\textbf{ēa}girt'> & pʰɑz\textbf{ej-ɑ-}ɡiɾtʰ & `falconer' &  \armenian{բազէակիրթ} 
				\\\hline 
			\end{tabular}
	}\end{table}
	
	
	As a last note, for words like [hʏlej-ɑɡɑn] `atomic' \armenian{հիւլէական} or [ej-utʰjʏn] `existence' \armenian{էութիւն} (from [e] `is' \armenian{է}), we transcribe the glide as a fully pronounced glide [j]. However, in HD's judgment, in careful speech this glide can be considerably weakened to either a transient glide [hʏleʲ-ɑɡɑn] or even a glottal stop [hʏleʔ-ɑɡɑn]. He also reports that the preference of a glottal stop feels more salient if the root sounds more clearly like a loanword: [seɾofpe] `seraph' \armenian{սերովբէ} and [seɾofpeʔ-ɑɡɑn] `seraphic' \armenian{սերովբէական}. 
	
	
	In HD's judgements, the   full glide form is quite common in Western Armenian. In contrast, VP informs us that Eastern Armenian uses a transient glide or glottal stop more often. In Armenian philology, the transcription of <ē.a> \armenian{է.ա} can be variably pronounced as [ejɑ], [eʲɑ], or [eʔɑ]. This makes it difficult to get reliable statistical data on this free variation. 
	
	
	
	\subsubsection{Stem final /i/}\label{section:syllable:VowelHiatus:Derived:I}
	When a stem-final /i/ precedes a vowel-initial derivational suffix, we   find various possible repair strategies: glide epenthesis, deleting the /i/, or merging the /i/ and the next vowel /ɑ/ into [e] (coalescence). The choice of strategy sometimes correlates with the morphological identity of the /i/, but also seems random.
	
	First, there is productive  derivational suffix \textit{-i} that can be added after virtually any verb (ending in /el/ or /ɑl/) to create an adjective (Table \ref{tab:vowel vowel i deriv glide eli}). This suffix can then undergo further derivational suffixation with either the nominalizer /-utʰjʏn/ or the adverbalizer /-oɾen/. Vowel hiatus between this suffix /i/ and the next vowel is repaired by glide epenthesis. 
	
	
	\begin{table}[H]
		\centering
		\caption{Glide epenthesis in derivation for  stems with deverbal suffix [-i]}
		\label{tab:vowel vowel i deriv glide eli}
		\resizebox{\textwidth}{!}{%
			\begin{tabular}{|l|llll|l| }
				\hline 
				+/-utʰjʏn/ & <badʒel> & bɑdˈʒel   & `to punish' & \armenian{պատժել} &   n=51 \\
				& <badʒel\textbf{i}> & bɑdʒeˈl-\textbf{i}   & `punishable' &  \armenian{պատժելի} & \\
				& <badʒel\textbf{iow}t'iwn> & bɑdʒel-\textbf{ij-u}ˈtʰjʏn   & `penalization' &  \armenian{պատժելիութիւն} & \\
				\hline 
				+/-oɾen/ & <xɣd͡ʒal> & χəʁˈd͡ʒɑl   & `to pity' & \armenian{խղճալ} &   n=6 \\
				& <xɣd͡ʒal\textbf{i}> & χəʁd͡ʒɑˈl-\textbf{i}   & `pitiful' &  \armenian{խղճալ} & \\
				& <xɣd͡ʒal\textbf{iō}ren> & χəʁd͡ʒɑˈl-\textbf{ij-o}ɾen   & `pitifully' &  \armenian{խղճալիօրէն} & \\
				\hline 
				
			\end{tabular}
	}\end{table}
	
	Thus, glide epenthesis is the norm for this suffix /i/ and its derivatives. However, in HD's judgment, careful speech allows replacing this inserted glide [j] in [bɑdʒel-\textbf{ij-u}tʰjʏn] with either a transient glide [ʲ] or a glottal stop: [bɑdʒel-\textbf{iʲ-u}tʰjʏn, bɑdʒel-\textbf{iʔ-u}tʰjʏn]. Based on judging the Eastern Armenian entries on English Wiktionary, the weak glide or glottal stop form seems to be more common in casual speech in Eastern Armenian. 
	
	Note that there are many words with a final /i/ that is a)   ambiguously a derivational suffix and b) is absent in some related forms. We set aside these words from discussion. For example, [ɑɾpʰi] and  [ɑɾpʰ] both mean `sun' \armenian{արփ, արփի}. Thus in a form like [ɑɾpʰ-ɑvoɾ] `luminous' \armenian{արփաւոր}, we have no way of knowing if this word was derived from [ɑɾpʰ] or from [ɑɾpʰi] with deletion.  
	
	
	Glide epenthesis is also attested in some roots (Table \ref{tab:vowel vowel i deriv glide mono}). There are five monosyllabic words that end in [i]. When a vowel suffix is added, we see glide epenthesis. 
	
	
	
	
	\begin{table}[H]
		\centering
		\caption{Glide epenthesis in derivation for monosyllabic  stems with final   [i]}
		\label{tab:vowel vowel i deriv glide mono}
		\begin{tabular}{|l|llll|  }
			\hline 
			+/-ɑX/ & <t͡s\textbf{i}> & ˈt͡s\textbf{i} & `horse' & \armenian{ձի} \\
			+/-ɑvoɾ/ & <t͡s\textbf{ia}wor> & t͡s\textbf{ij-ɑ}ˈvoɾ & `horseman' & \armenian{ձիաւոր}
			\\ \hline 
			+/-ɑX/ & <t\textbf{i}> & ˈtʰ\textbf{i} & `corpse' & \armenian{դի} \\
			+/-ɑɡ/ & <t\textbf{ia}g> & tʰ\textbf{iˈj-ɑ}ɡ & `corpse' & \armenian{դիակ}
			\\ \hline 
			+/-ɑ-/ & <t'\textbf{i}> & ˈtʰ\textbf{i} & `shovel' & \armenian{թի} \\
			+/-ɑ-/ & <t'\textbf{ia}t͡sowg> & tʰ\textbf{ij-ɑ-}ˈt͡suɡ & `paddle fish' & \armenian{թիաձուկ} \\
			& cf. <t͡sowg> & ˈt͡suɡ & `fish' & \armenian{ձուկ} 
			\\ \hline 
			+/-utʰjʏn/ & <m\textbf{i}> & ˈm\textbf{i} & `one' & \armenian{մի} \\
			& <m\textbf{iow}t'iwn> & m\textbf{ij-u}ˈtʰjʏn & `unity' & \armenian{միութիւն}
			\\ \hline 
			+/-oX/ & <l\textbf{i}> & ˈl\textbf{i} & `full' & \armenian{լի} \\
			+/-ov/ & <l\textbf{io}v> & l\textbf{iˈj-o}v & `fully' & \armenian{լիով}
			\\ \hline 
			
		\end{tabular}
	\end{table}
	
	
	As before, some of these tokens could be pronounced with a glottal stop in more formal speech, such as [mij-utʰjʏn] or [miʔ-utʰjʏn] `unity'. 
	
	There is likewise a monosyllabic root [di] \armenian{տի} which seems to be a bound root, because it is almost always found as part of a compound with the word [jezeɾkʰ] `edge', as [dij-ezeɾkʰ] `cosmos' \armenian{տիեզերք}. 
	
	Although glide epenthesis is the norm for the above words, there are morphemes that prefer vowel deletion (Table \ref{tab:vowel vowel i deriv eni delege}). The suffix /-eni/ that   used to derive plant names and other words. This suffix tends to delete before other vowel-initial derivational suffixes. 
	
	
	
	
	
	\begin{table}[H]
		\centering
		\caption{Vowel deletion in  derivation for stems with suffix [-eni] }
		\label{tab:vowel vowel i deriv eni delege}
		\resizebox{\textwidth}{!}{%
			\begin{tabular}{|l|llll|l|   }
				\hline 
				+/-ɑX/ & <t't'en\textbf{i}> & tʰətʰen\textbf{i} & `mulberry tree'  & \armenian{թթենի} & n=7 \\
				+/-ɑɡɑn/& <t't'en\textbf{a}gan>   & tʰətʰen-\textbf{ɑ}ˈɡɑn   & `moric' & \armenian{թթենական}& 
				\\ \hline 
				+/-ɑ-/ & <ayd͡zen\textbf{i}> & ɑjd͡zeˈn\textbf{i} & `goat's hair' & \armenian{այծենի} & n=4 \\
				& <ayd͡zen\textbf{a}kord͡z>   & ɑjd͡zen-\textbf{ɑ-}ˈkʰoɾd͡z & `chamois dresser' & \armenian{այծենագործ} & 
				\\ 
				& cf. <koɾd͡z> & ˈkʰoɾd͡z & `work' & \armenian{գործ} & 
				\\ \hline 
				+/-utʰjʏn/ & <vayren\textbf{i}> & vɑjɾeˈn\textbf{i} & `savage'   & \armenian{վայրենի}& n=1 \\
				& <vayren\textbf{ow}t'iwn> & vɑjɾen-\textbf{u}ˈtʰjʏn & `savageness' & \armenian{վայրենութիւն}&  \\ \hline 
				+/-oX/ & <epen\textbf{i}> & jepʰeˈn\textbf{i} & `ebony tree' & \armenian{եբենի}& n=2 \\
				+/-os/ & <epen\textbf{o}s> & jepʰeˈn-\textbf{o}s & `ebony' & \armenian{եբենոս} &  \\ \hline 
				
			\end{tabular}
	}\end{table}
	
	Although deletion is the norm for this suffix, we have found some instances where the vowel of  [-eni] is optionally deleted: [kʰɑχkeni] `townsman' \armenian{քաղքենի} can derive the word `middle class' with either deletion [kʰɑχken-utʰjʏn] \armenian{քաղքենութիւն} or glide epenthesis [kɑχkenij-utʰjʏn] \armenian{քաղքենիութիւն}. 
	
	Besides the above easily organized categories, most cases of final /i/ seem to behave randomly (Table \ref{tab:vowel vowel i deriv other glide}). For example, some words with final /i/ always used a glide in their derivatives in \citeauthor{kouyoumdjian-1970-DictionaryArmenianEnglish}. 
	
	\begin{table}[H]
		\centering
		\caption{Other stems with final /i/ that always use glide epenthesis in \citeauthor{kouyoumdjian-1970-DictionaryArmenianEnglish} }
		\label{tab:vowel vowel i deriv other glide}
		\resizebox{\textwidth}{!}{%
			\begin{tabular}{|l|llll|l|   }
				\hline 
				+/-ɑX/ & <naxaɣ\textbf{i}> & nɑχɑˈʁ\textbf{i} & `duodenum'   & \armenian{նախաղի} & n=11 \\
				+/-ɑɡɑn/& <naxaɣ\textbf{ia}gan> & nɑχɑʁ\textbf{ij-ɑ}ˈɡɑn & `duodenal'   & \armenian{նախաղիական}  & \\ 
				\hline 
				+/-ɑ-/ & <ar\textbf{i}> & ɑˈɾ\textbf{i} & `brave'   & \armenian{արի} & n=7 \\
				& <ar\textbf{ia}sird> & ɑɾ\textbf{ij-ɑ-}ˈsiɾd & `courageous'   & \armenian{արիասիրտ}  & \\ 
				& cf. <sird> & ˈsiɾd& `heart' & \armenian{սիրտ} & 
				\\ \hline 
				+/-utʰjʏn/ & <amowr\textbf{i}> & ɑmuˈɾ\textbf{i} & `bachelor'   & \armenian{ամուրի} & n=17 \\
				& <amowr\textbf{iow}t'iwn> & ɑmuɾ\textbf{ij-u}ˈtʰjʏn & `single life'   & \armenian{ամուրիութիւն}  & \\ 
				\hline 
				+/-uX/ & <k'aɣak'at͡s'\textbf{i}> & kʰɑʁɑkʰɑˈt͡s\textbf{i} & `citizen'   & \armenian{քաղաքացի} & n=1 \\
				+/-uhi/& <k'aɣak'at͡s'\textbf{iow}hi> & kʰɑʁɑkʰɑt͡s\textbf{ij-u}ˈhi & `citizeness'   & \armenian{քաղաքացիուհի}  & \\ 
				\hline 
				+/-eX/ & <sdnt\textbf{i}> & əstənˈtʰ\textbf{i} & `fosterling'   & \armenian{ստնդի} & n=1 \\
				+/-el/& <sdnt\textbf{ie}l> & əstəntʰ\textbf{iˈj-e}l & `to suckle'   & \armenian{ստնդիել}  & \\ 
				\hline 
				+/-oX/ & <k'an\textbf{i}> & kʰɑˈn\textbf{i} & `how many'   & \armenian{քանի} & n=2 \\
				+/-on/& <k'an\textbf{iō}n> & kʰɑn\textbf{iˈj-o}n & `how many'   & \armenian{քանիօն}  &
				\\ \hline 
			\end{tabular}
	}\end{table}
	
	For the above set of words, it's possible that some of these include a derivational suffix \textit{i}, and it is this suffix which prefers glide epenthesis. For example, the word  [ɑjɾi] `widow' \armenian{այրի} could be historically    derived from the word [ɑjɾ] `man' \armenian{այր}. This word takes glide epenthesis in its derivatives: [ɑjɾij-ɑnɑl] `to become a widow' \armenian{այրիանալ}. 
	
	In another set of words, the vowel /i/ is deleted in all the derivatives that are reported in \citeauthor{kouyoumdjian-1970-DictionaryArmenianEnglish} (Table \ref{tab:vowel vowel i deriv other delet}). 
	
	
	\begin{table}[H]
		\centering
		\caption{Other stems with final /i/ that always deletes in \citeauthor{kouyoumdjian-1970-DictionaryArmenianEnglish} }
		\label{tab:vowel vowel i deriv other delet}
		\resizebox{\textwidth}{!}{%
			\begin{tabular}{|l|llll|l|   }
				\hline 
				
				+/-ɑX/ & <tapn\textbf{i}> & tʰɑpʰˈn\textbf{i} & `laurel' &  \armenian{դաբնի}& n=29
				\\
				& <tapn\textbf{a}yin> & tʰɑpʰn\textbf{-ɑ}ˈjin & `of laurel' &   \armenian{դաբնային} & 
				\\ \hline 
				+/-ɑ-/ & <kaɣdn\textbf{i}> & kʰɑχtˈn\textbf{i} & `secret' &  \armenian{գաղտնի}& n=25
				\\
				& <kaɣdnn\textbf{a}bah> & kʰɑχtn\textbf{-ɑ-}ˈbɑh & `discreet' &   \armenian{գաղտնապահ} & 
				\\
				& cf. <bah> & ˈbɑh & `keeper' & \armenian{պահ} & 
				\\ \hline 
				+/-utʰjʏn/ & <ameh\textbf{i}> & ɑmeˈhi\textbf{i} & `wild' &  \armenian{ամեհի}& n=28
				\\
				& <ameh\textbf{ow}t'iwn> & ɑmeh\textbf{-u}ˈtʰjʏn & `wildness' &   \armenian{ամեհութիւն} & 
				\\ \hline 
				+/-uX/ & <barman\textbf{i}> & bɑɾmɑˈn\textbf{i} & `young man'& \armenian{պարմանի}& n=5
				\\
				+/-uhi/& <barman\textbf{ow}hi> & bɑɾmɑn\textbf{-u}ˈhi & `young woman'  &  \armenian{պարմանուհի} & 
				\\ \hline 
				+/-iX/ & <kadaɣ\textbf{i}> & ɡɑdɑˈʁ\textbf{i} & `mad'& \armenian{կատաղի}& n=1
				\\
				+/-il/& <kadaɣ\textbf{i}l> & ɡɑdɑˈʁ\textbf{-i}l & `to go mad' &  \armenian{կատաղիլ} & 
				\\ \hline 
				+/-eX/ & <vrat͡s'\textbf{i}> & vəɾɑˈt͡s\textbf{i} & `Georgian'& \armenian{վրացի}& n=10
				\\
				+/-eɾen/ & <vrat͡s'\textbf{e}rēn> & vəɾɑt͡s\textbf{-e}ˈɾen  & `Georgian language' &  \armenian{վրացերէն} & 
				\\ \hline 
				+/-oX/ & <asɣan\textbf{i}> & ɑsχɑˈn\textbf{i} & `thread'& \armenian{ասղանի}& n=4
				\\
				+/-ot͡s/ & <asɣan\textbf{o}t͡s'> & ɑsχɑˈn\textbf{-o}t͡s  & `needle-case'  &  \armenian{ասղանոց} & 
				\\ \hline 
				
			\end{tabular}
	}\end{table}
	
	Another set of words shows a special rule of fusing or merging /i/ with /ɑ/ to form [e] (Table \ref{tab:vowel vowel i deriv coalesnce only}).   We found 8 words which only had derivatives before /ɑ/, and here the /i/ and /ɑ/ merged to [e]. 
	
	
	\begin{table}[H]
		\centering
		\caption{Other stems with final /i/ where the /i/ coalesces with /ɑ/ to form [i]}
		\label{tab:vowel vowel i deriv coalesnce only}
		\begin{tabular}{|l|llll|    }
			\hline 
			& <ak\textbf{i}> & ɑˈkʰ\textbf{i} & `tail' & \armenian{ագի}
			\\
			+/-ɑvoɾ/ & <ak\textbf{e}wor> & ɑkʰ\textbf{-e}ˈvoɾ  & `tailed' & \armenian{ագեւոր} 
			\\
			& cf. <kord͡zawor> & kʰoɾd͡\textbf{z}-\textbf{ɑ}ˈvoɾ & `worker' & \armenian{գործաւոր}
			\\ \hline
			& <dar\textbf{i}> & dɑˈɾ\textbf{i} & `year' & \armenian{տարի}
			\\
			+/-ɑ-/ & <dar\textbf{e}kirk'> & dɑɾ\textbf{-e}ˈkʰiɾkʰ  & `annal' & \armenian{տարեգիրք} 
			\\
			& cf.<kirk'> &  ˈkʰiɾkʰ & `book' & \armenian{գիրք}
			\\ \hline
		\end{tabular}
	\end{table}
	
	Much more common are roots which seem to randomly pick one out of the three possible repair rules (Table \ref{tab:vowel vowel i deriv random}: glide epenthesis, deletion, /e/ coalescence). Although the coalescence rule is restricted to before /ɑ/, these roots can also randomly use either glide epenthesis or vowel deletion in this same  phonological context or in other phonological contexts. We found 35 such roots, many of which are quite high-frequency. 
	
	\begin{table}[H]
		\centering
		\caption{Other stems with final /i/ that randomly show coalescence, glide epenthesis, or deletion}
		\label{tab:vowel vowel i deriv random}
		\resizebox{\textwidth}{!}{%
			\begin{tabular}{|l|llll|    }
				\hline 
				& <kot\textbf{i}> & kʰoˈtʰ\textbf{i} & `leper' & \armenian{գոդի} \\
				/i-ɑnot͡s/$\rightarrow$[e] & <kot\textbf{e}not͡s'> & kʰotʰ\textbf{e}-ˈnot͡s & `Lazar-house'  & \armenian{գոդենոց} \\
				&cf.  <aɣp\textbf{a}not͡s'> & ɑχp-\textbf{ɑ}ˈnot͡s & `sewer'  & \armenian{աղբանոց} \\
				/i-utʰjʏn/$\rightarrow$[u]& <kot\textbf{ow}t'iwn> & kʰotʰ\textbf{-u}ˈtʰjʏn & `leprosy' & \armenian{գոդութիւն}
				\\ \hline 
				& <abag\textbf{i}> & ɑbɑˈɡ\textbf{i} & `glass' & \armenian{ապակի}
				\\
				/i-ɑ-/$\rightarrow$[e] & <abag\textbf{e}-kord͡z> & ɑbɑɡ\textbf{-e}ˈkʰoɾd͡z & `glass-maker' & \armenian{ապակեգործ}
				\\ & cf. <kord͡z> & ˈkʰoɾd͡z & `work'  & \armenian{գործ} 
				\\
				/i-ɑ-/$\rightarrow$[ɑ]& <abag\textbf{a}vad͡ʒaṙ> & ɑbɑɡ\textbf{-ɑ-}vɑˈd͡ʒɑɾ & `glass-seller' & \armenian{ապակավաճառ}
				\\ & cf. <vad͡ʒaṙ> & vɑˈd͡ʒɑɾ & `sale' & \armenian{վաճառ} 
				\\ \hline
				& <hok\textbf{i}> & hoˈkʰ\textbf{i} & `soul' & \armenian{հոգի}
				\\
				/i-ɑvoɾ/$\rightarrow$[e]& <hok\textbf{e}woṙ> & hokʰ\textbf{-e}vɑˈd͡ʒɑɾ & `spiritual'    & \armenian{հոգեւոր}
				\\ & cf. <t'ak\textbf{a}wor> & tʰɑkʰ-\textbf{ɑ}ˈvoɾ & `king' & \armenian{թագաւոր} \\
				/i-ɑnɑl/$\rightarrow$[ijɑ] & <hok\textbf{i}anal> & hokʰ\textbf{ij-ɑ}ˈnɑl & `to revive'  & \armenian{հոգիանալ}
				\\ \hline
				
				
			\end{tabular}
	}\end{table}
	
	
	In sum, a stem-final /i/ can display a quite random assortment of possible changes before a vowel-initial derivational suffix. Although some morphological categories show consistent behavior, many words seem to just randomly pick one of three possible repairs. The rule is used in some derivatives, but not others, and there is no clear semantic, morphological, or phonological rationale behind this variation. 
	
	
	\subsubsection{Stem final /o/}\label{section:syllable:VowelHiatus:Derived:O}
	
	Words with a final /o/ are rare. Thus, it is even rarer to find a /o/-final stem that takes a derivational suffix (Table \ref{tab:vowel vowel o deriv glide normal}).  The only cases we found in \citeauthor{kouyoumdjian-1970-DictionaryArmenianEnglish} were stems where the /o/ was spelled as <oy>.  The vowel sequence undergoes glide epenthesis.
	
	\begin{table}[H]
		\centering
		\caption{Glide epenthesis in derivation for  stems with final [o]}
		\label{tab:vowel vowel o deriv glide normal}
		\resizebox{\textwidth}{!}{%
			\begin{tabular}{|l|llll|l| }
				\hline 
				+/-ɑX/ & <martahad͡ʒ\textbf{oy}> & mɑɾtʰɑhoˈd͡ʒ\textbf{o} & `flatterer' & \armenian{մարդահաճոյ} & n=1
				\\
				+/-ɑnɑl/ & <martahad͡ʒ\textbf{oya}nal & mɑɾtʰɑhɑd͡ʒ\textbf{oj-ɑ}ˈɑ>   & `to flatter' &  \armenian{մարդահաճոյանալ} & 
				\\ \hline 
				+/-ɑ-/ & <tʃx\textbf{oy}> & tʰəʃˈχ\textbf{o} & `queen' & \armenian{դշխոյ} & n=1
				\\
				& <tʃx\textbf{oya}haw>  & tʰəʃχ\textbf{oj-ɑ}ˈhɑv  & `fat pullet' &  \armenian{դշխոյահաւ} & 
				\\
				& cf. <haw> & ˈhɑv & `chicken' & \armenian{հաւ} & 
				\\ \hline 
				+/-utʰjʏn/ & <vadapar\textbf{oy}> & vɑdɑpʰɑˈɾ\textbf{o} & `coward' & \armenian{վատաբարոյ} & n=12
				\\
				& <vadapar\textbf{oyow}t'iwn>  & vɑdɑpʰɑɾ\textbf{oj-u}ˈtʰjʏn  & `cowardice' &  \armenian{վատաբարոյութիւն} & 
				\\ \hline 
				+/-eX/ & <gog\textbf{oy}> & ɡoˈɡ\textbf{o} & `cocoa' & \armenian{կոկոյ} & n=1
				\\
				+/-eni/ & <gog\textbf{oye}ni>  & ɡoɡ\textbf{oj-e}ˈni & `coco tree' & \armenian{կոկոյենի} & 
				\\ \hline 
				
			\end{tabular}
	}\end{table}
	
	\subsubsection{Stem final /u/}\label{section:syllable:VowelHiatus:Derived:U}
	For words with final /u/, the most common vowel hiatus repair rule is turn the /u/ into [v] (Table \ref{tab:vowel vowel u deriv v normal}). We call this process /u/-devocalization. There many common words which show this process before virtually all types of vowels. 
	
	
	\begin{table}[H]
		\centering
		\caption{/u/-devocalization    in derivation for  stems with final [o]}
		\label{tab:vowel vowel u deriv v normal}
		\resizebox{\textwidth}{!}{%
			\begin{tabular}{|l|llll|l| }
				\hline 
				
				+/-ɑX/ & <owr\textbf{ow}> & uˈɾ\textbf{u} & `ghost' & \armenian{ուրու} &  n=11   \\
				+/-ɑɡɑn/ & <owr\textbf{owa}gan> & uɾ\textbf{v-ɑ}ˈɡɑn & `ghost' & \armenian{ուրուական} & \\ \hline 
				+/-ɑ-/ & <t͡s\textbf{ow}> & ˈt͡s\textbf{u} & `egg' & \armenian{ձու} &  n=14   \\
				& <t͡s\textbf{owa}gat'> & t͡sə\textbf{v-ɑ-}ˈɡɑtʰ & `egg-nog' & \armenian{ձուակաթ} & \\ \hline 
				+/-utʰjʏn/ & <erglez\textbf{ow}> & jeɾɡleˈz\textbf{u} & `bilingual' & \armenian{երկլեզու} &  n=2 \\
				& <erglez\textbf{owow}t'iwn> & jeɾɡlez\textbf{v-u}ˈtʰjʏn & `duplicity' & \armenian{երկլեզուութիւն} & \\ \hline 
				+/-iX/ & <lez\textbf{ow}> & leˈz\textbf{u} & `tongue' & \armenian{լեզու} &  n=4 \\
				+/-iɡ/ & <lez\textbf{owi}g> &lezˈ\textbf{v-i}ɡ & `little tongue' & \armenian{լեզուիկ} & \\
				\hline 
				+/-eX/ & <t͡ʃ'\textbf{ow}> & ˈt͡ʃ\textbf{u} & `travel' & \armenian{չու} &  n=7 \\
				+/-el/ & <t͡ʃ'\textbf{owe}l>&  t͡ʃəˈ\textbf{v-e}l & `to migrate' & \armenian{չուել} & \\
				\hline 
				+/oX/ & <t't'\textbf{ow}> & tʰəˈtʰ\textbf{u} & `sour' & \armenian{թթու} &  n=1  
				\\
				+/-od/ & <t't'\textbf{owo}d> & tʰətʰˈ\textbf{v-o}d & `sourish' &  \armenian{թթուոտ} & 
				\\ \hline 
				
			\end{tabular}
	}\end{table}
	
	Note that [v] can trigger schwa epenthesis to syllabify the consonant cluster. The [v] is variably devoiced to [f] after voiceless sounds; this is discussed in Section \S\ref{section:segmentalPhono:allphonLaryng:assiimlation:v}. 
	
	In the traditional orthography, both the /u/ and the devocalized [v] are spelled the same as <ow> \armenian{ու}. But in the reformed orthography as used for Eastern Armenian, the devocalized [v] is spelled as [v] \armenian{վ}. 
	
	The rule of /u/-devocalization is the default rule for handling /u/ before a vowel-initial derivational suffix. There are limited exceptions (Table \ref{tab:vowel vowel u deriv glide}). The word for `two' [jeɾɡu] and its derivatives like `twelve' [dɑsnəjeɾɡu] generally resist /u/-devocalization in their derivatives. Instead we get glide epenthesis. Although some of these derivatives can be pronounced with a [v] like [jeɾɡv-ɑɡɑn], the glide-based derivatives like [jeɾɡuj-ɑɡɑn] are significantly more commonly heard.  The word `pilot' [otʰɑt͡ʃu] `pilot' behaves the same. 
	
	
	\begin{table}[H]
		\centering
		\caption{Words that take glide epenthesis   for  stems with final [o]}
		\label{tab:vowel vowel u deriv glide}
		\begin{tabular}{|lllll| }
			\hline 
			&<erg\textbf{ow}>     & jeɾˈɡ\textbf{u} &  `two' & \armenian{երկու} \\
			&<erg\textbf{owa}gan>     & jeɾɡ\textbf{uj-ɑ}ˈɡɑn &  `dual' & \armenian{երկուական} \\
			&<erg\textbf{owow}t'iwn>     & jeɾɡ\textbf{uj-u}ˈtʰjʏn &  `bilocation' & \armenian{երկուութիւն} \\
			but& <erg\textbf{owo}reag>     & jeɾɡ\textbf{v-o}ɾˈjɑɡ &  `twin' & \armenian{երկուորեակ} \\
			\hline 
			&<ōtat͡ʃ'\textbf{ow}>     & otʰɑˈt͡ʃ\textbf{u} &  `pilot' & \armenian{օդաչու} \\
			&<ōtat͡ʃ'\textbf{owa}gan>     & otʰɑt͡ʃ\textbf{uj-ɑ}ˈɡɑn &  `aeronautic' & \armenian{օդաչուական} \\
			or &      & otʰɑt͡ʃ\textbf{v-ɑ}ˈɡɑn &    & \\\hline 
		\end{tabular}
	\end{table}
	
	Besides the above words, there are some morphological categories of words with final /u/. Some of these groups tend to show devocalization, while others arguably use deletion. 
	
	There are many words that end in the morpheme [-du], such as [hɑmɑɾ-ɑ-d\textbf{u}]  `accountable' \armenian{համարատու}. Here, we know this [-du] is some sort of compound-like suffix (a suffixoid)  because there's a word [hɑmɑɾ] `account'. This suffix/compound [-du] is used with the general meaning of `I give X or have X', and is related to the verb `to give' [dɑl] \armenian{տալ}. In general, these morpheme undergoes /u/-devocalization before the suffix /-utʰjʏn/: [hɑmɑɾ-ɑ-də\textbf{v-u}tʰjʏn] `report' \armenian{համարատուութիւն}. 
	
	Though we have found some cases of variation. For the word [t͡sɑjn-ɑ-du] `sonorous' \armenian{ձայնատու}, we found a  form  with devocalization in [t͡sɑjn-ɑ-də\textbf{v-u}tʰjʏn] `resonance'   \armenian{ձայնատուութիւն} , but also a form with deletion in  [t͡sɑjn-ɑ-d\textbf{-u}tʰjʏn] `aphonia' \armenian{ձայնատուութիւն}.
	
	On a last note, there are   many words that end in a suffix /-u/, especially as part of some suffix sequence /-ɑɾ-u/ where the /-ɑɾ/ is related to the verb [ɑɾnel] `to take': [pʰoɾt͡s] `attempt' \armenian{փորձ} derives [pʰoɾt͡s-ɑɾ-u]  `experienced' \armenian{փորձառու}. But for such suffixes, it is hard to determine how they behave with vowel hiatus.  When we come across a derivative like [pʰoɾt͡s-ɑɾ-ɑɡɑn] `experimental' \armenian{փորձառական}, it is difficult to know if the suffix /-u/ was deleted, or if it was just never added in the first place. 
	
	In sum, when a vowel-initial derivational suffix is added after an /u/, the default rule is devocalizing the /u/ to [v], but with some variation. 
	
	
	\subsection{Vowel hiatus repair in compounds and prefixoids}\label{section:syllable:VowelHiatus:CompPrefix}
	\textcolor{red}{write}
	Compound formation is a derivational process. Most `typical' compounds are formed by combining two stems with a vowel /-ɑ-/. This vowel triggers the same vowel hiatus rules as /ɑ/-initial derivational suffixes (\S\ref{section:syllable:VowelHiatus:CompPrefix:Typical}). This vowel is generally deleted before vowel-initial roots. But in some `atypical' compounds and with prefixoids (\S\ref{section:syllable:VowelHiatus:CompPrefix:Atypical}), the /-ɑ-/ is present before a root and triggers glottal stop epenthesis. 
	\subsubsection{Vowel hiatus repair in typical compounds} \label{section:syllable:VowelHiatus:CompPrefix:Typical}
	The most typical way to form a compound is to combine two stems with a linking vowel /-ɑ-/. If the first stem is V-final, then the vowel sequence between that stem and /-ɑ-/ is typically modified in some way, such as glide epenthesis. The various strategies for /V-ɑ-/ sequences is discussed in Section \S\ref{section:syllable:VowelHiatus:Derived}. But if the second  stem starts with a vowel, then the linking vowel /-ɑ-/ is typically omitted (Table \ref{tab:vowel compound normal}). 
	
	\begin{table}[H]
		\centering
		\caption{Vowel hiatus repair in typical compound constructions}
		\label{tab:vowel compound normal}
		\centering
		\begin{tabular}{|l| lll| }
			\hline 
			XC + CX &   ɑŋˈɡʏ\textbf{n} +  ˈ\textbf{kʰ}id͡z& `angle + line' &    \armenian{անկիւն,  գիծ}
			\\
			$\rightarrow$XC-ɑ-CX   & ɑŋɡʏ\textbf{n-ɑ-ˈkʰ}id͡z& `diagonal' &    \armenian{անկիւնագիծ}\\
			\hline 
			Xɑ + CX &   vəˈɡ\textbf{ɑ} +  ˈ\textbf{tʰ}uχt & `witness + paper' &    \armenian{վկայ,  թուղթ}
			\\
			$\rightarrow$Xɑj-ɑ-CX   & vəɡ\textbf{ɑj-ɑ-ˈtʰ}uχt & `certificate' &    \armenian{վկայաթուղթ}\\
			\hline 
			XC + VC & ˈmeɾ\textbf{t͡s} +\textbf{i}ˈmɑst & `close + meaning' & \armenian{մերձ, իմաստ} \\
			$\rightarrow$XC-VX & meɾ\textbf{t͡s-i}ˈmɑst & `rough meaning & ' \armenian{մերձիմաստ} 
			\\ \hline 
			XC + jeX & ˈvoχ\textbf{p} + ˈ\textbf{je}ɾkʰ & `lamentation + song' & \armenian{ողբ, երգ} \\
			$\rightarrow$ XC-eX & voχˈ\textbf{p-e}ɾkʰ & `tragic ballad' & \armenian{ողբերգ}  \\ \hline
			XC + voX & ˈkʰɑj\textbf{l} + ˈ\textbf{vo}ɾs & `wolf + hunt' & \armenian{գայլ,որս} \\
			$\rightarrow$ XC-oX & kʰɑjˈ\textbf{l-oɾ}s & `wolf-hunter'  & \armenian{գայլորս}  \\ \hline
		\end{tabular}
	\end{table}
	
	If the second stem's first two segments are [je]  \armenian{ե} or [vo]  \armenian{ո}, then the prescriptive rule is that that the second stem is treated as if it's vowel-initial. The [j] and [v] are absent. Adding the [j] or [v] however is attested in some special contexts (\S\ref{section:syllable:VowelHiatus:CompPrefix:Atypical}). 
	
	Very rarely, a compound is formed where a) the second stem starts with a consonant, but b) there is no linking vowel (Table \ref{tab:vowel compound no link}). The first stem can either end in a consonant or vowel.  It seems that the final vowel of the first stem does not undergo alternations. We use the qualifier `seems' because the relevant data is quite few, thus making it hard to form concrete generalizations. 
	
	
	\begin{table}[H]
		\centering
		\caption{Lack of vowel hiatus repair when there is no subsequent vowel in compounds}
		\label{tab:vowel compound no link}
		\centering
		\begin{tabular}{|l| lll| }
			\hline 
			Xɑ + CX &   ʃɑpʰʏˈʁ\textbf{ɑ} +  ˈ\textbf{t͡s}ev&  `sapphire + shape' & \armenian{շափիւղայ, ձեւ}
			\\
			$\rightarrow$Xɑ-CX   &   ʃɑpʰʏʁ-\textbf{ɑˈt͡s}ev      & `sapphire-shaped' & \armenian{շափիւղաձեւ} \\ 
			\hline 
			Xe + CX &   pʰɑˈz\textbf{e} +  ˈ\textbf{d͡ʒ}ɑnd͡ʒ& `hawk + fly' &    \armenian{բազէ, ճանճ} 
			\\
			$\rightarrow$Xe-CX   & pʰɑz\textbf{e-ˈd͡ʒ}ɑnd͡ʒ      & `hawk fly' &    \armenian{բազէճանճ}\\
			\hline 
			Xi + CX &   kʰeˈʁ\textbf{i} +  ˈ\textbf{d}un & `rudder + house'     &    \armenian{քեղի, տուն} 
			\\
			$\rightarrow$Xi-CX   & kʰeʁ\textbf{i-ˈd}un      & `wheelhouse'     &    \armenian{քեղիտուն}\\
			\hline 
			
		\end{tabular}
	\end{table}
	
	But if the first stem   ends in a vowel, while the second stem starts with a vowel, then the linking vowel is absent (Table \ref{tab:vowel compound no link , VV}). Vowel hiatus is usually repaired as if the second stem was a suffix. For example,   /u/ generally becomes [v] before both suffix-vowels and root-vowels.   /i/ tends to either delete or cause glide epenthesis, depending on the root.  /ɑ/ can trigger a glide or delete.  And /e/ tends to take glide epenthesis as well. 
	
	
	
	\begin{table}[H]
		\centering
		\caption{Vowel hiatus repair in compounds where first stem ends with vowel, and second one starts with vowel } 
		\label{tab:vowel compound no link , VV}
		\centering
		\begin{tabular}{|l| lll| }
			\hline 
			Xɑ + ɑX &   ɑɾˈkʰ\textbf{ɑ} +  \textbf{ɑ}ŋˈkʰəʁ  &  `king + vulture' & \armenian{արքայ, անգղ} 
			\\
			$\rightarrow$Xɑj-ɑX   &   ɑɾkʰ\textbf{ɑj-ɑ}ŋˈkʰəʁ      & `king vulture' & \armenian{արքայանգղ} \\ 
			\hline 
			Xɑ + əX &   pʰeˈs\textbf{ɑ} +  \textbf{ə}ŋˈɡeɾ  &  `groom + friend' & \armenian{փեսայ, ընկեր} 
			\\
			$\rightarrow$X-əX   &   pʰes\textbf{-ə}ŋˈɡeɾ      & `groomsman' & \armenian{փեսընկեր} \\ 
			\hline 
			Xu + əX &   t͡s\textbf{u} +  \textbf{ə}ŋɡɑˈl-it͡ʃ &  `egg + receiver' & \armenian{ձու,  ընկալիչ} 
			\\
			$\rightarrow$Xv-əX   &   t͡sə\textbf{v-ə}ŋˈɡɑl      & `ovary' & \armenian{ձուընկալ} \\ 
			\hline 
			Xi + oX &   jeɡeʁeˈt͡s\textbf{i} +  ˈ\textbf{o}ɾhˈnekʰ  & `church + blessing' &    \armenian{եկեղեցի, օրհնէք} 
			\\
			$\rightarrow$X-oX   & jeɡeʁet͡s\textbf{o}ɾhˈnekʰ      & `consecration' &  \armenian{եկեղեցօրհնէք}\\ \hline
			Xi + ɑX &   ˈt͡s\textbf{i} +  ˈ\textbf{ɑ}ɾˈt͡sɑɡ  & `horse + untied' &    \armenian{ձի, արձակ} 
			\\
			$\rightarrow$Xij-ɑX   & t͡s\textbf{ij-ɑ}ɾˈt͡sɑɡ     & `consecration' &  \armenian{ձիարձակ}\\
			\hline 
			Xi + oX &   ˈkʰ\textbf{i} +  \textbf{o}ˈʁi   & `juniper + gin' &    \armenian{գի, օղի} 
			\\
			$\rightarrow$Xij-oX   & kʰ\textbf{ij-o}ˈʁi     & `gin' &  \armenian{գիօղի}\\
			\hline 
			Xe + ɑX &   χɑhˈv\textbf{e} +   \textbf{ɑ}ˈmɑn & `coffee + pot'     &    \armenian{խահուէ, աման} 
			\\
			$\rightarrow$Xej-ɑX   & χɑhv\textbf{ej-ˈɑ}ˈmɑn      & `coffee-pot'     &    \armenian{խահուէաման}\\
			\hline 
			
		\end{tabular}
	\end{table}
	
	
	
	For some these cases, HD feels its possible to just use a glottal stop instead of glide epenthesis: [χɑhveʔ-ɑmɑn] `coffee-pot'. He likewise reports that using a glottal stop before a root `feels' more common than using a glottal stop before a suffix.  
	
	
	And as for the second stem, if the stem starts with [je,vo], then it is again treated as starting with [e,o] (Table \ref{tab:vowel compound no link vo je}). Vowel hiatus repair rules apply, as if the the second stem was a suffix. For example, /i/ and /ɑ/ can delete or trigger glide epenthesis.    Unfortunately, it is hard to know if this glide was epenthetic because of hiatus, vs. just retained from the second stem. 
	
	
	
	\begin{table}[H]
		\centering
		\caption{Vowel hiatus repair in compounds where first stem ends with vowel, and second one starts with [vo,je] } 
		\label{tab:vowel compound no link vo je}
		\centering
		\begin{tabular}{|l| lll| }
			\hline 
			Xi + voX &   kʰeˈɾ\textbf{i} +  \textbf{vo}ɾˈtʰi  &  `uncle + son' & \armenian{քեռի, որդի} 
			\\
			$\rightarrow$X-oX   &   kʰeɾ\textbf{-o}ɾˈtʰi      & `king vulture' & \armenian{քեռորդի} \\ 
			\hline 
			Xɑ + voX &   ɑɾˈkʰ\textbf{ɑ} +  \textbf{vo}ɾˈtʰi  &  `king + son' & \armenian{արքայ, որդի} 
			\\
			$\rightarrow$Xɑj-oX   &   ɑɾkʰ\textbf{ɑj-o}ɾˈtʰi      & `prince' & \armenian{արքայորդի} \\ 
			\hline 
			Xi + jeX &   ˈm\textbf{i} +  \textbf{je}ˈɾɑŋkʰ  &  `one + color' & \armenian{մի, երանգ} 
			\\
			$\rightarrow$Xij-eX   &   m\textbf{ij-e}ˈɾɑŋkʰ      & `monochromous' & \armenian{միերանգ} \\ 
			\hline 
			Xi + jeX &   vosˈk\textbf{i} +  \textbf{je}χˈt͡ʃʏɾ  &  `gold + horn' & \armenian{ոսկի, եղջիւր} 
			\\
			$\rightarrow$X-eX   &   vosk\textbf{-e}χˈt͡ʃʏɾ      & `golden horn' & \armenian{ոսկեղջիւր} \\ 
			\hline 
			
			Xɑ + jeX &   ɑɾˈkʰ\textbf{ɑ}  +  \textbf{je}ˈɾɑɡ  &  `king + vein' & \armenian{արքայ, երակ} 
			\\
			$\rightarrow$Xɑj-eX   &   ɑɾkʰ\textbf{ɑj-e}ˈɾɑɡ      & `basilisk' & \armenian{արքայերակ} \\ 
			\hline 
			
		\end{tabular}
	\end{table}
	
	
	Thus, when a compound has both a V-final first stem and a V-initial second stem, then it seems that the default pattern is to resolve the vowel hiatus by using the same rules  as if   the second stems is a derivational suffix. 
	
	However, we have found some bizarre cases of vowel hiatus repair in compounds that are hard to explain (Table \ref{tab:vowel compound no link e change weird}). There are some compounds where  the first stem ends in /i/, and this vowel becomes [e] before a vowel-initial stem. It seems that for such words, the linking vowel /-ɑ-/ was temporarily added creating an underlying three vowel-cluster /i-ɑ-V/. The /i-ɑ/ then merged or coalesced into /e/, and then the /e-V/ triggers a glottal stop. Surprisingly, a glide sounds quite bizarre here in HD's judgments. 
	
	
	\begin{table}[H]
		\centering
		\caption{Vowel hiatus repair in compounds where first stem ends with a derived [e], and second one starts with a vowel} 
		\label{tab:vowel compound no link e change weird} 
		\centering
		\begin{tabular}{|l| lll| }
			\hline 
			Xi + VX &   kʰɑˈɾ\textbf{i} +  \textbf{ɑ}ˈlʏɾ  &  `barley + flour' & \armenian{գարի,  ալիւր} 
			\\
			&   kʰɑɾ\textbf{eʔ-ɑ}ˈlʏɾ      & `barley flour' & \armenian{գարէալիւր} \\ 
			
			&   vosˈk\textbf{i} +  ˈ\textbf{o}ʁ  &  `gold + ring' & \armenian{ոսկի, օղ} 
			\\
			$\rightarrow$Xeʔ-VX   &   vosk\textbf{eˈʔ-o}ʁ      & `gold ring' & \armenian{ոսկէօղ} \\ 
			
			&   vosˈk\textbf{i} +  ˈ\textbf{ɑ}ɡən  &  `gold + gem' & \armenian{ոսկի, ակն} 
			\\
			$\rightarrow$Xeʔ-VX   &   vosk\textbf{eˈʔ-ɑ}ɡən      & `gold ring' & \armenian{ոսկեակն} \\ 
			\hline 
			
			
		\end{tabular}
	\end{table}
	
	\subsubsection{Vowel hiatus repair in prefixoids and atypical compounds}\label{section:syllable:VowelHiatus:CompPrefix:Atypical}
	The previous section discussed `typical' compounds. However, there are rare cases where a compound's second stem starts with a vowel, but a vowel linker /-ɑ-/ is inserted.  Here, the linking vowel  is phonologically unneeded but it is arbitrarily used because of the morphology. The hiatus between the linker and the second stem is repaired by a glottal stop. We discuss three such cases: numerals, prefixoids, and normal compounds. 
	
	One case comes from complex numerals where the linker is [ə]. Dialects and registers differ in whether this linker is pronounced before V-initial roots (Table \ref{tab:vowel compound numeral glottal}). This is discussed in \textcolor{red}{numeral morphology}.  HD's Western dialect prefers using this schwa even before a vowel, thus triggering a glottal stop. 
	
	
	\begin{table}[H]
		\centering
		\caption{Numerals with a glottal stop after the linker} 
		\label{tab:vowel compound numeral glottal} 
		\centering
		\begin{tabular}{|l    lll| }
			\hline 
			&  ˈdɑs +  ˈ\textbf{u}tʰə  &  `ten + eight' & \armenian{տաս,  ութը} 
			\\
			$\rightarrow$  &   dɑsn\textbf{-əˈʔ-u}tʰə      & `eighteen'  & \armenian{տասնըութը} \\ 
			&  ˈdɑs +  ˈ\textbf{i}nə  &  `ten + nine' & \armenian{տաս,  ինը} 
			\\
			&   dɑsn\textbf{-əˈʔ-i}nə      & `nineteen'  & \armenian{տասնըինը} \\ 
			
			\hline 
			
			
		\end{tabular}
	\end{table}
	
	
	
	
	
	Another common case are compounds that use prefixoids like [hɑm-ɑ-] `pan-', [hɑɡ-ɑ] `anti-' or [ɑmen-ɑ-] `most' (Table \ref{tab:vowel compound prefixoid glottal}). As explained in \textcolor{red}{prefixoid chapter}, the prefixoid is made up of a root-like prefix and a linking vowel /-ɑ-/. For some prefixoids, the linking vowel is usually absent before vowels. But there are arbitrary cases where the linking vowel is present, triggering a glottal stop. Some prefixoids like [ɑmen-ɑ-] exceptionally always take the linking vowel. 
	
	
	\begin{table}[H]
		\centering
		\caption{Prefixoids with a glottal stop after the linker} 
		\label{tab:vowel compound prefixoid glottal} 
		\centering
		\resizebox{\textwidth}{!}{%
			\begin{tabular}{|l| ll|ll| }
				\hline 
				& \multicolumn{2}{l|}{[hɑm-ɑ] `pan-', [hɑɡ-ɑ] `anti-'} &   \multicolumn{2}{l|}{[ɑmen-ɑ] `most'} 
				\\
				\hline 
				/ɑ/& \textbf{ɑ}ɾɑpʰɑˈɡɑn &    hɑm-\textbf{ɑʔ-ɑ}ɾɑpʰɑɡɑn-uˈtʰjʏn  
				& \textbf{ɑ}nuˈnov &   ɑmen-\textbf{ɑʔ-ɑ}nuˈnov 
				\\
				&`Arabian'   & `Pan-Arabism' & `renowned'  & `most renowned'  \\
				
				& \armenian{արաբական}  & \armenian{համաարաբականություն}
				& \armenian{անունով} & \armenian{ամենաանունով}
				\\
				\hline 
				/e/  &   \textbf{e}juˈtʰjʏn &   hɑm-\textbf{ɑʔ-e}juˈtʰjʏn
				&   \textbf{e}jɑˈɡɑn &   ɑmen-\textbf{ɑʔ-e}jɑˈɡɑn
				\\
				&  `existence'& `consubstantiality' 
				&  `essential'& `most essential' 
				
				\\
				& \armenian{էութիւն} & \armenian{համաէութիւն} & \armenian{էական} & \armenian{ամենաէական}
				\\ \hline 
				/i/& \textbf{i}sˈlɑm &    hɑm-\textbf{ɑʔ-i}slɑˈm-izm
				& \textbf{i}mɑsˈtun &   ɑmen-\textbf{ɑʔ-i}mɑsˈtun
				\\
				&`Islam'   & `Pan-Islamism' & `wise'  & `wisest'  \\
				
				& \armenian{իսլամ}  & \armenian{համաիսլամիզմ}
				& \armenian{իմաստուն} & \armenian{ամենաիմաստուն}
				\\
				\hline 
				/ə/& \textbf{ə}ŋɡeɾɑˈjin &    hɑɡ-\textbf{ɑʔ-ə}ŋɡeɾɑˈjin 
				& \textbf{ə}ntʰuˈnɑɡ &   ɑmen-\textbf{ɑʔ-ə}ntʰuˈnɑɡ
				\\
				&`social'   & `anti-social' & `capable'  & `most capable'  \\
				
				& \armenian{ընկերային}  & \armenian{հակաընկերային}
				& \armenian{ընդունակ} & \armenian{ամենաընդունակ}
				\\
				\hline 
				/o/& \textbf{o}ɾinɑˈɡɑn &    hɑm-\textbf{ɑʔ-o}ɾinɑˈɡɑn
				& \textbf{o}ɾtʰˈnjɑl  &   ɑmen-\textbf{ɑʔ-o}ɾtʰˈnjɑl  
				\\
				&`legitimate'   & `illegitmate' &`blessed' & `Most Blessed' \\
				
				& \armenian{օրինական}  & \armenian{հակաօրինական}
				& \armenian{օրհնեալ} & \armenian{ամենաօրհնեալ}
				\\
				\hline 
				/u/& \textbf{u}doˈbjɑ &    hɑɡ-\textbf{ɑʔ-ə}doˈbjɑ 
				& \textbf{u}ˈʒeʁ &   ɑmen-\textbf{ɑʔ-u}ˈʒeʁ
				\\
				&`utopia'   & `anti-utopia' & `strong'  & `strongest'  \\
				
				& \armenian{ուտոպիա}  & \armenian{հակաուտոպիա}
				& \armenian{ուժեղ} & \armenian{ամենաուժեղ}
				\\
				\hline 
			\end{tabular}
	}\end{table}
	
	Similarly, in a typical compound, the linking vowel is absent before a vowel. But in some arbitrary cases (Table \ref{tab:vowel compound compound glottal}), especially neologisms or technically vocabulary, the linking vowel is present. The vowel hiatus caused by the linking vowel and root triggers a glottal stop. Note that we know that these compounds are single words because a) they written as one word, b) they have final stress, and c) the first stem undergoes morphophonological alternations like vowel reduction (last two columns in Table \ref{tab:vowel compound compound glottal}). 
	
	\begin{table}[H]
		\centering
		\caption{Compounds with a glottal stop after the linker} 
		\label{tab:vowel compound compound glottal} 
		\centering
		\resizebox{\textwidth}{!}{%
			\begin{tabular}{|l| ll|ll| }
				\hline 
				& \multicolumn{2}{l|}{Without reduction} &   \multicolumn{2}{l|}{With reduction} 
				\\
				\hline 
				/ɑ/& əstɑˈmoks + \textbf{ɑ}ˈʁikʰ & əstɑmoks-\textbf{ɑʔ-ɑ}ʁikʰ-ɑˈjin
				& jeɾɑˈʒiʃt + \textbf{ɑ}ˈlikʰ & jeɾɑʒəʃt-\textbf{ɑʔ-ɑ}likʰ
				\\
				& `stomach + intestine' & `gastrointestinal'
				& `musician + wave' & `music channel'
				\\
				& \armenian{ստամոքս, աղիք} & \armenian{ստամոքսաաղիքային}
				& \armenian{երաժիշտ, ալիք} & \armenian{երաժշտաալիք}
				\\
				\hline 
				/e/& kʰəɾisˈtos + \textbf{e}t͡ʃ- & kʰəɾistos-\textbf{ɑˈʔ-e}t͡ʃ
				& ˈlujs + ˈ\textbf{e}t͡ʃ & lus-\textbf{ɑˈʔ-e}t͡ʃ
				\\
				& `Christ + $\sqrt{}$descend' & `Christ descent'
				& `light + page' & `illuminated page'
				\\
				& \armenian{Քրիստոս,  էջ} & \armenian{Քրիստոսաէջ}
				& \armenian{լոյս, էջ} & \armenian{լուսաէջ}
				\\
				\hline 
				/i/& iˈmɑst + ˈ\textbf{i}χt͡s  & imɑst-\textbf{ɑˈʔ-i}χt͡s
				& hənˈtʰiɡ + \textbf{i}ˈɾɑn  & həntk-\textbf{ɑʔ-i}ɾɑn-ˈjɑn
				\\
				& `meaning + wish' & `sensible'
				& `Indian + Iran' & `Indo-Iranic'
				\\
				& \armenian{իմաստ,  իղձ} & \armenian{իմաստաիղձ}
				& \armenian{հնդիկ,  Իրան} & \armenian{հնդկաիրանեան}
				\\
				\hline 
				/ə/& ɑʃˈχɑɾ + \textbf{ə}mpʰəɾˈnum  & ɑʃχɑɾ-\textbf{ɑʔ-ə}mpʰəɾˈnum
				& hənˈtʰiɡ + \textbf{ə}ŋˈɡujz  & həntk-\textbf{ɑʔ-ə}ŋˈɡujz 
				\\
				& `world + perception' & `worldview'
				& `Indian + walnut' & `coconut'
				\\
				& \armenian{աշխարհ, ըմբռնում} & \armenian{աշխարհաըմբռնում}
				& \armenian{հնդիկ,  ընկոյզ} & \armenian{հնդկընկույզ}
				\\
				\hline 
				/o/& ˈzɑɾtʰ + ˈ\textbf{o}ʁ  & zɑɾtʰ-\textbf{ɑˈʔ-o}ʁ
				& jeɾˈɡin + ˈ\textbf{o}tʰ  & jeɾɡn-\textbf{ɑʔ-o}tʰɑˈjin
				\\
				& `decoration + ear-ring' & `decorative ear-ring'
				& `heaven (archaic) + air' & `airborne'
				\\
				& \armenian{զարդ, օղ} & \armenian{զարդաօղ}
				& \armenian{երկին,  օդ} & \armenian{երկնաօդային}
				\\ \hline 
				/u/& ˈotʰ + \textbf{u}ˈʁi  & otʰ-\textbf{ɑʔ-u}ˈʁi
				& ˈmiʃt + \textbf{u}ˈɾɑχ  & məʃt-\textbf{ɑʔ-u}ˈɾɑχ
				\\
				& `air + way' & `airway'
				& `always + happy' & `ever-happy'
				\\
				& \armenian{օդ, ուղի} & \armenian{օդաուղի}
				& \armenian{միշտ,  ուրախ} & \armenian{մշտաուրախ}
				\\
				\hline 
			\end{tabular}
	}\end{table}
	
	When the prefixoid or compound uses the linker /ɑ/ and when the second stem starts [vo,je], then the [vo,je] surfaces in the compound (Table \ref{tab:vowel compound compound je vo linker}). 
	
	
	\begin{table}[H]
		\centering
		\caption{Compounds with a linker before [je,vo]} 
		\label{tab:vowel compound compound je vo linker} 
		\centering
		\resizebox{\textwidth}{!}{%
			\begin{tabular}{|l| ll|ll| }
				\hline 
				& \multicolumn{2}{l| }{/-ɑ/ + [je]}& \multicolumn{2}{l| }{/-ɑ/ + [vo]}
				\\ \hline
				Prefixoid & \textbf{je}ɾˈɡiɾ & hɑɡ-\textbf{ɑ-je}ɾɡˈɾ-jɑ & \textbf{vo}loɾdɑˈɡɑn & hɑɡ-\textbf{ɑ-vo}loɾdɑˈɡɑn 
				\\
				& `Earth'& `antichthon'  & `horizontal'& `antiperistaltic'  
				\\
				&  \armenian{երկիր} & \armenian{հակաերկրեայ} & \armenian{ոլորտական} & \armenian{հակաոլորտական}
				\\
				\hline 
				Prefixoid & \textbf{je}ɾˈɡɑɾ & ɑmen-\textbf{ɑ-je}ɾˈɡɑɾ & \textbf{vo}t͡ʃənt͡ʃɑˈt͡sum   & ɑmen-\textbf{ɑ-vo}t͡ʃənt͡ʃɑˈt͡sum 
				
				
				\\
				& `longest'& `longest'   & `annihilation' & `annihilation of all'  
				
				\\
				& \armenian{երկար}  & \armenian{ամենաերկար}& \armenian{ոչնչացում} &  \armenian{ամենաոչնչացում}
				\\
				\hline 
				Compound & 
				kʰəˈluχ + \textbf{je}ˈdev & kʰəlχ-\textbf{ɑ-je}ˈdev& ɑʁˈves + \textbf{vo}ɾˈsoɾtʰ 
				& ɑʁves-\textbf{ɑ-vo}ɾˈsoɾtʰ 
				\\
				& `head + back' & `back of head' & `fox + hunter' & `fox hunter'  
				
				\\
				& \armenian{գլուխ,  ետեւ}  & \armenian{գլխաետև}& \armenian{աղուէս,  որսորդ}  & \armenian{աղուէսաորսորդ}
				\\
				\hline 
			\end{tabular}
		}
	\end{table}
	
	
	
	
	For these `atypical' compounds, one can argue that the reason why there's a glottal stop is because speakers want to treat such as compounds as some grey area between a phrase and a word. In terms of phonological domains or prosody, one could argue that the glottal stop signifies a weak word boundary before the second stem. 
	
	Cross-linguistically, such phenomena have been modeled using recursive prosodic structure \textcolor{red}{cite, like greeks}. For a typical compound like [ɑɾkʰ\textbf{ɑj-ɑ}ŋɡəʁ] `king vulture' with glide epenthesis and a deleted linker  (Table \ref{tab:vowel compound no link}), the entire word is just one phonological word. See Section \S\ref{section:stress:regular:domain} for discussion on phonological words.  But for an atypical compound with an unneeded linker and a  glottal stop  like [əstɑmoks-\textbf{ɑʔ-ɑ}ʁikʰ-ɑˈjin] `gastrointestinal' (Table \ref{tab:vowel compound compound glottal}), the entire word would act as a phonological word, while the second stem would act as an extra internal phonological word. The first stem could also be treated as its own word (Representation \ref{rep:syll:vowelhiatusCompoundAtypica}). 
	
	\begin{representation}
		Prosodic structure of a typical compound vs. an atypical compound 
		\label{rep:syll:vowelhiatusCompoundAtypica}
		\centering
		\begin{tabular}{ll}
			One prosodic word in & Layered prosodic words in \\
			{}[ɑɾkʰ\textbf{ɑˈj-ɑ}ŋɡəʁ]  `king vulture'&  [əstɑmoks-\textbf{ɑʔ-ɑ}ʁikʰɑˈjin] `gastrointestinal'
			\\
			\begin{tikzpicture}[scale =1]
				\Tree    [.PWord  [.$\sigma$ ɑɾ ] [.$\sigma$ kʰ\textbf{ɑ} ] [.$\sigma$ ˈ\textbf{{j-ɑ}ŋ} ] [.$\sigma$ ɡəʁ ] 
				] 
			\end{tikzpicture}
			
			& 	
			\begin{tikzpicture}[scale =1]
				\Tree   [.PWord [ [.$\sigma$ əs ] ] [ [.$\sigma$ tɑ ] ] [ [.$\sigma$ mok ] ] [ [.$\sigma$ s\textbf{-ɑ} ] ] [.PWord   [.$\sigma$ \textbf{ʔ-ɑ} ] [.$\sigma$ ʁi ]  [.$\sigma$ kʰɑ ]   [.$\sigma$ ˈjin ] ]
				] 
		\end{tikzpicture}	\end{tabular}
	\end{representation}
	
	But it is an open question if such recursive prosodic structure is  truly applicable to atypical Armenian compounds. The main problem is that typical compounds show more refined word-internal domains, discussed in \textcolor{red}{compound prosody chapter}. Furthermore, the first stem stills shows morphophonological alternations (like vowel reduction), and these changes are arguably stem-level changes instead of word-level changes. Thus whatever prosodic boundary exists before the glottal stop, this boundary is not strong enough to prevent all stem-internal changes.
	
	\subsection{Vowel hiatus repair before regular inflectional suffixes}\label{section:syllable:VowelHiatus:Inf}
	There are few V-initial inflectional suffixes. Some of these suffixes are productive and can be easily added after any word:  \textit{-i} ({\gendat}), \textit{-e} ({\abl}), \textit{-ov} ({\ins}). One suffix is productive but can only take  monosyllabic roots: \textit{-eɾ} ({\pl}). Some suffixes are productive but can only follow C-final words: \textit{-ə} {\defgloss}, \textit{-əs} ({\possFsg}), \textit{-ətʰ} ({\possSsg})
	
	Besides these productive suffixes, there  is one suffix that is unproductive and can follow only a handful of roots \textit{-u} ({\gendat}). There are also some verbal inflectional suffixes like past \textit{-ɑ,-i} which we discuss in Section \S\ref{section:syllable:VowelHiatus:Irregular}.  
	
	
	Based on these suffixes, we can easily examine how vowel hiatus is handled between any type of V1 and a V2 that belongs to the set \{e, i, o\}. We find glide epenthesis is the norm. Unfortunately, we can't extensively examine how vowel hiatus is handled before an inflectional /ɑ,u/ because the data is too limited by the morphology. 
	
	\subsubsection{Stem-final /ɑ/}\label{section:syllable:VowelHiatus:Inf:A}
	
	First, consider words that end in /ɑ/ (Table \ref{tab:vow vowel a inf}). Before all V-initial inflection, glide epenthesis is the regular rule.  Irregularities are limited (\S\ref{section:syllable:VowelHiatus:Irregular}). 
	
	\begin{table}[H]
		\centering
		\caption{Glide epenthesis between /ɑ/ and V-initial inflection}
		\label{tab:vow vowel a inf}
		\resizebox{\textwidth}{!}{%
			\begin{tabular}{|l|lll|lll| }
				\hline 
				& \multicolumn{3}{l| }{Words ends in orthographic  <ay> \armenian{այ}}& \multicolumn{3}{l| }{Words ends in orthographic <a> \armenian{ա}}
				\\
				&        `king' & `rosy' & `monk'  & `Messiah' & `Europe' & `chemistry' \\
				& \armenian{արքայ} & \armenian{վարդեայ}& \armenian{աբեղայ} & \armenian{Մեսիա} & \armenian{Եւրոպա} & \armenian{քիմիա}
				\\
				\hline 
				& <ark'ay> & <varteay> & <apeɣay>  & <Mesia> & <Ewroba> & <k'imia> 
				\\
				& ɑɾˈkʰ\textbf{ɑ} & vɑɾtʰˈj\textbf{ɑ}& ɑpʰeˈʁ\textbf{ɑ} &  mesˈj\textbf{ɑ} & jevɾoˈb\textbf{ɑ}  & kʰimˈj\textbf{ɑ}  
				\\
				{\gendat} \textit{-i} &  ɑɾkʰ\textbf{ɑˈji}  &   vɑɾtʰj\textbf{ɑˈji}&ɑpʰeʁ\textbf{ɑˈji} &   mesj\textbf{ɑˈji} & jevɾob\textbf{ɑˈji}& kʰimj\textbf{ɑˈji}
				\\
				{\abl} \textit{-e}&  ɑɾkʰ\textbf{ɑˈje}  & vɑɾtʰj\textbf{ɑˈje}& ɑpʰeʁ\textbf{ɑˈje}&  mesj\textbf{ɑˈje} & jevɾob\textbf{ɑˈje}& kʰimj\textbf{ɑˈje}
				\\
				{\ins} \textit{-ov}&  ɑɾkʰ\textbf{ɑˈjov}  & vɑɾtʰj\textbf{ɑˈjov}& ɑpʰeʁ\textbf{ɑˈjov} &mesj\textbf{ɑˈjov} & jevɾob\textbf{ɑˈjov}& kʰimj\textbf{ɑˈjov}
				\\ \hline 
			\end{tabular}
	}\end{table}
	
	As is clear, glide epenthesis is the norm for repair vowel sequence of /ɑ/ plus essentially any type of inflectional vowel. This process applies after both roots like [ɑɾkʰɑ] `king', and after derivational suffixes like \textit{-jɑ} in [vɑɾt.jɑ] `rosy', derived from [vɑɾtʰ] `rose' \armenian{վարդ}. 
	
	
	
	\subsubsection{Stem-final /e/}\label{section:syllable:VowelHiatus:Inf:E}
	Given a words   that end in /e/ like [ɾo.be] `second', if a V-initial inflectional suffix like instrumental [-ov] is added, then we get glide epenthesis: [ɾo.be.jov]. Glide epenthesis is the rule before all V-initial inflectional suffixes (Table \ref{tab:vow vowel e inf}). 
	
	
	\begin{table}[H]
		\centering
		\caption{Glide epenthesis between /e/ and V-initial inflection}
		\label{tab:vow vowel e inf}
		\resizebox{\textwidth}{!}{%
			\begin{tabular}{|l|ll ll| }
				\hline 
				&        `second' & `prophet' &`made of bricks' & `Cairo'\\
				& \armenian{րոպէ}  & \armenian{մարգարէ}& \armenian{աղիւսէ} &  \armenian{Գահիրէ}
				\\
				\hline  
				& ɾo.ˈb\textbf{e} & mɑɾ.kʰɑ.ˈɾ\textbf{e} & ɑ.ʁʏ.ˈs\textbf{e} & kʰɑ.hi.ˈɾ\textbf{e}
				\\
				{\gendat} \textit{-i} &   ɾo.b\textbf{e.ˈji} & mɑɾ.kʰɑ.ɾ\textbf{e.ˈji}& ɑ.ʁʏ.s\textbf{e.ˈji} &  kʰɑ.hi.ɾ\textbf{e.ˈji}
				\\
				{\abl} \textit{-e}&  ɾo.b\textbf{e.ˈje} & mɑɾ.kʰɑ.ɾ\textbf{e.ˈje}& ɑ.ʁʏ.s\textbf{e.jˈe} &kʰɑ.hi.ɾ\textbf{e.ˈje}
				\\
				{\ins} \textit{-ov}&  ɾo.b\textbf{e.ˈjov}& mɑɾ.kʰɑ.ɾ\textbf{e.ˈjov} & ɑ.ʁʏ.s\textbf{e.ˈjov} &kʰɑ.hi.ɾ\textbf{e.ˈjov}
				\\ \hline 
			\end{tabular}
	}\end{table} 
	
	All words that end in /e/ take a glide /j/ before V-initial inflection. This rule applies across native words like [ɾobe] `second' and to loanwords like [kʰɑhiɾe] `Cairo'. Derivational suffixes also obey this rule, such as the derivational suffix -\textit{e} in [ɑʁʏse] `made of bricks', derived from [ɑʁʏs] `brick' \armenian{աղիւս}. 
	
	\subsubsection{Stem-final /i/}\label{section:syllable:VowelHiatus:Inf:I}
	
	When /i/ is before V-initial inflection, the norm is to get glide epenthesis in Western Armenian (Table \ref{tab:vow vowel i inf}). 
	
	\begin{table}[H]
		\centering
		\caption{Glide epenthesis between /i/ and V-initial inflection}
		\label{tab:vow vowel i inf}
		\resizebox{\textwidth}{!}{%
			\begin{tabular}{|l|ll l| lll| }
				\hline 
				& \multicolumn{3}{l|}{Roots with final /i/} &  \multicolumn{3}{l|}{Suffixes with final /i/} 
				\\
				&        `island' & `son'& `pigeon' &`queen' & `venerable' & `apple-tree'\\ 
				& \armenian{կղզի} & \armenian{որդի} & \armenian{աղաւնի}    & \armenian{թագուհի} & \armenian{յարգելի} & \armenian{խնձորենի}
				\\
				\hline  
				& ɡəʁˈz\textbf{i} & voɾˈtʰ\textbf{i}  & ɑʁɑvˈn\textbf{i} &  tʰɑkʰuˈh\textbf{i}&  hɑɾkʰeˈl\textbf{i} &  χənt͡soɾeˈn\textbf{i}
				\\
				{\gendat} \textit{-i} & ɡəʁz\textbf{iˈji} & voɾtʰ\textbf{iˈji} & ɑʁɑvn\textbf{iˈji}& tʰɑkʰuh\textbf{iˈji}    & hɑɾkʰel\textbf{iˈji}    & χənt͡soɾen\textbf{iˈji}    
				\\
				{\abl} \textit{-e}&  ɡəʁz\textbf{iˈje}  &  voɾtʰ\textbf{iˈje} &ɑʁɑvn\textbf{iˈje}  & tʰɑkʰuh\textbf{iˈje}    & hɑɾkʰel\textbf{iˈje}    & χənt͡soɾen\textbf{iˈje}    
				\\
				{\ins} \textit{-ov}& ɡəʁz\textbf{iˈjov}  & voɾtʰ\textbf{iˈjov} & ɑʁɑvn\textbf{iˈjov}  & tʰɑkʰuh\textbf{iˈjov}    & hɑɾkʰel\textbf{iˈjov}   & χənt͡soɾen\textbf{iˈjov}     
				\\ \hline 
			\end{tabular}
	}\end{table} 
	
	Glide epenthesis  applies after both roots and after all derivational suffixes. For example, glide epenthesis applies after: a) The feminine nominalizer \textit{-uhi} as in  [tʰɑkʰ-uhi]  `queen' derived from [tʰɑkʰ] `crown' \armenian{թագ}. b) The deverbal adjectivizer \textit{-i} as in [hɑɾkʰel-i] `venerable' derived from [hɑɾkʰel] `to respect \armenian{յարգել}. And c) the tree-naming suffix \textit{-eni} in [χənt͡soɾ-eni] `apple tree' derived from  [χənt͡soɾ] \armenian{խնձոր}. 
	
	There is likewise  an irregular word [t͡si] `horse' \armenian{ձի} which takes the irregular dative \textit{-u}. Here again we get glide epenthesis: [t͡si-ju]. 
	
	We emphasize that glide epenthesis is the norm in Western Armenian (Table \ref{tab:vow vowel i inf dialect}). In contrast in Eastern Armenian, stem-final /i/ usually deletes before V-initial inflection. In Eastern, the regular dative and ablative are /-i, it͡sʰ/. After /i/-final stems, the Eastern dative and ablative  are  /-u, ut͡sʰ/; Eastern likewise has a locative \textit{-um}. Many but not all \textit{i}-final words can lose their /i/ before these Eastern suffixes, while Western keeps the /i/. 
	
	
	\begin{table}[H]
		\centering
		\caption{Dialectal variation in how /i/ appears before  V-initial inflection}
		\label{tab:vow vowel i inf dialect}
		\begin{tabular}{|l|ll | ll| }
			\hline 
			&        `soul' & \armenian{հոգի} & `Azeri' & \armenian{ազերի
			}\\ 
			\hline  
			& Western & Eastern& Western & Eastern
			\\
			& hoˈkʰ\textbf{i} & hoˈkʰ\textbf{i} & ɑzeˈɾ\textbf{i}& ɑzeˈɾ\textbf{i}
			\\
			{\gendat} \textit{-i} (WA), \textit{-u} (EA) & hokʰ\textbf{iˈji}   & hoˈkʰ\textbf{u}   & ɑzeɾ\textbf{iˈji} & ɑzeɾ\textbf{iˈji} 
			\\
			{\abl} \textit{-e} (WA), \textit{-it͡sʰ, -ut͡sʰ} (EA)& hokʰ\textbf{iˈje}   & hoˈkʰ\textbf{ut͡sʰ}   & ɑzeɾ\textbf{iˈje} & ɑzeɾ\textbf{iˈjit͡sʰ}
			\\
			{\ins} \textit{-ov}&   hokʰ\textbf{iˈjov}   & hoˈkʰ\textbf{ov}   & ɑzeɾ\textbf{iˈjov} & ɑzeɾ\textbf{iˈjov}
			\\
			&   &hokʰ\textbf{iˈjov}     && 
			\\
			{\locgloss} \textit{-um}&    & hoˈkʰ\textbf{um}   &   & ɑzeɾ\textbf{iˈjum}
			\\
			&   &hokʰ\textbf{iˈjum}     && 
			\\ \hline 
		\end{tabular}
	\end{table} 
	
	\subsubsection{Stem-final /ə/}\label{section:syllable:VowelHiatus:Inf:Schwa}
	There are relatively few words that end in /ə/ and that can take inflection. One category of such words is the names for sounds,  like [pʰə] \armenian{բը} to mean the sound `p' or the letters \armenian{բ,փ} <p,p'>. Before V-initial inflection, we get glide epenthesis (Table \ref{tab:vow vowel schwa inf}). 
	
	
	\begin{table}[H]
		\centering
		\caption{Glide epenthesis between /ə/ and V-initial inflection}
		\label{tab:vow vowel schwa inf}
		\begin{tabular}{|l|ll ll| }
			\hline 
			& \armenian{բը} & \armenian{պը} & \armenian{մը}  & \armenian{հը} 
			\\
			\hline  
			& ˈpʰ\textbf{ə} & ˈb\textbf{ə} & ˈm\textbf{ə} & ˈh\textbf{ə}
			\\
			{\gendat} \textit{-i}  & pʰ\textbf{əˈji} & b\textbf{əˈji} & m\textbf{əˈji} & h\textbf{əˈji} 
			\\
			{\abl} \textit{-e} & pʰ\textbf{əˈje} & b\textbf{əˈje} & m\textbf{əˈje} & h\textbf{əˈje} 
			\\
			{\ins} \textit{-ov} & pʰ\textbf{əˈjov} & b\textbf{əˈjov} & m\textbf{əˈjov} & h\textbf{əˈjov} 
			\\ \hline 
		\end{tabular}
	\end{table} 
	
	\subsubsection{Stem-final /o/}\label{section:syllable:VowelHiatus:Inf:O}
	
	There are relatively few words that end in [o]. Before V-initial inflection, there is glide epenthesis (Table \ref{tab:vow vowel o inf}). 
	
	\begin{table}[H]
		\centering
		\caption{Glide epenthesis between /o/ and V-initial inflection}
		\label{tab:vow vowel o inf}
		\resizebox{\textwidth}{!}{%
			\begin{tabular}{|l|ll|ll| }
				\hline 
				& \multicolumn{2}{l| }{Words ends in orthographic  <ō> \armenian{օ}}& \multicolumn{2}{l| }{Words ends in orthographic <oy> \armenian{աy}}
				\\
				&        `zero' & `tango' & `collection' & `agreeable' \\
				& \armenian{զէրօ} &  \armenian{թանկօ} & \armenian{հաւաքածոյ} &  \armenian{հաճոյ}
				\\
				\hline 
				& <zērō> & <t'angō> & <hawak'ad͡zoy> & <had͡ʒoy> 
				\\
				& zeˈɾ\textbf{o}  & tʰɑŋˈɡ\textbf{o}  & hɑvɑkʰɑˈd͡z\textbf{o} & hɑˈd͡ʒ\textbf{o}    
				\\
				{\gendat} \textit{-i} &  zeɾ\textbf{oˈji}  &  tʰɑŋɡ\textbf{oˈji}  &  hɑvɑkʰɑd͡z\textbf{oˈji}  &  hɑd͡ʒ\textbf{oˈji} 
				\\
				{\abl} \textit{-e} &  zeɾ\textbf{oˈje}  &  tʰɑŋɡ\textbf{oˈje}  &  hɑvɑkʰɑd͡z\textbf{oˈje}  &  hɑd͡ʒ\textbf{oˈje} 
				\\
				{\ins} \textit{-ov} &  zeɾ\textbf{oˈjov}  &  tʰɑŋɡ\textbf{oˈjov}  &  hɑvɑkʰɑd͡z\textbf{oˈjov}  &  hɑd͡ʒ\textbf{oˈjov} 
				\\ \hline 
			\end{tabular}
	}\end{table}
	
	Glide epenthesis applies both after roots like [zeɾo] `zero', and after derivational suffixes like the nominalizer \textit{-o} in [hɑvɑvʰɑd͡z-o] `collection' that is added onto resultative participles like [hɑvɑkʰɑd͡z] `collected' \armenian{հաւաքած}. 
	
	
	Orthographically, loanwords with final [o] tend to be spelled as <ō> \armenian{օ}, while native words are spelled with final <oy> \armenian{ոյ}. The reason is diachronic, as   explained in Section \S\ref{section:syllable:VowelHiatus:Diachrony}. 
	
	
	
	\subsubsection{Stem-final /u/}\label{section:syllable:VowelHiatus:Inf:U}
	
	When a  stem-final /u/ is before a V-initial inflectional suffix, the most common repair in Western Armenian is glide epenthesis (Table \ref{tab:vow vowel u inf}).  
	
	\begin{table}[H]
		\centering
		\caption{Glide epenthesis between /u/ and V-initial inflection}
		\label{tab:vow vowel u inf}
		\begin{tabular}{|l|ll ll| }
			\hline 
			&        `owl' & `bee' &`sour' & `male' \\
			& \armenian{բու}  & \armenian{մեղու}& \armenian{թթու} &  \armenian{արու}
			\\
			\hline  
			& ˈpʰ\textbf{u} &  meˈʁ\textbf{u} &  tʰəˈtʰ\textbf{u} &  ɑˈɾ\textbf{u} 
			\\
			{\gendat} \textit{-i} & pʰ\textbf{uˈji} &  meʁ\textbf{uˈji} &  tʰətʰ\textbf{uˈji} &  ɑɾ\textbf{uˈji}   
			\\
			{\abl} \textit{-e}&   pʰ\textbf{uˈje} &  meʁ\textbf{uˈje} &  tʰətʰ\textbf{uˈje} &  ɑɾ\textbf{uˈje}
			\\
			{\ins} \textit{-ov}&  pʰ\textbf{uˈjov} &  meʁ\textbf{uˈjov} &  tʰətʰ\textbf{uˈjov} &  ɑɾ\textbf{uˈjov}
			\\ \hline 
		\end{tabular}
	\end{table} 
	
	
	There is however dialectal variation and some degree of lexical variation (Table \ref{tab:vow vowel u inf dialect}). In Eastern Armenian, the more typical rule is that /u/ becomes [v] (\textit{u}-devocalization) before V-initial inflection. The [v] can then trigger schwa epenthesis. However even in Eastern Armenian, there is lexical variation in that some words use devocalization, some use glide epenthesis, and some can do both. Data is from English Wiktionary via VP. In contrast in Western Armenian, glide epenthesis is the norm. 
	
	
	
	\begin{table}[H]
		\centering
		\caption{Dialectal variation in how /u/ appears before  V-initial inflection}
		\label{tab:vow vowel u inf dialect}
		\resizebox{\textwidth}{!}{%
			\begin{tabular}{|l|ll | ll|ll| }
				\hline 
				&        `flea' & \armenian{լու} & `pilot' & \armenian{օդաչու}  & `cat' & \armenian{կատու} 
				\\ 
				\hline  
				& Western & Eastern& Western & Eastern& Western & Eastern
				\\
				& ˈl\textbf{u}& ˈl\textbf{u} & otʰɑˈt͡ʃ\textbf{u} & otʰɑˈt͡ʃʰ\textbf{u} 
				& ɡɑˈd\textbf{u} & kɑˈt\textbf{u}
				\\
				{\gendat} \textit{-i} & l\textbf{uˈji}& l\textbf{əˈvi} & otʰɑt͡ʃ\textbf{uˈji} & otʰɑt͡ʃʰ\textbf{uˈʰi}
				& ɡɑd\textbf{uˈji} & kɑˈt\textbf{uˈʰi}
				\\
				& & & &  &&  kɑtˈ\textbf{vi}
				\\
				{\abl} \textit{-e} (WA), \textit{-it͡s} (EA) & l\textbf{uˈje} & l\textbf{əˈvit͡sʰ}& otʰɑt͡ʃ\textbf{uˈjit͡s} & otʰɑt͡ʃʰ\textbf{uˈʰit͡sʰ}
				& ɡɑd\textbf{uˈje} & kɑˈt\textbf{uˈʰit͡s}
				\\
				& & & &  &&  kɑtˈ\textbf{vit͡sʰ}
				\\
				{\ins} \textit{-ov}&    l\textbf{uˈjov}& l\textbf{əˈvov}& otʰɑt͡ʃ\textbf{uˈjov} & otʰɑt͡ʃʰ\textbf{uˈʰov} 
				& ɡɑd\textbf{uˈjov} & kɑˈt\textbf{uˈʰov}
				\\
				& & & &  &&  kɑtˈ\textbf{vov}
				\\ \hline 
			\end{tabular}
	}\end{table} 
	
	Note for the above Eastern words [otʰɑt͡ʃʰu] and  [kɑtu], we transcribe the inflected forms with a glide [j]. However, VP informs us that the glide in the above examples feels weaker than a full glide, thus potentially a transient glide. 
	
	In Western,  the  \textit{u}-devocalization process is   common when a stem-final /u/ precedes a derivational suffix (\S\ref{section:syllable:VowelHiatus:Derived:U}), but uncommon when it precedes an inflectional suffix. For example, [ɡɑdu] `cat' is inflected as dative [ɡɑdu-ji]  but is derived to [ɡɑdv-ɑɡɑn] `feline' \armenian{կատուական}. 
	
	However, there are some words like the word  `tongue' [lezu]    that show optionality in whether use devocalization or glide epenthesis (Table \ref{tab:vow vowel u inf lexical}).  In Western Armenian at least, the devocalized version   has a more archaic connotation. 
	
	\begin{table}[H]
		\centering
		\caption{Lexical  variation in how /u/ appears before  V-initial inflection}
		\label{tab:vow vowel u inf lexical}
		\begin{tabular}{|l|ll |ll|    }
			\hline 
			&       \multicolumn{2}{l}{ `tongue/language'} & \armenian{լեզու}&  
			\\ 
			\hline  
			& Western && Eastern  &
			\\
			& leˈz\textbf{u}&& leˈz\textbf{u}&
			\\
			{\gendat} \textit{-i} & lez\textbf{uˈji}& lezˈ\textbf{vi}& lez\textbf{uˈji}& lezˈ\textbf{vi}
			\\
			{\abl} \textit{-e} (WA), \textit{-it͡s} (EA)  & lez\textbf{uˈje}& lezˈ\textbf{ve}& lez\textbf{uˈjit͡sʰ}& lezˈ\textbf{vit͡sʰ}
			\\
			{\ins} \textit{-ov}  & lez\textbf{uˈjov}& lezˈ\textbf{vov}& lez\textbf{uˈjov}& lezˈ\textbf{vov}
			\\ \hline 
		\end{tabular}
	\end{table} 
	
	The devoicalized form is impressionistically more common in high-frequency collocations or sayings or proverbs (\ref{ex:lezu v idiom}). 
	
	\begin{exe}
		\ex \gll \textbf{lezv-i-tʰ} dɑɡ pʰɑn-mə ɡ-ɑ-$\emptyset$ \\
		tongue-{\gen}-{\possSsg} under thing-{\indf} exist-{\thgloss}-3{\sg}
		\\
		\trans Literally: `There is something under your thing'
		\\
		Meaning: `There is something that you want to say but you're having trouble  it.' \\
		\label{ex:lezu v idiom}
		\armenian{Լեզուիդ տակ բան մը կայ:}
	\end{exe}
	
	
	
	\subsection{Vowel hiatus repair in irregular inflection and  verbal inflection}\label{section:syllable:VowelHiatus:Irregular}
	When a vowel-final stem gets a vowel-initial regular inflectional suffix, then the normal way to resolve the vowel sequence is to use glide epenthesis (\S\ref{section:syllable:VowelHiatus:Inf}). However, irregular inflection uses different strategies. 
	
	
	There are a handful of irregular words that end in /ɑ/: [dəʁɑ] `boy' \armenian{տղայ} and [d͡ʒɑmpʰɑ] `road' \armenian{ճամբայ}. In standard speech, the vowel is irregularly deleted before irregular inflectional suffixes: [dəʁ-ot͡s] `boy-{\pl}.{\gen}' \armenian{տղոց} and [d͡ʒɑmpʰ-u] `road-{\gen}' \armenian{ճամբու}. But in colloquial speech, we can use regular inflectional suffix and potentially glide epenthesis: [dəʁɑ-neɾ-u] `boy-{\pl}-{\gen}' \armenian{տղաներու} and [d͡ʒɑmpʰɑj-i]  `road-{\gen}' \armenian{ճամբայի}. Fuller declension classes for these irregular words are found in  \textcolor{red}{cite chapter declesion irregular}. 
	
	For the irregular word [d͡ʒɑmpʰɑ] `road', it likewise has derived words where the [ɑ] is deleted: [d͡ʒɑmpʰ-oɾtʰ] `traveller' \armenian{ճամբորդ} or [d͡ʒɑmpʰ-el] `to send away' \armenian{ճամբել}. 
	
	The country-naming suffix \textit{-jɑ} \armenian{-իա} follows its own declension class: [əspɑn-jɑ] `Spain' \armenian{Սաանիա} but [əs]. In standard speech, the [ɑ] is deleted and replaced by a dative-genitive suffix \textit{-o}: [əspɑn-jo] `Spain-{\gen}' \armenian{Սպանիոյ}. In colloquial speech, we can use the regular suffix \textit{-i} and take a glide: [əspɑn-jɑj-i] \armenian{Սպանիայի}.  This declension class is discussed more in \textcolor{red}{cite chapter declesion irregular}. 
	
	Some words of time take an irregular dative-genitive suffix \textit{-vɑ}. This suffix deletes the stem-final vowel or changes it to a schwa. For example, [ɑɾdu] `morning' \armenian{արտու} but [ɑɾdə-vɑ] \armenian{արտուայ}, and [dɑɾi] `year' \armenian{տարի} but [dɑɾ-vɑ] \armenian{տարուայ}. This is discussed more in \textcolor{red}{cite chapter declesion irregular}. 
	
	Among verbs, we find very few cases of clear vowel hiatus repair. The clearest is when the past suffix \textit{-i} is added after the theme vowels [-e-, -ɑ-] and triggers a glide. For example for [-e-], [jeɾkʰ-e-n] `sing-{\thgloss}-3{\pl}' meaning `they sing' \armenian{երգեն}, and its past form [jeɾkʰ-ej-i-n] `sing-{\thgloss}-{\pst}-3{\pl}' meaning `if they were singing' \armenian{երգէին}. Similarly for the theme vowel [ɑ]: [ɡɑɾtʰ-ɑ-n] `read-{\thgloss}-3{\pl}' meaning `they read' \armenian{կարդան}, and its past form [ɡɑɾtʰ-ɑj-i-n] `read-{\thgloss}-{\pst}-3{\pl}' meaning `if they were reading' \armenian{կարդային}.
	
	An unclear case involves irregular past inflection. For irregular words in the past  perfective, the past markers [ɑ,i] never follow a theme.  For example, [pʰeɾ-e-n] `bring-{\thgloss}-3{\pl}' meaning `they bring'  \armenian{բերեն}  but [pʰəɾ-i-n] `bring-{\pst}-3{\pl}' meaning `they brought' \armenian{բերին}.
	It is unclear if this is because the [-i] suffix deletes theme vowels, or if the morphology just never  added the theme vowel here. Irregular verbal past marking is discussed more in \textcolor{red}{cite chapter verb irregular}.
	\subsection{Vowel hiatus repair before clitics}\label{section:syllable:VowelHiatus:Clitic}
	Armenian has few V-initial clitics (Table \ref{tab:vow vowel clitic}). These are the `also' clitic [ɑl] \armenian{ալ}, and the copula [e] \armenian{է} that can take on various inflected forms like present 1SG [em] \armenian{եմ}, past 1SG [eji] \armenian{էի}, and so on. When these clitics are added after a V-final word, we get glide epenthesis. 
	
	
	\begin{table}[H]
		\centering
		\caption{Glide epenthesis between before V-initial clitics}
		\label{tab:vow vowel clitic}
		\begin{tabular}{|l|lll |ll| }
			\hline 
			&      &   +/ɑl/ `also'  \armenian{ալ}  & +/e/  `is' \armenian{է} & & 
			\\
			\hline  
			/ɑ/ & vəˈɡ\textbf{ɑ}  & vəˈɡ\textbf{ɑjɑl} & vəˈɡ\textbf{ɑje} & `witness'  & \armenian{վկայ}
			\\ 
			/e/ & kʰəˈv\textbf{e}  &  kʰəˈv\textbf{ejɑl} &  kʰəˈv\textbf{eje} & `vote'  & \armenian{քուէ}
			\\ 
			/i/ & leˈʁ\textbf{i}  &  leˈʁ\textbf{ijɑl} &  leˈʁ\textbf{ije} & `bitter'  & \armenian{լեղի}
			\\ 
			/ə/ & ˈv\textbf{e}  &  ˈv\textbf{əjɑl} &    ˈv\textbf{əje} & `/v/ sound'  & \armenian{վը}
			\\ 
			/o/ & kiʰˈl\textbf{o}  &   kiʰˈl\textbf{ojɑl} &    kiʰˈl\textbf{oje} & `kilogram'  & \armenian{քիլոյ}
			\\ 
			/o/ & ɡəˈd͡z\textbf{u}  &   ɡəˈd͡z\textbf{ujɑl} &   ɡəˈd͡z\textbf{uje} & `spicy'  & \armenian{կծու}
			\\ \hline 
		\end{tabular}
	\end{table} 
	
	
	
	
	
	
	
	One exception is the indefinite suffix [mə] \armenian{մը}. Before a clitic, it changes its form to [mən]. See \textcolor{red}{indefinite floating} But in more archaic speech, the [mə] is reduced to [m].
	\textcolor{red}{find reference from an old grammar bcz impossible to find online [me] forms }
	
	
	\begin{exe}
		\ex \glll kʰit͡ʃ-\textbf{mən=ɑl} ɑnuʃeʁen  (modern: HD) \\ 
		kʰit͡ʃ-\textbf{m=ɑl} ɑnuʃeʁen  (archaic)\\ 
		little-{\indf}=also sweets \\
		\trans `And a  bit of sweets.'\footnote{For the [m-ɑl] form, the archaic sentence was taken from the title of an article from 1900; URL: \url{https://arar.sci.am/dlibra/publication/74562/edition/67396/}.  }
		\\ \armenian{Քիչ մըն ալ}
		\\ \armenian{Քիջ մ՚ալ անուշեղէն}
	\end{exe}
	
	\subsection{Overapplication of vowel hiatus repair}\label{section:syllable:VowelHiatus:Over}
	We went through various    rules for vowel hiatus repair like vowel deletion or glide epenthesis. In general, such rules apply when two vowels become next to each because of the morphology. But there are corners of the grammar where these rules overapply even though there's no vowel sequence. The overapplication is because of a mix of morphological and diachronic factors. The  corners are passives (\S\ref{section:syllable:VowelHiatus:Over:Passive}) and /j/-initial suffixes (\S\ref{section:syllable:VowelHiatus:Over:Ja}). 
	
	\subsubsection{Overapplication of glide epenthesis in passives}\label{section:syllable:VowelHiatus:Over:Passive}
	
	For passives, the passive suffix is a consonant /v/. But after /ɑ/-final roots, a glide is epenthesized and written in the orthography. For example, from the root   <vga\textbf{y}> [vəɡɑ] `witness' \armenian{վկայ}, we can form a passive verb <vga\textbf{yow}il> [vəɡɑ\textbf{j-v}-i-l] `to be testified' \armenian{վկայուիլ}. The reason this glide appears is because of two reasons, one diachronic and one synchronic. 
	
	The diachronic reason is that the modern passive suffix /v/ was historically a vowel */-u-/ in Classical Armenian. Thus vowel hiatus was created in Classical Armenian, and thus we got a glide. The synchronic reason is that the change from a vocalic morpheme */-u-/ to a consonantal morpheme [-v-] was accompanied by a huge reanalysis of passive morphophonology. Briefly put, passive stems try to resemble active stems, even if that means a glide is unnecessarily added: <vga\textbf{y}el> [vəɡɑ\textbf{j}-e-l] `to testify' \armenian{վկայել}. The need for resemblances affects many other morphophonological changes such as vowel reduction.  This passive problem is discussed in depth in \textcolor{red}{passive phonology chapter}. 
	
	\subsubsection{Vowel hiatus repair before /j/-initial suffixes} \label{section:syllable:VowelHiatus:Over:Ja}
	
	
	For /j/-initial suffixes, these suffixes are pronounced with an initial glide, but they are spelled with an initial vowel (Table \ref{tab:overapply j vowel hiatus}). For example, the suffix /-jɑ/ is spelled as <eay> in the word <aɣiws\textbf{eay}> [ɑʁʏs-\textbf{jɑ}]  `made of brick'  \armenian{աղիւսեայ}, derived from the word <aɣiws> [ɑʁʏs] `brick' \armenian{աղիւս}.  When added to a vowel-final stem, the vowel undergoes alternations as if the /j/ were a vowel. Such changes include deletion or de-vocalization. 
	
	\begin{table}[H]
		\centering
		\caption{Overapplication of vowel hiatus repair    before /j/-initial suffixes}
		\label{tab:overapply j vowel hiatus}
		\resizebox{\textwidth}{!}{%
			\begin{tabular}{|l|llll| }
				\hline 
				/ɑ-j/  &<giwt'er\textbf{ay}> & ɡʏtʰeˈɾ\textbf{ɑ} & `Cythera'  & \armenian{Կիւթերա}      
				\\
				$\rightarrow$[j] & <giwt'er\textbf{e}an>   & ɡʏtʰeɾ\textbf{-ˈj}ɑn & `Cytherean'  & \armenian{կիւթերեան}         \\
				\hline
				/u-j/  &<brd\textbf{ow}> & bəɾˈd\textbf{u} & `payprus'  & \armenian{պրտու}     
				\\
				$\rightarrow$[vj] & <brd\textbf{owe}ay>   & bəɾd\textbf{ˈv-j}ɑ & `made of papyrus'    &\armenian{պրտուեայ}       \\
				
				&<pazmalez\textbf{ow}> & pʰɑzmɑleˈz\textbf{u} & `polyglot'  & \armenian{բազմալեզու}      
				\\
				$\rightarrow$[vj] & <pazmalez\textbf{owe}an>   & pʰɑzmɑlez\textbf{ˈv-j}ɑn & `polyglot'  & \armenian{բազմալեզուեան}        \\
				\hline
				/i-j/  &<kin\textbf{i}> & kʰiˈn\textbf{i} & `wine'  & \armenian{գինի}     
				\\
				$\rightarrow$[j] & <kin\textbf{e}ag>   & kʰin\textbf{-ˈj}ɑɡ & `sour wine'  & \armenian{գինեակ}       \\
				&<əndan\textbf{i}> & əndɑˈn\textbf{i} & `familiar'  & \armenian{ընտանի}     
				\\
				$\rightarrow$[j] & <əndan\textbf{e}ōk'>   & əndɑn\textbf{-ˈj}okʰ & `familiarly'  & \armenian{ընտանեօք}       \\
				\hline
				/e-j/  &<ap'rotid\textbf{ē}> & ɑpʰɾotʰiˈd\textbf{e} & `Aphrodite'  & \armenian{Ափրոդիտէ}     
				\\
				$\rightarrow$[j] & <ap'rotid\textbf{e}an>   & ɑpʰɾotʰid\textbf{-ˈj}ɑn & `venereal'  & \armenian{ափրոդիտեան}       \\
				\hline
				
				
			\end{tabular}
	}\end{table}
	
	No such changes typically happen after /ɑ/ or /e/ (Table \ref{tab:overapply j glide epenthesis a e}). This is because, as explained in Section \S\ref{section:syllable:VowelHiatus:Derived}, such vowels usually take glide epenthesis before vowel-initial suffixes. The glide of the suffix here does the job of the epenthetic glide. 
	
	\begin{table}[H]
		\centering
		\caption{No application of of vowel hiatus repair   between /ɑ,e/ and   a  /j/-initial suffix}
		\label{tab:overapply j glide epenthesis a e}
		\begin{tabular}{|l|llll| }
			\hline 
			/ɑ-j/&<ark'\textbf{ay}> & ɑɾˈkʰ\textbf{ɑ} & `king'  & \armenian{արքայ} 
			\\
			$\rightarrow$[ɑj]  & <ark\textbf{aye}an>   & ɑɾkʰ\textbf{ɑ-ˈj}ɑn & `royal'  &  \armenian{արքայեան}     \\
			&<norənd͡z\textbf{ay}> & noɾənˈd͡z\textbf{ɑ} & `novice'  & \armenian{նորընծայ} 
			\\
			$\rightarrow$[ɑj]  & <norənd͡z\textbf{aye}al>   & noɾənd͡z\textbf{ɑ-ˈj}ɑl & `novice'  &  \armenian{նորընծայեալ}     \\
			&<yut\textbf{ay}> & huˈtʰ\textbf{ɑ} & `Judas'  & \armenian{Յուդայ} 
			\\
			$\rightarrow$[ɑj]  & <yut\textbf{aye}an>   & hutʰ\textbf{ɑ-ˈj}ɑn & `Judaical'  &  \armenian{յուդայեան}     \\
			
			\hline 
			
			/e-j/  &<hiwl\textbf{ē}> & hʏˈl\textbf{e} & `atom'  & \armenian{հիւլէ}     
			\\
			$\rightarrow$[ej] & <eōtn'hiwl\textbf{ēe}an>   & jotʰnəhʏl\textbf{e-ˈj}ɑn & `heptatomic'  & \armenian{եօթնհիւլէեան}        \\
			\hline
			
			
			
		\end{tabular}
	\end{table}
	
	
	We have found many words where   a /j/-initial suffix  was added to a /i/-final word, and then the /i/ is deleted (Table \ref{tab:overapply j vowel hiatus i deletion}). 
	
	
	\begin{table}[H]
		\centering
		\caption{Deletion of /i/ before a /j/-initial suffix}
		\label{tab:overapply j vowel hiatus i deletion}
		\resizebox{\textwidth}{!}{%
			\begin{tabular}{|llll|llll|  }
				\hline 
				<ʃok\textbf{i}> & ʃoˈkʰ\textbf{i} & `vapor' & \armenian{շոգի}
				& 
				<badan\textbf{i}> & bɑdɑˈn\textbf{i} & `adolescent' & \armenian{պատանի}
				\\
				<ʃok\textbf{e}ag> & ʃokʰ\textbf{-ˈj}ɑɡ & `light vapor' & \armenian{շոգեակ}
				&
				<badan\textbf{e}ag> & bɑdɑn\textbf{-ˈj}ɑɡ & `adolescent' & \armenian{պատանեակ}
				\\
				\hline 
				<yoṙ\textbf{i}> & hoˈɾ\textbf{i} & `evil' & \armenian{յոռի}
				&
				<dērun\textbf{i}> & deɾuˈn\textbf{i} & `dominical' & \armenian{տէրունի}
				\\
				<yoṙ\textbf{e}ag> & hoɾ\textbf{-ˈj}ɑɡ & `wicked' & \armenian{յոռեակ}
				&
				<dērun\textbf{e}an> & deɾun\textbf{-ˈj}ɑn & `dominical' & \armenian{տէրունեան}
				\\
				\hline 
				<d͡ʒd͡ʒ\textbf{i}> & d͡ʒəˈd͡ʒ\textbf{i} & `worm' & \armenian{ճճի}
				&
				<madan\textbf{i}> & mɑdɑˈn\textbf{i} & `ring' & \armenian{մատանի}
				\\
				<d͡ʒd͡ʒ\textbf{e}ag> & d͡ʒəd͡ʒ\textbf{-ˈj}ɑɡ & `animalcule' & \armenian{ճճեակ}
				&
				<madan\textbf{e}ag> & mɑdɑn\textbf{-ˈj}ɑɡ & `small ring' & \armenian{մատանեակ}
				\\
				\hline 
				<ort\textbf{i}> & voɾˈtʰ\textbf{i} & `son' & \armenian{որդի}
				&
				<gɣz\textbf{i}> & ɡəʁˈz\textbf{i} & `island' & \armenian{կղզի}
				\\
				<ort\textbf{e}ag> & voɾtʰ\textbf{-ˈj}ɑɡ & `little child' & \armenian{որդեակ}
				&
				<gɣz\textbf{e}ag> & ɡəʁ\textbf{ˈz-j}ɑɡ & `islet' & \armenian{կղզեակ}
				\\
				\hline
				
			\end{tabular}
	}\end{table}
	
	
	For /i/-final roots, many but not all of the above suffixes are [jɑɡ], such as [ʃokʰjɑɡ] `light vapor'. This suffix could be reanalyzed as just the root plus the diminutive suffix /ɑɡ/. The underlying form could thus be either /ʃokʰi-jɑɡ/ or /ʃokʰi-ɑɡ/. With the latter form, the /i/ becomes [j] before a vowel. Such a reanalysis however does not work for words that end  in sequences like [jɑn]. 
	
	
	
	
	
	As is clear, for some reason, the /j/-initial suffixes act like vowel-initial suffixes and they trigger vowel hiatus rules like /i/-deletion.  But why do they do this? The answer is likely diachronic. The fact these suffixes are spelled with an <e> suggests that in Classical Armenian, these /j/-initial suffixes were pronounced as */e/-initial suffixes. They would then transparently trigger vowel hiatus repair rules. Over time, the ancient */e/ became a modern /j/. This sound change required Armenian speakers to reanalyze these /j/-initial suffixes as arbitrarily requiring that the preceding vowel is deleted, de-vocalized, or be left unchanged.  
	
	One could perhaps argue that the data above suggests that modern speakers still treat the /j/-initial suffixes as underlyingly /e/-initial. That is, perhaps a suffix like [-jɑɡ] is actually /-eaɡ/. We don't think such an analysis is realistic however as we explain below. 
	
	
	In the \citeauthor{kouyoumdjian-1970-DictionaryArmenianEnglish} dictionary, we found around 1430 words which end in a /j/-initial suffix. Thus these suffixes are relatively common and productive. However, we only found 27 words such that a) the word had such a suffix, and b) the word was clearly derived from a vowel-final stem. Condition (b) is important. Such derived words are  all extremely-low frequency words, and they are often high-register technical words or liturgical words.  The Armenian child is unlikely to be systematically exposed to such derived words in their formative years.  Based on this statistical skew, we suspect that the Armenian child will at first treat the /j/-initial suffixes like [-jɑɡ] as truly just underlyingly /-jɑɡ/, without any abstraction. 
	
	In HD's impression, knowing how to morphologically form such derived words and how to pronounce them is something that the child is exposed to later in life at school or church. The child then learns that these /j/-initial suffixes for unknown reasons trigger vowel hiatus repair on the preceding vowel. 
	
	
	\subsection{Diachronic problems in glide epenthesis}\label{section:syllable:VowelHiatus:Diachrony}
	For vowel hiatus,  glide epenthesis is a common rule. However, glide epenthesis has an intricate connection with the history of glide deletion in Armenian.
	
	In modern Armenian, there are many native words that are pronounced with a final [ɑ] or [o], but that are spelled with a final glide: <ark'ay> [ɑɾkʰɑ] `king' \armenian{արքայ}  or  <xaʃoy> [χɑʃo]   `broth' \armenian{խաշոյ}. Such final glides are silent letters.  In contrast, relatively few words are spelled with a final <a> \armenian{ա}, <ō> \armenian{օ}, or <o> \armenian{ո}. For example,  <fransa> [fəɾɑnsɑ] `France' \armenian{Ֆրանսա}, [d͡zo] <d͡zo> `an interjection'  \armenian{ծօ},  <vet'o> [vetʰo] `veto' \armenian{վեթո}. Such words are usually either loanwords, interjections,  or have a country-naming suffix \textit{-jɑ}. 
	
	The reason for this state of affairs is because in Classical Armenian, these silent letters were pronounced. In the change from Classical Armenian to Modern Armenian, the final glide was deleted from polysyllabic words, while it was kept in monosyllabic words, or in compounds where the second stem was monosyllabic. In Table \ref{tab:diachrony ay oy}, we transcribe the Classical Words using the `pronunciation rules' that modern  Western Armenian speakers would use to read   Classical texts. We ultimately don't know how Classical Armenian was exactly pronounced. 
	
	\begin{table}[H]
		\centering
		\caption{Diachronic loss of final glides for polysyllabic words}
		\label{tab:diachrony ay oy}
		\resizebox{\textwidth}{!}{%
			\begin{tabular}{|l|ll|lll|}
				\hline   & `king'     & `broth'  & `Armenian' & `Latinized Armenian'  & `ram'\\
				& <ark'ay>  & <xaʃoy>  & <hay> & <ladinahay>  & <xoy> 
				\\ \hline 
				Classical         & [ɑɾˈkʰɑj] & [χɑˈʃoj] & [ˈhɑj] & [lɑdin-ɑ-ˈhɑj] & [ˈχoj]
				\\
				Modern & [ɑɾˈkʰɑ]& [χɑˈʃo]& [ˈhɑj] & [lɑdin-ɑ-ˈhɑj] & [ˈχoj]
				\\
				& \armenian{արքայ} & \armenian{խաշոյ} & \armenian{հայ} & \armenian{լատինահայ} & \armenian{խոյ}
				\\ \hline 
			\end{tabular}
	}\end{table}
	
	In contrast, Classical Armenian seems to not have many words that are spelled with a final <a> or <ō>. For <a>, one of the few examples we found on the Calfa dictionary was <alk'imia> [ɑlkʰimjɑ] `alchemy' \armenian{ալքիմիա}, an obvious loanword.\footnote{\url{https://dictionary.calfa.fr/entries/}}. For <ō>, this letter did not exist in Classical Armenian. The final <o> \armenian{ո} is also pretty rare in Classical: <ayo> [ɑjo] `yes' \armenian{այո}. 
	
	When a modern Western speaker uses a word like [ɑɾkʰɑ] `king',  the final silent glide <y> is  present in the spelling <ark'ay> \armenian{արքայ} but it not present in the lexical representation /ɑɾkʰɑ/. When a vowel-initial suffix is added, a glide is inserted: [ɑɾkʰɑj-i] `king-{\gen}' \armenian{արքայի}.    This complicated diachronic sequence of deleting a final glide and then inserting an epenthetic glide constitutes a type of rule inversion \citep{Vennemann-1972-RuleInversion}, which is common in historical linguistics. 
	
	Given these historical facts, some argue that words like [ɑɾkʰɑ] <ark'ay> `king'  are underlyingly glide-final /ɑɾkʰɑj/ in order to explain why the glide surfaces before vowels: [ɑɾkɑji] ({\gendat}) \textcolor{red}{cite vaux}.  But this psychologically unrealistic and unneeded for various reasons.  
	
	First, words that are spelled without a glide like [jevɾobɑ] <ewroba> `Europe' show the same epenthetic glide in inflection: [jevɾobɑji] ({\gendat}). They may differ before derivational morphology, but such variation is for independent reasons (\S\ref{section:syllable:VowelHiatus:Derived:A}). 
	
	Second, the \citeauthor{kouyoumdjian-1970-DictionaryArmenianEnglish} dictionary lists 294 words that end in an orthography <a> \armenian{ա} and pronounced as [ɑ], and 708 words that end in an orthographic <ay> \armenian{այ} and pronounced as [ɑ]. 
	
	Third, the reformed orthography does not spell these final deleted glides: reformed spelling <ark'a> \armenian{արքա} vs. traditional <ark'ay> \armenian{արքայ}. The orthographic reform was likely guided by the reformers intuition that these silent glides are phonologically absent.  
	
	A related fourth point is that some loanwords are variably spelled with a glide (like  a native word), or without a glide (like a typical loanword).  For example, [zeɾo] `zero' can be <zēroy> \armenian{զէրոյ} or <zērō> \armenian{զէրօ}.  There's no reason to suspect that these alternating spellings mean there's any difference in their phonology.   
	
	Fifth, this unwritten glide is completely invisible   to other inflectional suffixes. The definite suffix is /-ə/ after consonants, but /-n/ after vowels. Words like [ɑɾkʰɑ] are transparently treated as vowel-final and take the definite form [ɑɾkʰɑ-n]. The orthography deletes this glide before consonant-initial inflectional suffixes like the definite <ark'an> \armenian{արքան} and  the plural [ɑɾkʰɑ-neɾ] <ark'aner> \armenian{արքաներ}. 
	
	
	
	
	Thus, the Armenian child really has no reason to treat this glide as anything other than an inserted glide. The have no reason to postulate that most cases of final [ɑ] are underlyingly /ɑj/ with a rule of deleting final glides. 
	
	
	Note that this diachronic-synchronic problem       has an interesting interaction with   morphology. Consider the word <bargay> [bɑɾɡɑ] `Fatal Sister' \armenian{Պարկայ}, a Greek mythological character. The plural marker in Classical was \textit{-kʰ} \armenian{-ք}. The modern language uses this suffix as a nominalizer instead. When these nominalizer is added after a silent glide, speakers recognize the `learned' or `archaic' nature of this rule, and would pronounce the glide because they're reading the silent glide: <bargayk'> [bɑɾɡɑjkʰ] `The Fate sisters' \armenian{Պարկայք}.  



 