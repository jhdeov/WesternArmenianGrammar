\section{Old: todo: citations from previous work}

\todo{use these}

\subsection{syllable structure}
\footnote{There are a few known exceptions to the ban on complex onsets: consonant-glide clusters \citep[81]{Vaux-1998-ArmenianPhono}, borrowed proper names \citep{Baronian-2017-TwoProblemsArmenianPhono}, word-initial obstruent-rhotic clusters \citep{Tokhmɑkʰyan-1988-ConnectingArmenianSounds}, and word-initial obstruent clusters that are aspirated or fricated \citep{Hovakimyan-2016-EasternArmenianClusters}. The main exception to falling-sonority complex codas are stem-final appendixes \citep{VauxWolf-2009-Appendix}. This is discussed in S\ref{disssection: reduction: pstem  elsewhere: appendix}.} 

For lists of possible complex codas, see \citet[20]{Fairbanks-1948-PhonologyMorphoWestern} and \citet[23]{Vaux-1998-ArmenianPhono}. 

\subsection{Stem-final high vowels and vowel hiatus in Eastern Armenian}\label{disssection: reduction: pstem  elsewhere: vowel hiatus EA}

\todo{incorporate  For final /i/ in monosyllabic roots, both dialects resolve hiatus    with  glide-epenthesis in derivation and inflection: \textit{textipa{\t{dz}i}} `horse'; \textit{textipa{\t{dz}ij-avor}} `horseman'; \textit{textipa{\t{dz}ij-ov}} `horse\textsc{-inst} (EArm)'   Some earlier grammars  document optional glide formation in EArm: \textit{textipa{\t{dz}j-ov}} \citep[20]{Johnson-1954-EastArmGrammar}. For final /u/, monosyllabic and polysyllabic roots pattern together.} 

For final /i/,  both dialects generally use deletion in derivation. In   inflection, hiatus is repaired with glide epenthesis in WArm but with either  deletion (\ref{dissexample: hiatus ea i inf: delete}) or glide epenthesis (\ref{dissexample: hiatus ea i inf: epenthesis}) in EArm \cite[100ff]{Margaryan-1997-ArmenianPhonology}, though glide-epenthesis is increasingly preferred  \cite[140,197]{Sargsyan-1987-DoubletsNouns}. Just like in WArm (S\ref{disssection: reduction: strata elsewhere in WA: stem edge and hiatus}),   other hiatus repair rules like coalescence or glide formation can apply in derivation but {not} inflection. 



\begin{exe}
	
	\ex 
	\begin{multicols}{2}
		\begin{xlist}
			
			\ex \makebox[2.5cm][l]{WArm}\makebox[2.5cm][l]{EArm}\label{dissexample: hiatus ea i inf: delete}\\
			\makebox[2.5cm][l]{\textipa{jegeGe\t{ts}\'i}}\makebox[2.5cm][l]{\textipa{jekeGe\t{ts}ʰ\'i}}`church'\\
			\makebox[2.5cm][l]{\textipa{jegeGe\t{ts}-ɑɡ\'ɑn}}\makebox[2.5cm][l]{\textipa{jekeGe\t{ts}ʰ-ɑk\'ɑn}}`ecclesiastic'\\
			\makebox[2.5cm][l]{\textipa{jegeGe\t{ts}ij-\'ov}}\makebox[2.5cm][l]{\textipa{jekeGe\t{ts}ʰ-\'ov}}`\text{church-\textsc{inst}'}
			
			\ex \makebox[2cm][l]{WArm}\makebox[2cm][l]{EArm}\label{dissexample: hiatus ea i inf: epenthesis}\\
			\makebox[2cm][l]{\textipa{madɑn\'i}}\makebox[2cm][l]{\textipa{matɑn\'i}}`ring'\\
			\makebox[2cm][l]{\textipa{madan-\'el}}\makebox[2cm][l]{\textipa{matan-\'el}}`to seal'\\
			\makebox[2cm][l]{\textipa{madanij-\'ov}}\makebox[2cm][l]{\textipa{matanij-\'ov}}`ring-\textsc{inst}'
		\end{xlist}
		
	\end{multicols}
	
\end{exe}

Overapplication is also found  for stem-final  /u/. Both dialects generally use glide fortition to [v] in derivation.  In inflection, WArm uses glide-epenthesis while EArm also uses glide fortition for some  (\ref{dissexample: hiatus ea u inf: v}) but not all roots (\ref{dissexample: hiatus ea u inf: epenthesis}) (\citeauthor{Minassian-1980-EastArmenianGrammar} \citeyear[95]{Minassian-1980-EastArmenianGrammar}; \citeauthor{Margaryan-1997-ArmenianPhonology} \citeyear[103-7]{Margaryan-1997-ArmenianPhonology}).  




\begin{exe}
	
	\ex 
	\begin{multicols}{2}
		\begin{xlist}
			
			\ex \makebox[2.5cm][l]{WArm}\makebox[2.5cm][l]{EArm}\label{dissexample: hiatus ea u inf: v}\\
			\makebox[2.5cm][l]{\textipa{t@t\'u}}\makebox[2.5cm][l]{\textipa{tʰ@tʰ\'u}}`sour'\\
			\makebox[2.5cm][l]{\textipa{h\'ɑm}}\makebox[2.5cm][l]{\textipa{h\'ɑm}}`taste'\\
			\makebox[2.5cm][l]{\textipa{t@tv-a-h\'ɑm}}\makebox[2.5cm][l]{\textipa{tʰ@tʰv-a-h\'ɑm}}`\text{sour-tasting}'\\
			\makebox[2.5cm][l]{\textipa{t@tuj-\'i}}\makebox[2.5cm][l]{\textipa{tʰ@tʰv-\'i}}`\text{sour-\textsc{gen}'}
			
			\ex \makebox[2cm][l]{WArm}\makebox[2cm][l]{EArm}\label{dissexample: hiatus ea u inf: epenthesis}\\
			\makebox[2cm][l]{\textipa{jerɡ\'u}}\makebox[2cm][l]{\textipa{jerk\'u}}`two'\\\\
			\makebox[2cm][l]{\textipa{jergv-ɑɡ\'ɑn}}\makebox[2cm][l]{\textipa{jerkv-ɑk\'ɑn}}`dual'\\
			\makebox[2cm][l]{\textipa{jerguj-\'i}}\makebox[2cm][l]{\textipa{jerkuj-i}}`two-\textsc{gen}'
		\end{xlist}
		
	\end{multicols}
	
\end{exe}




To summarize, some EArm roots show the overapplication of  some stem-level hiatus repair rules (deletion, glide fortition)  in pre-inflectional vowel hiatus. This is analogous to pre-inflectional DHR. I argue that this due to the PStem.  I argue that the PStem expands before V-initial inflection   because of prosodic well-formedness, either from resyllabification (for DHR) or because the PStem cannot end   with vowel hiatus. PStem expansion triggers certain stem-level rules in EArm. In both dialects, the PStem is also  the site of appendix incorporation.  In sum, the Prosodic Stem   has independent support in Armenian. % It is the site of ain both dialects. .% The question of why the PStem is used for DHR is  answered  next.




