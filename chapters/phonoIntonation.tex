\chapter{Prosodic phonology and intonation}\label{chapter:intonation}
\textcolor{red}{pics, check out lena borise work}

This chapter looks at the phonology of phrases and sentences.  As   descriptive tools, we use basic tree structures from prosodic phonology \citep{Selkirk-1986-DerivedDomains,Nespor-Vogel-1986-ProsodicPhon} and basic autosegmental-metrical ToBI annotation \citep{Pierrehumbert-1980-phonologyPhoneticsEnglishIntonation,ladd-2008-intonationalPhonology,jun-2007-prosodicTypologyPhonologyIntonationPhrasing}. 

For prosodic phonology, the main idea is when syntactic structures (words, phrases, clauses) are pronounced, their pronunciation  forms specific groupings called prosodic constituents. Such constituents are demarcated by various prosodic or intonational cues like stress, pause, final duration, and pitch. For example, a sentence like (\ref{example:intonation:overview:basic}) is made up of four words, which are each pronounced as a separate prosodic word (with stress). The two syntactic phrases form two prosodic phrases $\phi$, each with phrasal stress (bold). The sentence is one intonational phrase, and the most prominent word (underlined) carries nuclear stress. The sentence ends in a falling pitch $\searrow$. 

\begin{exe}
	\ex \begin{xlist}
		\ex \glll (Adj \textbf{N})$_\phi$ (\underline{\textbf{N}} O)$_\phi$ $\searrow$\\
		ɡɑɾˈmir ɡɑˈ\textbf{du-n} \underline{bɑˈ\textbf{niɾ}} ɡeˈɾ-ɑ-v $\searrow$ \\
		red cat-{\defgloss} cheese eat.{\aorperf}-{\pst}-3{\sg} \\
		\trans `The red cat ate cheese.'  \label{example:intonation:overview:basic}
		\\ \armenian{Կարմիր կատուն պանիր կերաւ։}
	\end{xlist}
\end{exe}


In the above sentence, there is a clear isomorphism or match between the syntactic structure and the prosodic/phonological structure. But mismatches can occur in special circumstances. The formation of prosodic words was discussed in more depth in \S\ref{section:stress:regular:domain} in the context of lexical stress. But some words like the copula \textit{=e} `is' (\ref{example:intonation:overview:clitic}) are clitics because the syntax treats them as words, but the phonology treats them similarly to unstressed suffixes. As detailed in \S\ref{section:intonation:clitic}, such clitics are pronounced with the preceding word into a single larger prosodic word as demonstrated by their unambiguous resyllabification: [uˈɾɑ.χe].\footnote{Soviet Armenian literature generally doesn't designate any special category of clitics. Some grammars even seem to treat particles, pronouns, and clitics as being prosodic words \citep[25]{Soukyasyan-2004-ArmenianPhonology}. \textcolor{red}{re-read to check wich clitics he means}}


\begin{exe}
	\ex \glll (\textbf{N})$_\phi$ (\underline{\textbf{Adj}} Cop)$_\phi$ $\searrow$ \\
	ɡɑˈ\textbf{du-n} \underline{uˈ\textbf{ɾɑχ}} =e  $\searrow$ \\
	cat-{\defgloss} happy =is\\
	\trans `The   cat is happy.' \label{example:intonation:overview:clitic}
	\\ \armenian{Կատուն ուրախ է։}
\end{exe}

For clitics, resyllabification is quite unambiguous. In contrast, it seems there's no resyllabification across lexical words like [ɡɑ.du\underline{n.u}.ɾɑ...] from (\ref{example:intonation:overview:clitic}). Section \S\ref{section:intonation:resyll} goes over the resyllabification, with caveats on contradictory evidence. 

Moving to larger structures, Section \S\ref{section:intonation:phrase} discusses the formation of prosodic phrases from syntactic phrases. Usually an entire syntactic phrase forms a single prosodic phrase, but sometimes   a large syntactic phrase is broken up into smaller prosodic phrases. An interesting phenomenon is the location of phrasal stress in prosodic phrases (cf. \ref{example:intonation:overview:basic}). Noun phrases and adpositional phrases have phrasal stress on their last word (the noun or postposition), while verb phrases have stress on the preverbal word. Complications arise when verb phrases and noun phrases are combined together. 

The remaining sections look at the phonology of sentences. Section \S\ref{section:intonation:broadFocus} looks at the assignment of nuclear stress in typical broad-focus sentences. These are sentences where no individual word is more semantically important than the other. Briefly, nuclear stress is on the last prosodic phrase of the sentence. The last prosodic phrase is typically the verb phrase, and then nuclear stress is on the preverbal word. This word is often a direct object or indirect object.


Section \S\ref{section:intonation:focus} looks at the intonational of sentences. We look at declaratives, interrogatives, sentences with focused words, and negation. The different syntactic structures utilize different locations of nuclear stress,     post-focal deaccenting, and different distributions of sentence-final pitches. 

Section \S\ref{section:intonation:other} looks at other types of sentences that don't easily fit into the previous classification. Such sentences are subjunctive clauses, relative clauses with extraposition, imperatives, and vocatives. Each type of syntax has its own special phonological rules. 





\section{Clitics and particles}\label{section:intonation:clitic}
There are many   morphemes that can be (lazily) categorized as particles, as an umbrella term for anything that's not a noun, verb, adjective, adverb, or pronoun. These particles   are rather small in size (one syllable), and usually unstressed. Some of these can be easily classified as phonological clitics (\S\ref{section:intonation:clitic:overview}), while some seem to not be clitics (\S\ref{section:intonation:clitic:not}); another set of words seem to have been clitics in an earlier stage of the language but are no longer clitics now (\S\ref{section:intonation:clitic:old}). The stress behavior of such clitics was described earlier in \S\ref{section:stress:cliticc}. This section focuses on the more general prosodic structure of clitics. Note that we use underlining in this section for illustration/contrast, and not to mark nuclear stress. 

\todo{  Lists of such unstressable clitics can be found in \citet[78]{Margaryan-1997-ArmenianPhonology} and \citet[72]{Khanjian-2013-DissNegativeConcord}.}
\subsection{Particles that are clitics}\label{section:intonation:clitic:overview}



Armenian has a small set of clitics. Cross-linguistically, a clitic is a word that displays paradoxical behavior between the morphosyntax and the phonology \citep{Inkelas-1989-ProsodicLexicon,anderson-2005-aspectsTheoryClitic}. For the morphosyntax, a clitic is word-like in that it has some level of meaning that is word-like. But for the phonology, a clitic is suffix-like because it is pronounced as part of a larger word-like unit with an adjacent word. 

For example, when the English verb `is' is pronounced as [ɪz] and written as `is', then the verb acts as a non-clitic word. But when the verb is contracts as `s' [z], then the verb is now a clitic. The morphosyntax treats the verb as a word in both cases, but the phonology treats `is' as a suffix-like element when contracted. 

For Armenian, a small set of words are unambiguously treated as clitics. These elements or words are clitics because the morphosyntax treats them as having enough semantic content. They are spelled as separate words with a space, and speakers  simply `feel' that these are words. But the phonology treats them as suffix-like. The phonological properties are summarized in Table \ref{tab:phono property clitic}. 

\begin{table}[h]
	\centering
	\caption{Phonological properties of clitics} \label{tab:phono property clitic}
	\begin{tabular}{|l|llllll|}
		\hline        &Copula   & `also'   & `and'   & Q   & Prog.  & Subj.  \\
		& [e] &   [ɑl] &   [u] &   [mə] &    [ɡoɾ] &    [ne]    \\
		& \armenian{է} & \armenian{ալ} & \armenian{ու} & \armenian{մը} & \armenian{կոր} & \armenian{նէ}
		\\
		\hline 
		Unstressed? & \ding{51} & \ding{51} & \ding{51} & \ding{51} & \ding{51} & \ding{51}
		\\
		Resyllabified?  & \ding{51} & \ding{51}  & \ding{51}/\ding{55} & & & 
		\\
		Affects the definite?  & \ding{51} & \ding{51}  & \ding{51}/\ding{55} & & & 
		\\
		Can devoice? &  &  & & \ding{51} &    & 
		\\
		Can be   initial? & \ding{55} & \ding{55}  & \ding{51}    & \ding{55} & \ding{55}  & \ding{55} 
		\\ \hline 
	\end{tabular}
	
\end{table}

Note that copula can range from being just a single vowel as in the present 3SG [e], to having a coda [en] (present 3PL) or being bisyllabic [ejin] (past 3PL). All these inflected forms of the copula behave the same and are clitics. See \textcolor{red}{auxilialry chapter} for full paradigms of the copula/auxiliary. 


The first property is stress. In a typical situation, these elements are not stressed. They lean onto the preceding word which has stress. We show only one example here with the progressive [ɡoɾ] (\ref{clitic stress parameter in one clitic}), but we went through the stress properties of each of the above morphemes in \S\ref{section:stress:cliticc:one}. Note that in some clitic combinations, we can get stress on one of the clitics (\ref{clitic stress parameter in two clitic}) (\S\ref{section:stress:cliticc:cluster:Shift}). Stress is in bold. We underline the relevant clitic. 

\begin{exe}
	\ex \begin{xlist}
		\ex \gll ɡə-ləˈ\textbf{s-e-n} =\underline{ɡoɾ} \\
		{\ind}-listen-{\thgloss}-3{\pl} \underline{={\prog}} \\
		\trans `They are listening.'\label{clitic stress parameter in one clitic} \\
		\armenian{Կը լսեն կոր։}
		\ex \gll jetʰe ɡə-ləs-e-n =\underline{ˈ\textbf{ɡoɾ}} =ne\\
		if {\ind}-listen-{\thgloss}-3{\pl} =\underline{{\prog}}  ={\sbjv} \\
		\trans `If they are listening.'\label{clitic stress parameter in two clitic} \\
		\armenian{Եթէ կը լսեն կոր նէ։}
	\end{xlist}
\end{exe}

The second property is resyllabification. If clitic is vowel-initial, then it is syllabified with the preceding word.  If the preceding word ends in a consonant (\ref{clitic resyll parameter for C-final}), then this consonant is   pronounced as an onset.\footnote{In a clitic cluster however like in [dəˈ\textbf{χuɾ} =ɑl =e] `he is also sad', it is possible that the clitic consonant is ambisyllabic: [də.ˈ\textbf{χu}.ɾɑl.le]. } . If the preceding word ends in a vowel like [tʰɑzɑ] `fresh' or [ɑʁi] `salty', then we get glide epenthesis (\ref{clitic resyll parameter for V-final}); see more in \S\ref{section:syllable:VowelHiatus:Clitic}).


\begin{exe}
	\ex \begin{xlist}
		\ex \glll mɑɾtʰ-ə uˈ\textbf{ɾɑ}χ =\underline{e},  pʰɑjt͡s  dəˈ\textbf{χuɾ} =\underline{ɑl} =e\\
		[mɑɾ.tʰə u.ˈ\textbf{ɾɑ}. \underline{χe}, pʰɑjt͡s.  də.ˈ\textbf{χu}.  ɾɑ. le] \\
		man-{\defgloss} happy =is, but sad =\underline{also} =is \\
		\trans `The man is happy, but he is also sad.'\label{clitic resyll parameter for C-final} \\
		\armenian{Մարդը ուրախ է, բայց տխուր ալ է։}
		\ex \glll bibeɾ-ə tʰɑ\textbf{zɑ} =\underline{je},  pʰɑjt͡s  ɑˈ\textbf{ʁi} =\underline{jɑl} =e\\
		[bi.be.ɾə tʰɑ.ˈ\textbf{zɑ}. \underline{je}, pʰɑjt͡s.  ɑ.ˈ\textbf{ʁi}. \underline{jɑ}  .le] \\
		man-{\defgloss} happy =is, but sad =\underline{also} =is \\
		\trans `The pepper is fresh, but it is also salty.'\label{clitic resyll parameter for V-final} \\
		\armenian{Պիպերը թազա է, բայց աղի ալ է։}
	\end{xlist}
\end{exe}

The exception is the word `and', which can  syllabify either with the preceding word (\ref{clitic resyllab u vary resyll}) or on its own (\ref{clitic resyllab u vary alone}). In the latter case, the word is no longer acting as a clitic.

\begin{exe}
	\ex 
	\begin{xlist}
		\ex \gll ˈ\textbf{mɑɾtʰ} =\underline{u} ɡin, dəˈ\textbf{χɑ} =\underline{ju}  ɑχt͡ʃiɡ\\ 
		[ˈ\textbf{mɑɾ}. \underline{tʰu}. ɡin, də.ˈ\textbf{χɑ}. \underline{ju}. ɑχ.t͡ʃiɡ]  \\
		\label{clitic resyllab u vary resyll}
		\ex \glll  ˈ\textbf{mɑɾtʰ} =\underline{u} ɡin, dəˈ\textbf{χɑ} =u  ɑχt͡ʃiɡ\\  [ˈ\textbf{mɑɾtʰ}. \underline{u}. ɡin, də.ˈ\textbf{χɑ}. \underline{u}. ɑχ.t͡ʃiɡ] \\
		man \underline{and} woman, boy \underline{and} girl \\
		\trans `man and woman, boy and girl' \label{clitic resyllab u vary alone}\\
		\armenian{մարդ ու կին, տղայ ու աղջիկ}
	\end{xlist}
	
\end{exe}

The third property on definite allomorphy correlates with resyllabification. The definite suffix is \textit{-n} after vowels, and \textit{-n} after consonants (\ref{clitic def parameter no clitic}). But between a C-final word and V-initial clitic, the suffix is \textit{-n} instead of \textit{-ə} (\ref{clitic def parameter clitic is}, \ref{clitic def parameter clitic also}). See   \textcolor{red}{definite allomorphy chapter}. 


\begin{exe}
	\ex \begin{xlist}
		\ex \glll ɡɑˈ\textbf{du-n} jev ˈ\textbf{muɡ}-ə\\
		[ɡɑ.dun. jev. mu.ɡə] \\
		cat-{\defgloss} and mouse-{\defgloss} \\
		\trans `the cat and  the mouse'\label{clitic def parameter no clitic} \\
		\armenian{կատուն եւ մուկը}
		\ex \glll ɑsiɡɑ ˈ\textbf{muɡ}-n =\underline{e}  \\
		[ɑ.si.ɡɑ. ˈ\textbf{muɡ}. \underline{ne}] \\
		this mouse-{\defgloss}   =is \\
		\trans `This is the mouse.'\label{clitic def parameter clitic is} \\
		\armenian{Ասիկա մուկն է։}
		\ex \glll  ˈ\textbf{muɡ}-n =\underline{ɑl} des-ɑ-$\emptyset$\\
		[ˈ\textbf{muɡ}. \underline{nɑl} de.sɑ] \\
		mouse-{\defgloss}   =also see-{\pst}-1{\sg} \\
		\trans `I also saw the mouse.'\label{clitic def parameter clitic also} \\
		\armenian{Մուկն ալ տեսայ։}
	\end{xlist}
\end{exe}


The conjunction [u] is variably a clitic, as shown by how it can either  syllabify with the preceding word (\ref{clitic def u vary resyll}) or not (\ref{clitic def u vary alone}). When syllabified, it affects the definite suffix. 


\begin{exe}
	\ex 
	\begin{xlist}
		\ex \glll ˈ\textbf{muɡ}-n =\underline{u} ɡɑdu-n\\ 
		[ˈ\textbf{muɡ}. \underline{nu}. ɡɑ.dun]  \\
		mouse-{\defgloss}  \underline{and} cat-{\defgloss} \\
		\trans `the cat and the mouse' \label{clitic def u vary resyll}\\
		\armenian{մուկն ու կատուն} 
		\ex \glll  ˈ\textbf{muɡ}-ə =\underline{u} ɡɑdu-n\\ 
		[ˈ\textbf{mu}.ɡə. \underline{u}. ɡɑ.dun]  \\
		mouse-{\defgloss}  \underline{and} cat-{\defgloss} \\
		\trans `the cat and the mouse'  \label{clitic def u vary alone}\\
		\armenian{մուկը  ու կատուն} 
	\end{xlist}
	
\end{exe}


The fourth property is devoicing. The only relevant clitic is the progressive [ɡoɾ] because it starts with an obstruent (\ref{example:intonation:clitic:gor:gor}). This obstruent can optionally devoice after a voiceless obstruent (\ref{example:intonation:clitic:gor:kor}); suffixes seem to always devoice however (\S\ref{section:segmentalPhono:allphonLaryng:assiimlation:prog}). We're not sure how often the clitic devoices though. 

\begin{exe}
	\ex \begin{xlist}
		\ex \gll ɡə-ləˈ\textbf{s-e-s} =\underline{ɡoɾ} \\ 
		{\ind}-listen-{\thgloss}-2{\sg} =\underline{{\prog}} \\\label{example:intonation:clitic:gor:gor}
		\ex  \gll ɡə-ləˈ\textbf{s-e-s} =\underline{koɾ} \\ 
		{\ind}-listen-{\thgloss}-2{\sg} =\underline{{\prog}} \\
		\trans `You are listening.' \label{example:intonation:clitic:gor:kor}\\
		\armenian{Կը լսես կոր։}
	\end{xlist}
\end{exe}

The fifth final property is being able to stand alone. In general, these clitics cannot start a sentence. They are always pronounced after some word. The preceding word and the clitic are pronounced together. The exception is the word `and' [u] (\ref{example:intonation:clitic:stand:u}) which can start its own sentence or its own phrase after a pause. In this case, it is no longer acting as a clitic. 

\begin{exe}
	\ex \gll \underline{u} jeɾɾoɾtʰ kʰed-i-n ɑnun-ə diɡɾis e \\
	\underline{and} third river-{\gen}-{\defgloss} name-{\defgloss} Tigris is \\ 
	\trans Genesis 2:14 -- `The name of the third river is the Tigris' (NIV)
	\\ Literally: `And the name of the third river is Tigris.'\label{example:intonation:clitic:stand:u} \\
	\armenian{Ու երրորդ գետին անունը Տիգրիս է.} 
	
\end{exe}


Structurally, we treat clitics as somehow incorporating into the prosodic word of the preceding word (Representation \ref{rep:intonation:clitic:structure}). A suffix like ablative \textit{-e} takes stress and forms the last syllable of a prosodic word. In contrast, an unstressed clitic \textit{=e} is adjoined to the preceding word. The adjunction structure is arguably either a clitic group (CG: \citep{Nespor-Vogel-1986-ProsodicPhon,Vogel-2009-StatusCliticGroup} or a recursive prosodic word \citep{Ito-1989-ProsodicEpenthesis,Selkirk-1996-ProsodicFunctionWords,Booij-1996-CliticizationProsodicIntegrationDutch}. Both options have been proposed in the literature \textcolor{red}{cite vaux dolatian macak maybe khanjian}


\begin{representation}
	{Prosodic structure of  suffixes  vs. clitics}
	\label{rep:intonation:clitic:structure} 
	\begin{tabular}{| l| l|  l| }
		\hline 
		& suffix & clitic\\
		IPA:     & bɑniˈ\textbf{ɾ-e}  & bɑˈ\textbf{niɾ}=e
		\\
		Gloss:         & cheese-{\abl}& cheese=is
		\\
		Translation:         & `From cheese.'  & `It is cheese.'
		\\
		Orthography:     & \armenian{Պանիրէ։} & \armenian{Պանիր է։}
		\\
		Structure& 
		\begin{tikzpicture}[scale =1]
			\Tree    [.PWord  [.$\sigma$ bɑ ] [.$\sigma$ ni ] [.$\sigma$ ˈ\textbf{ɾe} ] ]
			] 
		\end{tikzpicture}
		& 
		\begin{tikzpicture}[scale =1]
			\Tree    [.CG [.PWord  [.$\sigma$ bɑ ] [.$\sigma$ ˈ\textbf{ni} ]  ] [ [.$\sigma$  ɾe ] ]	] 
		\end{tikzpicture}
		
		\begin{tikzpicture}[scale =1]
			\Tree    [.PWord' [.PWord  [.$\sigma$ bɑ ] [.$\sigma$ ˈ\textbf{ni} ]  ] [ [.$\sigma$  {ɾe} ] ]	] 
		\end{tikzpicture}
		\\ \hline 
	\end{tabular}
	
\end{representation} 






On a last note, we briefly mention the indefinite morpheme [-mə] (\ref{example:intonation:clitic:indf}). This morpheme has been called a clitic in the past because a) it is unstressed, b) it is spelled with a space \textcolor{red}{cite sigler}. But, its phonological behavior can also be explained if we treat this morpheme as a suffix. HD's speaker intuition is that this morpheme is more likely analyzable as just a suffix with a schwa. 

\begin{exe}
	\ex \glll Clitic: ˈ\textbf{mɑɾtʰ}=mə  \\ 
	Suffix: \textbf{mɑɾtʰ}-mə \\
	~ man-{\indf} \\
	\trans `a man' \label{example:intonation:clitic:indf}
	\\ \armenian{մարդ մը}
	
	
\end{exe}


\subsection{Particles that don't seem to be clitics}\label{section:intonation:clitic:not}
The previous section looked at morphemes that seem always be clitics (like the copula) or which by default act like clitics (the conjunction [u]). There are however some consonant-initial morphemes which seem to not be clitics, but we're unsure. 

There are some particles that are monosyllabic and unstressed (\ref{example:intonation:clitic:not}). Because they are consonant-initial, they cannot syllabify with the preceding word. But,   they don't seem to lean onto the preceding word  (\ref{example:intonation:clitic:not:vor:nolean}). And they can be as starting their sentence or phrase (\ref{example:intonation:clitic:not:vor:start}), with an optional pause.  

\begin{exe}
	\ex \label{example:intonation:clitic:not}
	\begin{xlist}
		\ex Complementizer [voɾ] `that' 
		\begin{xlist}
			\ex \gll kʰid-e-m \underline{voɾ}  dəχɑ-n uɾɑχ e \\
			know-{\thgloss}-1{\sg} \underline{that} boy-{\defgloss} happy  is \\
			\trans `I know that the boy is happy.' \label{example:intonation:clitic:not:vor:nolean}\\
			\armenian{Գիտեմ որ տղան ուրախ է։}
			\ex \gll  int͡ʃ ɡ-uz-e-m, \underline{voɾ}     uɾɑχ əll-ɑ-m \\
			what {\ind}-want-{\thgloss}-1{\sg},   \underline{that}  happy  be-{\thgloss}-1{\sg} \\
			\trans `What do I want? To be happy.' \label{example:intonation:clitic:not:vor:start}
			\\
			\armenian{Ի՞նչ կ՚ուզեմ։ Որ   ուրախ ըլլամ։}
		\end{xlist}
		\ex Conjunction [ɡɑm] `or' 
		\begin{xlist}
			\ex \gll t͡ʃuɾ \underline{ɡɑm} tʰej \\
			water \underline{or} tea \\
			\trans `water or tea' \\
			\armenian{Ջուր կամ թէյ։}
			\ex \gll \underline{ɡɑm} t͡ʃuɾ,  \underline{ɡɑm} tʰej \\
			\underline{or} water \underline{or} tea \\
			\trans `Either water or tea.' \\
			\armenian{Կամ ջուր կամ թէյ։}
		\end{xlist}
	\end{xlist}
\end{exe}

Other such particles of Armenian are listed in \textcolor{red}{chapter for particle lists}. 


It's not clear to us what phonological evidence can be used to treat the above morphemes as either always clitics, sometimes clitics, or never clitics. 
\subsection{Particles that used to be clitics}\label{section:intonation:clitic:old}
\textcolor{red}{write with those stuff like i-ver from the grammars, to show how theyre no longer clitics}
\section{Resyllabification across words}\label{section:intonation:resyll}
Within a word, a consonant-vowel sequence is always  syllabified as part of the same syllable. Their syllabification is likewise perceptually unambiguous. Similar unambiguity is found when a word-final consonant is syllabified with a following clitic. But across words, HD does not perceive resyllabification, but we're not sure what acoustic evidence is being used to create this perception. 

First consider suffixes and clitics. The segment [e] is either the stressed ablative suffix \textit{-e} (\ref{resyllabification:example: suff clitic:suff}), or an unstressed copula clitic \textit{=e} (\ref{resyllabification:example: suff clitic:clitc}). After a C-final word, both types of [e] take an onset. The forms are homophonous except for a difference in stress. 

\begin{exe}
	\ex 
	\begin{xlist}
		\ex \glll ʃuˈn-e \\
		[ʃu.ˈ{ne}] \\
		dog-{\abl}\\
		\trans `from a dog' \label{resyllabification:example: suff clitic:suff} \\
		\armenian{շունէ} \\
		<ʃown\textbf{ē}> 
		\ex \glll   ˈ{ʃu}n =e \\
		[ˈ{ʃu}. ne] \\
		dog =is \\
		\trans `It is a dog.' \label{resyllabification:example: suff clitic:clitc} \\
		\armenian{Շուն է։} \\
		<ʃown \textbf{ē}> 
	\end{xlist}
\end{exe}


In contrast, across words, our intuition is that there is no resyllabification, but we cannot find significant acoustic evidence against resyllabification. There likewise seems to be correlations with focus and stress. Note that after this point, we systematically use underlining for nuclear stress, and boldface for phrasal stress. 

Consider the following two-word sentences. Nuclear stress  is on the first word. The consonant in the [isu] sequence is either morphologically part of the first word   /is\#u/ (\ref{resyllabification:example: misuler:mis}), morphologically   part of the second word /i\#su/ (\ref{resyllabification:example: misuler:mi s}). 

\begin{exe}
	\ex 
	\begin{xlist}
		\ex \gll \underline{ˈ\textbf{mis}} uˈn-e-$\emptyset$-ɾ \\
		meat have-{\thgloss}-{\pst}-3{\sg} \\
		\trans `He had meat.' \label{resyllabification:example: misuler:mis} \\
		\armenian{Միս ունէր։}
		\ex \gll \underline{ˈ\textbf{mi}} suˈl-e-ɾ \\
		{\proh} whistle-{\thgloss}-2{\sg} \\
		\trans `Don't whistle!' \label{resyllabification:example: misuler:mi s} \\
		\armenian{Մի սուլեր։}
	\end{xlist}
\end{exe}

There are two pieces of evidence against resyllabification (holistic perception and articulation), and two pieces of evidence for resyllabification (no length differences and non-holistic perpcetion). 

For holistic perception, when these sentences are uttered as a whole, HD  perceives that the /s/ in (\ref{resyllabification:example: misuler:mis}) is a coda, while the /s/ in (\ref{resyllabification:example: misuler:mi s}) is a onset. In terms of articulation, he likewise `feels' that the /s/ in /is\#u/ is being articulated at the same time as /i/; while the /s/ in /i\#su/ is not articulated at the same time as /i/. 

However, when we examine the sound wave, we don't see a significant difference in the length of either the first vowel /i/ or the consonant /s/. Furthermore, when   only the substring [isu] is played back to HD, he cannot hear the difference between /i\#su/ and /is\#u/. 


Given this contradictory evidence, it's unclear if there is genuine phonetic resyllabification across words. If it exists, then perhaps HD's holistic perception data is because he is psychologically perceiving the word-initial boundary of the second word, thus creating the illusion of no resyllabification. If resyllabification does not exist, then perhaps the relevant acoustic cues are too subtle to easily find without doing a large sample of data and with more refined measurements. 

Similar contradiction is found for the sequence [ətʰə] below. Holistically, HD perceives no resyllabification. He feels that the /tʰ/ in /ətʰ\#ə/ (\ref{resyllabification:example: kirke t eri :et}) is articulated as a coda with a weaker   release than the onset /tʰ/ in /ə\#tʰə/ (\ref{resyllabification:example: kirke t eri :te}). But then we find no acoustic length difference, and the substring is perceived as homophonous when the substring is played back. 


\begin{exe}
	\ex 
	\begin{xlist}
		\ex \gll \underline{ˈ\textbf{kʰiɾ}kʰ-ətʰ}  əɾ-i-$\emptyset$ \\
		book-{\possSsg} do.{\aorperf}-{\pst}-1{\sg} \\
		\trans `I did your book.' \label{resyllabification:example: kirke t eri :et} \\
		\armenian{Գիրքդ ըրի։}
		\ex \gll \underline{ˈ\textbf{kʰiɾ}kʰ-ə}  tʰəɾ-i-$\emptyset$ \\
		book-{\defgloss} put.{\aorperf}-{\pst}-1{\sg} \\
		\trans `I put the book.' \label{resyllabification:example: kirke t eri :te} \\
		\armenian{Գիրքը դրի։}
	\end{xlist}
\end{exe}

For the two-words sentences above, stress was on the first word. When stress is on the second word, there is clearer evidence against   resyllabification. 

Consider the [əsu] sequence below where focus on the second word. When /s/ is part of the unfocused first  word /əs\#u/ (\ref{resyllabification:example: danage s u:es u}), it is perceived as a coda, not loud, and there is a slight glottal stop after it. In contrast, when /s/ is part of the focused second word /ə\#su/ (\ref{resyllabification:example: danage s u:e su}), the /s/ is an onset, louder (higher amplitude), and there's no glottal stop. 


\begin{exe}
	\ex 
	\begin{xlist}
		\ex \gll dɑˈ\textbf{nɑ}ɡ-əs  \underline{ˈ\textbf{u}ɾ=e} \\
		knife-{\possFsg} where=is\\
		\trans `Where is my knife?'   \label{resyllabification:example: danage s u:es u} \\
		\armenian{Դանակս ո՞ւր է։}
		\ex \gll dɑˈ\textbf{nɑ}ɡ-ə  \underline{ˈ\textbf{su}ɾ=e} \\
		knife-{\defgloss} sharp=is\\
		\trans `Is the knife sharp?   \label{resyllabification:example: danage s u:e su} \\
		\armenian{Դանակը սո՞ւր է։}
	\end{xlist}
\end{exe}


Similarly for [ətʰə], when the /tʰ/ is part of the first unfocused word (\ref{resyllabification:example: kirke t eri focus :et}), the aspiration is quite short and there's again a slight glottal stop. But when /tʰ/ is part of the second focused word (\ref{resyllabification:example: kirke t eri focus :te}), the /tʰ/ is an onset with a longer aspiration and no glottal stop. 

\begin{exe}
	\ex 
	\begin{xlist}
		\ex \gll ˈ\textbf{kʰiɾ}kʰ-ətʰ  \underline{ˈ\textbf{ə}ɾ-i-$\emptyset$} \\
		book-{\possSsg} do.{\aorperf}-{\pst}-1{\sg} \\
		\trans `I DID your book.' \label{resyllabification:example: kirke t eri focus :et} \\
		\armenian{Գիրքդ ըրի՛։}
		\ex \gll ˈ\textbf{kʰiɾ}kʰ-ə  \underline{ˈ\textbf{tʰə}ɾ-i-$\emptyset$} \\
		book-{\defgloss} put.{\aorother}-{\pst}-1{\sg} \\
		\trans `I PUT the book.' \label{resyllabification:example: kirke t eri focus :te} \\
		\armenian{Գիրքը դրի՛։}
	\end{xlist}
\end{exe}

Given all this evidence, it seems that when given a C-V sequence across two words, if Word2  is focused, then there is unambiguously no resyllabification. But if Word2 is not focused,  holistic perception and articulation suggests there is no resyllabification, but we haven't been able to find concrete acoustic evidence. 

Further, there is also morphological evidence against resyllabification. As explained earlier in (\S\ref{section:intonation:clitic:overview}), the definite suffix is \textit{-ə} after consonants (\ref{resyll across words: def: base}), but \textit{-n} between a C-final word and V-initial clitic (\ref{resyll across words: def: clitc}). V-initial clitics trigger the \textit{-n} form and this \textit{-n} is the onset of the clitic. But when this suffix precedes a V-initial word, the definite suffix doesn't change from \textit{-ə} to \textit{-n} (\ref{resyll across words: def: word}).

\begin{exe}
	\ex 
	\begin{xlist}
		\ex \glll ˈ\textbf{mu}ɡ-ə \\
		[ˈ\textbf{mu}.ɡə] \\
		mouse-{\defgloss} \\
		\trans `the mouse'  \label{resyll across words: def: base} \\
		\armenian{մուկը}
		\ex \glll ˈ\textbf{mu}ɡ-n =ɑl \\
		[ˈ\textbf{mu}ɡ. nɑl] \\
		mouse-{\defgloss} =also \\
		\trans `also the mouse' \label{resyll across words: def: clitc}  \\
		\armenian{մուկն  ալ}
		\ex \glll ˈ\textbf{mu}ɡ-ə ɑɾ-i-n \\
		[ˈ\textbf{mu}.ɡə. ɑ.ɾi] \\
		mouse-{\defgloss} take.{\aorperf}-{\pst}-3{\pl} \\
		\trans `They took the mouse.' \label{resyll across words: def: word}\\
		\armenian{Մուկը  առին։} 
	\end{xlist}
\end{exe}


The allomorphy thus suggests that  words generally don't syllabify with the preceding word.  Diachronically and cross-dialectally however, the definite suffix can be sensitive to the subsequent word. But this is not the case for the typical Western Armenian sentence. See \textcolor{red}{cite definite allomorphy chapter}
\section{Prosodic phrases and phrasal stress}\label{section:intonation:phrase}

Having established the formation of prosodic words, this section looks at the formation of prosodic phrases. In general, a syntactic phrase is changed into a prosodic phrase. Within a phrase, one element is perceived as more prominent than the others. This element is said to have phrasal stress. The word stress of other words in the phrase are said to have been demoted, such as via stress clash resolutions (\citealt[28]{Abeghyan-1933-Meter}; \citealt[24-7]{Fairbanks-1948-PhonologyMorphoWestern}; \textcolor{red}{vaux}). Tihs section focuses more on the following questions:
\begin{itemize}[noitemsep, topsep=0pt]
	\item Is the prosodic phrase always the same size as the original syntactic phrase?
	\item Where is the location of phrasal stress? 
\end{itemize}

For the first question, it seems that syntactic phrases are almost always mapped to single prosodic phrases, with minor complications in long noun phrases (\S\ref{section:intonation:phrase:noun} and in long verb phrases (\S\ref{section:intonation:phrase:verb:rec}, \S\ref{section:intonation:phrase:nounverb}). 

For the second question,   noun phrases assign stress on the last word (\S\ref{section:intonation:phrase:noun}). The same is for adpositional phrases (\S\ref{section:intonation:phrase:adpos}). Such prosodic phrases are thus right-headed. In contrast, verb phrases in an SOV sentence assign phrasal stress on the preverbal word, usually the object (\S\ref{section:intonation:phrase:verb}). Such non-final placement may be due to recursive prosodic phrasing. The interaction between noun-phrase final-stress and verb-phrase pre-final stress is quite complex (\S\ref{section:intonation:phrase:nounverb}). Some other types of syntactic phrases have their own unique stress rules (\S\ref{section:intonation:phrase:other}: adverbs (\S\ref{section:intonation:phrase:other:adv}), compound-like collocations (\S\ref{section:intonation:phrase:other:colloc}), and reduplication (\S\ref{section:intonation:phrase:other:redup}).

Note that throughout this section, we generally only mark phrasal stress (boldface), and not nuclear stress. This is most of the examples in this section are fragments and not complete sentences. Furthermore, marking nuclear stress in this section would be redundant because the last prosodic phrase has nuclear stress. 

\subsection{Phrasal stress in noun phrases}\label{section:intonation:phrase:noun}

For a noun phrase, the most typical pronunciation is to turn a noun phrase (NP) into a single prosodic phrase. The most perceptually prominent syllable is on the stressed syllable of the last word in the phrase 	\citep[24-5]{Abeghyan-1933-Meter}. This means that phrasal stress is on the final prosodic word of the prosodic phrase. Thus prosodic phrases are right-headed. Long noun phrases with multiple nouns can however be broken up in speech. 

First consider the basic order of the noun phrase. Left to right, we can have a genitive possessive pronoun, demonstrative, a numeral,  an adjective, and then the noun. Other permutations of the possessive pronoun and demonstrative are possible, usually with a sense of emphasis on the demonstrative.  The noun can end in either final lexical stress (\ref{prosodic phrase: noun: final stress}) or penultimate lexical stress  (\ref{prosodic phrase: noun: penult stress}). Each non-pronominal word has some perceptible word stress (=lexical stress), but the final noun has the strongest stress in bold. We use parentheses ()$_\phi$ to mark the edges of the prosodic phrase

\begin{exe}
	\ex 
	\begin{xlist}
		\ex \glll (Poss  Dem Num Adj \textbf{N})$_\phi$  \\
		im ˈɑjs jeɾˈɡu    ɡɑˈbujd  dupʰ-eˈ\textbf{ɾ-e-n}  \\
		I.{\gen} this two blue  box-{\pl}-{\abl}-{\defgloss} \\
		\trans `from my two blue boxes'  \label{prosodic phrase: noun: final stress}\\ 
		\armenian{իմ այս երկու կապոյտ տուփերէն} 
		\ex \glll (Poss  Dem Num Adj \textbf{N})$_\phi$  \\
		meɾ ˈɑjt ˈhiŋkʰ  ɡɑˈnɑnt͡ʃ  duˈ\textbf{pʰ-e}ɾ-ə\\
		we.{\gen} that five green box-{\pl}-{\defgloss}  \\
		\trans `our five green boxes' \label{prosodic phrase: noun: penult stress}\\ 
		\armenian{մեր այդ հինգ կանանչ  տուփերը}
		
	\end{xlist}
\end{exe}

The above noun phrase is obviously rather long. In HD's impression, the default pronunciation is to make this noun phrase be a single prosodic phrase without a phrase-internal pause. But one can also add a brief pause in the middle of the noun phrase  (\ref{example:intonation:phrase:noun:brekaup}), so that we get two prosodic phrases of almost equal length. Both phrases have phrasal stress on the last word. 

\begin{exe}
	\ex \glll (Poss  Dem \textbf{Num})$_\phi$  (Adj \textbf{N})$_\phi$  \\
	im ˈɑjs jeɾˈ\textbf{ɡu})$_\phi$ (ɡɑˈbujd dupʰ-eˈ\textbf{ɾ-e-n}) \\
	I.{\gen} this two blue  box-{\pl}-{\abl}-{\defgloss} \\
	\trans `from my two blue boxes' \label{example:intonation:phrase:noun:brekaup}\\ 
	\armenian{իմ այս երկու կապոյտ տուփերէն} 
	
\end{exe}

We similar prosodic phrasings for smaller noun phrases. For two-word phrases (\ref{tab:prosodic phrasing two-word noun phrase}), we get a single prosodic phrase with stress on the final noun. The type of pre-nominal modifier does not matter. 

\begin{table}[H]
	\centering
	\caption{Right-headed prosodic phrasing of two-word noun phrases}
	\label{tab:prosodic phrasing two-word noun phrase}
	\centering
	\resizebox{\textwidth}{!}{%
		\begin{tabular}{|ll| ll | ll | ll| ll | ll| }
			\hline 
			(Adj  & \textbf{N})$_\phi$  & 
			(Num & \textbf{N})$_\phi$ 
			& (Dem & \textbf{N})$_\phi$ 
			& (Poss & \textbf{N})$_\phi$
			\\
			\hline 
			ɑʁˈdod & {{dobˈ\textbf{ɾɑɡ}}} 
			&
			{jeˈɾekʰ}
			&  {muɾˈ\textbf{d͡ʒ-eɾ}} 
			&
			ɑs
			& {{ˈ\textbf{mɑɾ}tʰ-ə}} 
			&
			({iˈɾent͡s}
			& {{ɡɑˈ\textbf{du-n}}}) 
			\\
			dirty & bag
			&three&hammer-{\pl}
			&this&man-{\defgloss}
			&they.{\gen} &cat-{\defgloss}
			\\
			\multicolumn{2}{|l|}{`dirty bag'}
			&\multicolumn{2}{l|}{`three hammers'}
			&\multicolumn{2}{l|}{`this man'}
			&\multicolumn{2}{l|}{`their cat'}
			\\
			\multicolumn{2}{|l|}{\armenian{աղտոտ տոպրակ}}  
			& \multicolumn{2}{|l|}{\armenian{երեք  մուրճ}}  
			& \multicolumn{2}{|l|}{\armenian{աս  մարդը}}  
			& \multicolumn{2}{|l|}{\armenian{իրենց  կատուն}}  
			\\ \hline
		\end{tabular}
	}
\end{table}

A pre-nominal modifier can also be a quantifier (\ref{prosdic phrase:noun:quant n}). In general, quantifiers are treated like normal modifiers; the noun gets phrasal stress. 

\begin{exe}
	\ex \label{prosdic phrase:noun:quant n}
	\begin{xlist}
		\ex \glll (Quant \textbf{N})$_\phi$ \\
		ɑˈmen ɑɾˈ\textbf{du}) \\
		every morning \\ 
		\trans `every morning' \label{prosdic phrase:noun:quant n: q n}
		\\ \armenian{ամէն առտու}
		\ex \glll (Quant Adj \textbf{N})$_\phi$ \\
		pʰoˈloɾ hiˈvɑntʰ ɑʃɑɡeɾd-ˈ\textbf{ne}ɾ-ə \\
		all sick student-{\pl}-{\defgloss} \\ 
		\trans `all the sick students' \label{prosdic phrase:noun:quant n: q a n}
		\\ \armenian{բոլոր հիւանդ աշակերտները}
	\end{xlist}
\end{exe}

However, quantifiers can easily attract stress to themselves because, pragmatically, quantifiers introduce nuanced contextual information, such as the need to emphasize a certain number. Stress on the quantifier (\ref{prosdic phrase:noun:quant n:  a q n}) is however perceived more as sentential-stress (focus) rather than phrasal-stress

\begin{exe}
	\ex \glll (\underline{\textbf{Quant}}   {N})$_\phi$ \\
	\underline{pʰoˈ\textbf{loɾ}}  ɑʃɑɡeɾd-ˈ{ne}ɾ-ə   \\
	all  student-{\pl}-{\defgloss} \\ 
	\trans `ALL the  students' \label{prosdic phrase:noun:quant n:  a q n}
	\\ \armenian{Բոլոր հիւանդ աշակերտները։}
\end{exe}

Within a noun phrase, the pre-nominal modifier can be another noun phrase, such as a   genitive-marked possessor (\ref{prosdic phrase:noun:gen n: n n}). Both the possessor noun and the head noun can have their own other modifiers (\ref{prosdic phrase:noun:gen n: n a n},\ref{prosdic phrase:noun:gen n: a n n}). 



\begin{exe}
	\ex 
	\begin{xlist}
		\ex \glll (N-Gen \textbf{N})$_\phi$  \\
		ɡoˈv-i-n ɑɡˈ\textbf{ɾɑ-n}  \\
		cow-{\gen}-{\defgloss} tooth-{\defgloss} \\
		\trans `the cow's tooth' \label{prosdic phrase:noun:gen n: n n} 
		\\ \armenian{կովին ակռան}
		\ex \glll (N-Gen Adj \textbf{N})$_\phi$  \\
		ɡoˈv-i-n  χoˈʃoɾ ɑɡˈ\textbf{ɾɑ-n} \\
		cow-{\gen}-{\defgloss} huge tooth-{\defgloss} \\
		\trans `the cow's huge tooth' \label{prosdic phrase:noun:gen n: n a n} 
		\\ \armenian{կովին  խոշոր  ակռան}
		\ex \glll (Adj N-Gen   \textbf{N})$_\phi$  \\
		d͡ʒeɾˈmɑɡ  ɡoˈv-i-n  ɑɡˈ\textbf{ɾɑ-n} \\ 
		white cow-{\gen}-{\defgloss}   tooth-{\defgloss} \\
		\trans `the white cow's  tooth' \label{prosdic phrase:noun:gen n: a n n} 
		\\ \armenian{ճերմակ կովին  ակռան}
	\end{xlist}
\end{exe}

When the entire phrase is small (two or three words), HD feels that the entire noun phrase can be a single prosodic phrase. But when the noun phrase is large because both nouns have a modifier (\ref{prosdic phrase:noun:gen n: a n a n}), then HD feels that it is more natural to break up the entire noun phrase into two prosodic phrases, one for each noun.  Phrasal  stress is on the nouns, and there can be a  pause between the phrases. 

\begin{exe}
	\ex \glll (Adj \textbf{N-Gen})$_\phi$ (Adj   \textbf{N})$_\phi$  \\
	d͡ʒeɾˈmɑɡ  ɡo\textbf{ˈv-i-}n)$_\phi$ (χoˈʃoɾ  ɑɡˈ\textbf{ɾɑ-n}  \\ 
	white cow-{\gen}-{\defgloss}  huge  tooth-{\defgloss} \\
	\trans `the white cow's huge  tooth' \label{prosdic phrase:noun:gen n: a n a n} 
	\\ \armenian{ճերմակ կովին խոշոր  ակռան}
\end{exe}

For the above phrasing in (\ref{prosdic phrase:noun:gen n: a n a n}), HD feels that two prosodic phrases are used but not because of phonological length,  but because of semantic content. It is more accommodating for the speaker's and listener's perception to break this noun phrase into two phonological phrases, one per noun. The two nouns both have their own modifiers. By putting a pause between the noun phrases, it's easier for the speaker and hearer to mentally create an image of these two nouns and to then modify both. In contrast, if only one noun is modified, then it feels `easier' for the speaker and listener to mentally entertain the image of both nouns. 

Besides genitive possessors, a pre-nominal NP can also be an instrumental-marked noun phrase (\ref{prosdic phrase:noun:ins n: n n}). Such a phrase is translated to an English `with X' construction. Like genitive possessors, an instrumental noun and the head noun are by default phrased together (\ref{prosdic phrase:noun:ins n: n n}). Either of them can have their own modifiers (\ref{prosdic phrase:noun:ins n: n a n}, \ref{prosdic phrase:noun:ins n:  n a n}). But if both nouns have modifiers (\ref{prosdic phrase:noun:ins n: a n n}), then the default pronunciation is to create two prosodic phrases, one per noun. 


\begin{exe}
	\ex 
	\begin{xlist}
		\ex \glll (N-Ins \textbf{N})$_\phi$  \\
		ʒəbiˈd-ov dəˈ\textbf{ʁɑ-n}  \\ 
		smile-{\ins}-{\defgloss} boy-{\defgloss} \\
		\trans `the boy with the smile'  \label{prosdic phrase:noun:ins n: n n} 
		\\ \armenian{ժպիտով տղան}
		\ex \glll (N-Ins Adj \textbf{N})$_\phi$  \\
		ʒəbiˈd-ov uˈɾɑχ dəˈ\textbf{ʁɑ-n} \\ 
		smile-{\ins}-{\defgloss} happy boy-{\defgloss} \\
		\trans `the happy boy with the smile'  \label{prosdic phrase:noun:ins n: n a n} 
		\\ \armenian{ժպիտով ուրախ տղան}
		\ex \glll (Adj N-Ins  \textbf{N})$_\phi$  \\
		ɡɑɾˈmiɾ ʒəbiˈd-ov dəˈ\textbf{ʁɑ-n} \\ 
		red smile-{\ins}-{\defgloss}   boy-{\defgloss} \\
		\trans `the  boy with the red smile'  \label{prosdic phrase:noun:ins n:  n a n} 
		\\ \armenian{կարմիր ժպիտով  տղան}
		\ex \glll (Adj \textbf{N-Ins})$_\phi$  (Adj  \textbf{N})$_\phi$  \\
		ɡɑɾˈmiɾ ʒəbiˈ\textbf{d-ov}   uˈɾɑχ dəˈ\textbf{ʁɑ-n}  \\ 
		red smile-{\ins}-{\defgloss}  happy  boy-{\defgloss} \\
		\trans `the  happy boy with the red smile'  \label{prosdic phrase:noun:ins n: a n n} 
		\\ \armenian{կարմիր ժպիտով ուրախ տղան}
	\end{xlist}
\end{exe}


In sum, noun phrases are most typically pronounced as a single prosodic phrase with stress on the final noun. Deviations exist in special circumstances (pre-nominal nouns, focus-sensitive quantifiers). 


\subsection{Phrasal stress in adpositional phrases}\label{section:intonation:phrase:adpos}
An adpositional phrase (AP) is a phrase made up of an adposition and a noun. The adposition can be a preposition or a postposition. There are significantly more postpositions in the language, rather than prepositions. Prepositions lack case-marking, while most postpositions act like their own nouns and take case-suffixes.  In both types of adpositional phrases, we generally have final stress  \citep[26-7]{Abeghyan-1933-Meter}. 


First consider postpositional phrases (\ref{prosodic phrase: adp: post: a n p}). The postposition selects a noun phrase; the noun phrase often bears some case suffix. That noun phrase can have its own modifiers. The entire AP is a single prosodic phrase with stress on the final adposition. 

\begin{exe}
	\ex \label{prosodic phrase: adp: post: a n p}\begin{xlist}
		\ex \glll (Adj N \textbf{Post})$_\phi$ \\
		kʰeˈɾuɡ dəˈʁ-u-n kʰoˈ\textbf{v-e-n} \\
		chubby boy-{\gen}-{\defgloss} next-{\abl}-{\defgloss} \\ 
		\trans `from next to the chubby boy' \label{prosodic phrase: adp: post: a n p: final} 
		\\ \armenian{գէրուկ տղուն քովէն}
		\ex \glll (Adj N \textbf{Post})$_\phi$ \\
		kʰeˈɾuɡ dəˈʁ-u-n \textbf{kʰo}ˈv-ə) \\
		chubby boy-{\gen}-{\defgloss} next-{\defgloss} \\ 
		\trans `next to the chubby boy' \label{prosodic phrase: adp: post: a n p: final schwa} 
		\\ \armenian{գէրուկ տղուն քովը}
	\end{xlist}
\end{exe}

Although postpositions are function words, they are stressed because they're very noun-like. The morphosyntax places case-markers and other nominal inflectional suffixes for some but not all postpositions. The lexical phonology treats postpositions as simple prosodic words with their own  lexical stress. Thus the phrasal phonology also treats postpositional phrases as noun phrases, with final stress. 



There are dozens of postpositions in the language, and they all behave the same with regard to taking phrasal stress. Below, we go through some of the more common postpositions that can take inflectional suffixes. 

\begin{exe}
	\ex (N \textbf{Post})$_\phi$
	\begin{xlist}
		\ex \gll  kʰevoɾˈkʰ-i-n ˈ\textbf{mod} \\
		Kevork-{\gen}-{\defgloss} near \\ 
		\trans `near Kevork' 
		\\ \armenian{Գեւորգին մօտ}
		\ex \gll  ʃɑkʰɑˈɾ-e-n ˈ\textbf{zɑd} \\
		sugar-{\abl}-{\defgloss} besides \\ 
		\trans `besides sugar'
		\\ \armenian{շաքարէն զատ}
		\ex \gll  bɑɾdeˈz-e-n ɑnˈ\textbf{tʰin} \\
		garden-{\abl}-{\defgloss} beyond \\ 
		\trans `beyond the garden'
		\\ \armenian{պարտէզէն անդին}
		\ex \gll  ʃeŋˈkʰ-i-n ˈ\textbf{tʰe}m-ə \\
		building-{\gen}-{\defgloss} facing-{\defgloss} \\ 
		\trans `facing the building'
		\\ \armenian{շէնքին դէմը}
		\ex \gll  seʁɑˈn-i-n dɑˈ\textbf{ɡ}-e-n \\
		table-{\gen}-{\defgloss} under-{\abl}-{\defgloss} \\ 
		\trans `from under the table'
		\\ \armenian{սեղանին տակէն}
	\end{xlist}
\end{exe}

Although there are many postpositions, there are very few prepositions. None of them can take case suffixes. A typical preposotional phrase has final stress on the noun. 

\begin{exe}
	\ex
	\begin{xlist}
		\ex \glll (Pre Adj \textbf{N})$_\phi$ \\
		minˈt͡ʃev  ɡɑˈbujd ˈ\textbf{bɑ}d-ə \\
		until blue wall-{\defgloss} \\
		\trans `until the blue wall' 
		\\ \armenian{մինչեւ կապոյտ պատը}
		\ex \glll (Pre N-Gen \textbf{N})$_\phi$ \\
		tʰeˈbi  mɑɾoˈj-i-n ʃuˈ\textbf{ɡɑ-n} \\
		towards Maro-{\gen}-{\defgloss} store-{\defgloss} \\
		\trans `towards Maro's store'
		\\ \armenian{դէպի Մարոյին շուկան} 
	\end{xlist}
\end{exe}

For most prepositions like `until' [mint͡ʃev], the prepositional phrase takes final stress, on the noun. However,   there are some prepositions like `without' [ɑɾɑnt͡s] that take stress.



\begin{exe}
	\ex
	\begin{xlist}
		\ex \glll (\textbf{Pre} N)$_\phi$ \\
		ɑˈ\textbf{ɾɑnt͡s} bɑˈ{niɾ} \\
		without cheese\\
		\trans `without cheese'
		\\ \armenian{առանց պանիր} 
	\end{xlist}
\end{exe}

It seems that the  preposition [ɑɾɑnt͡s] `without' is unique in being able to take stress, and it seems to require stress. The exceptionality of this preposition is likely because of semantics. Negation and negation-like meanings often cause special intonational or prosodic effects. 


In sum, postpositional phrases place phrasal stress on the final postpositions. Most prepositional phrases place phrasal stress on the final noun. 


\subsection{Phrasal stress in verb phrases}\label{section:intonation:phrase:verb}

Noun phrases (NPs) and adpositional phrases (APs) have phrasal stress on the final word. But in verb phrases (VPs), the verb is final but avoids stress. Instead, phrasal stress is on the pre-verbal item.  We establish this generalization by going through simple cases of verb phrases with only two words (\S\ref{section:intonation:phrase:verb:two}), verb phrases with clitics and periphrasis (\S\ref{section:intonation:phrase:verb:clitic}), and recursively large verb phrases (\S\ref{section:intonation:phrase:verb:rec}). Note that the phrasal stress of verb phrases is closely tied with the nuclear stress of the sentence. The behavior of definite objects, ditransitives, and intransitives is discussed under nuclear stress in Section   \S\ref{section:intonation:broadFocus}. 

\subsubsection{Verb phrases with   two words}\label{section:intonation:phrase:verb:two}

Although Armenian is an SOV sentence, it is common to omit the subject, thus creating rather small sentences (\ref{prosodic phrease: verb: basic}). We establish the basics of verb-phrase stress with these smaller examples. Briefly, verbs avoid stress;   phrasal stress is on the preverbal word, usually the object. 


\begin{exe}
	\ex \label{prosodic phrease: verb: basic}
	\begin{xlist}
		\ex \glll (\textbf{N} V)$_\phi$ \\
		ɑɡɾɑ-ˈ\textbf{neɾ} uˈn-i-$\emptyset$\\ 
		tooth-{\pl} have-{\thgloss}-3{\sg} \\
		\trans `He has teeth.'  \label{prosodic phrease: verb: npl v}\\ 
		\armenian{Ակռաներ ունի։}
		\ex \glll (\textbf{N} V)$_\phi$ \\
		ɑɡˈ\textbf{ɾɑ}-mə uˈn-i-$\emptyset$\\ 
		tooth-{\indf} have-{\thgloss}-3{\sg} \\
		\trans `He has a  tooth.' \\ 
		\armenian{Ակռայ մը ունի։}
		
\end{xlist}\end{exe}

To reduce clutter, this section only marks phrasal stresses (boldface). We don't mark nuclear stress, which is essentially the last phrasal stress of the sentence. 


We know that phrasal stress is on the pre-verbal item because of two reasons. First, perceptually we can hear prominence on the pre-verbal item. Second, acoustically, the verb is deaccented or has lost its own lexical stress. Such deaccenting is called post-focal deaccenting or post-focal compression. 

\textcolor{blue}{wav form}


The preverbal item is usually a noun object, but it can range over other arguments or syntactic categories, such as locational nouns (\ref{example:inton:phrasa:verb:two:pos:loc}), adjectives (\ref{example:inton:phrasa:verb:two:pos:adj}),  or adverbs (\ref{example:inton:phrasa:verb:two:pos:adv}). 


\begin{exe}
	\ex 
	\begin{xlist}
		\ex \glll (\textbf{Loc} V)$_\phi$ \\
		ɑmeɾiˈ\textbf{ɡɑ} ɡ-ɑbˈɾ-i-n\\ 
		America  {\ind}-live-{\thgloss}-3{\pl} \\
		\trans `They live in  America.' \label{example:inton:phrasa:verb:two:pos:loc}\\ 
		\armenian{Ամերիկա կ՚ապրին։} 
		\ex \glll (\textbf{Adj} V)$_\phi$ \\
		d͡ʒeɾˈ\textbf{mɑɡ} jeˈʁ-ɑ-n\\ 
		white  become.{\aorperf}-{\thgloss}-3{\pl} \\
		\trans `They became white.' \label{example:inton:phrasa:verb:two:pos:adj}\\ 
		\armenian{Ճերմակ եղան։} 
		\ex \glll (\textbf{Adv} V)$_\phi$ \\
		ɑˈ\textbf{ɾɑkʰ} ɡ-eˈpʰ-e-n\\ 
		fast  {\ind}-cook-{\thgloss}-3{\pl} \\
		\trans `They  cook (things) quickly.' \label{example:inton:phrasa:verb:two:pos:adv} \\ 
		\armenian{Արագ կ՚եփեն։}  
		
\end{xlist}\end{exe}

\subsubsection{Verb phrases with   clitics or periphrasis}\label{section:intonation:phrase:verb:clitic} 
Some verb tenses use   an unstressed proclitic. If no object is present, then stress is on the verb (\ref{prosodic phrase: verb: bidi v}). If there is an object, then the object takes stress (\ref{prosodic phrase: verb: n bidi v}).  

\begin{exe}
	\ex \begin{xlist}
		\ex \glll (Pro \textbf{V})$_\phi$ \\
		bidi uˈ\textbf{d-e-m} \\ 
		{\fut} eat-{\thgloss}-1{\sg} \\ 
		\trans `I will eat.' \label{prosodic phrase: verb: bidi v}
		\\ \armenian{Պիտի ուտեմ։}
		\ex \glll (\textbf{N} Pro {V})$_\phi$ \\
		bɑˈ\textbf{niɾ} bidi uˈ{d-e-m} \\ 
		cheese {\fut} eat-{\thgloss}-1{\sg} \\ 
		\trans `I will eat cheese.' \label{prosodic phrase: verb: n bidi v}
		\\ \armenian{Պապնիր պիտի ուտեմ։}
	\end{xlist}
\end{exe}


Some verb tenses are periphrastic. The    verb is a participle, while inflection is on an enclitic auxiliary. The verb takes stress when no object is present (\ref{prosodic phrase: verb: v is}). If an object is present, stress can shift  to the object (\ref{prosodic phrase: verb: n v is}). However, these tenses often easily allow stress on the verb (\ref{prosodic phrase: verb: n v is deaccent}); the object is then deaccented as some type of given information. \textcolor{red}{cite nakipoglu}


\begin{exe}
	\ex \begin{xlist}
		\ex \glll (\textbf{V} Aux)$_\phi$ \\
		ɡeˈ\textbf{ɾ-ɑd͡z} =e-m \\ 
		eat-{\rptcp} =is-1{\sg} \\ 
		\trans `I have eaten.' \label{prosodic phrase: verb: v is}
		\\ \armenian{Կերած եմ։}
		\ex \gll (\textbf{N}  {V} Aux)$_\phi$   \\
		bɑˈ\textbf{niɾ} ɡeˈ{ɾ-ɑd͡z} =e-m \\ \label{prosodic phrase: verb: n v is}
		\ex \glll  (\textbf{N})  (\textbf{V} Aux)$_\phi$    \\
		bɑˈ\textbf{niɾ} ɡeˈ\textbf{ɾ-ɑd͡z} =e-m  \\ 
		cheese {\fut} eat-{\thgloss}-1{\sg} \\ 
		\trans `I have eaten cheese.'   \label{prosodic phrase: verb: n v is deaccent}
		\\ \armenian{Պանիր կերած եմ։}
	\end{xlist}
\end{exe}

\subsubsection{Recursive layering in verb phrases}\label{section:intonation:phrase:verb:rec}
It is rather paradoxical that when a noun phrase is pronounced, the phrasal stress is on the final element; whereas in a verb phrase, the phrasal stress is not on the final verb. We see this paradox clearly for larger verb phrases. The above verb phrases were all two-word phrases, but the pre-verbal argument can of course take modifiers (\ref{prosodic phrease: verb: mod n}). In such cases, phrasal stress stays on the pre-verbal item. 

\begin{exe}
	\ex \label{prosodic phrease: verb: mod n} 
	\begin{xlist}
		\ex \glll (Adj \textbf{N} V)$_\phi$ \\
		tʰeˈʁin ɑɡɾɑ-ˈ\textbf{neɾ} uˈn-i-$\emptyset$\\ 
		yellow tooth-{\pl} have-{\thgloss}-3{\sg} \\
		\trans `He has yellow teeth.'\label{prosodic phrease: verb: a npl v} \\ 
		\armenian{Դեղին ակռաներ ունի։}
		\ex \glll (Adj \textbf{N} V)$_\phi$ \\
		tʰeˈʁin  ɑɡˈ\textbf{ɾɑ}-mə uˈn-i-$\emptyset$\\ 
		tʰeˈʁin  tooth-{\indf} have-{\thgloss}-3{\sg} \\
		\trans `He has a yellow tooth.' \\ 
		\armenian{Դեղին ակռայ մը ունի։}
		
\end{xlist}\end{exe}

Syntactically, the above sentences consist of a noun phrase object inside a verb phrase; we use relatively simple trees and glosses for illustration. Prosodically, it is tempting to argue that such a sentence consists of actually two recursive layers of prosodic phrases (PPh), as in Representation \ref{rep:intonation:phrase:verb:rec}. One small prosodic phrase is created from the NP; this small prosodic phrase is contained in a larger prosodic phrase that is created from the VP. Note that PW is a prosodic word.  The PW with phrasal stress is in bold. 

\begin{representation}
	Hypothetical flat vs. recursive prosodic structure in a verb phrase\label{rep:intonation:phrase:verb:rec} 
	
	\begin{tabular}{|l|ll lll|}
		\hline 
		& \multicolumn{2}{l}{\textbf{N} + V (\ref{prosodic phrease: verb: npl v})}&  \multicolumn{3}{l|}{Adj + \textbf{N} + V (\ref{prosodic phrease: verb: a npl v})}
		\\
		\hline    Syntax & & & &&  \\& 
		\multicolumn{2}{l}{ 	\begin{tikzpicture}[scale = .9]
				\Tree [.VP  [.NP [.N ɑɡɾɑneɾ ] ] [ [.V uni ] ] ]
			\end{tikzpicture}
		}
		& 
		\multicolumn{3}{l|}{
			\begin{tikzpicture}[scale = .9]
				\Tree [.VP  [.NP [.Adj tʰeʁin ] [.N ɑɡɾɑ-neɾ ] ] [  [.V uni ] ] ]
			\end{tikzpicture}
		}
		\\ \hline
		Flat prosody && & & &  \\& 
		\multicolumn{2}{l}{
			\begin{tikzpicture}[scale = 1]
				\Tree [.PPh  [.\textbf{PW}  \edge[roof];ɑɡɾɑˈ\textbf{neɾ} ]  [.PW \edge[roof];uˈni ] ]
			\end{tikzpicture}
		}
		& 
		\multicolumn{3}{l|}{
			\begin{tikzpicture}[scale = 1]
				\Tree [.PPh  [.PW  \edge[roof];tʰeˈ{ʁin} ] [.\textbf{PW}  \edge[roof];ɑɡɾɑˈ\textbf{neɾ} ]  [.PW \edge[roof];uˈni ] ]
			\end{tikzpicture}
		}
		\\ \hline
		Recursive prosody &&& & &   \\ & 
		\multicolumn{2}{l}{
			\begin{tikzpicture}[scale = 1]
				\Tree [.PPh [.PPh  [.\textbf{PW}  \edge[roof];ɑɡɾɑˈ\textbf{neɾ} ]   ] [ [.PW \edge[roof];uˈni ] ] ]
			\end{tikzpicture}
		}
		& 
		\multicolumn{3}{l|}{
			\begin{tikzpicture}[scale = 1]
				\Tree [.PPh [.PPh  [.PW  \edge[roof];tʰeˈ{ʁin} ] [.\textbf{PW}  \edge[roof];ɑɡɾɑˈ\textbf{neɾ} ]   ] [ [.PW \edge[roof];uˈni ] ] ]
			\end{tikzpicture}
		}
		\\
		\hline 
		& {}[ɑɡɾɑˈ\textbf{neɾ} &  uˈni]& {}[tʰeˈʁin & ɑɡɾɑˈ\textbf{neɾ} & uˈni]
		\\
		& teeth & has & yellow&  teeth &  has 
		\\
		&  \multicolumn{2}{l}{`He has teeth'}& \multicolumn{3}{l|}{ `He has yellow teeth'} 
		\\
		\hline 
		
	\end{tabular}
\end{representation}

Conceptually, the recursive structure is more appealing because it provides a more unified representation of prosodic phrases. If phrasal stress is present in a prosodic phrase, then it is always at the right edge of the phrase. But empirically, it is an open question if there is acoustic evidence for or against this recursive structure. The two structures are distinguished by the presence of a phrase right-boundary )$_\phi$ between the noun and verb. Future acoustic evidence can be used to see if such a boundary truly exists. 


The use of recursive structure is appealing for       verb phrases, because here we see that phrasal stress is not at the right edge. Similar behavior is found with  complex predicates that are   made up of a word +  a light verb like `to do'; these are also discussed in \S\ref{section:intonation:broadFocus:complex}. 

This preverbal item can be a meaningless word like [mədiɡ]. This preverbal item gets stress if no object is present (\ref{prosodic phrase:verb: complex pred:xv}). Adding an object shifts stress to the object (\ref{prosodic phrase:verb: complex pred:nxv}). This object can have its own modifiers, and again we see stress on the object (\ref{prosodic phrase:verb: complex pred:anxv}). 


\begin{exe}
	\ex 
	\begin{xlist}
		\ex \glll (\textbf{X} V)$_\phi$ \\
		məˈ\textbf{diɡ} ɡ-əˈn-e-m\\ 
		X  {\ind}-do-{\thgloss}-1{\sg} \\
		\trans `I listen.' \label{prosodic phrase:verb: complex pred:xv}\\ 
		\armenian{Մտիկ կ՚ընեմ։} 
		\ex \glll (\textbf{N} X V)$_\phi$ \\
		jeɾkʰ-e\textbf{ˈɾu} mə{diɡ} ɡ-əˈn-e-m\\ 
		song-{\pl}-{\dat} X  {\ind}-do-{\thgloss}-1{\sg} \\
		\trans `I listen to songs.'\label{prosodic phrase:verb: complex pred:nxv} \\ 
		\armenian{Երգերու մտիկ կ՚ընեմ։} 
		\ex \glll (Adj \textbf{N} X V)$_\phi$ \\
		ˈhin jeɾkʰ-eˈ\textbf{ɾ-u} mə{diɡ} ɡ-əˈn-e-m\\ 
		old song-{\pl}-{\dat} X  {\ind}-do-{\thgloss}-1{\sg} \\
		\trans `I listen to old songs.' \label{prosodic phrase:verb: complex pred:anxv}\\ 
		\armenian{Հին երգերու մտիկ կ՚ընեմ։} 
\end{xlist}\end{exe}

Note that   in the rather large verb phrase in (\ref{prosodic phrase:verb: complex pred:anxv}), HD perceives some level of prominence on the preverb [məˈdiɡ], but the stress on the object is still stronger. 

As the size of the verb phrase goes from 2 to 4, we see that phrasal stress keeps shifting leftwards until it reaches the noun, and doesn't move further. Such iterative changes make sense if we again assumed that the syntax treated a complex predicate as some XP + V, and if the prosody created a recursive prosodic structure (Representation \ref{rep:intonation:phrase:verb:recrec}). 

\begin{representation}
	Hypothetical flat vs. recursive prosodic structure in a verb phrase with a complex predicate \label{rep:intonation:phrase:verb:recrec}
	
	\resizebox{\textwidth}{!}{%
		\begin{tabular}{|l|ll lll llll|}
			\hline 
			& \multicolumn{2}{l}{\textbf{X} + V (\ref{prosodic phrase:verb: complex pred:xv}) }&\multicolumn{3}{l}{    \textbf{N} + X +  V (\ref{prosodic phrase:verb: complex pred:nxv})}& \multicolumn{4}{l|}{Adj + \textbf{N} + X+  V  (\ref{prosodic phrase:verb: complex pred:anxv})}
			\\
			\hline    Syntax & & & &  & &&  & &  \\& 
			\multicolumn{2}{l}{
				\begin{tikzpicture}[scale = 1]
					\Tree [.VP  [.XP [.X mədiɡ ] ] [ [.V ɡənem ] ] ]
				\end{tikzpicture}
			}
			& 
			\multicolumn{3}{l}{
				\begin{tikzpicture}[scale = 0.8]
					\Tree [.VP  [.XP [.NP [.N jeɾkʰeɾu ] ] [ [.X mədiɡ ] ] ] [ [ [.V ɡənem ] ] ] ]
				\end{tikzpicture}
			}
			& 
			\multicolumn{4}{l|}{
				\begin{tikzpicture}[scale =  .8]
					\Tree [.VP  [.XP [.NP [.Adj hin ] [.N jeɾkʰeɾu ] ] [ [.X mədiɡ ] ] ] [ [ [.V ɡənem ] ] ] ]
				\end{tikzpicture}
			}
			\\ \hline
			Flat prosody && & &&  &&  &&  \\& 
			\multicolumn{2}{l}{
				\begin{tikzpicture}[scale = .8]
					\Tree [.PPh  [.\textbf{PW}  \edge[roof];məˈ\textbf{diɡ} ]  [.PW \edge[roof];ɡəˈnem ] ]
				\end{tikzpicture}
			}
			& 
			\multicolumn{3}{l}{
				\begin{tikzpicture}[scale = .8 ]
					\Tree [.PPh  [.\textbf{PW}  \edge[roof];jeɾkʰeˈ\textbf{ɾu} ] [.{PW}  \edge[roof];məˈ{diɡ} ]  [.PW \edge[roof];ɡəˈnem ] ]
				\end{tikzpicture}
			}
			& 
			\multicolumn{4}{l|}{
				\begin{tikzpicture}[scale = .8 ]
					\Tree [.PPh  [.{PW}  \edge[roof];ˈhin ]  [.\textbf{PW}  \edge[roof];jeɾkʰeˈ\textbf{ɾu} ] [.{PW}  \edge[roof];məˈ{diɡ} ]  [.PW \edge[roof];ɡəˈnem ] ]
				\end{tikzpicture}
			}
			\\ \hline
			Recur.  pros.  && &  & &&  &&  \\ & 
			\multicolumn{2}{l}{
				\begin{tikzpicture}[scale = .8]
					\Tree [.PPh [.PPh [.\textbf{PW}  \edge[roof];məˈ\textbf{diɡ} ]  ]  [ [.PW \edge[roof];ɡəˈnem ] ] ]
				\end{tikzpicture}
			}
			& 
			\multicolumn{3}{l}{
				\begin{tikzpicture}[scale = .8 ]
					\Tree [.PPh [.PPh [.PPh [.\textbf{PW}  \edge[roof];jeɾkʰeˈ\textbf{ɾu} ] ]  [  [.{PW}  \edge[roof];məˈ{diɡ} ] ] ]  [  [   [.PW \edge[roof];ɡəˈnem ] ]  ] ]
				\end{tikzpicture}
			}
			& 
			\multicolumn{4}{l|}{
				\begin{tikzpicture}[scale = .8 ]
					\Tree [.PPh [.PPh [.PPh [.{PW}  \edge[roof];ˈhin ]   [.\textbf{PW}  \edge[roof];jeɾkʰeˈ\textbf{ɾu} ] ]  [  [.{PW}  \edge[roof];məˈ{diɡ} ] ] ]  [  [   [.PW \edge[roof];ɡəˈnem ] ]  ] ]
				\end{tikzpicture}
			}
			\\\hline 
			& {}[meˈ\textbf{diɡ}&  ɡəˈnem]& {}[jeɾkʰeˈ\textbf{ɾu} &meˈ{diɡ} & ɡəˈnem] & {}[ˈhin & jeɾkʰeˈ\textbf{ɾu} &  meˈ{diɡ} &ɡəˈnem]
			\\
			& X& do& songs& X& do& old& songs& X & do
			\\
			& \multicolumn{2}{l}{`I listen'} & \multicolumn{3}{l}{`I listen to songs'} &\multicolumn{4}{l|}{`I listen to old songs'  }
			\\
			\hline 
			
		\end{tabular}
	}
\end{representation}



Similar evidence for recursive structuring is found in some periphrastic tenses. In some tenses, the verb is a participle, while inflection is on some light verb (\ref{prosodic phrase:v: v ellam: base}). Some tenses combine proclitics, participles, and light verbs (\ref{prosodic phrase:v: bidi v ellam: base}). When no object is present, stress is on the first verb (the participle).

\begin{exe}
	\ex \begin{xlist}
		\ex \glll  (\textbf{V})$_\phi$ (Comp \textbf{V} lightV)$_\phi$ \\ 
		ɡ-uˈ\textbf{z-e-m} voɾ ɡeˈ\textbf{ɾ-ɑd͡z} əlˈl-ɑ-m \\
		{\ind}-want{\thgloss}-1{\sg} that eat.{\aorperf}-{\rptcp} be-{\thgloss}-1{\sg} \\
		\trans `I want to have eaten.'  \label{prosodic phrase:v: v ellam: base} 
		\\ \armenian{Կ՚ուզեմ որ կերած ըլլամ։}
		\ex \glll  (Pro \textbf{V} lightV)$_\phi$ \\ 
		bidi ɡeˈ\textbf{ɾ-ɑd͡z} əlˈl-ɑ-m \\
		{\fut}   eat.{\aorperf}-{\rptcp} be-{\thgloss}-1{\sg} \\
		\trans `I will have eaten.'  \label{prosodic phrase:v: bidi v ellam: base} 
		\\ \armenian{Պիտի   կերած ըլլամ։}
	\end{xlist}
\end{exe}


If an object is present, this object is usually   in a separate prosodic phrase (\ref{prosodic phrase:v: v ellam: obj strnd}); the semantics of these complex tenses often imply that the object is some type of given information. 


\begin{exe}
	\ex \label{prosodic phrase:v: v ellam: obj strnd} \begin{xlist}
		\ex \glll  (\textbf{V})$_\phi$ Comp  \textbf{N})$_\phi$ (\textbf{V} lightV)$_\phi$ \\ 
		ɡ-uˈ\textbf{z-e-m} voɾ bɑˈ\textbf{ni}ɾ-ə ɡeˈ\textbf{ɾ-ɑd͡z} əlˈl-ɑ-m \\
		{\ind}-want{\thgloss}-1{\sg} that cheese-{\defgloss} eat.{\aorperf}-{\rptcp} be-{\thgloss}-1{\sg} \\
		\trans `I want to have eaten eaten.'  \label{prosodic phrase:v: v ellam: obj} 
		\\ \armenian{Կ՚ուզեմ որ պանիրը կերած ըլլամ։}
		\ex \glll  (\textbf{N})$_\phi$ (Pro \textbf{V} lightV)$_\phi$ \\ 
		bɑˈ\textbf{ni}ɾ-ə  bidi ɡeˈ\textbf{ɾ-ɑd͡z} əlˈl-ɑ-m \\
		cheese-{\defgloss} {\fut}   eat.{\aorperf}-{\rptcp} be-{\thgloss}-1{\sg} \\
		\trans `I will have eaten the cheese.'  \label{prosodic phrase:v: bidi v ellam: obj} 
		\\ \armenian{Պանիրը պիտի   կերած ըլլամ։}
	\end{xlist}
\end{exe}

The participle  data would make sense if we treat the Verb-Verb sequences as a verb phrase inside a verb phrase. The variable deaccenting of the object is due to the pragmatics on how such complex tenses are used. 

In sum, verb phrases place phrasal stress on the argument of the verb. This means that the prosodic phrase of a verb phrase does not have final stress; instead stress is on the preverbal item in most cases. Complex predicates can push this preverbal stress further leftwards. Such leftward shifting makes sense if we use recursive prosodic structures; but acoustic data is needed to verify such recursive structures. We stay agnostic for now. 


\subsection{Phrasal stress when noun phrases and verb phrases combine}\label{section:intonation:phrase:nounverb}
The previous sections established the prosodic phrase of a noun phrase is right-head (final stress), while the prosodic phrase of a verb phrase has stress on the non-final word. This section looks at how two types of phrases can be combined together. Different combinations cause different locations of phrasal stress. These combinations are subject-object clashes (\S\ref{section:intonation:phrase:nounverb:subObj}), nominalized infinitives (\S\ref{section:intonation:phrase:nounverb:inf}), nominalized participles (\S\ref{section:intonation:phrase:nounverb:participle}), and combinations of infinitives with finite verbs (\S\ref{section:intonation:phrase:nounverb:caus}).   

\subsubsection{Stress clash between subjects and objects}\label{section:intonation:phrase:nounverb:subObj}

When a sentence has both a subject NP and a  transitive verb phrase (\ref{prosodic phrase: noun verb: (a) s o v}), the two phrases are each turned into their own prosodic phrase. The phrasal stress of the verb phrase is perceived as the strongest stress of the sentence, i.e., as sentence stress or nuclear stress. Note that for simplicity,   we use flat prosodic phrases for the verb phrases in this section. 

\begin{exe}
	\ex \label{prosodic phrase: noun verb: (a) s o v} \begin{xlist}
		\ex \glll (\textbf{S})$_\phi$ (\textbf{N} V)$_\phi$ \\
		ɡɑˈ\textbf{du-n} muˈ\textbf{ɡ-eɾ} ɡ-uˈd-e-$\emptyset$ \\
		cat-{\defgloss} mouse-{\pl} {\ind}-eat-{\thgloss}-3{\sg} \\
		\trans `The cat eats mice.'
		\\ \armenian{Կատուն մուկեր կ՚ուտէ։}
		\ex \glll (Adj \textbf{S})$_\phi$ (\textbf{N} V)$_\phi$ \\
		ɑnoˈtʰi ɡɑˈ\textbf{du-n} muˈ\textbf{ɡ-eɾ} ɡ-uˈd-e-$\emptyset$ \\
		hungry cat-{\defgloss} mouse-{\pl} {\ind}-eat-{\thgloss}-3{\sg} \\
		\trans `The hungry cat eats mice.'
		\\ \armenian{Անօթի կատուն մուկեր կ՚ուտէ։}
	\end{xlist}
\end{exe}

A slight pause is perceptible between the subject and the verb phrase. If the object has no modifier (\ref{prosodic phrase: noun verb: (a) s o v}), then we see a stress clash between the subject and the object. If the object has a modifier (\ref{prosodic phrase: noun verb: (a) s a o v}), then there is no stress clash. 


\begin{exe}
	\ex \label{prosodic phrase: noun verb: (a) s a o v} \begin{xlist}
		\ex \glll (\textbf{S})$_\phi$ (Adj  \textbf{N} V)$_\phi$ \\
		ɡɑˈ\textbf{du-n} d͡ʒeɾˈmɑɡ muˈ\textbf{ɡ-eɾ} ɡ-uˈd-e-$\emptyset$ \\
		cat-{\defgloss} white mouse-{\pl} {\ind}-eat-{\thgloss}-3{\sg} \\
		\trans `The cat eats white mice.'
		\\ \armenian{Կատուն ճերմակ մուկեր կ՚ուտէ։}
		\ex \glll (Adj \textbf{S})$_\phi$ (Adj  \textbf{N} V)$_\phi$ \\
		ɑnoˈtʰi ɡɑˈ\textbf{du-n} d͡ʒeɾˈmɑɡ muˈ\textbf{ɡ-eɾ} ɡ-uˈd-e-$\emptyset$ \\
		hungry cat-{\defgloss} white mouse-{\pl} {\ind}-eat-{\thgloss}-3{\sg} \\
		\trans `The hungry cat eats white mice.'
		\\ \armenian{Անօթի կատուն ճերմակ մուկեր կ՚ուտէ։}
	\end{xlist}
\end{exe}

\subsubsection{Noun-based vs. verb-based stress in infinitival phrases}\label{section:intonation:phrase:nounverb:inf}
Verb phrases avoid placing stress on the verb. The cases so far focused on finite verb phrases (\ref{prosodic phrase: noun verb: s n v contrast}), meaning verb phrases which had some subject agreement or tense marking. Armenian likewise allows a verb phrase to lack any inflection, such as an infinitival phrase (\ref{prosodic phrase: noun verb: v n vinf}). Such infinitival phrases can be used as   the complement of a verb like `to like'. Such uses of infinitival phrases are analogous to English gerunds, as the translations show. 

\begin{exe}
	\ex \begin{xlist}
		\ex \glll (\textbf{N})$\phi$ (\textbf{N} V)$_\phi$ \\ 
		dəˈ\textbf{ʁɑ}-kʰ-ə bəˈ\textbf{nɑɡ} ɡə-ləˈv-ɑ-n \\
		boy-{\pl}-{\defgloss} plate {\ind}-wash-{\thgloss}-3{\pl} \\
		\trans `The boys wash dishes.'\label{prosodic phrase: noun verb: s n v contrast} 
		\\ \armenian{Տղաքը բնակ կը լուան։}
		\ex \glll (\textbf{V})$_\phi$ (\textbf{N} V-{\infgloss})$_\phi$ \\ 
		ɡə-siˈ\textbf{ɾ-e-m} bəˈ\textbf{nɑɡ} ləˈv-ɑ-l \\
		{\ind}-like-{\thgloss}-1{\sg}  plate wash-{\thgloss}-{\infgloss} \\
		\trans `I like washing dishes.'  \label{prosodic phrase: noun verb: v n vinf} 
		\\ \armenian{Կը սիրեմ բնակ լուալ։}
	\end{xlist}
\end{exe}

Note that when the infinitival phrase is selected by a verb like `to like' (\ref{prosodic phrase: noun verb: v n vinf}), the verb forms one prosodic phrase while the infinitival forms another. 

Verb phrases and infinitival phrases have analogous prosodic phrases. Both place stress on the non-verbal element by default (\ref{prosodic phrase: noun verb: s n v contrast}). However, infinitives can take nominal inflection, such as the definite suffix (\ref{prosodic phrase: noun verb: v inf infl ə}) or case suffixes (\ref{prosodic phrase: noun verb: v inf infl case}). When they do, the infinitival phrase acts like a noun phrase, and its prosodic phrase places stress on the inflected infinitive. 

\begin{exe}
	\ex \begin{xlist}
		\ex \glll  (N \textbf{V-{\infgloss}-Infl})$_\phi$ \\ 
		bəˈ{nɑɡ}  ləˈ\textbf{v-ɑ}-l-ə \\
		plate  wash-{\thgloss}-{\infgloss}-{\defgloss} \\
		\trans `the washing of plates'\label{prosodic phrase: noun verb: v inf infl ə} 
		\\ \armenian{բնակ լուալը} 
		\ex \glll  (N \textbf{V-{\infgloss}-Infl})$_\phi$ \\ 
		bəˈ{nɑɡ}  lə{v-ɑ}-ˈ\textbf{l-e-n} \\
		plate  wash-{\thgloss}-{\infgloss}-{\abl}-{\defgloss} \\
		\trans `from the washing of plates'\label{prosodic phrase: noun verb: v inf infl case} 
		\\ \armenian{բնակ լուալէն} 
	\end{xlist}
\end{exe}

Such inflected infinitival phrases are noun-like for both the prosody and the syntax. Such phrases can be used as subjects (\ref{prosodic phrase: noun verb: v inf infl ə adj v}) or other arguments/adjuncts (\ref{prosodic phrase: noun verb: v inf infl case v}). 

\begin{exe}
	\ex \begin{xlist}
		\ex \glll  (N \textbf{V-{\infgloss}-Infl})$_\phi$ (\textbf{Adj} V)$_\phi$\\ 
		bəˈ{nɑɡ}  ləˈ\textbf{v-ɑ}-l-ə ɡɑɾeˈ\textbf{voɾ} =e\\
		plate  wash-{\thgloss}-{\infgloss}-{\defgloss} important =is \\
		\trans `Washing  plates is important.'\label{prosodic phrase: noun verb: v inf infl ə adj v} 
		\\ \armenian{Բնակ լուալը կարեւոր է։} 
		\ex \glll  (N \textbf{V-{\infgloss}-Infl})$_\phi$ (Adv V)$_\phi$\\ 
		bəˈ{nɑɡ}  lə{v-ɑ}-ˈ\textbf{l-e-n}  ɑˈ\textbf{ɾɑkʰ} ɡə-zəzˈv-i-m\\
		plate  wash-{\thgloss}-{\infgloss}-{\abl}-{\defgloss} fast {\ind}-sick.of-{\thgloss}-1{\sg}\\
		\trans `I quickly get sick of washing dishes. \label{prosodic phrase: noun verb: v inf infl case v} 
		\\ \armenian{Բնակ լուալէն արագ կը զզուիմ։} 
	\end{xlist}
\end{exe}

To summarize,  when an infinitival phrase lacks any nominal inflection on the infinitive, then the phrase is pronounced as if it were a verb phrase. Phrasal stress is on the object, not the verb. But if the infinitival phrase gets nominal inflection on the infinitive, then the phonology treats this phrase like a noun phrase. Phrasal stress is on the verb. 
\subsubsection{Variable retention of preverbal stress in participle  clauses}\label{section:intonation:phrase:nounverb:participle}

Armenian syntax allows turning verb phrases into participle phrases, whether subject participles with \textit{-oʁ} or resultative participles with \textit{-ɑd͡z}. Such participle phrases sometimes maintain the stress patterns of the original verb phrase, but not always. 


Consider the finite verb phrase in (\ref{ex:prosodic phrase: participle: ogh: base vp}) where the verb has a stressed direct object. In  (\ref{ex:prosodic phrase: participle: ogh: v n}), the verb is   turned into a subject participle with \textit{-oʁ}, and the participle then acts as a pre-nominal modifier.  The original subject becomes the head noun, while the participle acts like an adjective and does not get phrasal stress.

\begin{exe}
	\ex \begin{xlist}
		\ex \glll (S)$_\phi$ (\textbf{O} V)$_\phi$ \\
		d͡zɑˈ\textbf{ɾɑ-n}    bəˈ\textbf{nɑɡ} ɡə-mɑkʰˈɾ-e-$\emptyset$\\
		servant-{\defgloss} plate {\ind}-clean-{\thgloss}-3{\sg} \\
		\trans `The servant washes plates.' \label{ex:prosodic phrase: participle: ogh: base vp} 
		\\ \armenian{Ծառան պնակ կը մաքրէ։}
		\ex \glll  (V-{\sptcp} \textbf{N})$_\phi$
		\\
		{mɑkʰˈɾ-oʁ}  {d͡zɑˈ\textbf{ɾɑ-n}}
		\\
		clean-{\sptcp} servant-{\defgloss}
		\\
		\trans `The cleaning servant  (the servant who cleans).'\label{ex:prosodic phrase: participle: ogh: v n} 
		\\ \armenian{մաքրող ծառան}
		
	\end{xlist}
\end{exe}

If the participle retains its direct object (\ref{ex:prosodic phrase: participle: ogh: base vp: o v n}), the direct object can optionally have its own phrasal stress too.  

\begin{exe}
	\ex \label{ex:prosodic phrase: participle: ogh: base vp: o v n}  \begin{xlist}
		\ex \gll (\textbf{N})$_\phi$ (V-{\sptcp} \textbf{N})$_\phi$
		\\
		{{bəˈ\textbf{nɑɡ}}} {mɑkʰˈɾ-oʁ} {{d͡zɑˈ\textbf{ɾɑ-n}}}\\
		plate clean-{\sptcp} servant-{\defgloss}
		\ex \glll (N V-{\sptcp} \textbf{N})$_\phi$
		\\
		{{bəˈ{nɑɡ}}} {mɑkʰˈɾ-oʁ} {{d͡zɑ\textbf{ɾɑ-n}}}
		\\
		plate clean-{\sptcp} servant-{\defgloss}
		\\
		\trans `The plate-cleaning servant (the servant who cleans plates).' 
		\\ \armenian{պնակ մաքրող ծառան}
	\end{xlist}
\end{exe}

If the head noun is deleted (\ref{example:inton:phrase:nounverb:oG:lone}), then the participle carries nominal inflection such as the definite suffix. The entire construction is again treated as a noun phrase with final stress. 

\begin{exe}
	\ex  \glll (N V-{\sptcp}-Infl)$_\phi$
	\\
	{{bəˈ{nɑɡ}}} {mɑkʰˈ\textbf{ɾ-o}ʁ-ə}
	\\
	plate clean-{\sptcp}-{\defgloss}
	\\
	\trans `The plate-cleaner  (the person who cleans plates).'   \label{example:inton:phrase:nounverb:oG:lone}
	\\ \armenian{պնակ մաքրողը}
\end{exe}

In contrast, the resultative participle   uses the suffix \textit{-ɑd͡z} (\ref{ex:prosodic phrase: participle: adz: base vp: o v n}). The original object becomes the head noun, while the original subject is a genitive possessor. The prosody of such constructions seems identical to normal noun phrases. If the genitive subject is unmodified (\ref{ex:prosodic phrase: participle: adz: base vp: o v n}), then we have one prosodic phrase; else we have two phrases (\ref{ex:prosodic phrase: participle: adz: base vp: a o v n}). 

\begin{exe}
	\ex \begin{xlist}
		\ex \glll (N-{\gen} V-{\rptcp} \textbf{N})$_\phi$
		\\
		{{dzɑɾɑˈ{j-i-n}}}  {mɑkʰˈɾ-ɑd͡z} {{bəˈ\textbf{nɑ}ɡ-ə}}
		\\
		servant-{\gen}-{\defgloss}  clean-{\rptcp} plate-{\defgloss} 
		\\
		\trans `the plate that the servant cleaned' \\
		Literary:  `the servant's cleaned plate'  \label{ex:prosodic phrase: participle: adz: base vp: o v n} 
		\\ \armenian{ծառային մաքրած պնակը}
		\ex \glll (A \textbf{N-{\gen}})$_\phi$ (V-{\rptcp} \textbf{N})$_\phi$
		\\
		t͡ʃəˈʁɑjn  {{dzɑɾɑˈ\textbf{j-i-n}}}   {mɑkʰˈɾ-ɑd͡z} {{bəˈ\textbf{nɑ}ɡ-ə}}
		\\
		angry servant-{\gen}-{\defgloss}  clean-{\rptcp} plate-{\defgloss} 
		\\
		\trans `the plate that the angry servant cleaned' \\
		Literary:  `the angry servant's cleaned plate'  \label{ex:prosodic phrase: participle: adz: base vp: a o v n} 
		\\ \armenian{ջղայն ծառային մաքրած պնակը}
	\end{xlist}
\end{exe}

The head noun can be deleted (\ref{ex:prosodic phrase: participle: adz: lone}), with nominal inflection on the participle. The construction gets final stress again. 


\begin{exe}
	\ex \label{ex:prosodic phrase: participle: adz: lone}
	\begin{xlist}
		\ex \glll (N-{\gen} V-{\rptcp}-Inf)$_\phi$
		\\
		{{dzɑɾɑˈ{j-i-n}}}  {mɑkʰˈ\textbf{ɾ-ɑ}d͡z-ə} 
		\\
		servant-{\gen}-{\defgloss}  clean-{\rptcp}-{\defgloss} 
		\\
		\trans `the thing that the servant cleaned' \\
		Literary:  `the servant's cleaned one'   
		\\ \armenian{ծառային մաքրածը}
		\ex \glll (A {N-{\gen}} (V-{\rptcp}-Inf)$_\phi$
		\\
		t͡ʃəˈʁɑjn  {{dzɑɾɑˈ{j-i-n}}}   {mɑkʰˈ\textbf{ɾ-ɑ}d͡z-ə} 
		\\
		angry servant-{\gen}-{\defgloss}  clean-{\rptcp}-{\defgloss} 
		\\
		\trans `the thing that the angry servant cleaned' \\
		Literary:  `the angry servant's cleaned one'  
		\\ \armenian{ջղայն ծառային մաքրածը}
	\end{xlist}
\end{exe}

It would be useful in the future to contrast the prosody of such structures against Turkish. Turkish likewise has participle clauses with subtle pronunciation rules \textcolor{red}{check gunes i think}. 


\subsubsection{Stress on verbs in verb-verb sequences}\label{section:intonation:phrase:nounverb:caus}
Simple verb phrases are generally pronounced with stress on the preverbal object. But certain sentence structures can combine an infinitive with a finite verb (analytical causatives and control verbs). In these  cases,    this simple generalization breaks down and we see stress on the embedded verb. 


Consider the basic transitive sentence in (\ref{ex: prosodic phrase: caus: base}) with one subject and one object. To create a causative meaning (\ref{ex: prosodic phrase: caus: caus: o}), Armenian uses an analytical construction where     the subject is turned dative, the verb is replaced by an infinitive, and then the verb `give' is added. 

\begin{exe}
	\ex \begin{xlist}
		\ex \glll (\textbf{S})$_\phi$ (\textbf{O} V)$_\phi$ \\
		mɑɾˈ\textbf{jɑ-n} nɑˈ\textbf{mɑɡ} ɡə-ɡɑɾˈtʰ-ɑ-$\emptyset$ \\
		Maria-{\defgloss} letter {\ind}-read-{\thgloss}-3{\sg} \\
		\trans `Maria reads letters.' \label{ex: prosodic phrase: caus: base} 
		\\ \armenian{Մարիան նամակ կը կարդայ։}
		\ex \glll (\textbf{S})$_\phi$  (\textbf{IO})$_\phi$  ({O} \textbf{V-{\infgloss}} V)$_\phi$ \\ 
		ɑ\textbf{ɾɑ-n} mɑɾjɑ-ˈ\textbf{ji-n} nɑˈ{mɑɡ} ɡɑɾˈ\textbf{tʰ-ɑ-l} ɡu-ˈd-ɑ-$\emptyset$ \\
		Ara-{\defgloss} Maria-{\dat}-{\defgloss} letter  read-{\thgloss}-{\infgloss}  {\ind}-give-{\thgloss}-3{\sg} \\
		\trans `Ara makes Maria read letters.'  \label{ex: prosodic phrase: caus: caus: o} 
		\\ \armenian{Արան Մարիային նամակ  կարդալ կու տայ։}
	\end{xlist}
\end{exe}

The object + infinitive sequence acts as an argument of `give'. Thus the the OVV sequence is parsed (O\textbf{V}V) with stress on the embedded verb. One could hypothesize that such a sequence involves a recursive prosodic structure: ((O\textbf{V})V). In HD's judgment, if the object has more inflectional material, then it feels common to parse the object as a separate phrase (\ref{ex: prosodic phrase: caus: caus: o pl rephrase}). 

\begin{exe}
	\ex  \label{ex: prosodic phrase: caus: caus: o pl rephrase} 
	\begin{xlist}
		\ex   \glll (\textbf{S})$_\phi$  (\textbf{IO})$_\phi$  (\textbf{O}  \textbf{V-{\infgloss}} V)$_\phi$ \\ 
		ɑ\textbf{ɾɑ-n}  mɑɾjɑ-ˈ\textbf{ji-n} nɑmɑɡ-ˈ{neɾ}  ɡɑɾˈ\textbf{tʰ-ɑ-l} ɡu-ˈd-ɑ-$\emptyset$ \\ 
		Ara-{\defgloss} Maria-{\dat}-{\defgloss} letter-{\pl}  read-{\thgloss}-{\infgloss}  {\ind}-give-{\thgloss}-3{\sg} \\ 
		\ex \glll (\textbf{S})$_\phi$  (\textbf{IO})$_\phi$  (\textbf{O})$_\phi$ (\textbf{V-{\infgloss}} V)$_\phi$ \\ 
		ɑ\textbf{ɾɑ-n}  mɑɾjɑ-ˈ\textbf{ji-n} nɑmɑɡ-ˈ\textbf{neɾ}  ɡɑɾˈ\textbf{tʰ-ɑ-l} ɡu-ˈd-ɑ-$\emptyset$ \\
		Ara-{\defgloss} Maria-{\dat}-{\defgloss} letter-{\pl}  read-{\thgloss}-{\infgloss}  {\ind}-give-{\thgloss}-3{\sg} \\
		\trans `Ara makes Maria read letters.'  
		\\ \armenian{Արան Մարիային նամակներ  կարդալ կու տայ։}
		
	\end{xlist}
\end{exe}

Thus, causativization turns the prosody of embedded verb phrases into something similar to the prosody of noun phrases. 

Similar transformations are found for sentences where a verb is a control verb like `want' and it selects an infinitival phrase. The basic order is finite-object-infinitive (\ref{ex:inton:phrase:verb:mixed:vinfvfin:foi}) with stress on the object. But other attested orders are object-finite-infinitive (\ref{ex:inton:phrase:verb:mixed:vinfvfin:ofi}) and object-infinitive-finite (\ref{ex:inton:phrase:verb:mixed:vinfvfin:oif}). In these latter cases, stress is on the pre-finite word. 

\begin{exe}
	\ex \begin{xlist}
		\ex \glll (\textbf{V-fin})$_\phi$ (\textbf{O} V-inf)$_\phi$  \\
		ɡ-uˈ\textbf{z-e-m} nɑˈ\textbf{mɑɡ} ɡɑɾˈtʰ-ɑ-l \\
		{\ind}-want-{\thgloss}-1{\sg} letter read-{\thgloss}-{\infgloss} \\
		\trans `I want to read letters.' \label{ex:inton:phrase:verb:mixed:vinfvfin:foi} \\
		\armenian{Կ՚ուզեմ նամակ կարդալ։}
		\ex \glll (\textbf{O} {V-fin} {V-inf})$_\phi$  \\
		nɑˈ\textbf{mɑɡ} ɡ-uˈ{z-e-m}  ɡɑɾˈ{tʰ-ɑ-l} \\
		letter {\ind}-want-{\thgloss}-1{\sg}  read-{\thgloss}-{\infgloss} \\
		\armenian{Նամակ կ՚ուզեմ  կարդալ։} \label{ex:inton:phrase:verb:mixed:vinfvfin:ofi} 
		\ex \glll ({O} \textbf{V-inf}  V-fin)$_\phi$  \\
		nɑˈ{mɑɡ}  ɡɑɾˈ\textbf{tʰ-ɑ-l} ɡ-uˈ{z-e-m}  \\
		letter  read-{\thgloss}-{\infgloss} {\ind}-want-{\thgloss}-1{\sg}  \\
		\armenian{Նամակ  կարդալ կ՚ուզեմ։} \label{ex:inton:phrase:verb:mixed:vinfvfin:oif} 
	\end{xlist}
\end{exe}

\subsection{Phrasal stress in other constructions}\label{section:intonation:phrase:other}
The previous section looked at the core types of syntactic phrases: noun phrases, adpositional phrases, verb phrases, and their combinations. This sections looks at other types of syntactic phrases and their prosodic structure: adverbs (\ref{section:intonation:phrase:other:adv}), compound-like collocations (\ref{section:intonation:phrase:other:colloc}), and reduplication (\ref{section:intonation:phrase:other:redup}). Such phrases don't fit neatly into the previous categories in terms of their syntax or stress. 

\subsubsection{Stress in adverbs}\label{section:intonation:phrase:other:adv}

Adverbs can cause peculiar changes to prosodic phrases.  It is rather difficult to make consistent generalizations on stress across all types of adverbs. Some adverbs induce their own special prosody because of their morphological or semantic properties. 

Some adverbs are generally pronounced as separate prosodic phrases (\ref{example:inton:phrase:other:adv:time}). Such adverbs include time adverbs whose meaning affects the general meaning of the verb.

\begin{exe}
	\ex \label{example:inton:phrase:other:adv:time} \begin{xlist}
		\ex \glll (\textbf{Adv})$_\phi$ (\textbf{V})$_\phi$ \\
		ɑjˈ\textbf{soɾ}   kʰəɾ-e-ˈ\textbf{t͡s-i-$\emptyset$} \\
		today   write-{\thgloss}-{\aorperf}-{\pst}-1{\sg} \\
		\trans `Today, I wrote (stuff).' 
		\\ \armenian{Այսօր գրեցի։}
		\ex \glll (\textbf{Adv})$_\phi$ (\textbf{O} V)$_\phi$ \\
		ɑjˈ\textbf{soɾ}  nɑˈ\textbf{mɑɡ} kʰəɾ-e-ˈt͡s-i-$\emptyset$ \\
		today letter write-{\thgloss}-{\aorperf}-{\pst}-1{\sg} \\
		\trans `Today, I wrote letters.' 
		\\ \armenian{Այսօր նամակ գրեցի։}
	\end{xlist}
\end{exe}

Manner adverbs come in two basic morphological types: simplex and complex. Simplex manner adverbs are just a root (\ref{example:inton:phrase:other:adv:mannerBase}). They generally are found preverbally, or before a bare object or locative. They attract phrasal stress away from the verb.


\begin{exe}
	\ex \label{example:inton:phrase:other:adv:mannerBase} \begin{xlist}
		\ex \glll (\textbf{Adv} V)$_\phi$ \\
		ɑ\textbf{ɾɑkʰ}  jeˈɡ-ɑ-$\emptyset$ \\
		quick   come.{\aorperf}-{\pst}-1{\sg} \\
		\trans `I quickly came.'
		\\ \armenian{Արագ եկայ։}
		\ex \glll (Adv  \textbf{Loc} V)$_\phi$ \\
		ɑ\textbf{ɾɑkʰ} ˈdun jeˈɡ-ɑ-$\emptyset$ \\
		quick   home come.{\aorperf}-{\pst}-1{\sg} \\
		\trans `I quickly came home.'
		\\ \armenian{Արագ  տուն եկայ։}
		\ex \glll (Adv   \textbf{O} V)$_\phi$ \\
		ɑ\textbf{ɾɑkʰ} ˈkʰiɾkʰ ɡə-ɡɑɾˈtʰ-ɑ-m \\
		quick   book {\ind}-read-{\thgloss}-1{\sg} \\
		\trans `I can quickly read books.'
		\\ \armenian{Արագ  գիրք կը կարդամ։}
\end{xlist}\end{exe}

In contrast, morphologically complex adverbs are instead parsed as their own separate prosodic phrase (\ref{example:inton:phrase:other:adv:mannerCOmp}).  


\begin{exe}
	\ex \label{example:inton:phrase:other:adv:mannerCOmp} \begin{xlist}
		\ex \glll (\textbf{Adv})$_\phi$ (\textbf{V})$_\phi$ \\
		ɑˈ\textbf{ɾɑ}k-oɾen   kʰəɾ-e-ˈ\textbf{t͡s-i-$\emptyset$} \\
		quick-{\advz}   write-{\thgloss}-{\aorperf}-{\pst}-1{\sg} \\
		\trans `Quickly, I wrote (stuff).' 
		\\ \armenian{Արագօրէն գրեցի։}
		\ex \glll (\textbf{Adv})$_\phi$ (\textbf{O} V)$_\phi$ \\
		ɑˈ\textbf{ɾɑ}k-oɾen   nɑˈ\textbf{mɑɡ} kʰəɾ-e-ˈt͡s-i-$\emptyset$ \\
		quick-{\advz} letter write-{\thgloss}-{\aorperf}-{\pst}-1{\sg} \\
		\trans `Quickly, I wrote letters.' 
		\\ \armenian{Արագօրէն նամակ գրեցի։}
		
\end{xlist}\end{exe}

Some adverbs like [ʃad] `many, very' induce special emphatic stress and naturally attract stress (\ref{example:inton:phrase:other:adv:shad}). This adverb can attract stress away from nouns, adjectives,  and verbs.  The special prosody of   [ʃɑd]    is likely because of its special semantics.


\begin{exe}
	\ex  \label{example:inton:phrase:other:adv:shad} Stress shift to the adverb [ʃɑd] \begin{xlist}
		\ex From nouns \begin{xlist}
			\ex \glll (\textbf{O} V)$_\phi$ \\
			nɑmɑɡ-ˈ\textbf{neɾ} uˈn-i-m \\
			letter-{\pl} have-{\thgloss}-1{\sg} \\
			\trans `I have letters.' 
			\\ \armenian{Նամակներ ունիմ։}
			\ex \glll (\textbf{Adv} O V)$_\phi$ \\
			ˈ\textbf{ʃɑd} nɑmɑɡ-ˈ{neɾ} uˈn-i-m \\
			many letter-{\pl} have-{\thgloss}-1{\sg} \\
			\trans `I have many letters.' 
			\\ \armenian{Շատ նամակներ ունիմ։}
		\end{xlist}
		\ex From adjectives \begin{xlist}
			\ex \glll (\textbf{Adj} V)$_\phi$ \\
			ˈ\textbf{hin} =e-n\\
			old =is-3{\pl} \\
			\trans `They are old.'
			\\ \armenian{Հին են։}
			\ex \glll (\textbf{Adv} Adj V)$_\phi$ \\
			ˈ\textbf{ʃɑd} ˈ{hin} =e-n\\
			very old is-3{\pl} \\
			\trans `They are very old.'
			\\ \armenian{Շատ հին են։}
		\end{xlist}
		\ex From verbs \begin{xlist}
			\ex \glll (\textbf{V})$_\phi$ \\
			ɡu-ˈ\textbf{l-ɑ-m} \\
			{\ind}-cry-{\thgloss}-1{\sg} \\
			\trans `I cry.'
			\\ \armenian{Կու լամ։}
			\ex \glll (\textbf{Adv}  V)$_\phi$ \\
			ˈ\textbf{ʃɑd} ɡu-ˈ{l-ɑ-m} \\ 
			very {\ind}-cry-{\thgloss}-1{\sg} \\
			\trans `I cry a lot.'
			\\ \armenian{Շատ կու լամ։}
		\end{xlist}
	\end{xlist}
\end{exe}


\subsubsection{Compound-like phrases}\label{section:intonation:phrase:other:colloc}

This section discusses collocational compounds (\ref{example:inton:phrase:other:colloc:advadv}). What we call a collocational compound is when two words are said together as a type of phrase, usually a common saying. These phrases often have some sort of `serial' meaning.  Such phrases typically consist of two words with identical syntactic category, such as both being adjectives/adverbs. Each word has a perceivable stress \citep[222]{Gharagulyan-1974-BookArmenianOrthoepy}; the two stresses are equivalent in prominence. We suspect that each word is parsed as its own prosodic phrase. 

\begin{exe}
	\ex \glll (\textbf{Adv})$_\phi$ (\textbf{Adv} V)$_\phi$ 
	\\
	jeɾˈ\textbf{ɡɑɾ} pʰɑˈ\textbf{ɾɑɡ} ɡə-χoˈs-i-n \\
	long thing {\ind}-speak-{\thgloss}-3{\pl} \\
	\trans `They talk too much about nothing.' \label{example:inton:phrase:other:colloc:advadv}
	\\ \armenian{Երկար-բարակ կը խօսին։}
	
\end{exe}

Such collocations can likewise be two verbs, whether as infinitives  (\ref{example:inton:phrase:other:colloc:infinf}) or finite verbs (\ref{example:inton:phrase:other:colloc:finfin}). 

\begin{exe}
	\ex \begin{xlist}
		\ex \glll ~ ~  (\textbf{V})$_\phi$ (\textbf{V})$_\phi$ 
		\\
		\textbf{t͡ʃ-e-m} siˈ\textbf{ɾ-e-ɾ} jeɾ\textbf{tʰ-ɑ-l} ˈ\textbf{kʰ-ɑ-l}  \\
		{\neggloss}-is-1{\sg} like-{\thgloss}-{\cn} go-{\thgloss}-{\infgloss} come-{\thgloss}-{\infgloss} \\
		\trans `I don't like coming and going (= moving around).'\label{example:inton:phrase:other:colloc:infinf}
		\\ \armenian{Չեմ սիրեր երթալ գալ։}
		\ex \glll ~ ~  (\textbf{V})$_\phi$ (\textbf{V})$_\phi$ 
		\\
		ɑˈ\textbf{men} ˈoɾ ɡ-eɾ\textbf{tʰ-ɑ-n} ɡu-ˈ\textbf{kʰ-ɑ-n}  \\
		every day {\ind}-go-{\thgloss}-3{\pl} {\ind}-come-{\thgloss}-3{\pl} \\
		\trans `They're coming and going (= visiting) every day.' \label{example:inton:phrase:other:colloc:finfin}
		\\ \armenian{Ամէն օր  կ՚երթան կու գան։}
	\end{xlist}
\end{exe}

\subsubsection{Reduplication}\label{section:intonation:phrase:other:redup}

Similar to collocations, echo reduplication creates two stress domains (\ref{example:inton:phrase:other:red:eecho})    \citep[14]{Margaryan-1997-ArmenianPhonology}. A word or phrase can be repeated, and the prefix  \textit{m-} replaces the word-initial onset of the second copy. Both copies have a perceivable primary stress; the two sound equivalent in prominence. We suspect that each copy is its own prosodic phrase. 

\begin{exe}
	\ex \label{example:inton:phrase:other:red:eecho}\begin{xlist}
		\ex \glll (\textbf{Adj})$_\phi$ (\textbf{Adj})$_\phi$ \\
		hɑŋˈ\textbf{kʰist} mɑŋˈ\textbf{kʰist}  ˈ\textbf{t͡ʃ-e-s}  bɑɾˈ{ɡ-i-ɾ} =ɡoɾ \\
		comfortable {\redgloss} {\neggloss}-is-2{\sg} sleep-{\thgloss}-{\cn} ={\prog}  \\
		\trans `You are not sleeping comfortably or whatever.' \\
		\armenian{Հանգիստ մանգիստ չես պարկիր կոր։}
		\ex \glll (\textbf{Adj})$_\phi$ (\textbf{Adj})$_\phi$ \\
		ɑˈ\textbf{tʰoɾ}  mɑˈ\textbf{tʰoɾ}  ˈ\textbf{betk} uˈn-i-m  \\
		chair {\redgloss} exist-{\thgloss}-3{\sg} need have-{\thgloss}-1{\sg} \\
		\trans `I need   chairs and whatever.'  \\
		\armenian{Աթոռ մաթոռ պէտք ունիմ։}
	\end{xlist}
\end{exe}


\section{Sentential stress in broad-focus contexts}\label{section:intonation:broadFocus}
The previous section cataloged phrasal stress:  the strongest stress   within a phrases. Building off of \citet{Dolatian-2022-InterfaceNuclearStressWesternArmenianTurkishPersian}. This section catalogs nuclear stress or sentential stress: the strongest stress within a sentence.  We concentrate on broad-focus contexts, meaning contexts where no single word is more semantically important than another. 

This section catalogues the different types of sentences, and how  nuclear stress is assigned. We go over  complex predicates (\S\ref{section:intonation:broadFocus:complex}), ditransitives (\S\ref{section:intonation:broadFocus:ditrans}), and definite direct objects (\S\ref{section:intonation:broadFocus:object}).  These are all unified in terms of how stress is on the last prosodic phrase. Typically the last phrase is the verb phrase, and  nuclear stress is either on the verb or a preverbal item within the verb phrase. Variations are attested for other transitive   orders like OVS (\S\ref{section:intonation:broadFocus:otherWordOrder}) and across classes of intransitives (\S\ref{section:intonation:broadFocus:intrans}).  


\subsection{Overview of nuclear stress}\label{section:intonation:broadFocus:overview}

When describing nuclear stress, we must be careful with the semantics or information structure of the sentence. This section describes nuclear stress in broad-focus or all-focus contexts. This is a context where all the information in a sentence is new, and none is more semantically salient than another. Such contexts arise in out-of-the-blue contexts, such as in the following dialogue (\ref{ex: nuclear:out of blue base dialog}). Syllables with phrasal stress are in boldface; underlining is for the word with  perceived nuclear stress. 

\begin{exe}
	\ex Broad-focus context \label{ex: nuclear:out of blue base dialog}
	\begin{xlist}
		\ex \gll \underline{ˈ\textbf{int͡ʃ}} jeʁ-ɑ-v \\ 
		what happen.{\aorperf}-{\pst}-3{\sg} \\
		\trans `What happened?'
		\\ \armenian{Ի՞նչ եղաւ։}
		\ex \glll (\textbf{Adv})$_\phi$, (\textbf{S})$_\phi$  (\textbf{\underline{O}} V)$_\phi$ \\
		\textbf{het͡ʃ}, d\textbf{əʁ}ɑ-kʰ-ə   \underline{bənɑɡ-ˈ\textbf{ne}ɾ-ə}  ləˈv-ɑ-t͡s-i-n \\
		nothing, boy-{\pl}-{\defgloss} plate-{\pl}-{\defgloss} wash-{\thgloss}-{\aorperf}-{\pst}-3{\pl} \\
		\trans `Nothing, the boys washed the dishes.'\label{ex: nuclear:out of blue base}
		\\ \armenian{Հէչ, տղաքը  պնակները   լուացին։}
	\end{xlist}
\end{exe}

In the broad-focus context in (\ref{ex: nuclear:out of blue base}), nuclear stress is on the preverbal object. In general, nuclear stress is on the rightmost phrasal stress in the sentence. Thus for all the sentences in Section \S\ref{section:intonation:phrase}, nuclear stress was on the last prosodic phrase. Native grammars generally never discuss broad-focus nuclear stress, just focus-induced nuclear stress \citep[23-4]{Abeghyan-1933-Meter} which we postpone to Section\S\ref{section:intonation:focus}. 





In simple transitive sentences, stress is by default on the object. The object can be a bare singular (\ref{ex: nuclear: object: bare}) or plural (\ref{ex: nuclear: object: pl}), indefinite (\ref{ex: nuclear: object: indf}) or definite (\ref{ex: nuclear: object: def}). The stressability of definite objects is discussed more in \S\ref{section:intonation:broadFocus:object}. 

\begin{exe}
	\ex \begin{xlist}
		\ex \glll (\textbf{S})$_\phi$ (\underline{\textbf{O}}) V)$_\phi$ \\
		mɑɾˈ\textbf{jɑ-n} \underline{nɑˈ\textbf{mɑɡ}} kʰəˈɾ-e-t͡s-$\emptyset$-$\emptyset$ \\
		Maria-{\defgloss} letter write-{\thgloss}-{\aorperf}-{\pst}-3{\sg} \\
		\trans `Maria wrote letters.' \label{ex: nuclear: object: bare} 
		\\ \armenian{Մարիան նամակ գրեց։}
		\ex \glll (\textbf{S})$_\phi$ (\underline{\textbf{O-{\pl}}}) V)$_\phi$ \\
		mɑɾˈ\textbf{jɑ-n} \underline{nɑmɑɡ-ˈ\textbf{neɾ}} kʰəˈɾ-e-t͡s-$\emptyset$-$\emptyset$ \\
		Maria-{\defgloss} letter-{\pl} write-{\thgloss}-{\aorperf}-{\pst}-3{\sg} \\
		\trans `Maria wrote letters.' \label{ex: nuclear: object: pl} 
		\\ \armenian{Մարիան նամակներ գրեց։}
		\ex \glll (\textbf{S})$_\phi$ (\underline{\textbf{O-{\indf}}}) V)$_\phi$ \\
		mɑɾˈ\textbf{jɑ-n} \underline{nɑˈ\textbf{mɑɡ}-mə} kʰəˈɾ-e-t͡s-$\emptyset$-$\emptyset$ \\
		Maria-{\defgloss} letter-{\pl} write-{\thgloss}-{\aorperf}-{\pst}-3{\sg} \\
		\trans `Maria wrote a letter.' \label{ex: nuclear: object: indf} 
		\\ \armenian{Մարիան նամակ մը գրեց։}
		\ex \glll (\textbf{S})$_\phi$ (\underline{\textbf{O-{\defgloss}}}) V)$_\phi$ \\
		mɑɾˈ\textbf{jɑ-n} \underline{nɑˈ\textbf{mɑ}ɡ-ə} kʰəˈɾ-e-t͡s-$\emptyset$-$\emptyset$ \\
		Maria-{\defgloss} letter-{\defgloss} write-{\thgloss}-{\aorperf}-{\pst}-3{\sg} \\
		\trans `Maria wrote the letter.' \label{ex: nuclear: object: def} 
		\\ \armenian{Մարիան նամակը գրեց։}
	\end{xlist}
\end{exe}

Acoustically, for a simple SOV sentence with nuclear stress on the object, we find that the verb is deaccented and lacks any prominence. This means that we have post-focal compression or post-focal deaccenting. Such acoustic data is discussed in \S\ref{section:intonation:focus:declarative} in the general context of intonation. 



\subsection{Complex predicate}\label{section:intonation:broadFocus:complex}


A special type of verb phrase is complex predicates. Such constructions consist of a verb that doesn't carry much lexical meaning like `to do'. The verb is preceded by a noun. The noun+verb combination semantically carries the verbal action. This construction can then takes its own object. Stress is on the (indirect) object if present (\ref{ex: nuclear stress: complex: base: oxv}); else on the preverbal word (\ref{ex: nuclear stress: complex: base: xv}). Their prosody was further likewise earlier discussed in \S\ref{section:intonation:phrase:verb:rec}. 



\begin{exe}
	\ex 
	\begin{xlist}
		\ex \glll (\textbf{S})$_\phi$ (\textbf{\underline{O}} X V)$_\phi$ \\
		mɑɾˈ\textbf{jɑ-n} \underline{ɑɾɑ-ˈ\textbf{ji-n}} məˈdiɡ əˈɾ-ɑ-v \\
		Maria-{\defgloss} Ara-{\dat}-{\defgloss} X do.{\aorperf}-{\pst}-3{\pst} \\
		\trans `Maria listened to Ara.' \label{ex: nuclear stress: complex: base: oxv}
		\\ \armenian{Մարիան Արային մտիկ ըրաւ։}
		\ex \glll (\textbf{S})$_\phi$ (\textbf{\underline{X}} V)$_\phi$ \\
		mɑɾˈ\textbf{jɑ-n} \underline{məˈ\textbf{diɡ}} əˈɾ-ɑ-v \\
		Maria-{\defgloss} X do.{\aorperf}-{\pst}-3{\pst} \\
		\trans `Maria listened.' \label{ex: nuclear stress: complex: base: xv}
		\\ \armenian{Մարիան մտիկ ըրաւ։}
	\end{xlist}
\end{exe}

The preverbal item can range from being a meaningless word (\ref{ex: nuclear stress: complex: base: xv}), a borrowed word (\ref{ex: nuclear stress: complex: category: loan}), a noun (\ref{ex: nuclear stress: complex: category: noun}), among other options. 



\begin{exe}
	\ex 
	\begin{xlist}
		\ex \glll (\textbf{S})$_\phi$ (\textbf{\underline{O}} X V)$_\phi$ \\
		mɑɾˈ\textbf{jɑ-n} \underline{tʰuχˈ\textbf{t-e}ɾ-ə} ˈpɾintʰ əˈɾ-ɑ-v \\
		Maria-{\defgloss} paper-{\pl}-{\defgloss} X do.{\aorperf}-{\pst}-3{\pst} \\
		\trans `Maria printed the papers.' \label{ex: nuclear stress: complex: category: loan}
		\\ \armenian{Մարիան թուղթերը} `print' \armenian{ըրաւ։}
		\ex \glll (\textbf{S})$_\phi$ ({\textbf{N}} V)$_\phi$ ( \textbf{Adj} V)$_\phi$ \\
		ˈ\textbf{kʰuj}n-əs {ˈ\textbf{t͡sujt͡s}}  ɡu-ˈd-ɑ-$\emptyset$ voɾ \underline{hiˈ\textbf{vɑntʰ}} =e-m\\
		color-{\possFsg} sign  {\ind}-give-{\thgloss}-3{\pst} that sick =is-1{\sg}\\
		\trans `My color shows that I am sick.' \label{ex: nuclear stress: complex: category: noun}
		\\ \armenian{Գոյնս ցոյց կու տայ որ հիւանդ եմ։}
	\end{xlist}
\end{exe}


Native grammars provide more lists of such constructions, mostly from Eastern Armenian  (\citealt[20]{Abeghyan-1933-Meter}; \citealt[75]{Margaryan-1997-ArmenianPhonology}; \citealt[153]{Sevak-2009-Coursebook}). But these are often also found in Western Armenian.  

\subsection{Stressability of objects}\label{section:intonation:broadFocus:object}

Section \S\ref{section:intonation:broadFocus:overview} included examples of nuclear stress on definite direct objects. Within the literature on prosody of Armenian and other related languages, it is often argued definite objects avoid getting nuclear stress unless they are focused \citep{Dolatian-2022-InterfaceNuclearStressWesternArmenianTurkishPersian}. Such a generalization is reported in Eastern Armenian \citep{KahnemuyipourMegerdoomian-2011-secondcliticvP}, Persian,  Turkish, and even Western Armenian \textcolor{red}{cite persian/turkish stress from my paper}


For Turkish, the ban on stressed definitie objects was reported in earlier work by \textcolor{red}{cite persian/turkish stress from my paper}. But more recent work  by \citet{Nakipoglu-2009-SemanticsTurkishAccDefStress} discovered that most contexts that linguists use to elicit definite objects often treat the object as given information. By being given information, the object is then predictably deaccented. This section replicates \citet{Nakipoglu-2009-SemanticsTurkishAccDefStress}'s work but with Western Armenian. We again find that definite objects can carry nuclear stress without being focused. 


First, consider the dialogue below. The question A (\ref{example:inton:broad:obj:snow:a}) and the answer B1  (\ref{example:inton:broad:obj:snow:b1}) set up the context of noise and snow. Sentence B2 (\ref{example:inton:broad:obj:snow:b2}) then introduces a definite object `the road' which is new information. The object is new information and takes nuclear stress; stress is not on the verb.   \citet{Nakipoglu-2009-SemanticsTurkishAccDefStress} \textcolor{red}{page number} discusses the pragmatics of this sentence in depth. Basically, the object is new and stressed because the question in A didn't presuppose the existence of roads. Stressing the verb is infelicitous (\ref{example:inton:broad:obj:snow:b2bad}). 


\begin{exe}
	\ex 
	\begin{xlist}
		
		\ex \gll A: \textbf{\underline{int͡}ʃ} ɡ-əlˈl-ɑ-$\emptyset$ =ɡoɾ? \underline{\textbf{int͡}}ʃ  =e  ɑjs ɑʁˈmuɡ-ə? \\
		~ what {\ind}-be-{\thgloss}-3{\sg} ={\prog}? what =is this noise-{\defgloss}?\\ 
		\trans `What is happening? What is this noise?'\label{example:inton:broad:obj:snow:a}
		\\ \armenian{Ի՞նչ կ՚ըլլայ կոր։ Ի՞նչ է այս աղմուկը։}
		\ex \gll  B1: jeˈɾeɡ kʰiˈ\textbf{ʃe}ɾ-ə ˈ\textbf{\underline{ʃɑd}} ˈt͡sujn jeˈɡ-eɾ =e-$\emptyset$-ɾ\\
		~ last night-{\defgloss} many snow come.{\aorother}-{\eptcp} =is-{\pst}-3{\sg} \\ 
		\trans `Last night it snowed a lot.' \label{example:inton:broad:obj:snow:b1}\\
		\armenian{Երէկ գիշեր շատ ձիւն եկեր էր։}
		\ex \glll  B2:  (\textbf{S})$_\phi$ (\underline{\textbf{O-{\defgloss}}} V)$_\phi$\\ ~ ɡɑɾɑvɑɾuˈ\textbf{tʰjʏ}n-ə \underline{d͡ʒɑmˈ\textbf{pʰɑ-n}}  ɡə-mɑkʰˈɾ-e-$\emptyset$=ɡoɾ  \\
		~ government-{\defgloss} road-{\defgloss} {\ind}-clean-{\thgloss}-3{\sg}={\prog}\\ 
		\trans `The government is cleaning the road.'\label{example:inton:broad:obj:snow:b2}
		\\ \armenian{Կառավարութիւնը ճամբան կը մաքրէ կոր։}
		\ex \gll  \#B2:  (\textbf{S})$_\phi$ ({\textbf{O-{\defgloss}}})$_\phi$ (\underline{\textbf{V}})$_\phi$
		\\
		~   ɡɑɾɑvɑɾuˈ\textbf{tʰjʏ}n-ə  d͡ʒɑmˈ\textbf{pʰɑ-n}  \underline{ɡə-mɑkʰˈ\textbf{ɾ-e-$\emptyset$}}=ɡoɾ\\\label{example:inton:broad:obj:snow:b2bad}
		
	\end{xlist}
	
\end{exe}

In contrast consider the dialogue below. Sentence A (\ref{example:inton:broad:obj:snowGive:a}) presupposes the existence of roads where people drive cars. Sentence B1 (\ref{example:inton:broad:obj:snowGive:b1}) explicitly introduces the definite object `the road' but it does not get nuclear stress. The object is treated as given information, because it is implied from sentence A. Stressing the object is infelicitous (\ref{example:inton:broad:obj:snowGive:b1bad}). 

\begin{exe}
	\ex 
	\begin{xlist}
		
		\ex \gll  A: jeɾeɡ kʰiˈ\textbf{ʃe}ɾ-ə ˈ\textbf{ʃɑd} ˈt͡sujn jeɡ-eɾ  =e-$\emptyset$-ɾ. tʰeɾevəs \underline{\textbf{t͡ʃ-em}} ɡərˈn-ɑ-ɾ  otʰo-ˈjov kʰoɾˈd͡z-i jeɾˈtʰ-ɑ-l   \\
		~  last night-{\defgloss} many snow come.{\aorother}-{\eptcp} =is-$\pst$-3{\sg}. perhaps {\neggloss}-is-1{\sg} can-{\thgloss}-{\cn}  car-{\ins}  work-{\dat} go-{\thgloss}-{\infgloss} \\
		\trans `Last night it snowed a lot. Perhaps I can't go to work by car.' \label{example:inton:broad:obj:snowGive:a}
		\\ \armenian{Երէկ գիշերը շատ ձիւն եկեր էր։ Թերեւս չեմ կրնար օթօյով գործի երթալ։}
		\ex \glll B1:  (\textbf{S})$_\phi$ (\textbf{O-{\defgloss}})$_\phi$ (\underline{\textbf{V}})$_\phi$\\
		~ ɡɑɾɑvɑɾuˈ\textbf{tʰjʏ}n-ə d͡ʒɑmˈ\textbf{pʰɑ-n} \underline{ɡə-mɑkʰˈ\textbf{ɾ-e-$\emptyset$}}=ɡoɾ \\ 
		~ government-{\defgloss} road-{\defgloss}   {\ind}-clean-{\thgloss}-3{\sg}={\prog} \\ 
		\trans `The government is cleaning the road.' \label{example:inton:broad:obj:snowGive:b1}
		\\ \armenian{Կառավարութիւնը ճամբան կը մաքրէ կոր։} 
		\ex \gll  \#B1  (\textbf{S})$_\phi$ ({\underline{\textbf{O-{\defgloss}}}} V)$_\phi$\\ 
		~ ɡɑɾɑvɑɾutˈ\textbf{ʰju}n-ə \underline{d͡ʒɑmˈ\textbf{pʰɑ-n}} ɡə-mɑkʰˈɾ-e-$\emptyset$=ɡoɾ \\ \label{example:inton:broad:obj:snowGive:b1bad}
		\ex \gll B2:  ɡərˈ\textbf{n-ɑ-s} \underline{otʰo-ˈ\textbf{jov}} jeɾˈtʰ-ɑ-l \\
		~   can-{\thgloss}-2{\sg} car-{\ins} go-{\thgloss}-{\infgloss}\\
		\trans `You can go by car.'\
		\\ \armenian{Կրնաս օթօյով երթալ։}
		
	\end{xlist}
	
\end{exe}

Thus, when special contexts permit, a definite object can be introduced as new information. When new, the object gets phrasal stress and nuclear stress.  Another type of context that allows stressed definite objects is narratives. In the narrative below (\ref{example:inton:broad:obj:narrative}), the speaker is narrating events as they happen. The subject Ara is doing actions in a sequence; each actions introduces a definite object and it gets stressed. 

\begin{exe}
	\ex Stress on definite objects in narratives (adapted from \textcolor{red}{nakipoluge source} \label{example:inton:broad:obj:narrative}
	\begin{xlist}
		
		\ex \glll (\textbf{S})$_\phi$ (\textbf{\underline{Loc}} V)$_\phi$ \\
		ɑˈ\textbf{ɾɑ-n} ˈ\textbf{\underline{dun}} jeˈɡ-ɑ-v   \\
		Ara house come.{\aorperf}-{\pst}-3{\sg} \\
		\trans `Ara came home.' 
		\\ \armenian{Արան տուն եկաւ։}
		\ex \glll (\textbf{N})$_\phi$ (\textbf{\underline{O}} V)$_\phi$ \\
		bɑjsɑˈ\textbf{ɡ-e-n}  \underline{pʰɑnɑˈ\textbf{li-n}} hɑˈn-e-t͡s-$\emptyset$-$\emptyset$ \\
		bag-{\abl}-{\defgloss}  key-{\defgloss}  take-{\thgloss}-{\aorperf}-{\pst}-3{\sg} \\
		\trans `He took the key out of his bag.'
		\\
		\armenian{Պայսակէն բանալին հանեց։}
		\ex \glll (\textbf{\underline{O}} V)$_\phi$ \\
		\underline{ˈ\textbf{tʰu}ɾ-ə} pʰɑˈt͡s-ɑ-v\\
		door-{\defgloss} open.{\aorperf}-{\pst}-3{\sg}\\
		\trans `He opened the door.' 
		\\ \armenian{Դուռը բացաւ։}
		\ex \glll (\textbf{Loc})$_\phi$ (\textbf{\underline{Adj}} V)$_\phi$, (\textbf{\underline{O}} V)$_\phi$ \\
		ˈ\textbf{neɾ}s-ə \underline{\textbf{ˈbɑʁ}} =e-$\emptyset$-ɾ,  \underline{bɑduˈ\textbf{hɑ}n-ə} kʰoˈt͡s-e-t͡s-$\emptyset$-$\emptyset$  \\
		inside-{\defgloss} cold =is-{\pst}-3{\sg},  window-{\defgloss} close-{\thgloss}-{\aorperf}-{\pst}-3{\sg}  \\
		\trans `It was cold inside, He closed the window.' 
		\\ \armenian{Ներսը պաղ էր.  պատուհանը գոցեց։}
		\ex \glll   (\textbf{\underline{O}} V)$_\phi$ \\
		\underline{vɑɾɑˈ\textbf{kʰuj}ɾ-ə} kʰoˈt͡s-e-t͡s-$\emptyset$-$\emptyset$\\
		curtain-{\defgloss} close-{\thgloss}-{\aorperf}-{\pst}-3{\sg} \\
		\trans `He drew the curtain.' \\
		\armenian{Վարագոյրը գոցեց։}
		\ex \glll   (\textbf{\underline{Loc}} V)$_\phi$,    (\textbf{\underline{O}} V)$_\phi$ \\
		\underline{pʰɑʁˈ\textbf{nikʰ}}  kʰəˈn-ɑ-t͡s-$\emptyset$-$\emptyset$,  \underline{ˈ\textbf{luj}s-ə} pʰɑˈt͡s-ɑ-v \\
		bathroom go.{\aorperf}-{\thgloss}-{\aorperf}-{\pst}-3{\sg},  light-{\defgloss}  open.{\aorperf}-{\pst}-3{\sg} \\
		\trans `He  went to the bathroom;  he  turned on the lights.'
		\\ \armenian{Բաղնիք գնաց, լոյսը բացաւ։}
		\ex \glll    (\textbf{\underline{O}} V)$_\phi$ \\
		\underline{oˈ\textbf{d͡ʒɑ}ɾ-ə} pʰəndˈɾ-e-t͡s-$\emptyset$-$\emptyset$  \\
		soap-{\defgloss} look.for-{\thgloss}-{\aorperf}-{\pst}-3{\sg}\\
		\trans `He looked for the soap.'
		\\ \armenian{Օճառը փնտռեց։}
		\ex \glll    (\textbf{\underline{O}} V)$_\phi$ \\
		\underline{ˈ\textbf{t͡seɾ}kʰ-ə} ləˈv-ɑ-t͡s-$\emptyset$-$\emptyset$ \\
		hand-\text{def}  wash-{\thgloss}-{\aorperf}-{\pst}-3{\sg}\\
		\trans `He washed his hands.'
		\\ \armenian{Ձեռքը լուաց։}
		\ex \glll   (C \textbf{\underline{V}})$_\phi$,  (\textbf{\underline{O}} V)$_\phi$ \\
		voɾbeˈsi lokʰˈ\underline{\textbf{n-ɑ-$\emptyset$}},   \underline{lokʰɑˈ\textbf{ɾɑ}n-ə} le-ˈt͡su-t͡s-$\emptyset$-$\emptyset$  \\
		for bathe-{\thgloss}-3{\sg}, bathtub-{\defgloss} fill-{\caus}-{\aorperf}-{\pst}-3{\sg} \\
		\trans `He  filled the bathtub to take a bath.'
		\\ \armenian{Որպէսզի լոգնայ, լոգարանը լեցուց։}
	\end{xlist}
\end{exe}

Thus, definite objects can get nuclear stress without needing focus. This is contrast to Eastern Armenian and Persian, where definite objects are reported to get stress only if focused. \textcolor{red}{ea persian literature}. 


\subsection{Ditransitive}\label{section:intonation:broadFocus:ditrans}
In a ditransitive sentence, the verb has two objects: a direct object (DO) and an indirect object (IO). The verb forms a prosodic phrase with the rightmost object (the preverbal object), and stress is on this object. The linear order between DOs and IOs is quite variable. See \textcolor{red}{ditransitive order} for their syntax. This section focuses on their prosody. 


First, consider the order IO+DO. The DO can be bare singular (\ref{example:inton:broad:ditrans:iodo:bare}) or plural (\ref{example:inton:broad:ditrans:iodo:pl}), indefinite (\ref{example:inton:broad:ditrans:iodo:indf}) or definite (\ref{example:inton:broad:ditrans:iodo:def}). The DO is phrased with the verb and takes stress. 

\begin{exe}
	\ex \begin{xlist}
		\ex \glll (\textbf{S})$_\phi$ (\textbf{IO})$_\phi$ (\textbf{DO} V)$_\phi$ \\
		ɑˈ\textbf{ɾɑ-n} ɡɑdu-ˈ\textbf{ji-n} \underline{bɑˈ\textbf{niɾ}}  dəˈv-ɑ-v \\
		Ara-{\defgloss} cat-{\dat}-{\defgloss} cheese give.{\aorperf}-{\pst}-3{\sg}  \\
		\trans `Ara gave some cheese to the cat.' \label{example:inton:broad:ditrans:iodo:bare}
		\\ \armenian{Արան կատուին պանիր տուաւ։}
		\ex \gll ɑˈ\textbf{ɾɑ-n} ɡɑdu-ˈ\textbf{ji-n} \underline{muˈ\textbf{ɡ-eɾ}}  dəˈv-ɑ-v \\
		Ara-{\defgloss} cat-{\dat}-{\defgloss} mouse-{\pl} give.{\aorperf}-{\pst}-3{\sg}  \\
		\trans `Ara gave some mice to the cat.' \label{example:inton:broad:ditrans:iodo:pl}
		\\ \armenian{Արան կատուին մուկեր տուաւ։}
		\ex \gll ɑˈ\textbf{ɾɑ-n} ɡɑdu-ˈ\textbf{ji-n} \underline{ˈ\textbf{muɡ}-mə}  dəˈv-ɑ-v \\
		Ara-{\defgloss} cat-{\dat}-{\defgloss} mouse-{\indf} give.{\aorperf}-{\pst}-3{\sg}  \\
		\trans `Ara gave a mouse to the cat.'\label{example:inton:broad:ditrans:iodo:indf}
		\\ \armenian{Արան կատուին մուկ մը տուաւ։}
		\ex \gll ɑˈ\textbf{ɾɑ-n} ɡɑdu-ˈ\textbf{ji-n} \underline{bɑˈ\textbf{ni}ɾ-ə}  dəˈv-ɑ-v \\
		Ara-{\defgloss} cat-{\dat}-{\defgloss} cheese-{\defgloss} give.{\aorperf}-{\pst}-3{\sg}  \\
		\trans `Ara gave the cheese to the cat.'\label{example:inton:broad:ditrans:iodo:def}
		\\ \armenian{Արան կատուին պանիրը տուաւ։}
	\end{xlist}
\end{exe}

The difference in perception between the IO and DO is stronger when the IO has a modifier (\ref{example:inton:broad:ditrans:iodo:mod}). The larger prosodic phrase of the IO causes a longer pause before the DO. 


\begin{exe}
	\ex \label{example:inton:broad:ditrans:iodo:mod} \begin{xlist}
		\ex \glll (Adj \textbf{IO})$_\phi$ (\textbf{DO} V)$_\phi$ \\
		ɑnoˈtʰi ɡɑdu-ˈ\textbf{ji-n} \underline{bɑˈ\textbf{niɾ}}  dəˈv-ɑ-v \\
		hungry cat-{\dat}-{\defgloss} cheese give.{\aorperf}-{\pst}-3{\sg}  \\
		\trans `He gave some cheese to the hungry cat.'
		\\ \armenian{Անօթի կատուին պանիր տուաւ։}
		\ex \gll ɑnoˈtʰi ɡɑdu-ˈ\textbf{ji-n} \underline{muˈ\textbf{ɡ-eɾ}}  dəˈv-ɑ-v \\
		hungry cat-{\dat}-{\defgloss} mouse-{\pl} give.{\aorperf}-{\pst}-3{\sg}  \\
		\trans `He gave some mice to the hungry cat.'
		\\ \armenian{Անօթի կատուին մուկեր տուաւ։}
		\ex \gll ɑnoˈtʰi ɡɑdu-ˈ\textbf{ji-n} \underline{ˈ\textbf{muɡ}-mə}  dəˈv-ɑ-v \\
		hungry cat-{\dat}-{\defgloss} mouse-{\indf} give.{\aorperf}-{\pst}-3{\sg}  \\
		\trans `He gave a mouse to the hungry cat.'
		\\ \armenian{Անօթի կատուին մուկ մը տուաւ։}
		\ex \gll ɑnoˈtʰi ɡɑdu-ˈ\textbf{ji-n} \underline{bɑˈ\textbf{ni}ɾ-ə}  dəˈv-ɑ-v \\
		hungry cat-{\dat}-{\defgloss} cheese-{\defgloss} give.{\aorperf}-{\pst}-3{\sg}  \\
		\trans `He gave the cheese to the hungry cat.'
		\\ \armenian{Անօթի կատուին պանիրը տուաւ։}
	\end{xlist}
\end{exe} 

For the DO+IO order (\ref{example:inton:broad:ditrans:doio}), the IO is phrased with the verb and gets stressed.  


\begin{exe}
	\ex\label{example:inton:broad:ditrans:doio}
	\begin{xlist}
		\ex \glll (\textbf{S})$_\phi$ (\textbf{DO})$_\phi$ (\textbf{IO} V)$_\phi$ \\
		ɑˈ\textbf{ɾɑ-n}  {bɑˈ\textbf{ni}ɾ-ə} \underline{ɡɑdu-ˈ\textbf{ji}-mə}  dəˈv-ɑ-v \\
		Ara-{\defgloss} cheese-{\defgloss} cat-{\dat}-{\indf} give.{\aorperf}-{\pst}-3{\sg}  \\
		\trans `Ara gave the cheese to a cat.'
		\\ \armenian{Արան պանիրը կատուի մը  տուաւ։}
		\ex \gll  ɑˈ\textbf{ɾɑ-n}  {bɑˈ\textbf{ni}ɾ-ə} \underline{ɡɑdu-neˈ\textbf{ɾ-u}}  dəˈv-ɑ-v \\
		Ara-{\defgloss} cheese-{\defgloss} cat-{\pl}-{\dat} give.{\aorperf}-{\pst}-3{\sg}  \\
		\trans `Ara gave the cheese to some cats.'
		\\ \armenian{Արան պանիրը կատուներու   տուաւ։}
		\ex  \gll ɑˈ\textbf{ɾɑ-n}  {bɑˈ\textbf{ni}ɾ-ə} \underline{ɡɑdu-ˈ\textbf{ji-n}}  dəˈv-ɑ-v \\
		Ara-{\defgloss} cheese-{\defgloss} cat-{\dat}-{\defgloss} give.{\aorperf}-{\pst}-3{\sg}  \\
		\trans `Ara gave the cheese to the cat.'
		\\ \armenian{Արան պանիրը կատուին  տուաւ։}
	\end{xlist}
\end{exe}

Again, if the DO is bigger \ref{example:inton:broad:ditrans:doio:mod}, we find a longer pause between the DO and IO. 


\begin{exe}
	\ex\label{example:inton:broad:ditrans:doio:mod}
	\begin{xlist}
		\ex \glll (Adj \textbf{DO})$_\phi$ (\textbf{IO} V)$_\phi$ \\
		tʰeˈʁin  {bɑˈ\textbf{ni}ɾ-ə} \underline{ɡɑdu-ˈ\textbf{ji}-mə}  dəˈv-ɑ-v \\
		yellow cheese-{\defgloss} cat-{\dat}-{\indf} give.{\aorperf}-{\pst}-3{\sg}  \\
		\trans `He gave the yellow cheese to a cat.'
		\\ \armenian{Դեղինեղին պանիրը կատուի մը  տուաւ։}
		\ex \gll  tʰeˈʁin  {bɑˈ\textbf{ni}ɾ-ə} \underline{ɡɑdu-neˈ\textbf{ɾ-u}}  dəˈv-ɑ-v \\
		yellow cheese-{\defgloss} cat-{\pl}-{\dat} give.{\aorperf}-{\pst}-3{\sg}  \\
		\trans `He gave the yellow cheese to some cats.'
		\\ \armenian{Դեղինեղին պանիրը կատուներու   տուաւ։}
		\ex  \gll tʰeˈʁin  {bɑˈ\textbf{ni}ɾ-ə} \underline{ɡɑdu-ˈ\textbf{ji-n}}  dəˈv-ɑ-v \\
		yellow cheese-{\defgloss} cat-{\dat}-{\defgloss} give.{\aorperf}-{\pst}-3{\sg}  \\
		\trans `He gave the yellow cheese to the cat.'
		\\ \armenian{Դեղինեղին պանիրը կատուին  տուաւ։}
	\end{xlist}
\end{exe}



In the above sentences, the first object was definite and could thus be placed earlier in the sentence. If the first object is indefinite,    we still find that only the second object gets stress, whether for IO+DO (\ref{example:inton:broad:ditrans:indf:iodo}) or DO+IO (\ref{example:inton:broad:ditrans:indf:doio}). 

\begin{exe}
	\ex \label{example:inton:broad:ditrans:indf:iodo}\begin{xlist}
		\ex \glll (\textbf{S})$_\phi$ (\textbf{IO})$_\phi$ (\textbf{DO} V)$_\phi$ \\
		ɑˈ\textbf{ɾɑ-n} ɡɑdu-ˈ\textbf{ji}-mə \underline{bɑˈ\textbf{niɾ}}  dəˈv-ɑ-v \\
		Ara-{\defgloss} cat-{\dat}-{\indf} cheese give.{\aorperf}-{\pst}-3{\sg}  \\
		\trans `Ara gave some cheese to a cat.'
		\\ \armenian{Արան կատուի մը պանիր տուաւ։}
		\ex \gll ɑˈ\textbf{ɾɑ-n} ɡɑdu-ˈ\textbf{ji}-mə \underline{muˈ\textbf{ɡ-eɾ}}  dəˈv-ɑ-v \\
		Ara-{\defgloss} cat-{\dat}-{\indf} mouse-{\pl} give.{\aorperf}-{\pst}-3{\sg}  \\
		\trans `Ara gave some mice   to a cat.'
		\\ \armenian{Արան կատուի մը մուկեր տուաւ։}
		\ex \gll ɑˈ\textbf{ɾɑ-n} ɡɑdu-ˈ\textbf{ji}-mə \underline{ˈ\textbf{muɡ}-mə}  dəˈv-ɑ-v \\
		Ara-{\defgloss} cat-{\dat}-{\indf} mouse-{\indf} give.{\aorperf}-{\pst}-3{\sg}  \\
		\trans `Ara gave a mouse   to a cat.'
		\\ \armenian{Արան կատուի մը մուկ մը տուաւ։}
		\ex \gll ɑˈ\textbf{ɾɑ-n} ɡɑdu-ˈ\textbf{ji}-mə \underline{bɑˈ\textbf{ni}ɾ-ə}  dəˈv-ɑ-v \\
		Ara-{\defgloss} cat-{\dat}-{\indf} cheese-{\defgloss} give.{\aorperf}-{\pst}-3{\sg}  \\
		\trans `Ara gave the cheese   to a cat.'
		\\ \armenian{Արան կատուի մը պանիրը տուաւ։}
	\end{xlist}
	\ex \label{example:inton:broad:ditrans:indf:doio}\begin{xlist}
		\ex \glll (\textbf{S})$_\phi$ (\textbf{DO})$_\phi$ (\textbf{IO} V)$_\phi$ \\
		ɑˈ\textbf{ɾɑ-n}  {ˈ\textbf{muɡ}-mə} \underline{ɡɑdu-ˈ\textbf{ji}-mə}  dəˈv-ɑ-v \\
		Ara-{\defgloss} mouse-{\indf} cat-{\dat}-{\indf} give.{\aorperf}-{\pst}-3{\sg}  \\
		\trans `Ara gave a mouse to a cat.'
		\\ \armenian{Արան մուկ մը կատուի մը  տուաւ։}
		\ex \gll  ɑˈ\textbf{ɾɑ-n}  {ˈ\textbf{muɡ}-mə} \underline{ɡɑdu-neˈ\textbf{ɾ-u}}  dəˈv-ɑ-v \\
		Ara-{\defgloss} mouse-{\indf} cat-{\pl}-{\dat} give.{\aorperf}-{\pst}-3{\sg}  \\
		\trans `Ara gave a mouse to some cats.'
		\\ \armenian{Արան մուկ մը  կատուներու   տուաւ։}
		\ex \gll  ɑˈ\textbf{ɾɑ-n}  {ˈ\textbf{muɡ}-mə} \underline{ɡɑdu-ˈ\textbf{ji-n}}  dəˈv-ɑ-v \\
		Ara-{\defgloss} mouse-{\indf} cat-{\dat}-{\defgloss} give.{\aorperf}-{\pst}-3{\sg}  \\
		\trans `Ara gave a mouse to the cat.'
		\\ \armenian{Արան մուկ մը  կատուին   տուաւ։}
	\end{xlist}
\end{exe}

The difference in prominence is again clearer if the first object is bigger (\ref{example:inton:broad:ditrans:indf:mod}). 


\begin{exe}
	\ex \label{example:inton:broad:ditrans:indf:mod} \begin{xlist}
		\ex \glll (Adj \textbf{IO})$_\phi$ (\textbf{DO} V)$_\phi$ \\
		ɑnoˈtʰi ɡɑdu-ˈ\textbf{ji}-mə \underline{bɑˈ\textbf{niɾ}}  dəˈv-ɑ-v \\
		hungry  cat-{\dat}-{\indf} cheese give.{\aorperf}-{\pst}-3{\sg}  \\
		\trans `He gave some cheese to a  hungry cat.'
		\\ \armenian{Անօթի կատուի մը պանիր տուաւ։}
		\ex \gll ɑnoˈtʰi ɡɑdu-ˈ\textbf{ji}-mə \underline{muˈ\textbf{ɡ-eɾ}}  dəˈv-ɑ-v \\
		hungry cat-{\dat}-{\indf} mouse-{\pl} give.{\aorperf}-{\pst}-3{\sg}  \\
		\trans `He gave some mice   to a cat.'
		\\ \armenian{Անօթի կատուի մը մուկեր տուաւ։}
		\ex \gll ɑnoˈtʰi ɡɑdu-ˈ\textbf{ji}-mə \underline{ˈ\textbf{muɡ}-mə}  dəˈv-ɑ-v \\
		hungry cat-{\dat}-{\indf} mouse-{\indf} give.{\aorperf}-{\pst}-3{\sg}  \\
		\trans `He gave a mouse   to a hungry cat.'
		\\ \armenian{Անօթի կատուի մը մուկ մը տուաւ։}
		\ex \gll ɑnoˈtʰi ɡɑdu-ˈ\textbf{ji}-mə \underline{bɑˈ\textbf{ni}ɾ-ə}  dəˈv-ɑ-v \\
		hungry cat-{\dat}-{\indf} cheese-{\defgloss} give.{\aorperf}-{\pst}-3{\sg}  \\
		\trans `He gave the cheese   to a  hungry cat.'
		\\ \armenian{Անօթի կատուի մը պանիրը տուաւ։}
	\end{xlist}
	\ex \begin{xlist}
		\ex \glll (Adj \textbf{DO})$_\phi$ (\textbf{IO} V)$_\phi$ \\
		bəzˈdiɡ  {ˈ\textbf{muɡ}-mə} \underline{ɡɑdu-ˈ\textbf{ji}-mə}  dəˈv-ɑ-v \\
		small  mouse-{\indf} cat-{\dat}-{\indf} give.{\aorperf}-{\pst}-3{\sg}  \\
		\trans `He gave a small mouse to a cat.'
		\\ \armenian{Պզտիկ մուկ մը կատուի մը  տուաւ։}
		\ex \gll  bəzˈdiɡ {ˈ\textbf{muɡ}-mə} \underline{ɡɑdu-neˈ\textbf{ɾ-u}}  dəˈv-ɑ-v \\
		small mouse-{\indf} cat-{\pl}-{\dat} give.{\aorperf}-{\pst}-3{\sg}  \\
		\trans `He gave a small mouse to some cats.'
		\\ \armenian{Պզտիկ մուկ մը  կատուներու   տուաւ։}
		\ex \gll bəzˈdiɡ   {ˈ\textbf{muɡ}-mə} \underline{ɡɑdu-ˈ\textbf{ji-n}}  dəˈv-ɑ-v \\
		small mouse-{\indf} cat-{\dat}-{\defgloss} give.{\aorperf}-{\pst}-3{\sg}  \\
		\trans `He gave a small mouse to the cat.'
		\\ \armenian{Պզտիկ մուկ մը  կատուին   տուաւ։}
	\end{xlist}
\end{exe}


In sum, in a ditransitive sentence, the second preverbal object is phrased with the verb and takes stress. 


\subsection{Other transitive word orders}\label{section:intonation:broadFocus:otherWordOrder}


In all the above sentences, the basic sentence structure is SOV with stress on the preverbal item. If the object of a transitive verb is omitted (SV) as in (\ref{example:inton:broad:otherOrder:sv}), then stress is on the verb. The transitive subject stays   in a separate prosodic phrase

\begin{exe}
	\ex \glll (\textbf{S})$_\phi$ (\textbf{\underline{V}})$_\phi$ \\
	mɑɾˈ\textbf{jɑ-n} \underline{kʰə\textbf{ɾ-e-t͡}s-$\emptyset$-$\emptyset$} \\
	Maria-{\defgloss}   write-{\thgloss}-{\aorperf}-{\pst}-3{\sg} \\
	\trans `Maria wrote (stuff).' \label{example:inton:broad:otherOrder:sv}
	\\ \armenian{Մարիան  գրեց։}
	
\end{exe}

SOV is the default order (\ref{example:inton:broad:otherOrder:sov}), but other word orders are logically possible, such as OVS  (\ref{example:inton:broad:otherOrder:ovs}) or SVO (\ref{example:inton:broad:otherOrder:svo}), but they each entail some type of shift in emphasis or deaccenting. For example, an OVS sentence implies that the subject is an afterthought. Here, nuclear stress is on the last phonological phrase (the subject). In SVO, each word is its own prosodic phrase, and stress is on the last one. 

\begin{exe}
	\ex \begin{xlist}
		\ex \glll  (\textbf{{S}})$_\phi$ (\underline{\textbf{O}} V)$_\phi$\\
		{mɑɾˈ\textbf{jɑ-n}}  \underline{nɑmɑɡ-ˈ\textbf{neɾ}} uˈn-i-$\emptyset$ \\
		Maria-{\defgloss}  letter-{\pl} have-{\thgloss}-3{\sg} \\
		\trans `Maria has letters.' (default order)\label{example:inton:broad:otherOrder:sov}
		\\ \armenian{Մարիան նամակներ ունի։}
		\ex \glll  (\textbf{{O}} V)$_\phi$ (\underline{\textbf{S}})$_\phi$\\
		nɑmɑɡ-ˈ\textbf{neɾ} uˈn-i-$\emptyset$ \underline{mɑɾˈ\textbf{jɑ-n}} \\
		letter-{\pl} have-{\thgloss}-3{\sg} Maria-{\defgloss} \\
		\trans `Maria has letters.' (`Maria' is an afterthought)\label{example:inton:broad:otherOrder:ovs}
		\\ \armenian{Նամակներ ունի Մարիան։}
		\ex \glll   (\textbf{{S}})$_\phi$    (\textbf{{V}})$_\phi$ (\underline{\textbf{O}})$_\phi$\\
		{mɑɾˈ\textbf{jɑ-n}} uˈ\textbf{n-i-$\emptyset$}  \underline{nɑmɑɡ-ˈ\textbf{neɾ}}  \\
		Maria-{\defgloss}   have-{\thgloss}-3{\sg} letter-{\pl}\\
		\trans `Maria has letters.' (`Maria' and `has' are treated as given)\label{example:inton:broad:otherOrder:svo}
		\\ \armenian{Մարիան  ունի նամակներ։}
	\end{xlist}
\end{exe}
\subsection{Intransitives and passives}\label{section:intonation:broadFocus:intrans}
The previous sections focused on (di)transitive sentences, where the verb placed stress on a preceding object. All types of objects received nuclear stress, whether bare or definite.  For other types of valency or voices, we find variation in stress placement. We focus on unaccusatives (\ref{section:intonation:broadFocus:intrans:unacc}), unergatives (\ref{section:intonation:broadFocus:intrans:unerg}), and passives (\ref{section:intonation:broadFocus:intrans:pass}). 

When the pre-verbal noun is indefinite, this noun gets nuclear stress regardless if it a transitive object, unaccusative subject, unergative subject, or passivized object. However, verbs vary in whether definite noun phrases get stressed. A definite transitive object or definite  unaccusative subject gets nuclear stress, while a definite unergative subject or definite passivized object does not get stress. 


\subsubsection{Unaccusative verbs}\label{section:intonation:broadFocus:intrans:unacc}
First consider unaccusative verbs. These are intransitive verbs where the subject semantically acts more as the undergoer of the verbal action, rather than the doer of the verbal action; the verbal action essentially happens. For example, the subject of `to come' happens to arrive. Other verbs like `to die' or `to fall' are also unaccusative. 

For transitive verbs, the object can be morphologically bare singular or plural, indefinite or definite; and it takes stress. Similarly,  the subject of an unaccusative verb can be morphologically bare  singular (\ref{example:inton:broad:intrans:unacc:bare}) vs.  plural (\ref{example:inton:broad:intrans:unacc:pl}),  indefinite (\ref{example:inton:broad:intrans:unacc:indf}) vs. or definite (\ref{example:inton:broad:intrans:unacc:def}). In all cases, stress is on the subject. 


\begin{exe}
	\ex \begin{xlist}
		\ex \glll (\textbf{\underline{S}} V)$_\phi$ \\
		\underline{nɑˈ\textbf{mɑɡ}} jeˈɡ-ɑ-v \\
		letter come.{\aorperf}-{\pst}-3{\sg} \\
		\trans `Some letters came.' \label{example:inton:broad:intrans:unacc:bare}
		\\\armenian{Նամակ եկաւ։}
		\ex \glll (\textbf{\underline{S-{\pl}}} V)$_\phi$ \\
		\underline{nɑmɑɡ-ˈ\textbf{neɾ}} jeˈɡ-ɑ-n \\
		letter-{\pl} come.{\aorperf}-{\pst}-3{\pl} \\
		\trans `Some letters came.'\label{example:inton:broad:intrans:unacc:pl}
		\\\armenian{Նամակներ եկան։}
		\ex \glll (\textbf{\underline{S-{\indf}}} V)$_\phi$ \\
		\underline{nɑˈ\textbf{mɑɡ}-mə} jeˈɡ-ɑ-v \\
		letter-{\indf} come.{\aorperf}-{\pst}-3{\sg} \\
		\trans `A letter  came.'\label{example:inton:broad:intrans:unacc:indf}
		\\\armenian{Նամակ  մը եկաւ։}
		\ex \glll (\textbf{\underline{S-{\defgloss}}} V)$_\phi$ \\
		\underline{nɑˈ\textbf{mɑ}ɡ-ə} jeˈɡ-ɑ-v \\
		letter-{\defgloss} come.{\aorperf}-{\pst}-3{\sg} \\
		\trans `The letter  came.'\label{example:inton:broad:intrans:unacc:def}
		\\\armenian{Նամակը   եկաւ։}
	\end{xlist}
\end{exe}

The above sentences used the verb `to come'. The same judgments are found for other unnaccusative verbs like  `to fall' (\ref{example:inton:broad:intrans:unacc:fall}) and `to die' (\ref{example:inton:broad:intrans:unacc:die}). 

\begin{exe}
	\ex \begin{xlist}
		\ex  \label{example:inton:broad:intrans:unacc:fall}\begin{xlist}
			\ex \glll (\textbf{\underline{S}} V)$_\phi$ \\
			\underline{bəˈ\textbf{nɑɡ}} iŋˈɡ-ɑ-v \\
			plate fall.{\aorperf}-{\pst}-3{\sg} \\
			\trans `Some plates fell.'
			\\\armenian{Բնակ ինկաւ։}
			\ex \glll (\textbf{\underline{S-{\pl}}} V)$_\phi$ \\
			\underline{bənɑɡ-ˈ\textbf{neɾ}} iŋˈɡ-ɑ-n \\
			plate-{\pl} fall.{\aorperf}-{\pst}-3{\pl} \\
			\trans `Some plates fell.'
			\\\armenian{Բնակներ ինկան։}
			\ex \glll (\textbf{\underline{S-{\indf}}} V)$_\phi$ \\
			\underline{bəˈ\textbf{nɑɡ}-mə} iŋˈɡ-ɑ-v \\
			plate-{\indf} fall.{\aorperf}-{\pst}-3{\sg} \\
			\trans `A plate  fell.'
			\\\armenian{Բնակ մը ինկաւ։}
			\ex \glll (\textbf{\underline{S-{\defgloss}}} V)$_\phi$ \\
			\underline{bəˈ\textbf{nɑ}ɡ-ə} iŋˈɡ-ɑ-v \\
			plate-{\defgloss} fall.{\aorperf}-{\pst}-3{\sg} \\
			\trans `The plate  fell.'
			\\\armenian{Բնակը ինկաւ։}
		\end{xlist}
		\ex  \label{example:inton:broad:intrans:unacc:die} \begin{xlist}
			\ex \glll (\textbf{\underline{S}} V)$_\phi$ \\
			\underline{ɡɑˈ\textbf{du}} meˈɾ-ɑ-v \\
			cat die-{\pst}-3{\sg} \\
			\trans `Some cats died.'
			\\\armenian{Կատու մեռաւ։}
			\ex \glll (\textbf{\underline{S-{\pl}}} V)$_\phi$ \\
			\underline{ɡɑdu-ˈ\textbf{neɾ}} meˈɾ-ɑ-n \\
			cat-{\pl} die-{\pst}-3{\pl} \\
			\trans `Some cats died.'
			\\\armenian{Կատուներ մեռան։}
			\ex \glll (\textbf{\underline{S-{\indf}}} V)$_\phi$ \\
			\underline{ɡɑˈ\textbf{du}-mə} meˈɾ-ɑ-v \\
			cat-{\indf} die-{\pst}-3{\sg} \\
			\trans `A cat  died.'
			\\\armenian{Կատու մը մեռաւ։}
			\ex \glll (\textbf{\underline{S-{\defgloss}}} V)$_\phi$ \\
			\underline{ɡɑˈ\textbf{du-n}} meˈɾ-ɑ-v \\
			cat-{\defgloss} die-{\pst}-3{\sg} \\
			\trans `The cat  died.'
			\\\armenian{Կատուն մեռաւ։}
			
		\end{xlist}
		
	\end{xlist}
\end{exe}

A special type of unaccusative verb is the verb `to exist' or [ɡɑ] (\ref{example:inton:broad:intrans:unacc:exist}). Stress is again on the subject.  


\begin{exe}
	\ex \label{example:inton:broad:intrans:unacc:exist}\begin{xlist}
		\ex \glll (\textbf{\underline{S}} V)$_\phi$ \\
		\underline{bɑdˈ\textbf{d͡ʒɑɾ}} ˈɡ-ɑ-$\emptyset$  \\
		reason exist-{\thgloss}-3{\sg} \\
		\trans `There's a reason.' 
		\\\armenian{Պատճառ կայ։}
		\ex \glll (\textbf{\underline{S-{\pl}}} V)$_\phi$ \\
		\underline{bɑdd͡ʒɑɾ-ˈ\textbf{neɾ}} ˈɡ-ɑ-n  \\
		reason-{\pl} exist-{\thgloss}-3{\pl} \\
		\trans `There are   reasons.'
		\\\armenian{Պատճառներ   կան։}
		\ex \glll (\textbf{\underline{S-{\indf}}} V)$_\phi$ \\
		\underline{bɑdˈ\textbf{d͡ʒɑɾ}-mə}  ˈɡ-ɑ-$\emptyset$  \\
		reason-{\indf}  exist-{\thgloss}-3{\sg} \\
		\trans `There's a reason.'
		\\\armenian{Պատճառ  մը կայ։}
		\ex \glll (\textbf{\underline{S-{\defgloss}}} V)$_\phi$ \\
		\underline{bɑdˈ\textbf{d͡ʒɑ}ɾ-ə}  ˈɡ-ɑ-$\emptyset$  \\
		reason-{\defgloss} exist-{\thgloss}-3{\sg} \\
		\trans `There's the reason.'
		\\\armenian{Պատճառը   կայ։}
	\end{xlist}
\end{exe}

In sum, an unaccusative subject gets stress like a transitive object. 
\subsubsection{Unergative verbs}\label{section:intonation:broadFocus:intrans:unerg}
In contrast to unaccusative verbs, an unergative verb is an intransitive verb where the subject is the doer of the verbal action. An example is the verb `to run'. A morphosyntactic contrast between unaccusative and unergative verbs is that the subject of an accusative can be morphologically bare (\ref{example:inton:broad:intrans:unerg:bareyes}), but the subject of an unergative is generally not (\ref{example:inton:broad:intrans:unerg:bareno}). 


\begin{exe}
	\ex \begin{xlist}
		\ex \glll (\textbf{\underline{S}} V)$_\phi$ \\
		\underline{ɡəɾˈ\textbf{jɑ}}  meˈɾ-ɑ-v \\
		tortoise die-{\pst}-3{\sg} \\
		\trans `Some tortoise died.'\label{example:inton:broad:intrans:unerg:bareyes}
		\\ \armenian{Կրիայ մեռաւ։}
		\ex \gll *?{ɡəɾˈ{jɑ}}  vɑˈz-e-t͡s-$\emptyset$-$\emptyset$ \\
		tortoise run-{\thgloss}-{\aorperf}-{\pst}-3{\sg} \\
		\trans `Some tortoise ran.'\label{example:inton:broad:intrans:unerg:bareno}
		\\ \armenian{Կրիայ վազեց։}
	\end{xlist}
\end{exe}

The subject of an unergative must have some type of inflectional suffix, such as a plural suffix (\ref{example:inton:broad:intrans:unerg:pl}), indefinite suffix  (\ref{example:inton:broad:intrans:unerg:indf}), or the definite suffix (\ref{example:inton:broad:intrans:unerg:def}). A bare plural and indefinite subject is stressed, but the definite subject is not stressed. 


\begin{exe}
	\ex \begin{xlist}
		\ex \glll (\textbf{\underline{S-{\pl}}} V)$_\phi$ \\
		\underline{{ɡəɾ{jɑ}-ˈ\textbf{neɾ}}}  vɑz-e-ˈt͡s-i-n \\
		tortoise-{\pl} run-{\thgloss}-{\aorperf}-{\pst}-3{\pl} \\
		\trans `Some tortoises ran.' \label{example:inton:broad:intrans:unerg:pl}
		\\ \armenian{Կրիաներ վազեցին։}
		\ex \glll (\textbf{\underline{S-{\indf}}} V)$_\phi$ \\
		\underline{{ɡəɾˈ\textbf{jɑ}-mə}}  vɑˈz-e-t͡s-$\emptyset$-$\emptyset$ \\ 
		tortoise-{\indf} run-{\thgloss}-{\aorperf}-{\pst}-3{\sg} \\
		\trans `A tortoise ran.'  \label{example:inton:broad:intrans:unerg:indf}
		\\ \armenian{Կրիայ մը վազեց։}
		\ex \glll (\textbf{{S-{\defgloss}}})$_\phi$ (\textbf{\underline{V}})$_\phi$ \\
		{ɡəɾˈ\textbf{jɑ-n}}   \underline{vɑˈ\textbf{z-e-t͡s-$\emptyset$-$\emptyset$}} \\ 
		tortoise-{\defgloss} run-{\thgloss}-{\aorperf}-{\pst}-3{\sg} \\
		\trans `The tortoise ran.' \label{example:inton:broad:intrans:unerg:def}
		\\ \armenian{Կրիան վազեց։}
	\end{xlist}
\end{exe}

A definite unergative subject is phrased separately from the verb. The same generalizations apply to other unergative verbs like `to shout' (\ref{example:inton:broad:intrans:unerg:shout}) or `to laugh' (\ref{example:inton:broad:intrans:unerg:laugh}). 

\begin{exe}
	\ex 
	\begin{xlist}
		\ex \label{example:inton:broad:intrans:unerg:shout}\begin{xlist}
			\ex \glll (\textbf{\underline{S-{\pl}}} V)$_\phi$ \\
			\underline{{mɑnuɡ-ˈ\textbf{neɾ}}}  boɾ-ɑ-ˈt͡s-i-n \\
			child-{\pl} shout-{\thgloss}-{\aorperf}-{\pst}-3{\pl} \\
			\trans `Some children shouted.'  
			\\ \armenian{Մանուկներ պոռացին։}
			\ex \glll (\textbf{\underline{S-{\indf}}} V)$_\phi$ \\
			\underline{{mɑˈ\textbf{nuɡ}-mə}}  boˈɾ-ɑ-t͡s-$\emptyset$-$\emptyset$ \\  
			child-{\indf} shout-{\thgloss}-{\aorperf}-{\pst}-3{\sg} \\
			\trans `A child shouted.'
			\\ \armenian{Մանուկ մը պոռաց։}
			\ex \glll (\textbf{{S-{\defgloss}}})$_\phi$ (\textbf{\underline{V}})$_\phi$ \\
			{{mɑˈ\textbf{nu}ɡ-ə}}   \underline{boˈ\textbf{ɾ-ɑ-t͡s-$\emptyset$-$\emptyset$}} \\ 
			child-{\defgloss} shout-{\thgloss}-{\aorperf}-{\pst}-3{\sg} \\
			\trans `The child shouted.'
			\\ \armenian{Մանուկը պոռաց։}
		\end{xlist}
		
		\ex \label{example:inton:broad:intrans:unerg:laugh} \begin{xlist}
			\ex \glll (\textbf{\underline{S-{\pl}}} V)$_\phi$ \\
			\underline{{ɡɑbiɡ-ˈ\textbf{neɾ}}}  χəntʰ-ɑ-ˈt͡s-i-n \\
			monkey-{\pl} laugh-{\thgloss}-{\aorperf}-{\pst}-3{\pl} \\
			\trans `Some monkeys laughed.'
			\\ \armenian{Կապիկներ խնդացին։}
			\ex \glll (\textbf{\underline{S-{\indf}}} V)$_\phi$ \\
			\underline{{ɡɑˈ\textbf{biɡ}-mə}}  χənˈtʰ-ɑ-t͡s-$\emptyset$-$\emptyset$ \\ 
			monkey-{\indf} laugh-{\thgloss}-{\aorperf}-{\pst}-3{\sg} \\
			\trans `A monkey laughed.'
			\\ \armenian{Կապիկ մը խնդաց։}
			\ex \glll (\textbf{{S-{\defgloss}}})$_\phi$ (\textbf{\underline{V}})$_\phi$ \\
			{{ɡɑˈ\textbf{bi}ɡ-ə}} \underline{χənˈ\textbf{tʰ-ɑ-t͡s-$\emptyset$-$\emptyset$}} \\ 
			monkey-{\defgloss} laugh-{\thgloss}-{\aorperf}-{\pst}-3{\sg} \\
			\trans `The monkey laughed.'
			\\ \armenian{Կապիկը խնդաց։}
		\end{xlist}
		
		
	\end{xlist}
\end{exe}

Thus, intransitives   avoid stress on definite subjects. Corroborating data also comes from relative clause extraposition (\S\ref{section:intonation:other:extraposition}). 

\subsubsection{Passive verbs}\label{section:intonation:broadFocus:intrans:pass}

A transitive active verb can be passivized by adding the passive suffix to the verb. The direct object gets promoted to the grammatical subject. The passivized object can be bare (\ref{example:inton:broad:intrans:pass:barepass}), just like an active object (\ref{example:inton:broad:intrans:pass:baretrans}). 


\begin{exe}
	\ex \begin{xlist}
		\ex \glll (\textbf{S})$_\phi$ (\textbf{\underline{O}} V)$_\phi$ \\
		ˈ\textbf{mɑɾ}tʰ-ə \underline{nɑˈ\textbf{mɑɡ}} kʰəˈɾ-e-t͡s-$\emptyset$-$\emptyset$ \\
		man-{\defgloss} letter write-{\thgloss}-{\aorperf}-{\pst}-3{\sg} \\
		\trans `The man wrote letters.' \label{example:inton:broad:intrans:pass:baretrans}
		\\ \armenian{Մարդը նամակ գրեց։}
		\ex \glll  (\textbf{\underline{S}} V-{\pass})$_\phi$ \\
		\underline{nɑˈ\textbf{mɑɡ}} kʰəɾ-v-e-ˈ{t͡s-ɑ-v} \\
		letter write-{\pass}-{\thgloss}-{\aorperf}-{\pst}-3{\sg} \\
		\trans `Some letter was written.' \label{example:inton:broad:intrans:pass:barepass}
		\\ \armenian{Նամակ գրուեցաւ։}
	\end{xlist}
\end{exe}

The passivized object can be bare singular (\ref{example:inton:broad:intrans:pass:barepass}) or   plural (\ref{example:inton:broad:intrans:pass:pl}), indefinite  (\ref{example:inton:broad:intrans:pass:indf}) or definite (\ref{example:inton:broad:intrans:pass:def}). The first three conditions place stress on the passivize object. But a definite passivized object  (\ref{example:inton:broad:intrans:pass:def}) prefers to be phrased separately from the verb, with stress on the verb. 

\begin{exe}
	\ex \begin{xlist}
		\ex \glll  (\textbf{\underline{S-{\pl}}} V-{\pass})$_\phi$ \\
		\underline{nɑ{mɑɡ}-ˈ\textbf{neɾ}} kʰəɾ-v-e-ˈ{t͡s-ɑ-n} \\
		letter-{\pl} write-{\pass}-{\thgloss}-{\aorperf}-{\pst}-3{\pl} \\
		\trans `Some letters were written.' \label{example:inton:broad:intrans:pass:pl}
		\\ \armenian{Նամակներ գրուեցան։}
		\ex \glll  (\textbf{\underline{S-{\indf}}} V-{\pass})$_\phi$ \\
		\underline{nɑˈ\textbf{mɑɡ}-mə} kʰəɾ-v-e-ˈ{t͡s-ɑ-v} \\
		letter-{\indf} write-{\pass}-{\thgloss}-{\aorperf}-{\pst}-3{\sg} \\
		\trans `A letter was  written.'\label{example:inton:broad:intrans:pass:indf}
		\\ \armenian{Նամակ մը  գրուեցաւ։}
		\ex \glll (\textbf{{S-{\defgloss}}})$_\phi$ (\textbf{\underline{V-{\pass}}})$_\phi$ \\
		{{nɑˈ\textbf{mɑ}ɡ-ə}} \underline{kʰəɾ-v-e-\textbf{t͡s-ɑ-v}} \\ 
		letter-{\defgloss} write-{\pass}-{\thgloss}-{\aorperf}-{\pst}-3{\sg} \\
		\trans `The letter was written.'\label{example:inton:broad:intrans:pass:def}
		\\ \armenian{Նամակը գրուեցաւ։} 
		
	\end{xlist}
\end{exe}

Some judgments seem constant for any passive verb, such as `to be killed' (\ref{example:inton:broad:intrans:pass:kill}). 


\begin{exe}
	\ex \label{example:inton:broad:intrans:pass:kill} \begin{xlist}
		\ex \glll  (\textbf{\underline{S}} V-{\pass})$_\phi$ \\
		\underline{tʰəʃnɑˈ\textbf{mi}} əspɑnnə-v-e-ˈ{t͡s-ɑ-v} \\
		enemy kill-{\pass}-{\thgloss}-{\aorperf}-{\pst}-3{\sg} \\
		\trans `Some enemy was killed.'
		\\ \armenian{Թշնամի սպաննուեցաւ։}
		\ex \glll  (\textbf{\underline{S-{\pl}}} V-{\pass})$_\phi$ \\
		\underline{tʰəʃnɑmi-ˈ\textbf{neɾ}} əspɑnnə-v-e-ˈ{t͡s-ɑ-n} \\
		enemy-{\pl} kill-{\pass}-{\thgloss}-{\aorperf}-{\pst}-3{\pl} \\
		\trans `Some enemies were killed.'
		\\ \armenian{Թշնամիներ սպաննուեցան։}
		\ex \glll  (\textbf{\underline{S-{\indf}}} V-{\pass})$_\phi$ \\
		\underline{tʰəʃnɑˈ\textbf{mi}-mə} əspɑnnə-v-e-ˈ{t͡s-ɑ-v} \\
		enemy-{\indf} kill-{\pass}-{\thgloss}-{\aorperf}-{\pst}-3{\sg} \\
		\trans `An enemy was  killed.'
		\\ \armenian{Թշնամի մը  սպաննուեցաւ։}
		\ex \glll (\textbf{{S-{\defgloss}}})$_\phi$ (\textbf{\underline{V-{\pass}}})$_\phi$ \\
		{{tʰəʃnɑˈ\textbf{mi-n}}} \underline{əspɑnnə-v-e-\textbf{t͡s-ɑ-v}} \\ 
		enemy-{\defgloss} kill-{\pass}-{\thgloss}-{\aorperf}-{\pst}-3{\sg} \\
		\trans `The enemy was killed.'
		\\ \armenian{Թշնամին սպաննուեցաւ։} 
		
	\end{xlist}
\end{exe}

Thus, intransitives and passives pattern together in avoiding stress on definite subjects. Corroborating data also comes from relative clause extraposition (\S\ref{section:intonation:other:extraposition}). 

\section{Intonation of declaratives, questions, and focus}\label{section:intonation:focus}

In contrast to a broad-focus context, a context is said to have narrow focus if some word has more semantic salience than others. In declarative sentences, this can be a word with negation or a word that is focused. In interrogatives or questions, this is the word that is being questioned. This section goes through the basic acoustics of declaratives in broad focus (\S\ref{section:intonation:focus:declarative}), and then an in-depth catalog of intonational contours for  narrow-focus contexts. These narrow-focus contexts involve negation (\S\ref{section:intonation:focus:declarative}), polar questions and their answers (\S\ref{section:intonation:focus:polar}), and  wh-questions and their answers (\S\ref{section:intonation:focus:wh}). We did not consider more complex types of questions such as choice questions or multiple-wh questions, but see \citet{ToparlakDolatian-202x-IntonationFocusMarkingWesternArmenian} for preliminary results.  

Throughout this section, we concentrate on marking the words which get nuclear stress (underlined) and on marking the sentence-final pitch (with arrows $\searrow$, $\nearrow$).   Building off of \citet{ToparlakDolatian-202x-IntonationFocusMarkingWesternArmenian}, we use simple autosegmental-metrical annotation for nuclear stress (H*) and for sentence-final pitches (L\%, H\%). 

Note that in many contexts, we find post-focal deaccenting after the focused or nuclear stress-bearing word. Impressionistically, we still perceive prosodic phrase boundaries and lexical stress, so we demarcate such phrases and lexical stresses.  But such impressions are likely just a psycholinguisitc illusions. 

\subsection{Declarative sentences with and without negation} \label{section:intonation:focus:declarative}

This section goes over the intonation of  basic declaratives. We find various general acoustic properties: 
\begin{itemize}[noitemsep,topsep=0pt]
	\item Declination: The pitch of the sentence decreases as we go from left to right. 
	\item Falling tone: Declarative sentences end in falling pitch L\%. 
	\item Nuclear stress: Nuclear stress is marked by a prominent high pitch H*. 
	\item Negation: Negated verbs attract nuclear stress.
	\item Post-focal deaccenting: After the word nuclear stress,  lexical stresses are weakened. 
\end{itemize}

We first go over basic SOV declaratives, in the positive and negative (\S\ref{section:intonation:focus:declarative:basicSOV}). We then discuss the intonation of periphrastic or cliticized verb phrases (\S\ref{section:intonation:focus:declarative:complex}), and then other word orders like OVS (\S\ref{section:intonation:focus:declarative:otherWord}). 

\subsubsection{Basic SOV declarative sentences}\label{section:intonation:focus:declarative:basicSOV}
A simple declarative sentence has positive polarity if there is no negation. For a simple SOV sentence (\ref{example:inton:focus:dec:basic:sov:pos}), we find stress on the object (H*), and then a sentence-final fall (L\%).


\begin{exe}
	\ex \glll   (\textbf{S})$_\phi$ (\underline{\textbf{O}} V)$_\phi$ $\searrow$ \\
	mɑɾˈ\textbf{jɑ-n} \underline{nɑˈ\textbf{mɑɡ}} uˈn-i-$\emptyset$ $\searrow$ \\
	Maria-{\defgloss} letter have-{\thgloss}-3{\sg}   \\
	\trans `Maria has some letters.' \label{example:inton:focus:dec:basic:sov:pos}
	\\ \armenian{Մարիան նամակ ունի։}
\end{exe}

The pitch track for this positive declarative is in (). Acoustically, the positive sentence has prominence H* on the object, the verb is deaccented via post-focal compression, and then the sentence ends in final L\%. The pitch of the subject seems higher than the pitch of the object. This suggests that Armenian has pitch downdrift or declension within a sentence. 

\textcolor{blue}{draw}



To create a negative sentence, the negation prefix \textit{t͡ʃ-} is added to the verb (\ref{example:inton:focus:dec:basic:sov:neg}). Stress then shifts to this negated verb. Although we're not completely sure, we suspect that the object and negated verb are now separated into two separate prosodic phrases. 

\begin{exe}
	\ex \glll   (\textbf{S})$_\phi$ (\textbf{O})$_\phi$ (\underline{\textbf{{\neggloss}-V}})$_\phi$ $\searrow$ \\
	mɑɾˈ\textbf{jɑ-n} {nɑˈ\textbf{mɑɡ}} \underline{ˈ\textbf{t͡ʃ-u}n-i-$\emptyset$}  $\searrow$ \\
	Maria-{\defgloss} letter {\neggloss}-have-{\thgloss}-3{\sg}   \\
	\trans `Maria does not have letters.' \label{example:inton:focus:dec:basic:sov:neg}
	\\ \armenian{Մարիան նամակ չունի։}
\end{exe}

Note that lexical stress is on the first syllable of the negated verb. For more prosodic data on negation in words, see \S\ref{section:stress:verb:negFinite}. 

Acoustically   for the negative version, stress H* is on the negated verb (in its first syllable). The rest of the verb is deaccented and we again end in a final L\%. 


\textcolor{blue}{draw}


\subsubsection{SOV declarative sentences with complex verb phrases}\label{section:intonation:focus:declarative:complex} 

There are other possible morphosyntactic permutations for declarative positive and negative sentences. These all show the same basic intonational shapes.  

Consider cliticized verbs. In a predicate sentence (S-Adj-V), stress is on the adjective while the verb is a copular enclitic (\ref{example:inton:focus:declarative:complex:adj:pos}). When the sentence is negated (\ref{example:inton:focus:declarative:complex:adj:neg}), the negation prefix is placed on the copula, and the copula becomes its own standalone word (a prosodic word) with nuclear stress. 


\begin{exe}
	\ex \begin{xlist}
		\ex \glll   (\textbf{S})$_\phi$ (\underline{\textbf{Adj}} Cop)$_\phi$ $\searrow$ \\
		mɑɾˈ\textbf{jɑ-n} \underline{ɡɑɾˈ\textbf{miɾ}} =e \\
		Maria-{\defgloss} red =is   \\
		\trans `Maria is red.' \label{example:inton:focus:declarative:complex:adj:pos}
		\\ \armenian{Մարիան կարմիր է։} 
		\ex \glll   (\textbf{S})$_\phi$ (\textbf{Adj})$_\phi$ (\underline{\textbf{{\neggloss}-Cop}})$_\phi$ $\searrow$ \\
		mɑɾˈ\textbf{jɑ-n} {ɡɑɾˈ\textbf{miɾ}} \underline{ˈ\textbf{t͡ʃ-e}} $\searrow$ \\
		Maria-{\defgloss} red   {\neggloss}-is \\ 
		\trans `Maria is not red.'  \label{example:inton:focus:declarative:complex:adj:neg}
		\\ \armenian{Մարիան կարմիր չէ։}
		
	\end{xlist}
\end{exe}

We see pitch-tracks in (). Note how the negated copula has a perceivable rise at the beginning of its only syllable, and then falls due to the sentence-final fall L\%. 

\textcolor{blue}{draw}

We can see the interaction between negation stress and the sentence-final fall when the copula is bisyllabic. In the positive (\ref{example:inton:focus:declarative:complex:adj:bi:pos}), the copula is unstressed, while the negative form (\ref{example:inton:focus:declarative:complex:adj:bi:neg}) has stress on the first syllable of the copula. 


\begin{exe}
	\ex \begin{xlist}
		\ex \glll   (\textbf{S})$_\phi$ (\underline{\textbf{Adj}} Cop)$_\phi$ $\searrow$ \\
		dəˈ\textbf{χɑ}-kʰ-ə \underline{ɡɑɾˈ\textbf{miɾ}} =e-ji-n  $\searrow$ \\
		boy-{\pl}{\defgloss} red =is-{\pst}-3{\pl}   \\
		\trans `The boys were red.' 
		\label{example:inton:focus:declarative:complex:adj:bi:pos}
		\\ \armenian{Տղաքը կարմիր էին։} 
		\ex \glll   (\textbf{S})$_\phi$ (\textbf{Adj})$_\phi$ (\underline{\textbf{{\neggloss}-Cop}})$_\phi$ $\searrow$ \\
		dəˈ\textbf{χɑ}-kʰ-ə  {ɡɑɾˈ\textbf{miɾ}} \underline{ˈ\textbf{t͡ʃ-e}-ji-n} $\searrow$ \\
		boy-{\pl}-{\defgloss} red   {\neggloss}-is-{\pst}-3{\pl} \\ 
		\trans `The boys were not red.' \label{example:inton:focus:declarative:complex:adj:bi:neg}
		\\ \armenian{Տղաքը կարմիր չէին։}
		
	\end{xlist}
\end{exe}

Notice how in the above constructions, the negated verb was the final word in the sentence. This is usually the case. But there are inflectional paradigms where the verb is periphrastic, and this creates sentences where the negated element is     earlier in the sentence. For example, consider the present perfect. This consists of a participle and the inflected auxiliary. In the positive (\ref{example:inton:focus:declarative:complex:periph:pos}), stress is early in the sentence on the preverbal object. But in the negative (\ref{example:inton:focus:declarative:complex:periph:neg}), the auxiliary is negated and placed before the verb. This negated auxiliary carries stress (\S\ref{section:stress:verb:negPeriph}). 


\begin{exe}
	\ex \begin{xlist}
		\ex \glll   (\textbf{S})$_\phi$ (\underline{\textbf{O}} V Aux)$_\phi$ $\searrow$ \\
		mɑɾˈ\textbf{jɑ-n} \underline{nɑ{mɑɡ}-ˈ\textbf{neɾ}} kʰəˈɾ-ɑd͡z =e  $\searrow$ \\
		Maria-{\defgloss} letter-{\pl} write-{\rptcp} =is \\
		\trans `Maria has written letters.' 
		\label{example:inton:focus:declarative:complex:periph:pos}
		\\ \armenian{Մարիան նամակներ գրած է։}
		\ex \glll   (\textbf{S})$_\phi$ ({\textbf{O}}) (\underline{\textbf{{\neggloss}-Aux}} V)$_\phi$ $\searrow$ \\
		mɑɾˈ\textbf{jɑ-n} {nɑ{mɑɡ}-ˈ\textbf{neɾ}} \underline{ˈ\textbf{t͡ʃ-e}} kʰəˈɾ-ɑd͡z   $\searrow$ \\
		Maria-{\defgloss} letter-{\pl} {\neggloss}-is write-{\rptcp}  \\
		\trans `Maria has not written letters.' 
		\label{example:inton:focus:declarative:complex:periph:neg}
		\\ \armenian{Մարիան նամակներ չէ գրած։}
		
	\end{xlist}
\end{exe}

Acoustically, the positive has stress H* on the object, while the participle and auxiliary are deaccented. In the negative, the auxiliary has stress H*, and causes deaccenting on the final participle. 

\textcolor{blue}{draw}

Other similarly long inflectional constructions include the simple future. The verb takes a future proclitic \textit{bidi}. In the positive (\ref{example:inton:focus:declarative:complex:fut:pos}), stress is on the preverbal object (here a locative noun). In the negative (\ref{example:inton:focus:declarative:complex:fut:neg}), the negation prefix is placed on the verb, and stress is on the verb. 


\begin{exe}
	\ex \begin{xlist}
		\ex \glll   (\textbf{S})$_\phi$ (\underline{\textbf{O}} Pro V)$_\phi$ $\searrow$ \\
		mɑɾˈ\textbf{jɑ-n} \underline{ˈ\textbf{dun}} bidi jeɾˈtʰ-ɑ-$\emptyset$   $\searrow$ \\
		Maria-{\defgloss} home {\fut} go-{\thgloss}-3{\sg} \\
		\trans `Maria will go home.'
		\label{example:inton:focus:declarative:complex:fut:pos}
		\\ \armenian{Մարիան տուն պիտի երթայ։} 
		\ex \glll   (\textbf{S})$_\phi$ ({\textbf{O}})$_\phi$ (Pro \underline{\textbf{{\neggloss}-V}})$_\phi$ $\searrow$ \\
		mɑɾˈ\textbf{jɑ-n} {\textbf{dun}} bidi \underline{ˈ\textbf{t͡ʃ-eɾ}tʰ-ɑ-$\emptyset$}     $\searrow$ \\
		Maria-{\defgloss} home  {\fut}  {\neggloss}-go-{\thgloss}-3{\sg} \\ 
		\trans `Maria will not go home.'
		\label{example:inton:focus:declarative:complex:fut:neg}
		\\ \armenian{Մարիան տուն պիտի չերթայ։} 
		
	\end{xlist}
\end{exe}

Pitch-tracks confirm our impressions. 

\textcolor{blue}{draw}

Some periphrastic constructions combine the main verb (as a participle) with a light verb. In the positive (\ref{example:inton:focus:declarative:complex:have:pos}), the participle    carries stress, while the light verb is deaccented. Note how the object usually does not carry nuclear stress for such complex tenses; this is because such tenses tend to imply that the object is given. In the negative (\ref{example:inton:focus:declarative:complex:have:neg}), the  light verb takes negation and stress. 

\begin{exe}
	\ex \begin{xlist}
		\ex \glll (\textbf{S})$_\phi$ (\textbf{O})$_\phi$ (Pro \textbf{\underline{V}} lightV)$_\phi$ $\searrow$ \\
		mɑɾˈ\textbf{jɑ-n} nɑmɑɡ-ˈ\textbf{neɾ}  bidi \underline{kʰəˈ\textbf{ɾ-ɑd͡}z} əlˈl-ɑ-$\emptyset$ $\searrow$\\
		Maria-{\defgloss} letter-{\pl} {\fut} write-{\rptcp} be-{\thgloss}-3{\sg}  \\
		\trans `Maria will have written some letters.' 
		\label{example:inton:focus:declarative:complex:have:pos}
		\\ \armenian{Մարիան նամակներ պիտի գրած ըլլայ։}
		\ex \glll (\textbf{S})$_\phi$ (\textbf{O})$_\phi$ (Pro V \underline{\textbf{{\neggloss}-lightV}})$_\phi$ $\searrow$ \\
		mɑɾˈ\textbf{jɑ-n} nɑmɑɡ-ˈ\textbf{neɾ}  bidi {kʰəˈɾ-ɑd͡z} \underline{ˈ\textbf{t͡ʃ-əl}l-ɑ-$\emptyset$}  $\searrow$\\
		Maria-{\defgloss} letter-{\pl} {\fut} write-{\rptcp} {\neggloss}-be-{\thgloss}-3{\sg}  \\
		\trans `Maria will not have   written some letters.' 
		\label{example:inton:focus:declarative:complex:have:neg}
		\\ \armenian{Մարիան նամակներ պիտի գրած չըլլայ։}
		
	\end{xlist}
\end{exe}

Pitch-tracks confirm our impressions. 


\textcolor{blue}{draw}

\subsubsection{Other word orders and function words}\label{section:intonation:focus:declarative:otherWord}

The previous examples were mostly all SOV sentences. Other word orders are possible such as OVS. In such sentences however, the negative verb takes stress and causes post-focal deaccenting on all subsequent words. The sentence then ends in a fall. 

To illustrate, the sentences below all postpone the subject till the end of the sentence (\ref{example:inton:focus:declarative:other:ovs}). Stress is still on the negated verb. The subject is deaccented and no longer bears perceivable phrasal stress. 

\begin{exe}
	\ex \label{example:inton:focus:declarative:other:ovs} \begin{xlist}
		\ex \glll (\textbf{O})$_\phi$ (\textbf{\underline{\neggloss-V}})$_\phi$  ({S})$_\phi$  $\searrow$ \\
		nɑmɑɡ-ˈ\textbf{neɾ} \underline{ˈ\textbf{t͡ʃ-u}n-i-$\emptyset$} mɑɾˈ{jɑ-n} $\searrow$ \\
		letter-{\pl} {\neggloss}-have-{\thgloss}-3{\sg} Maria-{\defgloss} \\
		\trans `Maria does not have letters.'
		\\ \armenian{նամակներ չունի Մարիան։}
		\ex \glll (\textbf{Adj})$_\phi$ (\textbf{\underline{{\neggloss}-Cop}})$_\phi$  ({S})$_\phi$  $\searrow$ \\
		ɡɑɾˈ\textbf{miɾ} \underline{ˈ\textbf{t͡ʃ-u}n-i-$\emptyset$} mɑɾˈ{jɑ-n} $\searrow$ \\
		red  {\neggloss}-have-{\thgloss}-3{\sg} Maria-{\defgloss} \\
		\trans `Maria is not red.'
		\\ \armenian{Կարմիր չունի Մարիան։}
		\ex \glll   ({\textbf{O}}) (\underline{\textbf{{\neggloss}-Aux}} V)$_\phi$ ({S})$_\phi$  $\searrow$ \\
		{nɑ{mɑɡ}-ˈ\textbf{neɾ}} \underline{ˈ\textbf{t͡ʃ-e}} kʰəˈɾ-ɑd͡z mɑɾˈ{jɑ-n}    $\searrow$ \\
		letter-{\pl} {\neggloss}-is write-{\rptcp}  Maria-{\defgloss} \\
		\trans `Maria has not written letters.' 
		\\ \armenian{Նամակներ չէ գրած Մարիան։}
	\end{xlist}
\end{exe}

The pitch tracks in () show that the subjects in the above sentences are all deaccented without any  prominent pitch. HD however still perceives the lexical stress of these subjects. This perception is likely just a psycholinguistic illusion. 

Finally, as a special category, consider negative sentences that include negation-related function words (negative polarity items or NPI) such as `any' [het͡ʃ] (\ref{example:inton:focus:declarative:other:func}). In such sentences, the negative word tends to also require negation on the verb. HD perceives almost equal levels of stress on both the NPI and the negated verb. The verb is higher, but the NPI is also quite high. 


\begin{exe}
	\ex \glll   (\textbf{S})$_\phi$ (\underline{\textbf{NPI}} {O})$_\phi$ (\underline{\textbf{{\neggloss}-V}})$_\phi$ $\searrow$ \\
	mɑɾˈ\textbf{jɑ-n}  \underline{\textbf{het͡ʃ}}  {nɑˈ{mɑɡ}} \underline{ˈ\textbf{t͡ʃ-u}n-i-$\emptyset$}  $\searrow$ \\
	Maria-{\defgloss} any letter {\neggloss}-have-{\thgloss}-3{\sg}   \\
	\trans `Maria does not have any letters.' \label{example:inton:focus:declarative:other:func}
	\\ \armenian{Մարիան հէչ նամակ չունի։}
\end{exe}

Acoustically, we find prominence H* on the NPI and the negated verb. The intervening object looks deaccented. 

\textcolor{blue}{draw}



\subsection{Polar questions and their answers}\label{section:intonation:focus:polar}

The previous section looked at declarative sentences. This section looks at polar questions. 
Polar questions are also called yes-no questions or interrogatives. These are questions like ``Did you read'' where the answer is expected to be either a `yes' or `no'. In Standard Armenian, a declarative and a polar question are distinguished only by intonation. Some morphological operations are attested in colloquial speech, but they are rather marginal in use; these are discussed in (\S\ref{section:intonation:phrase:nounverb:participle}).


Polar questions show the following general properties:
\begin{itemize}[noitemsep, topsep=0pt]
	\item Nuclear stress: the verb has nuclear stress H*. 
	\item Final rise: The sentence ends in a rising tone H\%.
\end{itemize}

These properties are found in basic polar questions where the sentence ends in a verb (\S\ref{section:intonation:focus:polar:finalV}), verbal enclitic, or light verb  (\S\ref{section:intonation:focus:polar:finalCl}). Non-final verbs are also stressed and create a high plateau  (\S\ref{section:intonation:focus:polar:finalN}). Negated verbs pattern the same as positives in polar questions  (\S\ref{section:intonation:focus:polar:negation}). Complications arise though for focused non-verbs  (\S\ref{section:intonation:focus:polar:contrastive}) and for questions with the stigmatized question particle  (\S\ref{section:intonation:focus:polar:particle}). 


\subsubsection{Basic polar questions with a final lexical verb}\label{section:intonation:focus:polar:finalV} 

We treat a  polar question as a `basic polar question' if there's no special emphasis on the non-verbal words.  The question asks where some sentence is true or not. The most basic word order is to make the verb final. 


First, consider the basic declarative and polar question below. The sentence is SOV. The declarative (\ref{example:intonation:focus:polar:basic:basic:dec}) has stress H* on the object, with a final fall. The polar question (\ref{example:intonation:focus:polar:basic:basic:polar}) instead places stress H* on the verb, with a strongly perceptible final rise H\%.  We use a ? in the top row of  (\ref{example:intonation:focus:polar:basic:basic:polar}) to explicitly represent questions. 


\begin{exe}
	\ex \begin{xlist}
		\ex \glll (\textbf{S})$_\phi$ (\textbf{\underline{O}} V)$_\phi$ $\searrow$ \\
		mɑɾˈ\textbf{jɑ-n} \underline{nɑmɑɡ-ˈ\textbf{neɾ}} uˈn-i-$\emptyset$ $\searrow$ \\
		Maria-{\defgloss} letter-{\pl} have-{\thgloss}-3{\sg} \\ 
		\trans `Maria has some letters.'
		\label{example:intonation:focus:polar:basic:basic:dec}
		\\ \armenian{Մարիան նամակներ ունի։}
		\ex \glll (\textbf{S})$_\phi$ (\textbf{{O}})$_\phi$ (\textbf{\underline{V}})$_\phi$ ?$\nearrow$ \\
		mɑɾˈ\textbf{jɑ-n} {nɑmɑɡ-ˈ\textbf{neɾ}} \underline{uˈ\textbf{n-i-$\emptyset$}} ?$\nearrow$ \\
		Maria-{\defgloss} letter-{\pl} have-{\thgloss}-3{\sg} \\ 
		\trans `Does Maria have some letters?'
		\label{example:intonation:focus:polar:basic:basic:polar}
		\\ \armenian{Մարիան նամակներ ունի՞։}
	\end{xlist}
\end{exe}

The pitch-tracks in () confirm the presence of the final rise in polar question. 

\textcolor{blue}{draw}


For the above polar question  (\ref{example:intonation:focus:polar:basic:basic:polar}), there isn't contrastive focus on the verb. The verb carries nuclear stress because we are questioning whether the sentence is true or not. To illustrate, a possible answer to this question is to say `no' and then to use either a negated verb (\ref{example:intonation:focus:polar:basic:basic:ans:neg}) or even a different subject (\ref{example:intonation:focus:polar:basic:basic:ans:subj}), object (\ref{example:intonation:focus:polar:basic:basic:ans:objj}), or verb (\ref{example:intonation:focus:polar:basic:basic:ans:verb}).    Nuclear stress is then on the new information (the focused word), and we get a sentence-final fall L\%. 

\begin{exe}
	\ex Negative answers to the polar question in  (\ref{example:intonation:focus:polar:basic:basic:polar}) 
	\begin{xlist}
		\ex \glll (No)$_\phi$ (\textbf{S})$\phi$ (\textbf{O})$_\phi$ (\underline{\textbf{{\neggloss}-V}})$_\phi$ $\searrow$  \\
		ˈ\textbf{vot͡}ʃ,  mɑɾˈ\textbf{jɑ-n} nɑmɑɡ-ˈ\textbf{neɾ} \underline{ˈ\textbf{t͡ʃ-u}n-i-$\emptyset$} $\searrow$ \\
		no, Maria-{\defgloss} letter-{\pl} {\neggloss}-have-{\thgloss}-3{\sg} \\
		\trans `No, Maria does \underline{NOT} have letters.' \\
		\label{example:intonation:focus:polar:basic:basic:ans:neg}
		\armenian{Ոչ, Մարիան նամակներ չունի։}
		\ex \glll (No)$_\phi$ (\underline{\textbf{S}})$\phi$ ({O} V)$_\phi$ $\searrow$  \\
		ˈ\textbf{vot͡}ʃ,  \underline{ɑ\textbf{ɾɑ-}n}  nɑmɑɡ-ˈ{neɾ} {uˈn-i-$\emptyset$} $\searrow$ \\
		no, Ara-{\defgloss} letter-{\pl}  have-{\thgloss}-3{\sg} \\
		\trans `No, \underline{ARA} has    letters.' \\
		\label{example:intonation:focus:polar:basic:basic:ans:subj}
		\armenian{Ոչ, Արան նամակներ  ունի։}
		\ex \glll (No)$_\phi$ ({\textbf{S}})$\phi$ (\underline{\textbf{O}} V)$_\phi$ $\searrow$  \\
		ˈ\textbf{vot͡}ʃ, mɑɾ\textbf{jɑ-}n  \underline{seʁɑn-ˈ\textbf{neɾ}} {uˈn-i-$\emptyset$} $\searrow$ \\
		no, Maria-{\defgloss} table-{\pl}  have-{\thgloss}-3{\sg} \\
		\trans `No, Maria  has \underline{TABLES}.' \\
		\label{example:intonation:focus:polar:basic:basic:ans:objj}
		\armenian{Ոչ, Մարիան սեղաններ  ունի։}
		\ex \glll (No)$_\phi$ (\textbf{S})$\phi$ (\textbf{O})$_\phi$ (\underline{\textbf{V}})$_\phi$ $\searrow$  \\
		ˈ\textbf{vot͡}ʃ,  mɑɾˈ\textbf{jɑ-n} nɑmɑɡ-ˈ\textbf{neɾ} \underline{d͡zɑˈ\textbf{χ-e-t͡s-$\emptyset$-$\emptyset$}}  $\searrow$ \\
		no, Maria-{\defgloss} letter-{\pl} sell-{\thgloss}-{\aorperf}-{\pst}-{\sg} \\
		\trans `No, Maria  \underline{SOLD}  letters.' \\
		\label{example:intonation:focus:polar:basic:basic:ans:verb}
		\armenian{Ոչ, Մարիան նամակներ ծախեց։}
	\end{xlist}
\end{exe}

Acoustically, the pitch-tracks show the presence of a sentence-final L\% for these negative answers. The focused word carries stress H*, and subsequent words are deaccented. Word     stress are however still perceivable, probably as an illusion.  

\textcolor{blue}{draw}

\subsubsection{Basic polar questions with a final non-lexical verb}\label{section:intonation:focus:polar:finalCl} 

In the above sentences, the sentence ended in a simple lexical verb like `to have'. We saw a perceivable rise H\% on this final verb. Matters get more complicated when the final verb is a clitic copula or auxiliary. 

First consider the copula cases. In a declarative S-Adj-V sentence (\ref{example:intonation:focus:polar:basic:cl:dec}), there is stress on the adjective and then a final fall L\%. The verb is a cliticized unstressed copula. In contrast in the polar question (\ref{example:intonation:focus:polar:basic:cl:polar}), there is a sentence-final rise H\%. 


\begin{exe}
	\ex \begin{xlist}
		\ex \glll   (\textbf{S})$_\phi$ (\underline{\textbf{Adj}} Cop)$_\phi$ $\searrow$ \\
		mɑɾˈ\textbf{jɑ-n} \underline{ɡɑɾˈ\textbf{miɾ}} =e  $\searrow$\\
		Maria-{\defgloss} red =is   \\
		\trans `Maria is red.'
		\label{example:intonation:focus:polar:basic:cl:dec}
		\\ \armenian{Մարիան կարմիր է։} 
		\ex \glll   (\textbf{S})$_\phi$ (\underline{\textbf{Adj}} Cop)$_\phi$ ?$\nearrow$ \\
		mɑɾˈ\textbf{jɑ-n} {\underline{ɡɑɾˈ\textbf{miɾ}}} =e  ?$\nearrow$ \\
		Maria-{\defgloss} red   =is \\ 
		\trans `Is Maria    red?' 
		\label{example:intonation:focus:polar:basic:cl:polar}
		\\ \armenian{Մարիան կարմի՞ր է։}
		
	\end{xlist}
\end{exe}

Impressionistically for the polar question (\ref{example:intonation:focus:polar:basic:cl:polar}), nuclear stress is on the pre-copula adjective [ɡɑɾˈmir =e] `red =is' . This is reflected in the orthography because the question marker \armenian{՞} is placed on the adjective: \armenian{կարմի՞ր է} <garmiˀr ē>. Acoustically however, it seems that the sentence-final rise starts in the stressed vowel of the verb [ɡɑɾˈmiɾ] but then only reaches its limit in the final auxiliary. The pitch tracks show this complication. 

\textcolor{blue}{draw}

If the copula is larger, such as being bisyllabic, we again see the same pattern (\ref{example:intonation:focus:polar:basic:cl:bi:dec}). Impressionistically for the polar question (\ref{example:intonation:focus:polar:basic:cl:bi:polar}), nuclear stress is on the pre-copula adjective, while the sentence ends in a rise H\%. 


\begin{exe}
	\ex \begin{xlist}
		\ex \glll   (\textbf{S})$_\phi$ (\underline{\textbf{Adj}} Cop)$_\phi$ $\searrow$ \\
		də\textbf{χɑ}kʰ-ə  \underline{ɡɑɾˈ\textbf{miɾ}} =e-ji-n $\searrow$ \\
		boy-{\pl}-{\defgloss} red =is-{\pst}-3{\pl}   \\
		\trans `The boys were  red.'
		\label{example:intonation:focus:polar:basic:cl:bi:dec}
		\\ \armenian{Տղաքը կարմիր էին։} 
		\ex \glll   (\textbf{S})$_\phi$ (\underline{\textbf{Adj}} Cop)$_\phi$ ?$\nearrow$ \\
		də\textbf{χɑ}kʰ-ə  {\underline{ɡɑɾˈ\textbf{miɾ}}} =e-ji-n  ?$\nearrow$ \\
		boy-{\pl}-{\defgloss} red   =is-{\pst}-3{\pl} \\ 
		\trans `Were the boys red?'
		\label{example:intonation:focus:polar:basic:cl:bi:polar}
		\\ \armenian{Տղաքը կարմի՞ր էին։}
		
	\end{xlist}
\end{exe}

With a longer clitic, we find the same intonational contours. The sentence-final rise starts in the stressed syllable of the pre-copula word. The rise reaches its limit in the clitic. 



\textcolor{blue}{draw}

The same generalizations are again found with complex periphrastic tenses. In (\ref{example:intonation:focus:polar:basic:cl:light:dec}), the main verb is a participle while inflection is on a light verb. In the declarative, stress is on the participle. In the polar question as well (\ref{example:intonation:focus:polar:basic:cl:light:polar}), we perceive stress on the participle  and then a sentence-final rise H\%. 


\begin{exe}
	\ex \begin{xlist}
		\ex \glll (\textbf{S})$_\phi$ (\textbf{O})$_\phi$ (Pro \textbf{\underline{V}} lightV)$_\phi$ $\searrow$ \\
		mɑɾˈ\textbf{jɑ-n} nɑmɑɡ-ˈ\textbf{neɾ}  bidi \underline{kʰəˈ\textbf{ɾ-ɑd͡}z} əlˈl-ɑ-$\emptyset$ $\searrow$\\
		Maria-{\defgloss} letter-{\pl} {\fut} write-{\rptcp} be-{\thgloss}-3{\sg}  \\
		\trans `Maria will have written some letters.' 
		\label{example:intonation:focus:polar:basic:cl:light:dec}
		\\ \armenian{Մարիան նամակներ պիտի գրած ըլլայ։}
		\ex \glll (\textbf{S})$_\phi$ (\textbf{O})$_\phi$ (Pro \textbf{\underline{V}} lightV)$_\phi$ ?$\nearrow$ \\
		mɑɾˈ\textbf{jɑ-n} nɑmɑɡ-ˈ\textbf{neɾ}  bidi \underline{kʰəˈ\textbf{ɾ-ɑd͡}z} əlˈl-ɑ-$\emptyset$ ?$\nearrow$\\
		Maria-{\defgloss} letter-{\pl} {\fut} write-{\rptcp} be-{\thgloss}-3{\sg}  \\
		\trans `Will Maria   have written some letters?' 
		\label{example:intonation:focus:polar:basic:cl:light:polar}
		\\ \armenian{Մարիան նամակներ պիտի գրա՞ծ ըլլայ։}
		
	\end{xlist}
\end{exe}

The acoustic patterns are again the same. The rise starts in the final syllable of the participle, reaches its peak during the light verb, and stays constant. 


\textcolor{blue}{draw}





\subsubsection{Basic polar questions with a final non-verb}\label{section:intonation:focus:polar:finalN}

In the previous sentences, the polar question ended in either a verb or a verb-like element such as a copula or light verb. Such SOV constructions are the default ways to form polar questions. However, it is possible to have other word orders such as OVS. In such constructions, nuclear stress stays on the verb, but the sentence-final rise H\% continues from the sentence-medial verb all the way to the end of the sentence. 

To illustrate, consider the two polar questions below. The default word order (\ref{example:intonation:focus:polar:otherorder:sov}) is SOV with a rise on the verb. An alternative word order is OVS (\ref{example:intonation:focus:polar:otherorder:ovs}). The subject is treated as some type of less important information, such as a afterthought or topic. The stress is still on the verb, and there is still a sentence-final rise. 


\begin{exe}
	\ex \begin{xlist}
		\ex \glll (\textbf{S})$_\phi$ (\textbf{O})$_\phi$ (\textbf{\underline{V}})$_\phi$ ?$\nearrow$ \\
		mɑɾˈ\textbf{jɑ-n} nɑmɑɡ-ˈ\textbf{neɾ} \underline{uˈ\textbf{n-i-$\emptyset$}} ?$\nearrow$ \\
		Maria-{\defgloss} letter-{\pl} have-{\thgloss}-3{\sg} \\
		\trans `Does Maria have some letters?' 
		\label{example:intonation:focus:polar:otherorder:sov}
		\\ \armenian{Մարիան նամակներ ունի՞։}
		\ex \glll (\textbf{O})$_\phi$ (\textbf{\underline{V}})$_\phi$  ({S})$_\phi$ ?$\nearrow$ \\
		nɑmɑɡ-ˈ\textbf{neɾ} \underline{uˈ\textbf{n-i-$\emptyset$}} mɑɾˈ{jɑ-n}  ?$\nearrow$ \\
		letter-{\pl} have-{\thgloss}-3{\sg} Maria-{\defgloss}  \\
		\trans `Does Maria have some letters?' 
		\label{example:intonation:focus:polar:otherorder:ovs}
		\\ \armenian{Նամակներ ունի՞ Մարիան։}
		
	\end{xlist}
\end{exe}

Acoustically, we find an interesting pattern for the OVS question (\ref{example:intonation:focus:polar:otherorder:ovs}). The rise starts on the verb's final syllable. The pitch reaches its peak by the beginning of the post-verbal word. The pitch then stays high until the end of the sentence. Because of this high plateau, HD perceives that there is no phrasal stress  after the focused word; lexical stress is recoverable via just knowing the word. 

\textcolor{blue}{draw}

The continuation of the rise from the verb till the end does not care about lexical stress. For example, in the polar questions below, the subject has penultimate lexical stress because it has a final schwa (\ref{example:intonation:focus:polar:otherorder:schwa:sov}). In the OVS polar question (\ref{example:intonation:focus:polar:otherorder:schwa:ovs}), the rise continues from the verb till the end of the sentence, even into the schwa. 


\begin{exe}
	\ex \begin{xlist}
		\ex \glll (\textbf{S})$_\phi$ (\textbf{O})$_\phi$ (\textbf{\underline{V}})$_\phi$ ?$\nearrow$ \\
		mɑɾˈ\textbf{jɑ}m-ə nɑmɑɡ-ˈ\textbf{neɾ} \underline{uˈ\textbf{n-i-$\emptyset$}} ?$\nearrow$ \\
		Mariam-{\defgloss} letter-{\pl} have-{\thgloss}-3{\sg} \\
		\trans `Does Mariam have some letters?' 
		\label{example:intonation:focus:polar:otherorder:schwa:sov}
		\\ \armenian{Մարիամը նամակներ ունի՞։}
		\ex \glll (\textbf{O})$_\phi$ (\textbf{\underline{V}})$_\phi$  ({S})$_\phi$ ?$\nearrow$ \\
		nɑmɑɡ-ˈ\textbf{neɾ} \underline{uˈ\textbf{n-i-$\emptyset$}} mɑɾˈ{jɑ}m-ə   ?$\nearrow$ \\
		letter-{\pl} have-{\thgloss}-3{\sg} Mariam-{\defgloss}  \\
		\trans `Does Mariam have some letters?' 
		\label{example:intonation:focus:polar:otherorder:schwa:ovs}
		\\ \armenian{Նամակներ ունի՞ Մարիամը։}
		
	\end{xlist}
\end{exe}

The pitch-tracks again confirm this impression. For the OVS question (\ref{example:intonation:focus:polar:otherorder:schwa:ovs}), the subject has perceived non-final lexical stress, but there are no pitch  differences in the word at all. The perception of lexical stress is thus likely just an illusion. 

\textcolor{blue}{draw}

In the above sentences, the verb phrase consists of just an object and a lexical verb. If the verb is a   clitic copula, we again find the same patterns. S-Adj-V is the typical order (\ref{example:intonation:focus:polar:otherorder:enc:sov}), but Adj-V-S is possible.  For the Adj-V-S order (\ref{example:intonation:focus:polar:otherorder:enc:ovs}), again stress is   on the pre-copula  word. The sentence-final rise starts from the final syllable of this word and continues to the final syllable. 

\begin{exe}
	\ex \begin{xlist}
		\ex \glll   (\textbf{S})$_\phi$ (\underline{\textbf{Adj}} Cop)$_\phi$ ?$\nearrow$ \\
		mɑɾˈ\textbf{jɑ}m-ə {\underline{ɡɑɾˈ\textbf{miɾ}}} =e  ?$\nearrow$ \\
		Mariam-{\defgloss} red   =is \\ 
		\trans `Is Mariam    red?' 
		\label{example:intonation:focus:polar:otherorder:enc:sov}
		\\ \armenian{Մարիամը կարմի՞ր է։}
		\ex \glll (\underline{\textbf{Adj}} Cop)$_\phi$    ({S})$_\phi$  ?$\nearrow$ \\
		{\underline{ɡɑɾˈ\textbf{miɾ}}} =e  mɑɾˈ{jɑ}m-ə  ?$\nearrow$ \\
		red   =is  Mariam-{\defgloss}  \\ 
		\trans `Is Mariam    red?' 
		\label{example:intonation:focus:polar:otherorder:enc:ovs}
		\\ \armenian{Կարմի՞ր է Մարիամը։}
		
	\end{xlist}
\end{exe}

The pitch-tracks again confirm this impression. 

\textcolor{blue}{draw}



\subsubsection{Basic polar questions with negation} \label{section:intonation:focus:polar:negation}
The previous polar questions all had the verb in the positive. When the verb is negative, we find similar patterns in terms of nuclear stress and final stress. Basically, the negated verb attracts nuclear stress, and the sentence-final rise starts in this negative verb and continues till the end of the sentence. However, sometimes we find that the negated verb severely weakens or deaccents the subsequent words. 


First consider a basic SOV sentence with a negated verb. In both the declarative (\ref{example:intonation:focus:polar:neg:dec}) and polar question forms (\ref{example:intonation:focus:polar:neg:pol}),  nuclear stress is on the negated verb. The declarative has a final fall, while the polar question has a final rise. 

\begin{exe}
	\ex \begin{xlist}
		\ex \glll   (\textbf{S})$_\phi$ (\textbf{O})$_\phi$ (\underline{\textbf{{\neggloss}-V}})$_\phi$ $\searrow$ \\
		mɑɾˈ\textbf{jɑ-n} {nɑmɑɡ-ˈ\textbf{neɾ}} \underline{ˈ\textbf{t͡ʃ-u}n-i-$\emptyset$}  $\searrow$ \\
		Maria-{\defgloss} letter-{\pl} {\neggloss}-have-{\thgloss}-3{\sg}   \\
		\trans `Maria does not have letters.' 
		\label{example:intonation:focus:polar:neg:dec}
		\\ \armenian{Մարիան նամակներ չունի։}
		\ex \glll   (\textbf{S})$_\phi$ (\textbf{O})$_\phi$ (\underline{\textbf{{\neggloss}-V}})$_\phi$ ?$\nearrow$ \\
		mɑɾˈ\textbf{jɑ-n} {nɑmɑɡ-ˈ\textbf{neɾ}} \underline{ˈ\textbf{t͡ʃ-u}n-i-$\emptyset$}  ?$\nearrow$ \\
		Maria-{\defgloss} letter-{\pl} {\neggloss}-have-{\thgloss}-3{\sg}   \\
		\trans `Does Maria  not have letters?' 
		\label{example:intonation:focus:polar:neg:pol}
		\\ \armenian{Մարիան նամակներ չունի՞։}
		
	\end{xlist}
\end{exe}



Impressionistically, the lexical stress of the negated verb is on the first syllable. But acoustically for the polar question, the final rise of the sentence is quite more significant than the prominence of the first syllable. 

\textcolor{blue}{draw}

The intonational contours are the same if we have a post-verbal element. In an OVS polar question (\ref{example:intonation:focus:polar:neg:ovs}), the negated verb still has nuclear stress with a significant rise, and we again have a sentence-final rise. Again, there is no prominent phrasal stress after the negated verb. 

\begin{exe}
	\ex\label{example:intonation:focus:polar:neg:ovs} \begin{xlist}
		\ex \glll   (\textbf{O})$_\phi$ (\underline{\textbf{{\neggloss}-V}})$_\phi$ (\textbf{S})$_\phi$  ?$\nearrow$ \\
		{nɑmɑɡ-ˈ\textbf{neɾ}} \underline{ˈ\textbf{t͡ʃ-u}n-i-$\emptyset$} mɑɾˈ\textbf{jɑ-n}   ?$\nearrow$ \\
		letter-{\pl} {\neggloss}-have-{\thgloss}-3{\sg}   Maria-{\defgloss}  \\
		\trans `Does Maria  not have letters?' 
		\\ \armenian{Նամակներ չունի՞ Մարիան։}
		\ex \glll   (\textbf{O})$_\phi$ (\underline{\textbf{{\neggloss}-V}})$_\phi$ (\textbf{S})$_\phi$  ?$\nearrow$ \\
		{nɑmɑɡ-ˈ\textbf{neɾ}} \underline{ˈ\textbf{t͡ʃ-u}n-i-$\emptyset$} mɑɾˈ{jɑ}m-ə   ?$\nearrow$ \\
		letter-{\pl} {\neggloss}-have-{\thgloss}-3{\sg}   Mariam-{\defgloss}  \\
		\trans `Does Mariam  not have letters?' 
		\\ \armenian{Նամակներ չունի՞ Մարիամը։}
		
	\end{xlist}
\end{exe}


The pitch-tracks illustrate this contour. The rise starts on the negated verb and continues till the end. However, HD perceives that the subject is rather quiet or low in amplitude.  For polar questions, HD perceived that a post-verbal subject is quieter after a negative verb than after a positive verb.  We call this quieting effect `post-negative weakening'.   It's unclear if this post-negative weakening is a true acoustic process (and a type of post-focal deaccenting), vs. just an illusion triggered by knowing the semantic significance of negation. 

Similar contours are    also found with non-lexical verbs.  Consider S-Adj-V (\ref{example:intonation:focus:polar:neg:cop:sov}) and Adj-V-S (\ref{example:intonation:focus:polar:neg:cop:ovsfull}) polar questions with a negated copula. We likewise include an Adj-V-S  sentence with a final schwa (\ref{example:intonation:focus:polar:neg:cop:ovsschwa})  for easier contrast later.    All  versions have stress on the negated copula, and a sentence-final rise. 

\begin{exe}
	\ex \begin{xlist}
		\ex \glll   (\textbf{S})$_\phi$ (\textbf{Adj})$_\phi$ (\underline{\textbf{{\neggloss}-Cop}})$_\phi$ ?$\nearrow$ \\
		mɑɾˈ\textbf{jɑ-n} {ɡɑɾˈ\textbf{miɾ}} \underline{ˈ\textbf{t͡ʃ-e}} ?$\nearrow$ \\
		Maria-{\defgloss} red   {\neggloss}-is \\ 
		\trans `Isn't Maria   red?' 
		\label{example:intonation:focus:polar:neg:cop:sov}
		\\ \armenian{Մարիան կարմիր չէ՞։}
		\ex \glll    (\textbf{Adj})$_\phi$ (\underline{\textbf{{\neggloss}-Cop}})$_\phi$ ({S})$_\phi$ ?$\nearrow$ \\
		{ɡɑɾˈ\textbf{miɾ}} \underline{ˈ\textbf{t͡ʃ-e}} mɑɾˈ{jɑ-n}  ?$\nearrow$ \\
		red   {\neggloss}-is Maria-{\defgloss} \\ 
		\trans `Isn't Maria     red?' 
		\\ \armenian{Կարմիր չէ՞ Մարիան։}
		\label{example:intonation:focus:polar:neg:cop:ovsfull}
		\ex \glll    (\textbf{Adj})$_\phi$ (\underline{\textbf{{\neggloss}-Cop}})$_\phi$ ({S})$_\phi$ ?$\nearrow$ \\
		{ɡɑɾˈ\textbf{miɾ}} \underline{ˈ\textbf{t͡ʃ-e}} mɑɾˈ{jɑ}m-ə  ?$\nearrow$ \\
		red   {\neggloss}-is Mariam-{\defgloss} \\ 
		\trans `Isn't Mariam red?'
		\label{example:intonation:focus:polar:neg:cop:ovsschwa}
		\\ \armenian{Կարմիր չէ՞ Մարիամը։}
	\end{xlist}
\end{exe}

The pitch-tracks in () show that this rise starts from the negated copula up until the end of the sentence. The continous rise and high plateau cause the loss of phrasal stresses after the negated verb. 

\textcolor{blue}{draw}

Periphrastic tenses further show this consistent pattern. In (\ref{example:intonation:focus:polar:neg:periph:sov}), the verb phrase is made up of a negated auxiliary and a participle. As a polar question, stress is on the negated auxiliary  and there is a final rise. The final word is usually   the participle, but we can also have a post-posed subject (\ref{example:intonation:focus:polar:neg:periph:ovs}). 

\begin{exe}
	\ex \begin{xlist}
		\ex \glll   (\textbf{S})$_\phi$ ({\textbf{O}}) (\underline{\textbf{{\neggloss}-Aux}} V)$_\phi$ ?$\nearrow$ \\
		mɑɾˈ\textbf{jɑ-n} {nɑ{mɑɡ}-ˈ\textbf{neɾ}} \underline{ˈ\textbf{t͡ʃ-e}} kʰəˈɾ-ɑd͡z   ?$\nearrow$ \\
		Maria-{\defgloss} letter-{\pl} {\neggloss}-is write-{\rptcp}  \\
		\trans `Hasn't Maria   written letters?'
		\label{example:intonation:focus:polar:neg:periph:sov}
		\\ \armenian{Մարիան նամակներ չէ՞ գրած։}
		\ex \glll  ({\textbf{O}}) (\underline{\textbf{{\neggloss}-Aux}} V)$_\phi$   ({S})$_\phi$ ?$\nearrow$ \\
		{nɑ{mɑɡ}-ˈ\textbf{neɾ}} \underline{ˈ\textbf{t͡ʃ-e}} kʰəˈɾ-ɑd͡z mɑɾˈ{jɑ-n}    ?$\nearrow$ \\
		letter-{\pl} {\neggloss}-is write-{\rptcp} Maria-{\defgloss}   \\
		\trans `Hasn't Maria written letters?' 
		\label{example:intonation:focus:polar:neg:periph:ovs}
		\\ \armenian{Նամակներ չէ՞ գրած Մարիան։}
		
	\end{xlist}
\end{exe}

The intonational of these sentences is the same as before. We see a rise from the auxiliary onto the final verb. The high plateau causes the loss of any  subsequent phrasal prominence. 

\textcolor{blue}{draw}






\subsubsection{Contrastive polar questions} \label{section:intonation:focus:polar:contrastive}

The previous sections focused on polar questions where the entire sentence was being questioned. In contrast, what we call a `contrastive polar question' is when some specific word in the sentence is being questioned. Such questions are like English `Did you read the BOOK?' where we question whether a book was read vs. some other entity. 

For Armenian, we can create a contrastive polar question by questioning any word, such as the subject or object. The questioned word gets a significant rise. After this word, the pitch continues to rise until the end of the sentence. Sometimes the final syllable of the sentence also has a rise, but sometimes the final syllable has a fall. 


\textcolor{red}{idk whats the most common term for this}

First consider subject focus. The subject can have lexical stress on   the last syllable (\ref{example:intonation:focus:polar:contrastive:subj:full}), or even on the penultimate syllable if the word ends in a schwa (\ref{example:intonation:focus:polar:contrastive:subj:schwa}). The polar question enhances the lexical stress of the subject, and we have a high rising plateau after this word. 

\begin{exe}
	\ex \begin{xlist}
		\ex \glll (\textbf{\underline{S}})$_\phi$ (\textbf{O} V)$_\phi$ ?$\nearrow$\\
		\underline{mɑɾˈ\textbf{jɑ-n}} nɑmɑɡ-ˈ{neɾ} uˈn-i-$\emptyset$ ?$\nearrow$ \\
		Maria-{\defgloss} letter-{\pl} have-{\thgloss}-3{\sg} \\
		\trans `Does \underline{Maria} have letters (as opposed to someone else).'
		\label{example:intonation:focus:polar:contrastive:subj:full}
		\\ \armenian{Մարիա՞ն նամակներ ունի։}
		\ex \glll (\textbf{\underline{S}})$_\phi$ ({O} V)$_\phi$ ?$\nearrow$\\
		\underline{mɑɾˈ\textbf{jɑ}m-ə} nɑmɑɡ-ˈ{neɾ} uˈn-i-$\emptyset$ ?$\nearrow$ \\
		Mariam-{\defgloss} letter-{\pl} have-{\thgloss}-3{\sg} \\
		\trans `Does \underline{Mariam} have letters (as opposed to someone else).'
		\label{example:intonation:focus:polar:contrastive:subj:schwa}
		\\ \armenian{Մարիա՞մը նամակներ ունի։}
	\end{xlist}
\end{exe}

The pitch-tracks confirm this impression. The rise starts rather late in the stressed syllable of the subject, before any schwa. The rise continues up until the end of the sentence. The post-subject sentences seem to have equivalent levels of prominence (a high plateau), thus we don't mark any phrasal stresses after the subject. 

\textcolor{blue}{draw}

Sometimes, HD would keep the rise on the final syllable, but he would also sometimes have a fall on the final syllable. Such a final fall was also attested in \citet{ToparlakDolatian-202x-IntonationFocusMarkingWesternArmenian}. For HD, the use of a final rise seems more common; in contrast, \citet{ToparlakDolatian-202x-IntonationFocusMarkingWesternArmenian}'s consultants seem to prefer a final fall. 

\textcolor{blue}{draw}


Similar patterns arise for object focus. The object has lexical stress on the rightmost non-schwa: final in (\ref{example:intonation:focus:polar:contrastive:obj:full}), non-final in (\ref{example:intonation:focus:polar:contrastive:obj:schwa}). When questioned, this syllable is enhanced and we hear rising intonation. 


\begin{exe}
	\ex \begin{xlist}
		\ex \glll (\textbf{{S}})$_\phi$ (\textbf{\underline{O}} V)$_\phi$ ?$\nearrow$\\
		{mɑɾˈ\textbf{jɑ-n}} \underline{nɑmɑɡ-ˈ\textbf{neɾ}} uˈn-i-$\emptyset$ ?$\nearrow$ \\
		Maria-{\defgloss} letter-{\pl} have-{\thgloss}-3{\sg} \\
		\trans `Does {Maria} have \underline{letters} (as opposed to something else).'
		\label{example:intonation:focus:polar:contrastive:obj:full}
		\\ \armenian{Մարիան նամակնե՞ր ունի։}
		\ex \glll (\textbf{{S}})$_\phi$ (\textbf{\underline{O}} V)$_\phi$ ?$\nearrow$\\
		{mɑɾˈ\textbf{jɑ-n}} \underline{nɑmɑɡ-ˈ\textbf{ne}ɾ-ə} uˈn-i-$\emptyset$ ?$\nearrow$ \\
		Maria-{\defgloss} letter-{\pl}-{\defgloss} have-{\thgloss}-3{\sg} \\
		\trans `Does {Maria} have \underline{the letters} (as opposed to something else).'
		\label{example:intonation:focus:polar:contrastive:obj:schwa}
		\\ \armenian{Մարիան նամակնե՞րը ունի։}
	\end{xlist}
\end{exe}

The pitch-tracks again confirm this impression. 

\textcolor{blue}{draw}

\subsubsection{Polar questions with the question particles}\label{section:intonation:focus:polar:particle}

The previous examples were all polar questions as formed in standard speech. The most typical constructions are to simply modify the intonation of the sentence, without adding any new question morphemes. This section looks at some question morphemes that are sometimes used. 


In standard speech, there are some words that function as question participles like [ɑɾtʰjokʰ] `perhaps' (\ref{example:intonation:focus:polar:particile:artjok}). This particle often has irregular stress on the first syllable. This particle is generally restricted to the beginning of the sentence. But this participle is not often used. Furthermore, even when it is used, it does not change the intonational contour of the question. 

\begin{exe}
	\ex \glll (\textbf{Q})$_\phi$ (\textbf{S})$_\phi$ (\textbf{O})$_\phi$ (\underline{\textbf{V}})$_\phi$  ?$\nearrow$ \\
	ˈ\textbf{ɑɾtʰ}jokʰ mɑɾˈ\textbf{jɑ-n} nɑmɑɡ-ˈ\textbf{neɾ} \underline{uˈ\textbf{n-i-$\emptyset$}} ?$\nearrow$ \\
	perhaps Maria-{\defgloss} letter-{\pl} have-{\thgloss}-3{\sg} \\
	\trans `Perhaps does Maria have letters?'
	\label{example:intonation:focus:polar:particile:artjok}
	\\ \armenian{Արդեօք Մարիան նամակներ ունի՞։}
\end{exe}


The pitch tracks show that there is some rise on this sentence-initial particle, but then the sentence has the typical sentence-final rise. 

\textcolor{blue}{draw}


In colloquial speech, speakers sometimes use the question particle \textit{mə}. This morpheme is quite stigmatized because it borrowed from Turkish. See \S\ref{section:stress:cliticc} and \S\ref{section:intonation:clitic:overview} for data on how this morpheme  is a clitic. 

In polar questions, this particle tends to be restricted to the sentence-final position. In an SOV polar question, the absence of this particle triggers a sentence-final rise (\ref{example:intonation:focus:polar:particile:me:sov:no}). But in contrast, the presence of this particle greatly weakens this final rise (\ref{example:intonation:focus:polar:particile:me:sov:yes}). This final rise is close to a sentence-final fall or perhaps just a level tone --. We're not sure. 


\begin{exe}
	\ex \begin{xlist}
		\ex \glll (\textbf{S})$_\phi$ (\textbf{O})$_\phi$ (\underline{\textbf{V}})$_\phi$  ?$\nearrow$ \\
		mɑɾˈ\textbf{jɑ-n} nɑmɑɡ-ˈ\textbf{neɾ} \underline{uˈ\textbf{n-i-$\emptyset$}} ?$\nearrow$ \\
		Maria-{\defgloss} letter-{\pl} have-{\thgloss}-3{\sg} \\
		\trans `Does Maria have letters?'
		\label{example:intonation:focus:polar:particile:me:sov:no}
		\\ \armenian{Մարիան նամակներ ունի՞։}
		\ex \glll (\textbf{S})$_\phi$ (\textbf{O})$_\phi$ (\underline{\textbf{V}} Q)$_\phi$  ?--  \\
		mɑɾˈ\textbf{jɑ-n} nɑmɑɡ-ˈ\textbf{neɾ} \underline{uˈ\textbf{n-i-$\emptyset$}} =mə  ?-- \\
		Maria-{\defgloss} letter-{\pl} have-{\thgloss}-3{\sg} ={\q} \\
		\trans `Does Maria have letters?'
		\label{example:intonation:focus:polar:particile:me:sov:yes}
		\\ \armenian{Մարիան նամակներ ունի՞ մը։}
	\end{xlist}
\end{exe}

Nuclear stress is on the verb in both sentences above. But the pitch-tracks in () show that  there is a stark difference in pitch levels based on the presence or absence of the question particle. Without the particle, the verb has a high rise, but the presence of the particle severely weakens this rise. 


Similar weakening is found for SOV sentences with negation. The lack of a question particle triggers a perceptible sentence-final rise (\ref{example:intonation:focus:polar:particile:me:neg:no}). The presence of this particle severely weakens this rise (\ref{example:intonation:focus:polar:particile:me:neg:yes}), into perhaps just a level tone.  


\begin{exe}
	\ex \begin{xlist}
		\ex \glll (\textbf{S})$_\phi$ (\textbf{O})$_\phi$ (\underline{\textbf{{\neggloss}-V}})$_\phi$  ?$\nearrow$ \\
		mɑɾˈ\textbf{jɑ-n} nɑmɑɡ-ˈ\textbf{neɾ} \underline{ˈ\textbf{t͡ʃ-u}n-i-$\emptyset$} ?$\nearrow$ \\
		Maria-{\defgloss} letter-{\pl} {\neggloss}-have-{\thgloss}-3{\sg} \\
		\trans `Doesn't Maria have letters?'
		\label{example:intonation:focus:polar:particile:me:neg:no}
		\\ \armenian{Մարիան նամակներ չունի՞։}
		\ex \glll (\textbf{S})$_\phi$ (\textbf{O})$_\phi$ (\underline{\textbf{{\neggloss}-V}} Q)$_\phi$  ?-- \\
		mɑɾˈ\textbf{jɑ-n} nɑmɑɡ-ˈ\textbf{neɾ} \underline{ˈ\textbf{t͡ʃ-u}n-i-$\emptyset$}  =mə ?--\\
		Maria-{\defgloss} letter-{\pl} {\neggloss}-have-{\thgloss}-3{\sg} {\q} \\
		\trans `Doesn't Maria have letters?'
		\label{example:intonation:focus:polar:particile:me:neg:yes}
		\\ \armenian{Մարիան նամակներ չունի՞ մը։}
	\end{xlist}
\end{exe}

The pitch-tracks in () again show that the presence of this question particle weakens the final pitch rise. 

For space, we do not go through every possible syntactic construction that can use this question particle. Essentially any polar question can be modified to include this particle. Because this particle is stigmatized, it is difficult to know if HD's use of a weakened final rise is a general characteristic of this particle \textit{across} speakers, or if there is significant inter-speaker variability. What makes matters more difficult is that speakers consciously avoid using this particle because of social stigma. Such stigma would prevent eliciting the relevant data from speakers, and it likewise lowers the chance of finding this particle in recorded natural speech. 




\subsection{Wh-questions  and focused answers}\label{section:intonation:focus:wh}

This section looks at wh-questions, which are questions that use wh-words or question words like `Who are you?'. These questions show the following properties: 
\begin{itemize}[noitemsep,topsep=0pt]
	\item Nuclear stress: The wh-word gets nuclear stress H*. 
	\item Post-focal deaccenting: After the wh-word, all words lose their stress. 
	\item Final rise: The sentence ends in a final rise H\%.
\end{itemize}

The answer sentence to a wh-questions shows the first two properties (stress and post-focal deaccenting). The answers end in a final fall L\%. 

We go over subject wh-questions (\S\ref{section:intonation:focus:wh:subj}) and object-questions (\S\ref{section:intonation:focus:wh:obj}). However, we did find some possible length restrictions on the final rise (\S\ref{section:intonation:focus:wh:length}. 

\subsubsection{Subject questions   }\label{section:intonation:focus:wh:subj}

In  a subject wh-question (\ref{example:intontion:focus:wh:subj:basic:q}), the subject of the sentence is an interrogative pronoun `who' [ov]. The question places nuclear stress on the wh-word, and then we have  a sentence-final rise. 

\begin{exe}
	
	\ex \glll(\underline{\textbf{who}})$_\phi$ ({O} V)$_\phi$ ?$\nearrow$ \\
	\underline{ˈ\textbf{ov}} nɑmɑɡ-ˈ\textbf{neɾ} uˈn-i-$\emptyset$ ?$\nearrow$ \\
	who letter-{\pl} have-{\thgloss}-3{\sg} \\
	\trans `Who has letters?' 
	\label{example:intontion:focus:wh:subj:basic:q}
	\\ \armenian{Ո՞վ նամակներ ունի։}
\end{exe}

Acoustically, in the question, the stress on the wh-word causes the loss of stress in all subsequent words. This process of post-focal deaccenting is quite robust. The final syllable of the sentence is unstressed and gets a sentence-final rise H\%.  

\textcolor{blue}{draw}

In the corresponding answer, stress is on the subject, and we end in a sentence-final fall. For contrast, we show two possible subjects: one with final stress (\ref{example:intontion:focus:wh:subj:basic:ans:full}) and with penultimate stress (\ref{example:intontion:focus:wh:subj:basic:ans:schwa}). 

\begin{exe}
	\ex \begin{xlist}
		\ex \glll (\underline{\textbf{S}})$_\phi$ ({O} V)$_\phi$  $\searrow$ \\
		\underline{mɑɾˈ\textbf{jɑ-n}} nɑmɑɡ-ˈ{neɾ} uˈn-i-$\emptyset$  $\searrow$ \\
		Maria-{\defgloss} letter-{\pl} have-{\thgloss}-3{\sg} \\
		\trans `\underline{Maria} has letters.' 
		\label{example:intontion:focus:wh:subj:basic:ans:full}
		\\ \armenian{Մարիան նամակներ ունի։}
		\ex \glll (\underline{\textbf{S}})$_\phi$ ({O} V)$_\phi$  $\searrow$ \\
		\underline{mɑɾˈ\textbf{jɑ}m-ə} nɑmɑɡ-ˈ{neɾ} uˈn-i-$\emptyset$  $\searrow$ \\
		Mariam-{\defgloss} letter-{\pl} have-{\thgloss}-3{\sg} \\
		\trans `\underline{Mariam} has letters.' 
		\label{example:intontion:focus:wh:subj:basic:ans:schwa}
		\\ \armenian{Մարիամը նամակներ ունի։}
		
	\end{xlist}
\end{exe}


In the answer to this question, post-focal deaccenting applies after the subject, and we get a final fall L\%.  Nuclear stress is on the stressed syllable of the subject, regardless of that syllable is final or non-final. 

\textcolor{blue}{draw}

The above wh-question shows three basic components for subject questions: stress on the subject, post-focal deaccenting, and a sentence-final rise. It seems these properties are consistent across all possible subject wh-questions. 

For example, consider a subject wh-question that ends in a clitic verb (\ref{example:intontion:focus:wh:subj:cl:q}). Stress is on the subject as expected, and there is a sentence-final rise. This rise is on the clitic even though the clitic doesn't bear lexical stress. 

\begin{exe}
	\ex \begin{xlist}
		
		
		\ex \glll(\underline{\textbf{who}})$_\phi$ ({Adj} V)$_\phi$ ?$\nearrow$ \\
		\underline{ˈ\textbf{ov}} ɡɑɾˈ{miɾ} =e ?$\nearrow$ \\
		who red =is\\ 
		\trans `Who is red?'
		\label{example:intontion:focus:wh:subj:cl:q}
		\\ \armenian{Ո՞վ կարմիր է։}
		\ex \glll (\underline{\textbf{S}})$_\phi$ ({O} V)$_\phi$ $\searrow$ \\
		\underline{mɑɾˈ\textbf{jɑ-n}}  ɡɑɾˈ{miɾ} =e   $\searrow$ \\
		Maria-{\defgloss} red =is\\ 
		\trans `\underline{Maria} is red.' 
		\label{example:intontion:focus:wh:subj:cl:ans:full}
		\\ \armenian{Մարիան կարմիր է։} 
		\ex \glll (\underline{\textbf{S}})$_\phi$ ({O} V)$_\phi$  $\searrow$ \\
		\underline{mɑɾˈ\textbf{jɑ}m-ə} ɡɑɾˈ{miɾ} =e   $\searrow$ \\
		Mariam-{\defgloss} red =is\\ 
		\trans `\underline{Mariam} is red.' 
		\label{example:intontion:focus:wh:subj:cl:ans:schwa}
		\\ \armenian{Մարիամը կարմիր է։} 
		
		
	\end{xlist}
\end{exe}

The pitch-tracks show that the sentence-final rise of the question starts on the final syllable, even though it is a clitic without lexical stress. Both the question and answers (\ref{example:intontion:focus:wh:subj:cl:ans:full}, \ref{example:intontion:focus:wh:subj:cl:ans:schwa}) have stress on the subject, followed by post-focal deaccenting. 


\textcolor{blue}{draw}

Periphrastic tenses show the same intonational contours (\ref{example:intontion:focus:wh:subj:perip}). The sentence ends in a  verb plus clitic, yet we see a sentence-final rise in the question (\ref{example:intontion:focus:wh:subj:perip:q}). 


\begin{exe}
	\ex \label{example:intontion:focus:wh:subj:perip} \begin{xlist}
		
		
		\ex \glll(\underline{\textbf{who}})$_\phi$ ({O} V Aux)$_\phi$ ?$\nearrow$ \\
		\underline{ˈ\textbf{ov}} nɑmɑɡ-ˈneɾ kʰəˈɾ-ɑd͡z =e ?$\nearrow$ \\
		who letter-{\pl} write-{\rptcp} =is\\ 
		\trans `Who has written letters?'
		\label{example:intontion:focus:wh:subj:perip:q}
		\\ \armenian{Ո՞վ նամակներ գրած է։}
		\ex \glll (\underline{\textbf{S}})$_\phi$ ({O} V Aux)$_\phi$ $\searrow$ \\
		\underline{mɑɾˈ\textbf{jɑ-n}}  nɑmɑɡ-ˈneɾ kʰəˈɾ-ɑd͡z =e $\searrow$ \\
		Maria-{\defgloss} letter-{\pl} write-{\rptcp} =is\\ 
		\trans `\underline{Maria}  has written letters.' 
		\\ \armenian{Մարիան  նամակներ գրած է։} 
		\ex \glll (\underline{\textbf{S}})$_\phi$ ({O} V Aux)$_\phi$  $\searrow$ \\
		\underline{mɑɾˈ\textbf{jɑ}m-ə} nɑmɑɡ-ˈneɾ kʰəˈɾ-ɑd͡z  =e $\searrow$ \\
		Mariam-{\defgloss} letter-{\pl} write-{\rptcp} =is\\ 
		\trans `\underline{Mariam}     has written letters.' 
		\\ \armenian{Մարիամը  նամակներ գրած է։} 
		
		
	\end{xlist}
\end{exe}

In the corresponding answers, nuclear stress is on the stressed syllable of the subject, followed by post-focal deaccenting, and then finally a sentence-final fall L\%. 

\textcolor{blue}{draw}

Larger periphrastic structures again show the same intonation. In (\ref{example:intontion:focus:wh:subj:peripbig}), the verb phrase consist of a proclitic, verb, and a light verb. In both the question and answer, the subject gets focus, subsequent words lose prominence, and there is a sentence-final rise. 



\begin{exe}
	\ex \label{example:intontion:focus:wh:subj:peripbig}\begin{xlist}
		
		
		\ex \glll(\underline{\textbf{who}})$_\phi$ ({O})$_\phi$ (Pro V lightV)$_\phi$ ?$\nearrow$ \\
		\underline{ˈ\textbf{ov}} nɑmɑɡ-ˈneɾ bidi kʰəˈɾ-ɑd͡z  əlˈl-ɑ-$\emptyset$ ?$\nearrow$ \\
		who letter-{\pl} {\fut}  write-{\rptcp} be-{\thgloss}-3{\sg}\\ 
		\trans `Who will have written letters?'
		\\ \armenian{Ո՞վ նամակներ պիտի գրած ելլայ։}
		\ex \glll (\underline{\textbf{S}})$_\phi$  ({O})$_\phi$ (Pro  V lightV)$_\phi$ $\searrow$ \\
		\underline{mɑɾˈ\textbf{jɑ-n}}  nɑmɑɡ-ˈneɾ  bidi kʰəˈɾ-ɑd͡z  əlˈl-ɑ-$\emptyset$ $\searrow$ \\
		Maria-{\defgloss} letter-{\pl}  {\fut} write-{\rptcp} be-{\thgloss}-3{\sg}\\ 
		\trans `\underline{Maria}  will have written letters.' 
		\\ \armenian{Մարիան  նամակներ պիտի գրած ելլայ։} 
		\ex \glll (\underline{\textbf{S}})$_\phi$  ({O})$_\phi$ (Pro  V lightV)$_\phi$  $\searrow$ \\
		\underline{mɑɾˈ\textbf{jɑ}m-ə} nɑmɑɡ-ˈneɾ bidi  kʰəˈɾ-ɑd͡z    əlˈl-ɑ-$\emptyset$   $\searrow$ \\
		Mariam-{\defgloss} letter-{\pl} {\fut} write-{\rptcp}  be-{\thgloss}-3{\sg} \\ 
		\trans `\underline{Mariam}     will have written letters.' 
		\\ \armenian{Մարիամը  նամակներ պիտի գրած ելլայ։} 
		
		
	\end{xlist}
\end{exe}

Acoustically, the pitch-tracks show that the final rise in the wh-question is strictly limited to the final syllable. The answers have a sentence-final fall. Both questions and answers have post-focal deaccenting after the subject. 

\textcolor{blue}{draw}


\subsubsection{Object questions} \label{section:intonation:focus:wh:obj}

Object wh-questions show similar intoniatinoal properties as subject wh-questions. First, nuclear stress is on the object, then we have post-focal deaccenting, and the sentence ends a final-rise on the final syllable. 

First consider basic SOV sentences (\ref{example:intontion:focus:wh:obj:basic}). In both the question and answer form, the object   gets stress, and the verb is deaccented. The question has a final rise while the declarative answer has a final fall. 

\begin{exe}
	\ex \label{example:intontion:focus:wh:obj:basic}\begin{xlist}
		\ex \glll (\textbf{S})$_\phi$ (\underline{\textbf{what}} V)$_\phi$ ?$\nearrow$ \\
		mɑrˈ\textbf{jɑ-n} \underline{ˈ\textbf{int͡ʃ}} uˈn-i-$\emptyset$ ?$\nearrow$ \\
		Maria-{\defgloss} what have-{\thgloss}-3{\sg} \\ 
		\trans `What does Maria have?' 
		\\ \armenian{Մարիան ի՞նչ ունի։}
		\ex \glll (\textbf{S})$_\phi$ (\underline{\textbf{O}} V)$_\phi$  $\searrow$ \\
		mɑrˈ\textbf{jɑ-n} \underline{nɑmɑɡ-ˈ\textbf{neɾ}} uˈn-i-$\emptyset$  $\searrow$ \\
		Maria-{\defgloss} letter-{\pl} have-{\thgloss}-3{\sg} \\ 
		\trans `Maria has \underline{letters}.'
		\label{example:intontion:focus:wh:obj:basic:ans:full}
		\\ \armenian{Մարիան նամակներ  ունի։}
		\ex \glll (\textbf{S})$_\phi$ (\underline{\textbf{O}} V)$_\phi$  $\searrow$ \\
		mɑrˈ\textbf{jɑ-n} \underline{nɑmɑɡ-ˈ\textbf{ne}ɾ-ə} uˈn-i-$\emptyset$  $\searrow$ \\
		Maria-{\defgloss} letter-{\pl}-{\defgloss} have-{\thgloss}-3{\sg} \\ 
		\trans `Maria has \underline{the letters}.'
		\label{example:intontion:focus:wh:obj:basic:ans:schwa}
		\\ \armenian{Մարիան նամակները  ունի։}
	\end{xlist}
\end{exe}

For the answers, nuclear stress is on the stressed syllable of the object. This syllable can be final (\ref{example:intontion:focus:wh:obj:basic:ans:full}) or non-final (\ref{example:intontion:focus:wh:obj:basic:ans:schwa}).  The pitch-tracks show all these properties. 

\textcolor{blue}{draw}


Unlike subject wh-questions, object wh-questions allow more flexibility in their syntax. For example, the above sentences are SOV, but OVS orders are also possible (\ref{example:intontion:focus:wh:obj:ovs}). In such constructions, stress is still on the object, and there is still a rise on the final syllable. 


\begin{exe}
	\ex \label{example:intontion:focus:wh:obj:ovs} \begin{xlist}
		\ex \glll  (\underline{\textbf{what}} V)$_\phi$  ({S})$_\phi$ ?$\nearrow$ \\
		\underline{ˈ\textbf{int͡ʃ}} uˈn-i-$\emptyset$ mɑrˈ{jɑ}m-ə  ?$\nearrow$ \\
		what have-{\thgloss}-3{\sg} Mariam-{\defgloss} \\ 
		\trans `What does Mariam have?' 
		\label{example:intontion:focus:wh:obj:ovs:q}
		\\ \armenian{Ի՞նչ ունի Մարիամը։}
		\ex \glll  (\underline{\textbf{what}} V)$_\phi$  ({S})$_\phi$ ?$\nearrow$ \\
		\underline{ˈ\textbf{int͡ʃ}} uˈn-i-$\emptyset$ mɑrˈ{jɑ-n}   ?$\nearrow$ \\
		what have-{\thgloss}-3{\sg} Maria-{\defgloss} \\ 
		\trans `What does Maria have?' 
		\label{example:intontion:focus:wh:obj:ovs:ans:full}
		\\ \armenian{Ի՞նչ ունի Մարիան։}
		\ex \glll  (\underline{\textbf{O}} V)$_\phi$  ({S})$_\phi$ $\searrow$ \\
		\underline{nɑmɑɡ-ˈ\textbf{neɾ}} uˈn-i-$\emptyset$   mɑrˈ{jɑ-n} $\searrow$ \\
		letter-{\pl} have-{\thgloss}-3{\sg} Maria-{\defgloss}  \\ 
		\trans `Maria has \underline{letters}.'
		\label{example:intontion:focus:wh:obj:ovs:ans:schwa}
		\\ \armenian{Նամակներ  ունի Մարիան։}
	\end{xlist}
\end{exe}


Acoustically, we again find post-focal deaccenting after the object. This deaccenting causes the loss of phrasal prominence, but we can still perceive lexical stresses (probably just an illusion).  The final syllable of the sentence gets a rise H\%, regardless if that syllable has lexical stress (\ref{example:intontion:focus:wh:obj:ovs:ans:full}) or not (\ref{example:intontion:focus:wh:obj:ovs:q}) (= is or isn't a schwa). 

\textcolor{blue}{draw}

Longer sentences can be formed with periphrastic tenses (\ref{example:intontion:focus:wh:obj:periph}). In the sentences below, the verb phrase has proclitic, verb, and light verb. The SOV order creates a long sequence of deaccenting words after the focused object. 

\begin{exe}
	\ex \label{example:intontion:focus:wh:obj:periph} \begin{xlist}
		\ex \glll (\textbf{S})$_\phi$ (\underline{\textbf{what}})$_\phi$ (Pro V lightV)$_\phi$ ?$\nearrow$ \\
		mɑɾˈ\textbf{jɑ-n} \underline{ˈ\textbf{int͡ʃ}} bidi kʰəˈɾ-ɑd͡z əlˈl-ɑ-$\emptyset$ ?$\nearrow$ \\
		Maria-{\defgloss} what {\fut} write-{\rptcp} be-{\thgloss}-3{\sg} \\ 
		\trans `What will Maria have written?'
		\\ \armenian{Մարիան ի՞նչ պիտի գրած ըլլայ։}
		\ex \glll (\textbf{S})$_\phi$ (\underline{\textbf{O}})$_\phi$ (Pro V lightV)$_\phi$  $\searrow$ \\
		mɑɾˈ\textbf{jɑ-n} \underline{nɑmɑɡ-ˈ\textbf{neɾ}} bidi kʰəˈɾ-ɑd͡z əlˈl-ɑ-$\emptyset$  $\searrow$ \\
		Maria-{\defgloss} letter-{\pl} {\fut} write-{\rptcp} be-{\thgloss}-3{\sg} \\ 
		\trans `Maria will have written \underline{letters}.'
		\label{example:intontion:focus:wh:obj:periph:ans:full}
		\\ \armenian{Մարիան նամակներ  պիտի գրած ըլլայ։}
		\ex \glll (\textbf{S})$_\phi$ (\underline{\textbf{O}})$_\phi$ (Pro V lightV)$_\phi$  $\searrow$ \\
		mɑɾˈ\textbf{jɑ-n} \underline{nɑmɑɡ-ˈ\textbf{ne}ɾ-ə} bidi kʰəˈɾ-ɑd͡z əlˈl-ɑ-$\emptyset$  $\searrow$ \\
		Maria-{\defgloss} letter-{\pl}-{\defgloss} {\fut} write-{\rptcp} be-{\thgloss}-3{\sg} \\ 
		\trans `Maria will have written \underline{the letters}.'
		\label{example:intontion:focus:wh:obj:periph:ans:schwa}
		\\ \armenian{Մարիան նամակները  պիտի գրած ըլլայ։}
	\end{xlist}
\end{exe}

Acoustically, nuclear stress is realized as a pitch rise on the stressed syllable of the object, regardless if that syllable is word-final (\ref{example:intontion:focus:wh:obj:periph:ans:full}) or not (\ref{example:intontion:focus:wh:obj:periph:ans:schwa}). We then find deaccenting. The question ends in a final rise, while the answer in a fal fall.  

Such periphrastic tenses can be reverted to an OVS form (\ref{example:intontion:focus:wh:obj:periph:ovs}). Again, such inversion does not affect the intonation. The object gets stressed, subsequent words get deaccented, and the sentence-final syllable gets a rise in the question while a fall in the answer. 


\begin{exe}
	\ex \label{example:intontion:focus:wh:obj:periph:ovs} \begin{xlist}
		\ex \glll (\underline{\textbf{what}})$_\phi$ (Pro V lightV)$_\phi$   ({S})$_\phi$  ?$\nearrow$ \\
		\underline{ˈ\textbf{int͡ʃ}} bidi kʰəˈɾ-ɑd͡z əlˈl-ɑ-$\emptyset$ mɑɾˈ\textbf{jɑ}m-ə  ?$\nearrow$ \\
		what {\fut} write-{\rptcp} be-{\thgloss}-3{\sg} Mariam-{\defgloss}  \\ 
		\trans `What will Mariam have written?'
		\\ \armenian{Ի՞նչ պիտի գրած ըլլայ Մարիամը։}
		\ex \glll (\underline{\textbf{what}})$_\phi$ (Pro V lightV)$_\phi$   ({S})$_\phi$  ?$\nearrow$ \\
		\underline{ˈ\textbf{int͡ʃ}} bidi kʰəˈɾ-ɑd͡z əlˈl-ɑ-$\emptyset$ mɑɾˈ\textbf{jɑ-n}  ?$\nearrow$ \\
		what {\fut} write-{\rptcp} be-{\thgloss}-3{\sg} Maria-{\defgloss}  \\ 
		\trans `What will Maria have written?'
		\\ \armenian{Ի՞նչ պիտի գրած ըլլայ Մարիան։}
		\ex \glll(\underline{\textbf{O}})$_\phi$ (Pro V lightV)$_\phi$   ({S})$_\phi$  $\searrow$ \\
		\underline{nɑmɑɡ-ˈ\textbf{neɾ}} bidi kʰəˈɾ-ɑd͡z əlˈl-ɑ-$\emptyset$ mɑɾˈ{jɑ-n}  $\searrow$ \\
		letter-{\pl} {\fut} write-{\rptcp} be-{\thgloss}-3{\sg} Maria-{\defgloss}  \\ 
		\trans `Maria will have written \underline{letters}.'
		\\ \armenian{Նամակներ  պիտի գրած ըլլայ Մարիան։}
		
	\end{xlist}
\end{exe}

Pitch-tracks again confirm these impressions. The focused object has a prominent rise, followed by a post-focal deaccenting on all subsequent words. The question has a final rise on the sentence-final syllable. 



\textcolor{blue}{draw}

\subsubsection{Distance restrictions on final rises}\label{section:intonation:focus:wh:length}
All the above sentences had multiple syllables after focused word. These syllables were part of at least one lexical word. The sentence was long enough to easily allow a rise on the focused word, and then a rise on the final syllable.  However, when only one syllable (a clitic) follows the object focus, HD tends to not have any rise on this final syllable. 

To illustrative, consider SOV     sentences where the verb is a clitic copula (\ref{example:inton:focus:wh:length:cl:sov}).  

\begin{exe}
	\ex \begin{xlist}
		\ex \glll (\textbf{S})$_\phi$ (\underline{\textbf{who}} Cop)$_\phi$ ?$\nearrow$ \\
		mɑɾˈ\textbf{jɑ}m-ə  \underline{ˈ\textbf{ov}} =e ?$\nearrow$ \\
		Mariam-{\defgloss} who =is \\
		\trans `Who is Mariam?'
		\label{example:inton:focus:wh:length:cl:sov}
		\\ \armenian{Մարիամը ո՞վ է։}
		\ex \glll (\underline{\textbf{who}} Cop)$_\phi$ ({S})$_\phi$  ?$\nearrow$ \\
		\underline{ˈ\textbf{ov}} =e mɑɾˈ{jɑ}m-ə   ?$\nearrow$ \\
		who  =is  Mariam-{\defgloss} \\
		\trans `Who is Mariam?'
		\label{example:inton:focus:wh:length:cl:ovs}
		\\ \armenian{Ո՞վ է Մարիամը։}
		
	\end{xlist}
\end{exe}

The pitch-tracks show that the object has prominence.  But   for the SOV question (\ref{example:inton:focus:wh:length:cl:sov}),   the final clitic seems to have no prominence at all, not even a final rise. HD still perceives a sentence-final rise, but this rise may actually be anchored onto the non-final wh-word instead. The lack of a sentence-final rise may be because this clitic is too close to the focused object.  In contrast, the OVS sentence (\ref{example:inton:focus:wh:length:cl:ovs}) shows a sentence-final rise. 

When the clitic is longer, it seems that is then easier to create a sentence-final rise. Consider the sentences in (\ref{example:inton:focus:wh:length:cl:long}). 



\begin{exe}
	\ex \label{example:inton:focus:wh:length:cl:long} \begin{xlist}
		\ex \glll (\textbf{S})$_\phi$ (\underline{\textbf{who}} Cop)$_\phi$ ?$\nearrow$ \\
		dəʁˈ\textbf{ɑ}kʰ-ə  \underline{ˈ\textbf{ov}} =e-ji-n ?$\nearrow$ \\
		boy-{\pl}-{\defgloss} who =is-{\pst}-3{\pl} \\
		\trans `Who were the boys?'
		\\ \armenian{Տղաքը ո՞վ էին։}
		\ex \glll (\underline{\textbf{who}} Cop)$_\phi$ ({S})$_\phi$  ?$\nearrow$ \\
		\underline{ˈ\textbf{ov}} =e-ji-n dəʁˈ{ɑ}kʰ-ə   ?$\nearrow$ \\
		who =is-{\pst}-3{\pl} boy-{\pl}-{\defgloss}  \\
		\trans `Who were the boys?'
		\\ \armenian{Ո՞վ էին տղաքը։}
		
	\end{xlist}
\end{exe}

The pitch-tracks show a sentence-final rise in both the SOV and OVS orders.  






\subsection{Summary of   focus intonation and cross-dialectal differences}\label{section:intonation:focus:summary}
\textcolor{red}{write after recordings in later stages}

, or negated verb. Complications
In the base case, the verb is final stressed syllable of the sentence (\S\ref{section:intonation:focus:polar:finalV}). Here, H* and H\% are the same. But there are cases where the  verb is a final unstressed clitic  or an unstressed light verb (\S\ref{section:intonation:focus:polar:finalCl}). In this case, stress H* is on the final stressable syllable, and we see a continuous rise from that syllble until the end of the sentence. 

If the verb is non-final as in OVS (\S\ref{section:intonation:focus:polar:finalN}, the verb still gets stress H*. The rise H\%  starts from the verb and continues till the end of the sentence. 

All these patterns are generally the same when the verb is negated (\S\ref{section:intonation:focus:polar:negation}), though there may be some level of post-focal deaccenting. 

When a non-verb is focused in a polar question (\S\ref{section:intonation:focus:polar:contrastive}, that non-verb gets stress H*, and then we see a high plateau. The sentence ends in a rise H\% for HD but there is variation. 

\section{Prosodic structure of other syntactic structures}\label{section:intonation:other}
This section goes over the intonation of syntactic structures that don't easily fit into the previous sections and their categories. These constructions     are subjunctive clauses with the clitic \textit{=ne} (\S\ref{section:intonation:other:subjunctive}), relative clauses with extraposition (\S\ref{section:intonation:other:extraposition}), imperatives (\S\ref{section:intonation:other:imperative}), and vocatives (\S\ref{section:intonation:other:vocative}). 

\subsection{Subjunctive clauses and the subjunctive clitic} \label{section:intonation:other:subjunctive}

In a subjunctive or subordinate clause, the subjunctive clitic \textit{=ne} is optional but its presence triggers speciaj   intonational effects.  See \S\ref{section:stress:cliticc} and \S\ref{section:intonation:clitic:overview} for data on how this morpheme  is a clitic.

For a subordinate clause like an `if-clause', the clause can end with (\ref{example:inton:other:subj:basic:yes}) or without (\ref{example:inton:other:subj:basic:no}) the subjunctive marker \textit{=ne} after the verb. The two sentences are synonymous, but they have different intonational effects on the first verb. We underline and mark the word with the most prominent stress in the if-clause. 

\begin{exe}
	\ex \begin{xlist}
		\ex \gll jetʰe \underline{jeɾˈtʰ-ɑ-n}, uɾɑχ ɡ-əll-ɑ-n  \\
		if go-{\thgloss}-3{\pl}, happy {\ind}-be-{\thgloss}-3{\pl} \\
		\trans `If they go, they'll be happy.'
		\label{example:inton:other:subj:basic:no}
		\\ \armenian{Եթէ երթան, ուրախ կ՚ըլլան։}
		\ex \gll jetʰe \underline{jeɾˈtʰ-ɑ-n} =ne, uɾɑχ ɡ-əll-ɑ-n  \\
		if go-{\thgloss}-3{\pl} ={\sbjv}, happy {\ind}-be-{\thgloss}-3{\pl} \\
		\trans `If they go, they'll be happy.'
		\label{example:inton:other:subj:basic:yes}
		\\ \armenian{Եթէ երթան նէ, ուրախ կ՚ըլլան։}
	\end{xlist}
\end{exe}


Essentially, whenever the \textit{=ne} is added, the preceding syllable is perceivably more prominent (\ref{example:inton:other:subj:basic:yes}) than when the \textit{=ne} is absent (\ref{example:inton:other:subj:basic:yes}). The syllable has a perceivably higher pitch. 

\textcolor{blue}{draw}

If the if-clause has an object, the object typically gets stress (\ref{example:inton:other:subj:obj:no}). If the \textit{=ne} is added (\ref{example:inton:other:subj:obj:yes}), then stress visibly shifts to the verbal syllable that precedes the \textit{=ne}. 


\begin{exe}
	\ex \begin{xlist}
		\ex \gll jetʰe \underline{ˈdun} jeɾtʰ-ɑ-n, uɾɑχ ɡ-əll-ɑ-n  \\
		if home go-{\thgloss}-3{\pl}, happy {\ind}-be-{\thgloss}-3{\pl} \\
		\trans `If they go home, they'll be happy.'
		\label{example:inton:other:subj:obj:no}
		\\ \armenian{Եթէ տուն երթան, ուրախ կ՚ըլլան։}
		\ex \gll jetʰe dun \underline{jeɾˈtʰ-ɑ-n} =ne, uɾɑχ ɡ-əll-ɑ-n  \\
		if home go-{\thgloss}-3{\pl} ={\sbjv}, happy {\ind}-be-{\thgloss}-3{\pl} \\
		\trans `If they go home, they'll be happy.'
		\label{example:inton:other:subj:obj:yes}
		\\ \armenian{Եթէ տուն երթան նէ, ուրախ կ՚ըլլան։}
	\end{xlist}
\end{exe}

The shift in prominence is visible from the pitch-tracks. 


\textcolor{blue}{draw}



The subjunctive clitic \textit{=ne} quite regularly shifts stress to its preceding syllable. For example, the progressive clitic \textit{=ɡoɾ} is typically unstressed (\ref{example:inton:other:subj:gor:no}). But if the progressive is before the subjunctive (\ref{example:inton:other:subj:gor:yes}), then the progressive gets stress. 


\begin{exe}
	\ex \begin{xlist}
		\ex \gll jetʰe \underline{ɡ-eɾˈtʰ-ɑ-n} =ɡoɾ, uɾɑχ ɡ-əll-ɑ-n  \\
		if {\ind}-go-{\thgloss}-3{\pl} ={\prog}, happy {\ind}-be-{\thgloss}-3{\pl} \\
		\trans `If they are going, they'll be happy.'
		\label{example:inton:other:subj:gor:no}
		\\ \armenian{Եթէ կ՚երթան կոր, ուրախ կ՚ըլլան։}
		\ex \gll jetʰe ɡeɾtʰ-ɑ-n \underline{=ˈɡoɾ} =ne, uɾɑχ ɡ-əll-ɑ-n  \\
		if {\ind}-go-{\thgloss}-3{\pl} ={\prog} ={\sbjv}, happy {\ind}-be-{\thgloss}-3{\pl} \\
		\trans `If they are going, they'll be happy.'
		\label{example:inton:other:subj:gor:yes}
		\\ \armenian{Եթէ կ՚երթան կոր նէ, ուրախ կ՚ըլլան։}
	\end{xlist}
\end{exe}

Again the pitch-tracks show this shift in pitch prominence. 

\textcolor{blue}{draw}


Unfortunately, the subjunctive marker \textit{=ne} is quite stigmatized yet common in colloquial speech. This makes it difficult to easily elicit data on the marker.  It is an open question if this shift in prominence is because the marker induces semantic focus on the preceding word, or if this is merely a lexical idiosyncrasy of this marker. 


\subsection{Relative clauses and extraposition}\label{section:intonation:other:extraposition}

We go over the basic prosodic structure of relative clauses. A major property of such clauses is that they are obligatorily extraposed if a)   the head noun is preverbal, and b) the head noun is in the same prosodic phrase as the verb. We first provide a brief overview of the phenomenon (\S\ref{section:intonation:other:extraposition:overview}), then we catalog contexts for extraposition from objects (\S\ref{section:intonation:other:extraposition:object}) and subjects (\S\ref{section:intonation:other:extraposition:subj}). We tease apart phrasing and stress in  \S\ref{section:intonation:other:extraposition:gen}. 

\subsubsection{Overview of relative clause prosody and extraposition}\label{section:intonation:other:extraposition:overview}


Nouns can be modified with relative clauses. Such clauses are pronounced as separate stress domains (\ref{example:inton:other:rel:basic}).   

\begin{exe}
	\ex 	\label{example:inton:other:rel:basic} \begin{xlist}
		\ex 	\glll  ({{\textbf{N}})$_\phi$}  (that {\textbf{O}} V)$_\phi$ \\
		{{ɡɑˈ\textbf{du}-mə}}   voɾ {{bɑ\textbf{ˈni}ɾ-ə}} {ɡeˈɾ-ɑ-v} 
		\\
		cat-{\indf} that cheese-{\defgloss} eat.{\aorperf}-{\pst}-3{\sg}
		\\
		\trans `A   cat who ate the cheese.'
		\\\armenian{Կատու մը որ  պանիրը կերաւ։}
		\ex 	\glll  (Adj {{\textbf{N}})$_\phi$}  (that {\textbf{O}} V)$_\phi$ \\
		{ɡɑɾˈmiɾ} {{ɡɑˈ\textbf{du}-mə}}  voɾ {{bɑ\textbf{ˈni}ɾ-ə}} {ɡeˈɾ-ɑ-v} 
		\\
		red cat-{\indf} that cheese-{\defgloss} eat.{\aorperf}-{\pst}-3{\sg}
		\\
		\trans `A red cat who ate the cheese.'
		\\\armenian{Կարմիր կատու մը որ  պանիրը կերաւ։}
	\end{xlist}
	
	
\end{exe}

We use the term `stress domain' out of agnosticism. It's not obvious to us if relative clauses are necessarily separate intonational phrases, or if they're just separate prosodic phrases. We suspect that they're just separate prosodic phrases. The evidence is that there seems to be a constant decrease or declination in pitch as we move through the sentence. Thus, it does not seem that relative clauses trigger a reset or re-start in pitch levels. More systematic acoustic data is however needed. 

\textcolor{blue}{draw}




Such clauses can be added either directly after the noun, or after an intervening verb.  An interesting correlation between stress and relative clauses is extraposition. Consider a transitive (S)OV sentence. If the subject is modified with a relative clause (\ref{example:inton:other:rel:extrapos:nosubj}), then the relative clause must be adjacent to the subject. In contrast, if the object is modified (\ref{example:inton:other:rel:extrapos:noadj}), then the object is extraposed or placed after the verb. 

\begin{exe}
	\ex \begin{xlist}
		\ex \glll  ({{\textbf{S}}})$_\phi$  (that {\textbf{O}} V)$_\phi$ (\textbf{O} V)$_\phi$ \\
		{{ɡɑˈ\textbf{du}-mə}} voɾ   {{bɑˈ\textbf{ni}ɾ-ə}} {ɡeˈɾ-ɑ-v}  mɑɾjɑˈ\textbf{m-i-n} χɑˈd͡z-ɑ-v
		\\
		cat-{\indf} that   cheese-{\defgloss} eat.{\aorperf}-{\pst}-3{\sg}  Mariam-{\dat}-{\defgloss} bite-{\pst}-3{\sg}
		\\
		\trans `A cat who ate  the cheese bit Mariam.'
		\label{example:inton:other:rel:extrapos:nosubj}
		\\\armenian{Կատու մը որ   պանիրը կերաւ Մարիամին խածաւ։}
		
		\ex \glll  ({{\textbf{O}}} V)$_\phi$  (that {\textbf{O}} V)$_\phi$ \\
		{{ɡɑˈ\textbf{du}-mə}} {uˈ{n-i-m}}   voɾ {{bɑˈ\textbf{ni}ɾ-ə}} {ɡeˈɾ-ɑ-v} 
		\\
		cat-{\indf} have-{\thgloss}-1{\sg} that cheese-{\defgloss} eat.{\aorperf}-{\pst}-3{\sg} 
		\\
		\trans `I have a   cat who ate the cheese.'
		\label{example:inton:other:rel:extrapos:noadj}
		\\\armenian{Կատու մը ունիմ որ  պանիրը կերաւ։}
	\end{xlist}
	
\end{exe}

For such SOV sentences, the subject's relative clause must be adjacent to the subject (\ref{example:inton:other:rel:extrapos:nosubj}). If the relative clause was extraposed to after the verb (\ref{example:inton:other:rel:extrapos:nosubj:extraposWrong}), then the relative clause incorrectly modifies the object.

\begin{exe}
	\ex \glll  \#({{\textbf{S}}})$_\phi$   (\textbf{O} V)$_\phi$  (that {\textbf{O}} V)$_\phi$\\
	{{ɡɑˈ\textbf{du}-mə}}  mɑɾjɑˈ\textbf{m-i-n} χɑˈd͡z-ɑ-v voɾ   {{bɑˈ\textbf{ni}ɾ-ə}} {ɡeˈɾ-ɑ-v} 
	\\
	cat-{\indf}  Mariam-{\dat}-{\defgloss} bite-{\pst}-3{\sg} that   cheese-{\defgloss} eat.{\aorperf}-{\pst}-3{\sg} 
	\\
	\trans Intended: `A cat who ate  the cheese bit Mariam.'\\
	Actual: `A cat bit Mariam who ate the cheese.' 
	\label{example:inton:other:rel:extrapos:nosubj:extraposWrong}
	\\\armenian{Կատու մը Մարիամին խածաւ որ պանիրը կերաւ։}
	
\end{exe}

In contrast, if the relative clause modifies the direct object in an SOV sentence, then the relative clause must be extraposed or placed after the verb (\ref{example:inton:other:rel:extrapos:noadj}). If the relative clause was placed next to the noun (\ref{example:inton:other:rel:extrapos:no}), then there is a connotation that the object is given  topicalized information, and there's a significant pause before the main verb.

\begin{exe}
	\ex   \glll  ({\textbf{O}})$_\phi$ (that {\textbf{O}} V)$_\phi$ (V)$_\phi$   \\
	{{ɡɑˈ\textbf{du}-mə}}  voɾ {{bɑˈ\textbf{ni}ɾ-ə}} {ɡeˈɾ-ɑ-v}   {uˈ\textbf{n-i-m}} 
	\\
	cat-{\indf} that cheese-{\defgloss} eat.{\aorperf}-{\pst}-3{\sg}  have-{\thgloss}-1{\sg} 
	\\
	\trans `A   cat who ate the cheese, I have.'
	\label{example:inton:other:rel:extrapos:no}
	\\\armenian{Կատու մը  որ  պանիրը կերաւ, ունիմ։}
\end{exe}

\subsubsection{Extraposition of objects}\label{section:intonation:other:extraposition:object}

For direct objects, the need for extraposition seems consistent. Extraposition applies for   direct objects  with modifiers (\ref{example:inton:other:rel:extrapos:yes}) and   for definite objects (\ref{example:inton:other:rel:extrapos:def}). 


\begin{exe}
	\ex \begin{xlist}
		\ex \glll  (Adj {{\textbf{O-{\indf}}}} V)$_\phi$  (that {\textbf{O}} V)$_\phi$ \\
		{ɡɑɾˈmiɾ} {{ɡɑˈ\textbf{du}-mə}} {uˈ{n-i-m}} voɾ {{bɑˈ\textbf{ni}ɾ-ə}} {ɡeˈɾ-ɑ-v} 
		\\
		red cat-{\indf} have-{\thgloss}-1{\sg}  that cheese-{\defgloss} eat.{\aorperf}-{\pst}-3{\sg} 
		\\
		\trans `I have a red cat who ate the cheese.'
		\label{example:inton:other:rel:extrapos:yes}
		\\\armenian{Կարմիր կատու մը ունիմ որ  պանիրը կերաւ։}
		\ex \glll ({{\textbf{O-{\defgloss}}}} V)$_\phi$  (that {\textbf{O}} V)$_\phi$ \\ 
		{{ɡɑˈ\textbf{du-n}}} {uˈ{n-i-m}} voɾ {{bɑˈ\textbf{ni}ɾ-ə}} {ɡeˈɾ-ɑ-v} 
		\\
		cat-{\defgloss} have-{\thgloss}-1{\sg}  that cheese-{\defgloss} eat.{\aorperf}-{\pst}-3{\sg} 
		\\
		\trans `I have the   cat who ate the cheese.'
		\label{example:inton:other:rel:extrapos:def}
		\\\armenian{Կատուն ունիմ որ  պանիրը կերաւ։}
	\end{xlist}
\end{exe}

Note that the above generalizations are for pre-verbal objects. If the object is post-verbal (\ref{example:inton:other:rel:svo}), then no extraposition is  needed because the noun and relative clause are already adjacent. 

\begin{exe}
	\ex \glll  (\textbf{V})$_\phi$  ({{\textbf{O-{\indf}}}})$_\phi$  (that {\textbf{O}} V)$_\phi$ \\
	uˈ\textbf{n-i-m}	 {{ɡɑˈ\textbf{du}-mə}}  voɾ {{bɑˈ\textbf{ni}ɾ-ə}} {ɡeˈɾ-ɑ-v} 
	\\
	have-{\thgloss}-1{\sg}  cat-{\indf}    that cheese-{\defgloss} eat.{\aorperf}-{\pst}-3{\sg} 
	\\
	\trans `I have  a cat   who ate the cheese.'
	\label{example:inton:other:rel:svo}
	\\\armenian{Ունիմ կատու մը   որ  պանիրը կերաւ։}
\end{exe}




The generalization is that extraposition applies so that the preverbal object and the verb can be parsed into a single prosodic phrase. We see this generalization likewise in ditransitives, where there are  two objects (\ref{example:inton:other:rel:extrapos:ditrans:base}). The first object can get a non-extraposed relative clause (\ref{example:inton:other:rel:extrapos:ditrans:preN}). 

\begin{exe}
	\ex \begin{xlist}
		\ex \glll (\textbf{IO})$_\phi$ (\textbf{DO} V)$_\phi$ \\
		dəˈ\textbf{ʁ-u}-mə ɡɑˈ\textbf{du}-mə dəv-i-$\emptyset$ \\
		boy-{\dat}-{\indf} cat-{\indf} give.{\aorperf}-{\pst}-1{\sg} \\
		\trans `I gave a cat to a boy.' 
		\label{example:inton:other:rel:extrapos:ditrans:base}
		\\ \armenian{Տղու մը կատու մը տուի։}
		\ex \glll (\textbf{IO})$_\phi$ (that \textbf{Adj} Cop)$_\phi$ (\textbf{DO} V)$_\phi$ \\
		dəˈ\textbf{ʁ-u}-mə  voɾ u\textbf{ˈrɑχ} =e-$\emptyset$-ɾ ɡɑˈ\textbf{du}-mə dəv-i-$\emptyset$ \\
		boy-{\dat}-{\indf} that happy =is-{\pst}-3{\sg} cat-{\indf} give.{\aorperf}-{\pst}-1{\sg} \\
		\trans `I gave a cat to a boy who was happy.' 
		\label{example:inton:other:rel:extrapos:ditrans:preN}
		\\ \armenian{Տղու մը որ ուրախ էր, կատու մը տուի։}
	\end{xlist}
\end{exe}



But the second and immediately preverbal object requires extraposition (\ref{example:inton:other:rel:extrapos:ditrans:extrapos}). If the preverbal object moves its location, then there is no extraposition (\ref{example:inton:other:rel:extrapos:ditrans:reorder}).  

\begin{exe}
	\ex \begin{xlist}
		\ex \glll (\textbf{IO})$_\phi$ (\textbf{DO} V)$_\phi$ (that \textbf{Adj} Cop)$_\phi$  \\
		dəˈ\textbf{ʁ-u}-mə  ɡɑˈ\textbf{du}-mə dəv-i-$\emptyset$  voɾ ɡɑɾ\textbf{ˈmiɾ} =e-$\emptyset$-ɾ \\
		boy-{\dat}-{\indf}  cat-{\indf} give.{\aorperf}-{\pst}-1{\sg} that red =is-{\pst}-3{\sg} \\
		\trans `I gave a cat that is red to a boy.'
		\label{example:inton:other:rel:extrapos:ditrans:extrapos}
		\\ \armenian{Տղու մը կատու մը տուի որ կարմիր էր։}
		\ex \glll (\textbf{DO})$_\phi$ (that \textbf{Adj} Cop)$_\phi$ (\textbf{IO} V)$_\phi$   \\
		ɡɑˈ\textbf{du}-mə   voɾ ɡɑɾ\textbf{ˈmiɾ} =e-$\emptyset$-ɾ  dəˈ\textbf{ʁ-u}-mə  dəv-i-$\emptyset$\\
		cat-{\indf} that red =is-{\pst}-3{\sg} boy-{\dat}-{\indf}   give.{\aorperf}-{\pst}-1{\sg}  \\
		\trans `I gave a cat that is red to a boy.'
		\label{example:inton:other:rel:extrapos:ditrans:reorder}
		\\ \armenian{Կատու մը  որ կարմիր էր, տղու մը տուի։}
		
	\end{xlist}
\end{exe}


Object wh-questions show obligatory extraposition (\ref{example:inton:other:rel:extrapos:obj:wh}). The object is generally preverbal, it and takes focus. The relative clause is extraposed. 


\begin{exe}
	\ex \glll (\textbf{S})$_\phi$ (\underline{\textbf{what}} V)$_\phi$ (that {Adj} Cop)$_\phi$  \\
	dəˈ\textbf{ʁɑ-n}   \underline{ˈ\textbf{int͡ʃ}} uˈn-i-$\emptyset$  voɾ ɡɑɾ{ˈmiɾ} =e \\
	boy-{\defgloss} what have-{\thgloss}-1{\sg} that red =is \\
	\trans `What does the boy have that is red?'
	\label{example:inton:other:rel:extrapos:obj:wh}
	\\ \armenian{Տղան ի՞նչ ունի որ կարմիր է։}
	
\end{exe}

If the wh-word and relative clause stayed adjacent (\ref{example:inton:other:rel:extrapos:obj:whever}), then the sentence is not easily interpreted as a wh-question. The sentence is instead a declarative and the `what that' sequence is reinterpreted as a free relative `whatever'. 


\begin{exe}
	\ex 		\label{example:inton:other:rel:extrapos:obj:whever}
	\begin{xlist}
		\ex \glll (\textbf{S})$_\phi$ ({\textbf{what}})$_\phi$ (that \textbf{Adj} Cop)$_\phi$ (\textbf{V})$_\phi$ \\
		dəˈ\textbf{ʁɑ-n}   {ˈ\textbf{int͡ʃ}} voɾ ɡɑɾ\textbf{ˈmiɾ} =e uˈ\textbf{n-i-$\emptyset$}   \\
		boy-{\defgloss} what that red =is  have-{\thgloss}-1{\sg} \\
		\trans `The boy has whatever that is read.'
		\\ \armenian{Տղան ինչ  որ կարմիր  է ունի։}
		\ex \glll (\textbf{S})$_\phi$ (\textbf{V})$_\phi$ ({\textbf{what}})$_\phi$ (that \textbf{Adj} Cop)$_\phi$  \\
		dəˈ\textbf{ʁɑ-n}    uˈ\textbf{n-i-$\emptyset$} {ˈ\textbf{int͡ʃ}} voɾ ɡɑɾ\textbf{ˈmiɾ} =e   \\
		boy-{\defgloss} have-{\thgloss}-1{\sg}  what that red =is  \\
		\trans `The boy has whatever that is read.'
		\\ \armenian{Տղան ունի ինչ  որ կարմիր  է ։}
		
	\end{xlist}
\end{exe}


\subsubsection{Extraposition of subjects}\label{section:intonation:other:extraposition:subj}
This correlation between prosodic phrasing and extraposition is also found in intransitive subjects. For an unaccusative verb, the norm is that the subject is part of the prosodic phrase of the verb, regardless if the subject is indefinite (\ref{example:inton:other:rel:extrapos:unacc:indef}) or definite (\ref{example:inton:other:rel:extrapos:unacc:def}). Extraposition is again the norm.

\begin{exe}
	\ex \begin{xlist}
		\ex \glll (\textbf{S} V)$_\phi$  (that \textbf{Adj} Cop)$_\phi$  \\
		ɡɑˈ\textbf{du}-mə jeˈɡ-ɑ-v voɾ ɡɑɾ\textbf{ˈmiɾ} =e-$\emptyset$-ɾ  \\
		cat-{\indf} come.{\aorperf}-{\pst}-3{\sg} that red =is-{\pst}-3{\sg} \\
		\trans `A cat came that was red.'
		\label{example:inton:other:rel:extrapos:unacc:indef}
		\\ \armenian{Կատու մը եկաւ որ կարմիր էր։}
		\ex \gll ɑjtʰ ɡɑˈ\textbf{du-n} jeˈɡ-ɑ-v voɾ ɡɑɾ\textbf{ˈmiɾ} =e-$\emptyset$-ɾ  \\
		that cat-{\defgloss} come.{\aorperf}-{\pst}-3{\sg} that red =is-{\pst}-3{\sg} \\
		\trans `That cat came that was red.'
		\label{example:inton:other:rel:extrapos:unacc:def}
		\\ \armenian{Այդ կատուն եկաւ որ կարմիր էր։}
	\end{xlist}
\end{exe}

Note that the definite form in (\ref{example:inton:other:rel:extrapos:unacc:def}) includes a demonstrative. Without an additional modifier like a demonstrative, it feels infelicitous to add a relative to the definite subject.

This correlation is clearer in unergatives (\ref{example:inton:other:rel:extrapos:unerg:indef}) and passives (\ref{example:inton:other:rel:extrapos:pass:indef}). If the subject is indefinite, then it is phrased by the verb and triggers relative clause extraposition. 


\begin{exe}
	\ex \begin{xlist}
		\ex \glll (\textbf{S} V)$_\phi$  (that \textbf{Adj} Cop)$_\phi$  \\
		ɡɑˈ\textbf{du}-mə vɑˈz-e-t͡s-$\emptyset$-$\emptyset$ voɾ ɡɑɾ\textbf{ˈmiɾ} =e-$\emptyset$-ɾ  \\
		cat-{\indf} run-{\thgloss}-{\aorperf}-{\pst}-3{\sg} that red =is-{\pst}-3{\sg} \\
		\trans `A cat ran that was red.'
		\label{example:inton:other:rel:extrapos:unerg:indef}
		\\ \armenian{Կատու մը վազեց որ կարմիր էր։}
		\ex \gll%l (\textbf{S} V)$_\phi$  (that \textbf{Adj} Cop)$_\phi$  \\
		zinˈ\textbf{voɾ}-mə əspɑnnə-v-e-ˈt͡s-ɑ-v  voɾ ɡɑɾ\textbf{ˈmiɾ} =e-$\emptyset$-ɾ  \\
		soldier-{\indf} kill-{\pass}-{\thgloss}-{\aorperf}-{\pst}-3{\sg} that red =is-{\pst}-3{\sg} \\
		\trans `A soldier was killed   who was red.'
		\label{example:inton:other:rel:extrapos:pass:indef}
		\\ \armenian{Զինուոր մը սպաննուեցաւ որ կարմիր էր։}
	\end{xlist}
\end{exe}

But for unergatives (\ref{example:inton:other:rel:extrapos:unerg:def}) and passives (\ref{example:inton:other:rel:extrapos:pass:def}),   the definite tends to be phrased separately. So extraposition is not needed. 


\begin{exe}
	\ex \begin{xlist}
		\ex \glll ( \textbf{S})$_\phi$  (that \textbf{Adj} Cop)$_\phi$   (\textbf{V})$_\phi$\\
		ɑjtʰ ɡɑˈ\textbf{du-n}  voɾ ɡɑɾ\textbf{ˈmiɾ} =e-$\emptyset$-ɾ  vɑˈ\textbf{z-e-t͡s-$\emptyset$-$\emptyset$}\\
		that  cat-{\indf}  that red =is-{\pst}-3{\sg} run-{\thgloss}-{\aorperf}-{\pst}-3{\sg}\\
		\trans `That cat that was red ran.'
		\label{example:inton:other:rel:extrapos:unerg:def}
		\\ \armenian{Այդ կատուն  որ կարմիր էր վազեց։}
		\ex \gll ɑjtʰ zinˈ\textbf{vo}ɾ-ə   voɾ ɡɑɾ\textbf{ˈmiɾ} =e-$\emptyset$-ɾ  əspɑnnə-v-e-ˈ\textbf{t͡s-ɑ-v}\\
		that soldier-{\indf}  that red =is-{\pst}-3{\sg} kill-{\pass}-{\thgloss}-{\aorperf}-{\pst}-3{\sg}\\
		\trans `That soldier   who was red was killed.'
		\label{example:inton:other:rel:extrapos:pass:def}
		\\ \armenian{Այդ զինուորը    որ կարմիր էր սպաննուեցաւ։}
	\end{xlist}
\end{exe}

Adding an extraposed relative clause (\ref{example:inton:other:rel:extrapos:unergpass:extra})  is either unacceptable or creates the sense of an afterthought.


\begin{exe}
	\ex \label{example:inton:other:rel:extrapos:unergpass:extra} \begin{xlist}
		\ex \glll ?\#( \textbf{S} V)$_\phi$  (that \textbf{Adj} Cop)$_\phi$  \\ 
		ɑjtʰ ɡɑˈ\textbf{du-n} vɑˈz-e-t͡s-$\emptyset$-$\emptyset$  voɾ ɡɑɾ\textbf{ˈmiɾ} =e-$\emptyset$-ɾ  \\
		that  cat-{\indf} run-{\thgloss}-{\aorperf}-{\pst}-3{\sg} that red =is-{\pst}-3{\sg} \\
		\trans `That cat ran, that was red.'
		\\ \armenian{Այդ կատուն  վազեց որ կարմիր էր ։}
		\ex \gll ?\#ɑjtʰ zinˈ\textbf{vo}ɾ-ə  əspɑnnə-v-e-ˈ{t͡s-ɑ-v} voɾ ɡɑɾ\textbf{ˈmiɾ} =e-$\emptyset$-ɾ  \\
		that soldier-{\indf} kill-{\pass}-{\thgloss}-{\aorperf}-{\pst}-3{\sg} that red =is-{\pst}-3{\sg} \\
		\trans `That soldier   was killed, who was red.'
		\\ \armenian{Այդ զինուորը  սպաննուեցաւ   որ կարմիր էր։}
	\end{xlist}
\end{exe}


In transitive sentences, a SOV order does not allow extraposing a  relative clause from the subject (\ref{example:inton:other:rel:extrapos:nosubj:extraposWrong}). But in an OSV order, we can extrapose the relative clause from the subject. Such subject constructions are discussed more in \textcolor{red}{cite subject incorporation} in the context of subject incorporation.  


\begin{exe}
	\ex \glll   ({{\textbf{O}}})$_\phi$   (\textbf{S} V)$_\phi$  (that {\textbf{O}} V)$_\phi$\\
	mɑɾjɑˈ\textbf{m-i-n} {{ɡɑˈ\textbf{du}-mə}}   χɑˈd͡z-ɑ-v voɾ   {{bɑˈ\textbf{ni}ɾ-ə}} {ɡeˈɾ-ɑ-v} 
	\\
	Mariam-{\dat}-{\defgloss} cat-{\indf}   bite-{\pst}-3{\sg} that   cheese-{\defgloss} eat.{\aorperf}-{\pst}-3{\sg} 
	\\
	\trans   `Mariam was bit by a cat who ate the cheese.' 
	\label{example:inton:other:rel:extrapos:OSV}
	\\\armenian{Մարիամին  կատու մը խածաւ որ պանիրը կերաւ։}
	
\end{exe} 

\subsubsection{General role of prosodic phrasing}\label{section:intonation:other:extraposition:gen}

In all the above sentences, a unifying factor for the extraposition contexts was that a) the noun and verb were in the same prosodic phrase, and b) the noun had phrasal stress. Data from focus show that  the first property (prosodic phrasing) is the primary factor behind extraposition. The stress correlations are the effects of such phrasing. For example, consider the wh-question and answer in (\ref{example:inton:other:rel:extrapos:whS} ). In these SOV sentences, the subject has focus, while the object is modifed with an extraposed relative clause. Subject focus causes deaccenting on all subsequent words (\S\ref{section:intonation:focus:wh:subj}).


\begin{exe}
	\ex \label{example:inton:other:rel:extrapos:whS} 
	\begin{xlist}
		
		\ex \glll (\underline{\textbf{S}})$_\phi$ ({{{O}}} V)$_\phi$  (that {{O}} V)$_\phi$ \\
		\underline{ˈ\textbf{ov}} {{ɡɑˈ{du}-mə}} {uˈ{n-i-$\emptyset$}}   voɾ {{bɑˈ{ni}ɾ-ə}} {ɡeˈɾ-ɑ-v} 
		\\
		who	cat-{\indf} have-{\thgloss}-3{\sg} that cheese-{\defgloss} eat.{\aorperf}-{\pst}-3{\sg} 
		\\
		\trans `Who has a   cat who ate the cheese?'
		\\\armenian{Ո՞վ կատու մը ունի որ  պանիրը կերաւ։}
		
		\ex \glll (\underline{\textbf{S}})$_\phi$ ({{{O}}} V)$_\phi$  (that {{O}} V)$_\phi$ \\
		\underline{mɑɾˈ\textbf{jɑ-n}} {{ɡɑˈ{du}-mə}} {uˈ{n-i-$\emptyset$}}   voɾ {{bɑˈ{ni}ɾ-ə}} {ɡeˈɾ-ɑ-v} 
		\\
		Maria-{\defgloss} 	cat-{\indf}  have-{\thgloss}-3{\sg}  that cheese-{\defgloss} eat.{\aorperf}-{\pst}-3{\sg} 
		\\
		\trans `\underline{Maria} has a   cat who ate the cheese'
		\\\armenian{Մարիան  կատու մը ունի որ  պանիրը կերաւ։}
	\end{xlist}
\end{exe}

Even though the object is unstressed, it still requires extraposition. If the object and clause were adjacent before the verb (\ref{example:inton:other:rel:extrapos:whMore}), then that creates a connotation that the object is somehow topicalized, or that the verb has some level of focus. 

\begin{exe}
	\ex \label{example:inton:other:rel:extrapos:whMore} 
	\begin{xlist}
		
		\ex \glll (\underline{\textbf{S}})$_\phi$ ({{{O}}})$_\phi$  (that {{O}} V)$_\phi$  (V)$_\phi$\\
		\underline{ˈ\textbf{ov}} {{ɡɑˈ{du}-mə}} voɾ {{bɑˈ{ni}ɾ-ə}} {ɡeˈɾ-ɑ-v}  {uˈ{n-i-$\emptyset$}}   
		\\
		who	cat-{\indf} that cheese-{\defgloss} eat.{\aorperf}-{\pst}-3{\sg}   have-{\thgloss}-3{\sg}  
		\\
		\trans `For a cat who ate the cheese, who has it?'
		\\\armenian{Ո՞վ կատու մը  որ  պանիրը կերաւ ունի։}
		
		\ex \glll (\underline{\textbf{S}})$_\phi$ ({{{O}}} V)$_\phi$  (that {{O}} V)$_\phi$ \\
		\underline{mɑɾˈ\textbf{jɑ-n}} {{ɡɑˈ{du}-mə}}    voɾ {{bɑˈ{ni}ɾ-ə}} {ɡeˈɾ-ɑ-v}   {uˈ{n-i-$\emptyset$}}
		\\
		Maria-{\defgloss} 	cat-{\indf}   that cheese-{\defgloss} eat.{\aorperf}-{\pst}-3{\sg}   have-{\thgloss}-3{\sg} 
		\\
		\trans `For a cat who ate the cheese, \underline{Maria} has it.'
		\\\armenian{Մարիան կատու մը  որ  պանիրը կերաւ ունի։}
	\end{xlist}
\end{exe}


We also find traces of this phonologically-conditioned extraposition with instrumental-marked noun phrases (\ref{ex:inton:extrapose:ins}). Such phrases act as modifiers. Data is limited, but it seems they show similar extraposition patterns. These modifiers extrapose to not break up the prosodic phrasing between the verb and the pre-verbal word. We illustrate with an unaccusative verb `to exist' that must   be phrased with its subject (\ref{ex:inton:extrapose:ins:extrapos}).  Lack of extraposition creates a strong connotation of  topicalization (\ref{ex:inton:extrapose:ins:noextrapose}). 

\begin{exe}
	\ex\label{ex:inton:extrapose:ins}
	\begin{xlist}
		\ex \glll (\textbf{\underline{S}} V)$_\phi$ (Adj \textbf{N-{\ins}})$_\phi$ \\
		ˈ\textbf{mɑɾtʰ}-mə  ˈɡ-ɑ-$\emptyset$ ɡɑˈnɑnt͡ʃ ɑt͡ʃk-eˈ\textbf{ɾ-ov} \\ 
		person-{\indf}  exist-{\thgloss}-3{\sg} green eye-{\pl}-{\ins} \\
		\trans `There's a man with green eyes.' 
		\label{ex:inton:extrapose:ins:extrapos}
		\\ \armenian{Մարդ մը կայ կանանչ աչքերով։}
		\ex \glll (\textbf{\underline{S}})$_\phi$ (Adj \textbf{N-{\ins}})$_\phi$  (\textbf{V})$_\phi$\\
		ˈ\textbf{mɑɾtʰ}-mə  ɡɑˈnɑnt͡ʃ ɑt͡ʃk-eˈ\textbf{ɾ-ov}   ˈ\textbf{ɡ-ɑ-$\emptyset$} \\ 
		person-{\indf} green eye-{\pl}-{\ins}  exist-{\thgloss}-3{\sg} \\
		\trans `A man with green eyes, he exists.' 
		\label{ex:inton:extrapose:ins:noextrapose}
		\\ \armenian{Մարդ մը  կանանչ աչքերով կայ։}
	\end{xlist}
\end{exe}



\textcolor{red}{perhaps do N+postposition + RC գացի մարդուն քովը որ ուրախ էր}

\todo{update with extraposition slide materials. mention verb focus contradiction}
\subsection{Imperatives}\label{section:intonation:other:imperative}

Imperative sentences have relatively simple morphosyntax. A declarative SOV sentence (\ref{example:inton:other:imp:basic:dec}) is changed to an imperative sentence (\ref{example:inton:other:imp:basic:imp}) by using imperative morphology on the verb. The imperative verb attracts the nuclear stress of the sentence away from the object. HD still perceives some level of prominence on the object,  suggesting that the verb forms its own separate prosodic phrase. 

\begin{exe}
	\ex \begin{xlist}
		\ex \glll (\underline{\textbf{O}} V)$_\phi$ \\
		\underline{nɑˈ\textbf{mɑɡ}} ɡə-kʰəˈɾ-e-$\emptyset$ \\
		letter {\ind}-write-{\thgloss}-3{\sg} \\
		\trans `He writes letters.'
		\label{example:inton:other:imp:basic:dec}
		\\ \armenian{Նամակ կը գրէ։}
		\ex \glll ({\textbf{O}})$_\phi$ ({\textbf{\underline{V}}})$_\phi$ !  \\
		{nɑˈ\textbf{mɑɡ}} \underline{kʰəˈ\textbf{ɾ-e-$\emptyset$}} \\
		letter  write-{\thgloss}-2{\sg} \\
		\trans `Write letters!.'
		\label{example:inton:other:imp:basic:imp}
		\\ \armenian{Նամակ  գրէ՛։}
	\end{xlist}
\end{exe}

Acoustically, it seems that the imperative verb tends to have a higher pitch H* in the sentence. The stressed syllable of the verb is likewise significantly longer when it is imperative verb. 

\textcolor{blue}{draw}


The imperative verb is typically sentence-final. When it is sentence-medial, it still attracts stress (\ref{example:inton:other:imp:medial:imp}) . 


\begin{exe}
	\ex\begin{xlist}
		\ex \glll (\underline{\textbf{O}} V)$_\phi$ \\
		\underline{nɑˈ\textbf{mɑ}ɡ-ə} ɡə-kʰəˈɾ-e-$\emptyset$ \\
		letter-{\defgloss} {\ind}-write-{\thgloss}-3{\sg} \\
		\trans `He writes the letter.'
		\\ \armenian{Նամակը կը գրէ։}
		\ex \glll ({\textbf{O}})$_\phi$ ({\textbf{\underline{V}}})$_\phi$ !  \\
		{nɑˈ\textbf{mɑ}ɡ-ə} \underline{kʰəˈ\textbf{ɾ-e-$\emptyset$}} \\
		letter-{\defgloss}  write-{\thgloss}-2{\sg} \\
		\trans `Write the letter!.'
		\\ \armenian{Նամակը  գրէ՛։}
		\ex \glll ({\textbf{\underline{V}}})$_\phi$  ({\textbf{O}})$_\phi$ !  \\
		\underline{kʰəˈ\textbf{ɾ-e-$\emptyset$}} {nɑˈ\textbf{mɑ}ɡ-ə}\\
		write-{\thgloss}-2{\sg} letter-{\defgloss}  \\
		\trans `Write the letter!.'
		\label{example:inton:other:imp:medial:imp}\\ \armenian{Գրէ՛ Նամակը։}
	\end{xlist}
\end{exe}

Acoustically, it seems that the sentence-medial imperative tends to trigger post-focal deaccenting. 

\textcolor{blue}{draw}


More wide-scale acoustic data is needed to check if there any differences between imperative intonation and verb focus. 



\subsection{Vocatives}\label{section:intonation:other:vocative}

A vocative sentence is when  the name of a person is called out in an utterance. Contrast (\ref{example:inton:other:voc:basic:list}) where the name of a person `Mariam' simply said, as if from a list, vs. (\ref{example:inton:other:voc:basic:voc}) where the name is called out in order to catch the attention of the person. 

\begin{exe}
	\ex \begin{xlist}
		\ex \gll mɑɾˈ\textbf{jɑm} \\
		Mariam \\
		\trans `Mariam.' (reading from list)
		\label{example:inton:other:voc:basic:list}
		\\ \armenian{Մարիամ։}
		\ex \gll ˈ\textbf{mɑɾ}jɑm  !\\
		Mariam \\
		\trans `Mariam!' (calling out to a person named Mariam)
		\label{example:inton:other:voc:basic:voc}
		\\ \armenian{Մարիամ։}
		
	\end{xlist}
\end{exe}


In Armenian, there isn't any special vocative morphology. Instead,  vocative names usually get initial stress \citep[133]{Vaux-1998-ArmenianPhono}. Vocative stress can also go on initial schwas (\ref{example:inton:other:voc:schwa:voc}) \citep[220]{Gharagulyan-1974-BookArmenianOrthoepy}. 


\begin{exe}
	\ex \begin{xlist}
		\ex \gll məɡəɾˈ\textbf{dit͡ʃ} \\
		Megerdich \\
		\trans `Megerdich.' (reading from list)
		\\ \armenian{Մկրտիչ։}
		\ex \gll ˈ\textbf{mə}ɡəɾdit͡ʃ  !\\
		Megerdich \\
		\trans `Megerdich!' (calling out to a person named Megerdich)
		\label{example:inton:other:voc:schwa:voc}
		\\ \armenian{Մկրտիչ։}
		
	\end{xlist}
\end{exe}


Initial stress is the norm (\citealt[336,338]{Adjarian-1971-LiakatarPhono}, \citealt[76]{Margaryan-1997-ArmenianPhonology}). But depending on the specific name, there are reports of final stress or word-medial stress, and initial stress. \citet[17]{Johnson-1954-EastArmGrammar} likewise documents some syllable-size restrictions on the positioning of irregular stress in the vocatives of an Eastern Armenian speaker. We suspect that such variation is mostly due to extra-linguistic factors and social conventions. For example, when someone wants to greet a person, and if that name is sentence-final  \ref{example:inton:other:voc:greet}), then   name of the person is often elongated, thus creating final stress. 

\begin{exe}
	\ex \gll pʰɑɾev mɑɾˈ\textbf{jɑm} \\
	hello Mariam \\
	\trans `Hello, Mariam.' 
	\label{example:inton:other:voc:greet}
	\\ \armenian{Բարեւ Մարիամ։}
\end{exe}

This is in contrast to hypocoristics which almost always take initial stress (\S\ref{section:stress:prestress:hypocoristic}). 


Data is too limited to fully understand why there is such reported variation. Data is likewise to limited to know what are the exact acoustic differences between vocative stress vs. other types of focus. 

