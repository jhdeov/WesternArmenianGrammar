
\chapter{Orthography}\label{chapter:ortho}
This chapter goes over the orthographic system of Western Armenian. We focus on describing the basic elements of Armenian orthography. We do not go in depth in explaining the relationship between orthography and phonology in this chapter. We instead refer readers to specialized sections within different phonology chapters.  This is because understanding the linguistic use of various orthographic elements depends on understanding the phonology. More information of Armenian orthography can be found in the references of  more specialized sources \citep{sanjian-1996-armenianAlphabet}. 



As an overview, Section \S\ref{section:ortho:letter} goes over the basic letters of the Armenian alphabet. In Section \S\ref{section:ortho:systems}, we describe the history of the Armenian script and past spelling reforms. 

Section  \S\ref{section:ortho:punctuation} goes over punctuation symbols in Western Armenian. Some symbols are placed at the end of phrases and words, while some symbols are `infixed' or inserted inside of a word. These infixed symbols target the stressed or focused syllable in both declaratives and interrogatives (\S\ref{section:stress:ortho}). 


Section \S\ref{section:ortho:mismatch} goes over some a set of mismatches between the orthography and phonology. Although Armenian orthography is quite close to the surface pronunciation of words, there are mismatches and homophony in the voicing of consonants. These are due to diachronic sound changes. 


\section{Letters}\label{section:ortho:letter}
The Armenian script was invented in around the 5\textsuperscript{th} century by saint Mesrop Mashtots, an Armenian clergyman. The script is written left-to-right. The script originally had only 36 letters, but an additional two letters \armenian{օ}  <ō>,  \armenian{ֆ} <f>  were added in the additional ages. The digraph \armenian{ու} <ow> is often treated as an additional symbol because its primary pronunciation is the vowel /u/.  

The native name of the Armenian script is  in (\ref{example:ortho:name}). Morphologically, the word `alphabet' is a compound of the first two letters of the script (\armenian{ա} /ɑjpʰ/,  \armenian{բ} /pʰen/) that are connected with the conjunction /u/ `and'. 



\begin{exe}
	\ex \gll hɑjeren-i ɑjpʰupʰen-ə
	\\
	Armenian.language-{\gen} alphabet-{\defgloss}
	\\
	\trans	`the Armenian alphabet'
	\label{example:ortho:name}
	\\
	\armenian{հայերէնի այբուբենը}
\end{exe}

Throughout this grammar, we utilize the transliteration scheme in Table \ref{tab:letters}.  Outside of this orthography chapter and of other orthography-based sections, we generally skip providing transliterations and just provide the IPA transcription. 

Note that the transliteration system that we use in this book is not a standardized transliterations. Most existing transliterations of the Armenian script are based on the pronunciation rules of Classical or Eastern Armenian. Thus these transliteration systems are unsuitable to Western Armenian phonemes. As for those systems which were designed for Western Armenian, many of these systems are difficult to use because either a) they don't have a 1-to-1 symbol association, or b) they represent affricates with difficult-to-remember symbols such as <č> for \armenian{չ} [t͡s]. 


\begin{table}[]
	\caption{Letters   of Armenian script}\label{tab:letters}
	{%\resizebox{1\textwidth}{!}{%
		\begin{tabular}{| ll|ll|ll| }
			\hline
			Uppercase & Lowercase & \multicolumn{2}{l|}{Name} & Transliteration  &  Pronunciation
			\\
			\armenian{Ա}  & \armenian{ա}  & \armenian{այբ}  & ɑjpʰ & <a> &/ɑ/  \\
			\armenian{Բ}  & \armenian{բ}  & \armenian{բեն}  & pʰen  & <p> & /pʰ/ \\
			\armenian{Գ}  & \armenian{գ}  & \armenian{գիմ}  & kʰim & <k> & /kʰ/ \\
			\armenian{Դ}  & \armenian{դ}  & \armenian{դա}   & tʰɑ&<t> & /tʰ/  \\
			\armenian{Ե}  & \armenian{ե}  & \armenian{եչ}   & jetʃʰ & <e> & /e/, /je/  \\
			\armenian{Զ}  & \armenian{զ}  & \armenian{զա}   & zɑ & <z> & /z/   \\
			\armenian{Է}  &  \armenian{է}  & \armenian{է}    & e &  <ē>& /e/ \\
			\armenian{Ը}  & \armenian{ը}  & \armenian{ըթ}   & ətʰ & <ə> & /ə/ \\
			\armenian{Թ}  & \armenian{թ}  & \armenian{թօ}   & tʰo & <t'>   & /tʰ/\\
			\armenian{Ժ}  & \armenian{ժ}  & \armenian{ժէ}   & ʒe & <ʒ> & /ʒ/ \\
			\armenian{Ի}  & \armenian{ի}  & \armenian{ինի}  & ini  & <i> & /i/ \\
			\armenian{Լ}  & \armenian{լ}  & \armenian{լիւն} & lʏn   & <l>& /l/\\
			\armenian{Խ}  & \armenian{խ}  & \armenian{խէ}   & χe   & <x>& /χ/ \\
			\armenian{Ծ}  & \armenian{ծ}  & \armenian{ծա}   & d͡zɑ  & <d͡z>& /d͡z/  \\
			\armenian{Կ}& \armenian{կ}& \armenian{կեն}&  ɡen &  <g> &  /ɡ/ \\
			\armenian{Հ}& \armenian{հ}& \armenian{հօ}&  ho   &  <h>&  /h/ \\
			\armenian{Ձ}& \armenian{ձ}& \armenian{ձա}&  t͡sɑ &  <t͡s>&  /t͡sʰ/  \\
			\armenian{Ղ}& \armenian{ղ}& \armenian{ղատ}&  ʁɑd &  <ɣ> &  /ʁ/   \\
			\armenian{Ճ}& \armenian{ճ}& \armenian{ճէ}&  d͡ʒe  &  <d͡ʒ> &  /d͡ʒ/ \\
			\armenian{Մ}& \armenian{մ}& \armenian{մեն}&  men &   <m>  &  /m/  \\
			\armenian{Յ}& \armenian{յ}& \armenian{յի}&  hi   &  <y> &  /j/, /h/, silent \\
			\armenian{Ն}& \armenian{ն}& \armenian{նու}&  nu &  <n> &  /n/  \\
			\armenian{Շ}& \armenian{շ}& \armenian{շա}&  ʃɑ   &  <ʃ> &  /ʃ/ \\
			\armenian{Ո}& \armenian{ո}& \armenian{ո}&  vo   &  <o>&  /o/, /vo/ \\
			\armenian{Չ} & \armenian{չ}  & \armenian{չա} &  t͡ʃɑ &  <t͡ʃ'>  &  /t͡ʃ/ \\
			\armenian{Պ}& \armenian{պ}& \armenian{պէ}&  be   &  <b> &  /b/\\
			\armenian{Ջ}& \armenian{ջ}& \armenian{ջէ}&  t͡ʃe  &  <t͡ʃ> &  /t͡ʃ/ \\
			\armenian{Ռ}& \armenian{ռ}& \armenian{ռա}&  ɾɑ   &  <\.{r}> &  /ɾ/\\
			\armenian{Ս}& \armenian{ս}& \armenian{սէ}&  se   &  <s>&  /s/ \\
			\armenian{Վ}& \armenian{վ}& \armenian{վեւ}&  vev  &  <v>&  /v/ \\
			\armenian{Տ}& \armenian{տ}& \armenian{տիւն}&  dʏn   &  <d> &  /d/\\
			\armenian{Ր}& \armenian{ր}& \armenian{րէ}&  ɾe   &  <r>&  <ɾ> \\
			\armenian{Ց}& \armenian{ց}& \armenian{ցօ}&  t͡so&  <t͡s'> &  /t͡s/  \\
			\armenian{Ւ}& \armenian{ւ}& \armenian{հիւն}&  hʏn  &  <w> &  /v/ \\
			\armenian{Փ}& \armenian{փ}& \armenian{փիւր}&  pʰʏɾ &  <p'>&  /pʰ/ \\
			\armenian{Ք}& \armenian{ք}& \armenian{քէ}&  kʰe  &  <k'> &  /kʰ/\\
			\armenian{Օ}& \armenian{օ}& \armenian{օ}&  o    &  <ō> &  /o/ \\
			\armenian{Ֆ}& \armenian{ֆ}& \armenian{ֆէ}&  fe   &  <f> &  /f/ \\
			\armenian{ՈՒ}& \armenian{ու}& \armenian{ու}&  u &  <ow> &  /u/, /v/
			\\ \hline	\end{tabular}
	}
\end{table}

Armenian has graphemes or letters for every phonemic consonant and vowel. Vowels show some complications however (Table \ref{tab:vowel letter associations}).   For all but the schwa, a pronounced vowel is always written in the orthography in some manner or another.  For the schwa, some instances of a pronounced schwa are written with \armenian{ը} <ə>, while some are not written at all. 



\begin{table}[H]
	\centering
	\caption{Vowel-to-letter associations for non-schwas}
	\label{tab:vowel letter associations}
	{%\resizebox1\textwidth}{!}{%
		\begin{tabular}{|l| ll| llll| }
			\hline     Vowel & \multicolumn{2}{l|}{Letter(s)} & \multicolumn{4}{l|}{Example}  \\
			/ɑ/&\armenian{ա} & <a> & \armenian{\textbf{ա}փ}  & <\textbf{a}p'> & [\textbf{ɑ}pʰ] & `palm'
			\\
			/e/&\armenian{ե} & <e> & \armenian{\textbf{ե}րգ}  & <\textbf{e}rk> & [j\textbf{e}ɾkʰ] & `song'
			\\
			& \armenian{է} & <ē> & \armenian{\textbf{է}շ} & <\textbf{ē}ʃ> & [\textbf{e}ʃ] & `donkey'
			\\
			/o/ & \armenian{ո} & <o> & \armenian{\textbf{ո}չ} & <\textbf{o}t͡ʃ'> & [v\textbf{o}t͡ʃ] &`no'
			\\
			& \armenian{օ} & <ō> & \armenian{\textbf{օ}ձ} & <\textbf{ō}t͡s> & [\textbf{o}t͡s] &`snake'
			\\
			/i/&\armenian{ի} & <i> & \armenian{\textbf{ի}մ} & <\textbf{i}m> & [\textbf{i}m] & `my'
			\\
			/u/&\armenian{ու} & <ow> & \armenian{\textbf{ու}ս}  & <\textbf{ow}s> & [\textbf{u}s] & `shoulder'
			\\
			/ə/ & \armenian{ը} & /ə/ & \armenian{\textbf{ը}ստ} & <\textbf{ə}sd> & [\textbf{ə}st] & `according to' 
			\\
			& $\emptyset$ & & \armenian{դրամ}  & <tram> & [t\textbf{ə}ɾɑm] & `money'
			\\ \hline
		\end{tabular}
	}
\end{table}

The fast majority of words with a pronounced schwa do not represent the schwa in the orthography. We discuss this asymmetry in Section \S\ref{section:segmentalPhono:vowel:schwa:unwritten} within the context of the schwa's phonology.  In brief, when a schwa is unwritten, the orthography reflects the origins of the schwa as being epenthetic, a reduced vowel, or a syncopated vowel. 

\section{Writing system and spelling reforms}\label{section:ortho:systems}


Western Armenian differs from Eastern Armenian because Western Armenian uses a more conservative spelling system called the Classical Orthography, Traditional Orthography, or Mesropian Orthography.  Eastern Armenian instead uses the Reformed Orthography based on Soviet-era spelling reforms. Examples in Table \ref{tab:spelling system} illustrate some of the differences.  The Reformed system removed silent letters in words. Depending on the word, word-medial   /e/ can be written with either grapheme \armenian{է}  or \armenian{ե}; similarly word-medial vowel /o/ can be written with either \armenian{ո} or \armenian{օ}. The Reformed system removed this unpredictability by uniformity using \armenian{ե}, \armenian{ո} for word-medial /e, o/.  

\begin{table}[H]
	\centering
	\caption{Example of differences across spelling systems}
	\label{tab:spelling system}
	{%\resizebox1\textwidth}{!}{%
		\begin{tabular}{|l|lllll| }
			\hline 
			Classical spelling & \armenian{ծառայ} &\armenian{լեռ} & \armenian{տէր}  & \armenian{մոմ} & \armenian{մօտ}
			\\
			Transliteration & <d͡za\.{r}ay> & <le\.{r}> & <dēr> & <mom> & <mōm> 
			\\
			Reformed spelling& \armenian{ծառա} &\armenian{լեռ} & \armenian{տեր}  & \armenian{մոմ} &\armenian{մոտ}
			\\
			Transliteration& <d͡za\.{r}a> & <le\.{r}> & <der> & <mom> & <mom> 
			\\
			Pronunciation& [d͡zɑrɑ] & [ler] & [deɾ] & [mom] & [mom] 
			\\ \hline 
		\end{tabular}
	}
\end{table}


The tradition spelling system included many more types of unpredictability between the orthography and pronunciation. We don't survey these unpredictabilities because they have limited connection to the synchronic phonology of Armenian. But they are useful for learners of the Armenian script. These unpredictabilities  are amply documented in various teaching grammars of Armenian (\textcolor{red}{bucket}).

The above unpredictability are ultimately due to various sound changes from Classical Armenian to modern Western Armenian. For example, word-final glides in Classical Armenian were lost in polysyllabic words (\textcolor{red}{cite macak?}). This loss created the silent letter \armenian{յ} as in the word <d͡za\.{r}ay> [d͡zɑɾɑ] `servant' above. For midvowels, the graphemes \armenian{է} <ē> and \armenian{ե}  <e> were originally pronounced as different midvowels in Classical Armenian (\textcolor{red}{cite macak?}). The  <ē> form may have been a tense or long version /eː/, while  <e> was a plain vowel /e/.  Eventually, the two types of midvowels merged into just /e/, thus creating unpredictable spellings in Modern Armenian. 




The spelling reform removed essentially all types of unpredictability, surveyed in \citet[12ff]{DumTragut-2009-ArmenianReferenceGrammar}. Because Western Armenian is spoken in the Armenian diaspora, Western Armenian publications and literature never adopted the soviet spelling reforms. 

\textcolor{red}{digraphs? might as well}


\section{Punctuation symbols}\label{section:ortho:punctuation}
Armenian utilizes a small set of punctuation symbols. One set of symbols is placed at the end of phrases and clauses. One set is placed before or between words . And one set is  placed inside words.  

This chapter provides a simple overview of the types of symbols. The use of these symbols do not significantly differ from Eastern Armenian.  For more in-depth discussion of Armenian punctuation and their orthographic rules, see \citet[ch5]{DumTragut-2009-ArmenianReferenceGrammar}

For the word-final symbols in Table \ref{tab:punctuation final}, these symbols are used for ending clauses and sentences. Their uses are largely the same in Armenian as they are in other European languages. 



\begin{table}[H]
	\centering
	\caption{Phrase-final or clause-final punctuation symbols}
	\label{tab:punctuation final}
	\begin{tabular}{|l|ll| l| }
		\hline
		Symbol & \multicolumn{2}{l|}{Name}& English analog     
		\\
		\hline		\armenian{,}  & \armenian{ստորակէտ} & əstoɾɑɡed & comma
		\\
		\armenian{՝} & \armenian{բութ}  & pʰutʰ &   semicolon
		\\
		
		.  & \armenian{միջակէտ} & mit͡ʃɑɡed & colon
		\\
		\armenian{։}   & \armenian{վերջակէտ} & veɾt͡ʃɑɡed & period
		\\ 		\hline
		
	\end{tabular}
\end{table}



Table \ref{tab:punctuation around} shows the set of punctuation symbols that are placed at the edges of words. These include the Armenian analog of apostrophes and brackets.

\begin{table}[H]
	\centering
	\caption{Punctuation that is placed around words or at edges}
	\label{tab:punctuation around}
	\begin{tabular}{|l|ll| l|}
		\hline	Symbol & \multicolumn{2}{l|}{Name}& English analog     
		\\		\hline
		\armenian{«  »} & \armenian{չակերտ} & t͡ʃɑɡeɾd & brackets
		\\
		\armenian{՚ } & \armenian{ապաթարց}  & ɑbɑtʰɑɾt͡s &  apostrophe
		\\
		\hline
	\end{tabular}
\end{table}

Finally, Armenian has punctuation symbols that are infixed to placed inside the word (Table \ref{tab:punctuation infix}). These symbols are placed on the stressed vowel of the word which has the strongest prominence in the sentence. In other words, these markers are placed on the syllable that carries nuclear stress or sentential stress.  See also Section \S\ref{section:stress:ortho} for discussion on stress and orthography. 

\begin{table}[H]
	\centering
	\caption{Punctuation symbols that are infixed inside the word}
	\label{tab:punctuation infix}
	\begin{tabular}{|l|ll| l| }
		\hline	Symbol & \multicolumn{2}{l|}{Name}& English analog     
		\\ 		\hline
		\armenian{՞}    & \armenian{պարոյկ} &  bɑɾʏg &question mark
		\\
		\armenian{՜}       & \armenian{երկար}  & jeɾɡɑɾ & exclamation mark
		\\
		\armenian{՛} & \armenian{շեշտ} & ʃeʃt & emphasis mark    \\
		\hline
	\end{tabular}
\end{table}

The stressed vowel is typically the final non-schwa vowel (\ref{example: question mark final}). But the symbol can be placed further inward if the word has irregular non-final stress (\ref{example: question mark irregular}). The examples below illustrate with the question mark symbol \armenian{՞}, which we transliterate as \textsuperscript{?}. The marker is placed on the word which carries the nuclear stress of the sentence.  We underline this syllable.

\begin{exe}
	\ex \begin{xlist} 
		\ex \gll mɑɾjɑ-n uˈ\underline{ɾɑχ} e 
		\\
		Maria-{\defgloss} happy is
		\\
		\trans	`Is Maria happy?' \label{example: question mark final}
		\\
		Orthography: \armenian{Մարիան ուրա՞խ է։}
		\\
		Transliteration: <Marian owɾa\textsuperscript{?}x ē>
		\ex \gll ˈ\underline{voɾ}kʰɑn g-uz-e-s 
		\\
		how.much {\ind}-want-{\thgloss}-2{\sg}
		\\
		\trans	`How much do you want? \label{example: question mark irregular}
		\\
		Orthography: \armenian{Ո՞րքան կ՚ուզես։}
		\\
		Transliteration: <O\textsuperscript{?}rk'an g'owzes> 
	\end{xlist}
\end{exe}


\textcolor{red}{idk if its worth going through examples of each marker}



\section{Orthography-phonology mismatches}\label{section:ortho:mismatch}

Armenian orthography does not exactly match the surface pronunciations of words, but it is fairly close. As said in Section \S\ref{section:ortho:systems}, the traditional spelling system creates various types of homophony and unpredictability. This section overviews a significant areas of mismatch between orthography and pronunciation. Some of these mismatches are only present in Eastern Armenian, while some are present in both. 

Unless otherwise specified, the Eastern examples are from English Wiktionary. The Wiktionary examples   are heavily moderated and are reliable for Eastern Armenian. For transliterations,  we adopt ISO 9985 to transliterate the words for Eastern Armenian,\footnote{\url{https://www.translitteration.com/transliteration/en/armenian-eastern-classical/iso-9985/}} while our own transliteration for Western Armenian.
\subsection{Homophony in voiceless letters}\label{section:ortho:mismatch:homophony}

In Western Armenian, stops and affricates have a 2-way laryngeal contrast. Stops and affricates are phonologically either voiced or voiceless.\footnote{Acoustically, the actual correlates of voice vary by geographic region. This is overviewed in Section \S\ref{section:segmentalPhono:cons:stop}. } However the orthography displays a 3-way contrast between the graphemes for stops and affricates. Each voiced sound has one corresponding voiced grapheme, but each voiceless sound   has   two homophones letters. Table \ref{tab:homophony stops transltieration} illustrates for all stops and affricates.  


\begin{table}[H]
	\centering
	\caption{Homophonous letters for stops and affricates}
	\label{tab:homophony stops transltieration}
	\begin{tabular}{|l|l|l|llll| }
		\hline 
		Letter & Trans. & Pron. & \multicolumn{4}{l|}{Example } \\
		
		\hline 
		\armenian{բ}& <p>& /pʰ/ & \armenian{բառ} &  <pa\.{r}> & [pʰɑɾ] & `word' 
		\\
		
		\armenian{փ}& <p'>& /pʰ/ & \armenian{փառք}& <p'a\.{r}k'> & [pʰɑɾkʰ] & `glory' 
		\\
		\armenian{պ}& <b>& /b/ & \armenian{պար}& <bar> & [bɑɾ] & `dance' 
		
		\\
		\hline 
		\armenian{դ}& <t>& /tʰ/ & \armenian{դեր} &  <ter> & [tʰeɾ] & `role' 
		\\
		
		\armenian{թ}& <t'>& /tʰ/ & \armenian{թեւ}& <t'ew> & [tʰev] & `wing' 
		\\
		\armenian{տ}& <d>& /d/ & \armenian{տեղ}& <deɣ> & [deʁ] & `place' 
		
		\\
		\hline 
		\armenian{գ}& <k>& /kʰ/ & \armenian{գամ} &  <kam> & [kʰɑm] & `nail' 
		\\
		
		\armenian{ք}& <k'>& /kʰ/ & \armenian{քար}& <k'ar> & [kʰɑɾ] & `rock' 
		\\
		\armenian{կ}& <g>& /ɡ/ & \armenian{կար}& <gar> & [ɡɑɾ] & `string' 
		
		\\
		\hline 
		\armenian{ձ}& <t͡s>& /t͡s/ & \armenian{ձագ} &  <t͡sak> & [t͡sɑkʰ] & `cub' 
		\\
		
		\armenian{ց}& <t͡s'>& /t͡s/ & \armenian{ցաւ}& <t͡s'aw> & [t͡sɑv] & `pain' 
		\\
		\armenian{ծ}& <d͡z>& /d͡z/ & \armenian{ծակ}& <d͡zaɡ> & [d͡zɑɡ] & `hole' 
		
		\\
		\hline 
		
		\armenian{ջ}& <t͡ʃ>& /t͡ʃ/ & \armenian{ջուր} &  <t͡ʃowr> & [t͡ʃuɾ] & `water' 
		\\
		
		\armenian{չ}& <t͡ʃ'>& /t͡ʃ/ & \armenian{չու}& <t͡ʃ'ow> & [t͡ʃu] & `flight' 
		\\
		\armenian{ճ}& <d͡ʒ>& /d͡ʒ/ & \armenian{ճուտ}& <d͡ʒowd> & [d͡ʒud] & `chick'
		
		\\
		\hline 
		
	\end{tabular}
\end{table}.

For a given voiceless sound like [pʰ], there are two homophonous graphemes \armenian{բ}, \armenian{փ} <p, p'>. For a speaker of Western Armenian, the choice of grapheme is unpredictable and has no phonological correlation.  The homophony is a cause for common spelling errors for Western Armenian.

The voiceless homophony is because in Classical Armenian, the different graphemes did reflect different voicing quality (Table \ref{tab:ortho:stops}). In  modern Western Armenian stops/affricates show a two-way contrast between voiced and voiceless. But Classical  stops/affricates had a 3-way contrast between voiced, voiceless unaspirated, and voiceless aspirated.  The 3-way contrast from Classical Armenian survived into Eastern Armenian, but not Western.  The examples below illustrate  with the labial stops. 


\begin{table}[H]
	\centering
	\caption{Labial stops in  Western Armenian (WA)  vs. Eastern Armenian (EA)  and Classical Armenian (CA)}
	\label{tab:ortho:stops}
	\begin{tabular}{|l| l l| lll|  }
		\hline 
		
		Letter & \multicolumn{2}{l|}{Example} & \multicolumn{2}{l}{Pronunciation} & 
		\\
		& & & WA & EA/CA & 
		\\
		\hline 
		\armenian{բ}&& <p>& /pʰ/ &/b/ &
		\\
		& \armenian{բառ} &  <pa\.{r}>  & [pʰɑɾ] & [bɑr] & `word'
		\\
		
		\armenian{փ}&& <p'>& /pʰ/ &/pʰ/ &
		\\
		& \armenian{փառք}& <p'a\.{r}k'> & `glory' & [pʰɑɾkʰ]  & [pʰɑrkʰ] 
		\\
		\armenian{պ}& &<b>& /b/ & /p/ &\\
		&\armenian{պար}& <bar> & `dance' & [bɑɾ] & [pɑɾ] 
		\\ \hline 
	\end{tabular}
	
\end{table} 

When the 3-way contrast changed into a 2-way contrast for Western Armenian, this created homophony for the voiceless letters.  In terms of the exact sound change from Classical to Western, the Classical voiceless unaspirates stayed voiceless unaspirated: \armenian{փ} /pʰ/ $\rightarrow$  /pʰ/. But the voiced and unaspirated sounds switched voiced. The Classical voiceless unaspirates became voiced: \armenian{պ} /p/ $\rightarrow$ /b/. And the Classical voiced became voiceless unaspirated: \armenian{բ} /b/ $\rightarrow$ /pʰ/. This diachronic process is quite complicated and there has bee a lot of work in diachronic phonology to  explain such a change \textcolor{red}{CITE}p{Baronian and other folks}. 

\subsection{Mismatches in voicing of clusters}\label{section:ortho:mismatch:clusters}

The second area of mismatches comes from voicing in clusters. In some mono-morphemic words, the orthography shows consonants clusters with  non-agreeing voice. In pronunciation though, these clusters are pronounced with identical voicing. Table \ref{tab:ortho cluster voice} illustrates with clusters of a sibilant and an orthographic voiced stop, and clusters of a voiced fricative and voiceless stop.\footnote{Note that fricative-stop clusters cause de-aspiration of the stop. See Section \S\ref{section:segmentalPhono:allphonLaryng:postFricDeasp}.}

\begin{table}[H]
	\centering
	\caption{Orthographic mismatches in voicing of consonant clusters}
	\label{tab:ortho cluster voice}
	\begin{tabular}{|llll| }
		\hline 		\armenian{սպասել} & 
		<\textbf{sb}asel> & [ə\textbf{sp}ɑsel] & `to wait'
		\\
		\armenian{հաստ} 
		&<ha\textbf{sd}> & [hɑ\textbf{st}] & `thick'
		\\
		\armenian{սկիզբ} & 
		<\textbf{sɡ}izb> &  [ə\textbf{sk}isp] & `beginning'
		\\
		\hline 
		\armenian{զբօսանք} 
		& <\textbf{zp}ōsank'> & 
		[ə\textbf{sp}osɑnkʰ] & `recess'
		\\
		\armenian{աղբ} & 
		<a\textbf{ɣp}> & [ɑ\textbf{χp}] & `trash'
		\\       
		\armenian{խեղդել} & 
		<xe\textbf{ɣt}el> & [χe\textbf{χt}el] & `to strangle'
		\\
		\armenian{յաղթել} 
		& <ya\textbf{ɣt'}el> & [hɑ\textbf{χt}el] & `to win'
		\\
		\armenian{ազգ} 
		& <a\textbf{zk}> & [ɑ\textbf{sk}] & `nation'
		\\
		\armenian{աղքատ}
		& <a\textbf{ɣk'}ad> & [ɑ\textbf{χk}ɑd] & `poor'
		\\
		\hline 
	\end{tabular}
\end{table} 

For the clusters above, there is no synchronic evidence that the cluster is composed of consonants with different voicing. That is, for a word like \armenian{ազգ} <azk> [ɑsk] `nation', there is no synchronic evidence that the fricative [s] is derived from an underlying /z/. In all the above words, the cluster is part of a single morpheme, and the consonant never alternates   in voicing. That is, for a word like [ɑsk], the fricative is never pronounced as [z] in any morphologically-related word. 

Within Armenian philology, there is a lot of work in cataloging words with such mismatches between the spelling and pronunciation, mostly for Eastern Armenian (\citealt[242-4]{Adjarian-1971-LiakatarPhono}; \citealt[29]{Minassian-1980-EastArmenianGrammar};  \citealt[62,74]{Soukyasyan-2004-ArmenianPhonology}; \citealt[21]{Avetisyan-2007-ComparativeGrammar};  \citealt[30]{Avetisyan-2011-ComparativePhonoEastWest}). See \citet[59]{Hovhannisyan-2014-ArmenianSyllable} and \citet[185]{Gharagulyan-1974-BookArmenianOrthoepy} for a summary and systematic catalog.  Teaching grammars likewise provide pedagogical tips on how to spell these clusters (\citealt[75]{Ezekyan-2007-Armenian}; \citealt[92]{Sevak-2009-Coursebook}). But synchronically though, these clusters are just residues of sound changes from Classical to modern Armenian. They do not reflect modern Western morphology or phonology. 

For example, for  \armenian{ազգ} <azk> [ɑsk], the orthography reflects the fact this word is a reflex of Classical [ɑzɡ] where the cluster was voiced. The orthography matches the classical pronunciation. Eventual sound changes caused the grapheme \armenian{գ} /k/ to switch from being a voiced segment /ɡ/ in Classical to a voiceless /kʰ/ in Western. Once this change occurred, adjacent fricatives had to assimilate in voicing: CA [ɑzɡ] $\rightarrow$ [ɑsk], not [*ɑzk]. 

\subsection{Post-rhotic devoicing}\label{section:ortho:mismatch:rhoticdevoicing}
In Eastern Armenian, there are words which are orthographically written with a final voiced stops and affricates but which are pronounced with a voiceless aspirated form (Table \ref{tab:eastern rhotic devoicing}). This orthography mismatch is especially common after the rhotic /ɾ/.  

\textcolor{red}{cite}

\begin{table}[H]
	\centering
	\caption{Eastern Armenian words with orthography-phonology mismatch for voicing after rhotics}   \label{tab:eastern rhotic devoicing}
	
	
	\begin{tabular}{|l|ll| ll| l| }
		\hline Spelling & \multicolumn{2}{l|}{Transliteration} & \multicolumn{2}{l|}{Pronunciation} & Meaning  \\
		& EA & WA & EA & WA & 
		\\
		\hline       \armenian{նուրբ} & <now\textbf{rb}>   &  <nowrp>
		& [ˈnu\textbf{ɾpʰ}] & [ˈnuɾpʰ] & `gentle'  \\
		\armenian{բարդ} & <ba\textbf{rd}>  &  <part>
		& [ˈbɑ\textbf{ɾtʰ}] & [ˈpʰɑɾtʰ] & `complex'  \\
		\armenian{երգ} & <e\textbf{rg}>  &  <erk>
		& [ˈje\textbf{ɾkʰ}] & [ˈjeɾkʰ] & `song'  \\
		\armenian{բարձ} & <ba\textbf{rj}>  &  <part͡s>
		& [ˈbɑ\textbf{ɾt͡sʰ}] & [ˈpʰɑɾt͡s] & `pillow'  \\
		\armenian{վերջ} & <ve\textbf{rǰ}>&    <vert͡ʃ>
		& [ˈve\textbf{ɾt͡ʃʰ}] & [ˈveɾt͡ʃ] & `end' 	 \\
		
		\hline   
	\end{tabular}
	
\end{table}

Because of how frequent this mismatch some, many philologists and phonologists have argued that Eastern Armenian has an allophonic rule of changing devoicing final voiced stops and affricates after rhotics. 

But this rule is not true allophony in Eastern Armenian (Table \ref{tab:eastern rhotic devoicing no}). One can find words   where devoicing does not apply. There are likewise no cases of this rule applying in derived environments.  The most likely scenario is that this orthography-phonology mismatch in Eastern Armenian is just a diachronic change and not a synchronically active rule.

\begin{table}[H]
	\centering
	\caption{Eastern Armenian words where final voiced stops or affricates surface after rhotics}\label{tab:eastern rhotic devoicing no}
	
	\begin{tabular}{|l|ll| ll| l| }
		\hline Spelling & \multicolumn{2}{l|}{Transliteration} & \multicolumn{2}{l|}{Pronunciation} & Meaning  \\
		& EA & WA & EA & WA & 
		\\
		\hline       \armenian{բորբ} &
		<bo\textbf{rb}>   &  <porp>
		& [ˈbo\textbf{ɾb}] & [ˈboɾpʰ] & `bright'  \\
		\armenian{թակարդ} &
		<t’aka\textbf{rd}>  &  <t'agart>
		& [tʰɑˈkɑ\textbf{ɾd}] & [tʰɑˈɡɑɾtʰ] & `trap'  \\
		\armenian{գորգ} 
		& <go\textbf{rg}>  &  <kork>
		& [ˈɡo\textbf{ɾɡ}] & [ˈkʰoɾkʰ] & `carpet'  \\
		\armenian{մերձ} &
		<me\textbf{rj}>  &  <mert͡s>
		& [ˈme\textbf{ɾd͡z}] & [ˈmeɾt͡s] & `near'  \\
		\armenian{կամուրջ} &
		<kamow\textbf{rǰ}>&    <gamowrt͡ʃ>
		& [kɑˈmu\textbf{ɾd͡ʒ}] & [ɡɑˈmuɾt͡ʃ] & `bridge' 	 \\
		
		\hline   
	\end{tabular} 
	
\end{table}

It is possible that this this rule of devoicing stops/affricates after /ɾ/ is a diachronic sound change that is in progress in Eastern Armenian (Table \ref{tab:eastern rhotic devoicing maybe}) \citep{Asatryan-1976-EasternVoicedPlosivePronunciation,Avetisyan-2005-PronunciationChangesArmenianOrthoepy}. For example, Vahagn Petrosyan informs us that some words are prescriptively pronounced with a final voiced stop, but  this voiced sound is often colloquially devoiced.   

\begin{table}[H]
	\centering
	\caption{Eastern Armenian words where final voiced stops or affricates are variably devoiced after rhotics}\label{tab:eastern rhotic devoicing maybe}
	
	\begin{tabular}{|l|ll| lll| l| }
		\hline Spelling & \multicolumn{2}{l|}{Transliteration} & \multicolumn{3}{l|}{Pronunciation} & Meaning  \\
		& EA & WA & Std. EA& Coll. EA & WA & 
		\\
		\hline   \armenian{կախարդ} &
		<kaɣa\textbf{rd}>  &  <gaɣart>
		& [kɑˈʁɑ\textbf{ɾd}]& [kɑˈʁɑ\textbf{ɾt}ʰ] & [ɡɑˈʁɑɾtʰ] & `witch'  \\
		\armenian{լուրջ} &
		<k’ow\textbf{rǰ}>&    <k'owrt͡ʃ>
		& [ˈkʰu\textbf{ɾd͡ʒ}] & [ˈkʰu\textbf{ɾt͡ʃʰ}] & [kʰluɾt͡ʃ] & `rag' 	 \\
		
		\hline   
	\end{tabular} 
	
\end{table}

\textcolor{red}{read the above EA references again for data for other stuff}

No such issue arises in Western Armenian (Table \ref{tab:wester rhotic devoicing no}). Word-final voiced stops and affricates exist after rhotics. 

\begin{table}[H]
	\centering
	\caption{No post-rhotic devoicing in Western Armenian   }\label{tab:wester rhotic devoicing no}
	
	\begin{tabular}{|l ll   l| }
		\hline Spelling &  Transliteration (WA)   & Pronunciation (WA) & Meaning  \\
		
		\hline       \armenian{կերպ} &
		<ge\textbf{rb}>
		& [ˈɡe\textbf{ɾb}] & `form'  \\
		\armenian{աւարտ} &
		<awa\textbf{rd}>
		& [ɑˈvɑ\textbf{ɾd}] & `end'  \\
		\armenian{մերկ} 
		&  <me\textbf{rg}>
		& [ˈme\textbf{ɾɡ}] & `naked'  \\
		\armenian{գործ} &
		<ko\textbf{rd͡z}>
		& [ˈkʰo\textbf{ɾd͡z}] & `work'  \\
		\armenian{սուրճ} &
		<sow\textbf{rd͡ʒ}>
		& [ˈsu\textbf{ɾd͡ʒ}] & `coffee' 	 \\
		
		\hline   
	\end{tabular} 
	
	
	
	
	
\end{table}

\subsection{Deletion of <h> after rhotics}\label{section:ortho:mismatch:hdeletion}
In modern Armenian, the orthography has a letter \armenian{հ} for the sound /h/.  But this sound is   deleted in some words. This deletion is obligatory and likely due to a diachronic rule. 


There are some roots which are orthographically spelled with a   rhotic + <h>.  Some of these roots pronounce the <h>, and some don't. Table  \ref{tab:rhotic h dropping dont pronounce}  lists such roots where the <h> is not pronounced. For these roots, this <h>  was likely pronounced in earlier stages of the language.  The sound /h/ is absent both when the word is said in isolation and when suffixes are added. 

\begin{table}[H]
	\centering
	\caption{Words where the letter <h> is not pronounced after rhotics}
	\label{tab:rhotic h dropping dont pronounce}
	\begin{tabular}{| lllll| }
		\hline
		Root  & \armenian{աշխարհ} 
		& <aʃxa\textbf{rh}> & ɑʃˈχɑ\textbf{ɾ} & `world'  \\
		$\rightarrow$  & \armenian{աշխարհային} 
		& <aʃxa\textbf{rh}ayin> & ɑʃχɑ\textbf{ɾ}-ɑˈjin & `worldly'   \\ 
		$\rightarrow$  & \armenian{աշխարհի} 
		& <aʃxa\textbf{rh}ayin> & ɑʃχɑˈ\textbf{ɾ}-i & `world-{\gen}'  \\ \hline
		Root  & \armenian{խոնարհ} 
		& <xona\textbf{rh}> & χoˈnɑ\textbf{ɾ} & `humble'  \\
		$\rightarrow$  & \armenian{խոնարհում} 
		& <xona\textbf{rh}owm> & χonɑˈ\textbf{ɾ}-um & `conjugation'   \\ 
		$\rightarrow$  & \armenian{խոնարհէ} 
		& <xona\textbf{rh}ē> & χonɑˈ\textbf{ɾ}-e & `humble-{\abl}'  \\ \hline
		Root  & \armenian{ճանապարհ} 
		& <d͡ʒanaba\textbf{rh}> & d͡ʒɑnɑˈbɑ\textbf{ɾ} & `road'  \\
		$\rightarrow$  & \armenian{ճանապարհորդ} 
		& <d͡ʒanaba\textbf{rh}ort> & d͡ʒɑnɑbɑˈ\textbf{ɾ}-oɾtʰ & `traveller'   \\ 
		$\rightarrow$  & \armenian{ճանապարհներ} 
		& <d͡ʒanaba\textbf{rh}ner> & d͡ʒɑnɑbɑ\textbf{ɾ}-ˈner & `road-{\pl}'  \\ \hline
		Root  & \armenian{շնորհ} 
		& <ʃno\textbf{rh}> & ʃəˈno\textbf{ɾ} & `grace'  \\
		$\rightarrow$  & \armenian{շնորհակալ} 
		& <ʃno\textbf{rh}owm> & ʃəno\textbf{ɾ}-ɑˈɡɑl & `thankful'   \\ 
		$\rightarrow$  & \armenian{շնորհով} 
		& <ʃno\textbf{rh}ov> & ʃənoˈ\textbf{ɾ}-ov & `grace-{\ins}'  \\ \hline
		Root  & \armenian{խորհուրդ} 
		& <xo\textbf{rh}owrt> & χoˈ\textbf{ɾ}uɾtʰ & `advice'  \\
		$\rightarrow$  & \armenian{խորհրդաւոր} 
		& <xo\textbf{rh}rtawor> & χo\textbf{ɾ}əɾtʰ-ɑˈvoɾ & `wise'   \\ 
		$\rightarrow$  & \armenian{խորհուրդներ} 
		& <xo\textbf{rh}owrtner> & χo\textbf{ɾ}uɾtʰ-ˈneɾ & `advice-{pl}'  \\ 
		\hline
	\end{tabular}
\end{table}

For the above words, the absence of a pronounced /h/ is not a phonological rule but an orthography-phonology mismatch. Such a rule of deleting an /h/ after a rhotic must have been a diachronic rule, and it is not a synchronic rule. Evidence for this is that there are roots where the <rh> sequence is pronounced as /ɾh/ (Table \ref{tab:rhotic h dropping do pronounce}). 

\begin{table}[H]
	\centering
	\caption{Words where the letter <h> is   pronounced after rhotics}
	\label{tab:rhotic h dropping do pronounce}
	\begin{tabular}{| llll| }
		\hline
		\armenian{ժպիրհ} & <ʒbi\textbf{rh}>  & ʒəˈbi\textbf{ɾh} & `insolent'
		\\
		\armenian{նիրհ} 
		& <ni\textbf{rh}> &  ˈni\textbf{ɾh} & `light slumber'
		\\
		\armenian{արհեստ} & <a\textbf{rh}esd> & ɑ\textbf{ɾˈh}est &`handicraft'
		\\
		\armenian{արհամարհ} & <a\textbf{rh}amarh> & ɑ\textbf{ɾh}ɑˈmɑɾ & `despicable'
		\\
		\armenian{զարհուր} & <za\textbf{rh}ur> & zɑ\textbf{ɾˈh}uɾ & `terrifying'
		\\ \hline
	\end{tabular}
\end{table}

There is likewise the bound root \armenian{օրհն} <ōrhn> where the <h> is typically pronounced as a /tʰ/ in all its derivatives (Table \ref{tab:rhotic h dropping orhnel}). 


\begin{table}[H]
	\centering
	\caption{Words where the letter <h> is   pronounced as [tʰ]}
	\label{tab:rhotic h dropping orhnel}
	\begin{tabular}{| lllll| }
		\hline
		\armenian{օրհնել} & <ō\textbf{rh}nel>
		& o\textbf{ɾtʰ}ˈn-e-l & `to bless' & $\sqrt{}$-{\thgloss}-{\infgloss}
		\\
		\armenian{օրհնեալ} & <ō\textbf{rh}neal>
		& o\textbf{ɾtʰ}ˈn-jɑl & `blessed' & $\sqrt{}$-{\adjz}
		\\
		\armenian{օրհնութիւն} & <ō\textbf{rh}nut'iwn>
		& o\textbf{ɾtʰ}n-uˈtʰjʏn & `blessing' & $\sqrt{}$-{\nmlz}
		\\
		
		\hline
	\end{tabular}
\end{table}

In sum, the above data is just an orthography-phonology mismatch, and is not a synchronic phonological rule of /h/ deletion. 
