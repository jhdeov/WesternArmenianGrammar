\chapter{Segmental phonology}\label{chapter:segmentalPhono}
This chapter goes over the segmental phonology of Western Armenian (henceforth Armenian or WA). We focus on providing the basic phoneme inventory of Armenian. 


We document the set of attested allophonic processes in Armenian. For consonants, these processes involve changes in the laryngeal or voicing quality of obstruents, i.e., voicing assimilation. For vowels, there is little known about any allophonic alternations. There are some reports of vowel rounding for the underlying sequence /ju/. We briefly list segmental processes that have been reported in previous grammars but which seem to be either unsystematic or obsolete in modern speech. 
.
This section focuses more so on the phonology of segments, and not on their phonetics or acoustics. For an overview of segmental phonetics, see \citet{Seyfarth-JIPAArmenian}


For ease of reference, Figures \ref{tab:cons inventory} and \ref{fig:vowel inventory} provide the consonant inventory and vowel inventory. 

\begin{figure}[H]
	\centering 
	\caption{Consonant inventory}\label{tab:cons inventory}
	\resizebox{1\textwidth}{!}{%
  \begin{tabular}{| l| lllllllll| }
  	\hline 
  	& Bilabial & Labio & Dental & Alveolar & Post- & Palatal & Velar & Uvular & Glottal \\
  	& & -dental & & & alveolar & & & & \\
  	\hline 
  	Plosive & pʰ b & & tʰ d & & & & kʰ ɡ & & (ʔ) \\
  	Affricate & & & t͡s d͡z & &t͡ʃ d͡ʒ & & & & \\
  	Nasal & m & & & n & & & (ŋ) & & \\
  	Tap & & & & & & & & & \\
  	Fricative & & f v & & s z & ʃ ʒ & & & χ ʁ & h \\
  	Rhotic & & & & ɾ & & & & & \\
  	Lateral & & & & l & & & & & \\
  	Glide &. (w) & & & & & j & & &
  	\\ \hline
  \end{tabular}
	}
\end{figure}

The consonant inventory includes [ŋ] in parenthesis. This sound is not a contrastive phoneme in Armenian; it is derived from /n/ when /n/ precedes a velar stop (\S\ref{section:segmentalPhono:nasalPlace}). The glide [w] is in parenthesis because it is restricted to non-nativized loanwords (\S\ref{section:segmentalPhono:cons:glide}). The glottal stop [ʔ] is epenthesized in vowel hiatus in some morphological constructions (\S\ref{section:syllable:VowelHiatus}). 


For vowels, we include the sound /ʏ/ even though this sound is often interchangeable with the sound sequence /uj/. We include the schwa /ə/. For the mid vowels, many past phonological studies of Armenian treated these vowels as lax /ɛ, ɔ/. We treat them as tense /e, o/. We discuss this difference in Section \S\ref{section:segmentalPhono:vowel:canonical}. We include /œ/ as a marginal phoneme.

\begin{figure}[H]
	\centering
	\caption{Vowel inventory}
	\label{fig:vowel inventory}
	\begin{tikzpicture}[scale=2]
  \large
  \tikzset{
  	vowel/.style={fill=white, anchor=mid, text depth=0ex, text height=1ex},
  	dot/.style={circle,fill=black,minimum size=0.4ex,inner sep=0pt,outer sep=-1pt},
  }
  \coordinate (hf) at (0,2); % high front
  \coordinate (hb) at (2,2); % high back
  \coordinate (lf) at (1,0); % low front
  \coordinate (lb) at (2,0); % low back
  \def\V(#1,#2){barycentric cs:hf={(3-#1)*(2-#2)},hb={(3-#1)*#2},lf={#1*(2-#2)},lb={#1*#2}}
  
  % Draw the horizontal lines first.
  \draw (\V(0,0)) -- (\V(0,2));
  \draw (\V(1,0)) -- (\V(1,2));
  \draw (\V(2,0)) -- (\V(2,2));
  \draw (\V(3,0)) -- (\V(3,2));
  
  % Place all the unrounded-rounded pairs next, on top of the horizontal lines.
  \path (\V(0,0)) node[vowel, left] {i} node[dot] {};
  % \path (\V(0,1)) node[vowel, left] {ɨ} node[vowel, right] {ʉ} node[dot] {};
  \path (\V(0,2)) node[vowel, right] {u} node[dot] {};
  \path (\V(0.5,0.4)) node[vowel, right] {ʏ} node[dot] {};
  % \path (\V(0.5,1.6)) node[vowel, left] { } node[vowel, right] {ʊ} node[dot] {};
  \path (\V(1,0)) node[vowel, left] {e}   node[dot] {};
  %\path (\V(1,0)) node[vowel, left] {e} node[vowel, right] {ø} node[dot] {};
  % \path (\V(1,1)) node[vowel, left] {ɘ} node[vowel, right] {ɵ} node[dot] {};
  \path (\V(1,2)) node[vowel, right] {o} node[dot] {};
  \path (\V(2,0))   node[vowel, right] {œ} node[dot] {};
  % \path (\V(2,0)) node[vowel, left] {ɛ} node[vowel, right] {œ} node[dot] {};
  % \path (\V(2,1)) node[vowel, left] {ɜ} node[vowel, right] {ɞ} node[dot] {};
  % \path (\V(2,2)) node[vowel, left] {ʌ} node[vowel, right] {ɔ} node[dot] {};
  % \path (\V(2.5,0)) node[vowel, left] {æ} node[vowel, right] { } node[ ] {};
  % \path (\V(3,0)) node[vowel, left] {a} node[vowel, right] {ɶ} node[dot] {};
  \path (\V(3,2)) node[vowel, left] {ɑ} node[dot] {};
  
  % Draw the vertical lines.
  \draw (\V(0,0)) -- (\V(3,0));
  \draw (\V(0,1)) -- (\V(3,1));
  \draw (\V(0,2)) -- (\V(3,2));
  
  % Place the unpaired symbols last, on top of the vertical lines.
  \path (\V(1.5,1)) node[vowel] {ə};
  % \path (\V(2.5,1)) node[vowel] {ɐ};
	\end{tikzpicture}
\end{figure}

We go over each type of segment below. 

\textcolor{red}{add frequency of letters}
\section{Consonants}\label{section:segmentalPhono:cons}

\subsection{Stops}\label{section:segmentalPhono:cons:stop}
Western Armenian has a 2-way distinction for stops: phonologically voiced vs. phonologically voiceless (Table \ref{tab:stop minimal pair}). Stops can have one of 3 places of articulation: bilabial, coronal (dental), and velar. Near-minimal pairs are below for word-initial, intervocalic, and word-final stops.

\begin{table}
	\centering
	\caption{Near-minimal pairs for stops}
	\label{tab:stop minimal pair}
	{%\resizebox{1\textwidth}{!}{%
  	\begin{tabular}{|l|ll|ll|ll| }
    \hline
    & \multicolumn{2}{l|}{Initial}& \multicolumn{2}{l|}{Intervocalic}& \multicolumn{2}{l|}{Final}
    \\ \hline
    /pʰ/ & ˈ\textbf{pʰ}ɑt͡s& `open' &ʃɑˈ\textbf{pʰ}ɑtʰ & `week' & ˈt͡ʃɑ\textbf{pʰ}& `size'
    \\
    & & \armenian{բաց} & & \armenian{շաբաթ} & & \armenian{չափ}
    \\ 
    &\textbf{pʰ}ɑˈɾi & `kind' &pʰɑ\textbf{pʰ}ɑˈkʰel & `to hope' & ɑˈɾɑ\textbf{pʰ} & `Arab'
    \\
    & &\armenian{բարի} & &\armenian{փափաքել} & & \armenian{արաբ}
    \\
    /b/ & ˈ\textbf{b}ɑʁ & `cold' & bɑˈ\textbf{b}ɑ & `dad' & ˈɡɑ\textbf{b} & `link' 
    \\
    & & \armenian{պաղ} & & \armenian{պապա} & & \armenian{կապ}
    \\
    & \textbf{b}ɑʃtel & `to worship'& ɑ\textbf{b}ɑˈɡi& `glass'& bɑˈɾɑ\textbf{b}& `empty'
    \\
    & & \armenian{պաշտել}& & \armenian{ապակի}& & \armenian{պարապ}
    \\
    /tʰ/ &ˈ\textbf{tʰ}ɑs & `lesson' & tʰɑˈ\textbf{tʰ}ɑɾ & `pause' & ˈpʰɑ\textbf{tʰ} & `duck'
    \\
    & & \armenian{դաս} & & \armenian{դադար} & & \armenian{բադ}
    \\
    &\textbf{tʰ}ɑˈdel &`to judge' &ɑ\textbf{tʰ}ɑˈmɑntʰ &`diamond' & ɑɾˈd͡zɑ\textbf{tʰ} & `silver'
    \\
    & & \armenian{դատել}& & \armenian{ադամանդ} & & \armenian{արծաթ}
    \\
    /d/ & ˈ\textbf{d}ɑɾ & `letter' & ɡɑˈ\textbf{d}ɑg & `joke'& ˈmɑ\textbf{d} & `finger'
    \\
    & & \armenian{տառ} & & \armenian{կատակ} & & \armenian{մատ}
    \\
    &\textbf{d}ɑˈɾi & `year'&bɑ\textbf{d}ɑˈni &`teenager' & ɑˈzɑ\textbf{d} & `free'
    \\
    & & \armenian{տարի}& & \armenian{պատանի}& & \armenian{ազատ}
    \\
    /kʰ/ & ˈ\textbf{kʰ}ɑm & `nail' & ʃɑˈ\textbf{kʰ}ɑɾ & `sugar'& ˈtʰɑ\textbf{kʰ} & `crown'
    \\
    & & \armenian{գամ} & & \armenian{շաքար} & & \armenian{թագ}
    \\
    & \textbf{kʰ}ɑˈvɑtʰ&`cup' &ɑ\textbf{kʰ}ɑˈɾɑɡ & `farm' & ɑˈɾɑ\textbf{kʰ} & `fast' 
    \\
    & &\armenian{գաւաթ} & &\armenian{ագարակ} & & \armenian{արագ}
    \\
    /ɡ/ & ˈ\textbf{ɡ}ɑtʰ & `milk' & t͡ʃɑˈ\textbf{ɡ}ɑtʰ & `forehead' & ˈpʰɑ\textbf{ɡ} & `closed'
    \\
    & & \armenian{կաթ} & & \armenian{ճակատ} & & \armenian{փակ}
    \\
    &\textbf{ɡ}ɑˈɾɑb & `swan' &dɑ\textbf{ɡ}ɑˈvin & `still'& ɑˈɾɑ\textbf{ɡ} & `proverb'
    \\
    
    & &\armenian{կարապ} & & \armenian{տակաւին}& & \armenian{առակ}
    \\
    \hline
  	\end{tabular}
  }
	\end{table}
	
	
	In terms of articulation, the coronal series /tʰ, d/ is usually pronounced with tongue-tip touching the back of the teeth, i.e. a dental articulation. Dental articulation is previously reported for both Western Armenian and Eastern Armenian \textcolor{red}{CITE}p{add stuff}. A more narrow phonetic transcription would transcribe these consonants as [t̪ʰ, d̪]. We opt for simpler [tʰ, d].
	
	In terms of acoustics, the distinction between the phonologically voiced and voiceless stops varies by geographic region and by influence of other languages. Depending on the region where Western Armenian is spoken, the phonological voiced-voiceless distinction /D-Tʰ/ is acoustically manifested by either prevoicing vs. unaspiration [D-T], unaspiration vs. aspiration [T-Tʰ], or prevoicing vs. aspiration [D-Tʰ]. Table \ref{tab:stop acoustic variation} illustrates. 
	
	\begin{table}[H]
  \centering
  \caption{Acoustic variation of stops}
  \label{tab:stop acoustic variation}
  \begin{tabular}{|ll|lll|}
  	\hline 
  	\multicolumn{2}{|l|}{Phonological value} & [D-Tʰ] & [D-T] & [T-Tʰ] \\
  	\hline 
  	Voiced & /b/ & [b] & [b] & [p]
  	\\
  	Voiceless & /pʰ/ & [pʰ] & [p] & [pʰ]
  	\\
  	\hline Voiced & /d/ & [d] & [d] & [t]
  	\\
  	Voiceless & /tʰ/ & [tʰ] & [t] & [tʰ]
  	\\
  	\hline Voiced & /ɡ/ & [ɡ] & [ɡ] & [k]
  	\\
  	Voiceless & /kʰ/ & [kʰ] & [k] & [kʰ]
  	\\
  	\hline
  	Region & & Turkey & Lebanon & USA
  	\\
  	\hline\end{tabular}
  
	\end{table} 
	
	The earliest work on Western Armenian was done by Hrachia Adjarian in the late 19\textsuperscript{th} century \citep{Adjarian-1899-ArmenianExplosives}. He collected acoustic data on speakers of Armenian across the Ottoman Empire. His work likewise one of the earliest work to utilize what is now called Voice Onset Time (VOT) \citep{braun-2013-earlyCaseVOTAdjarian}. His speakers had a [D-Tʰ] distinction, whereby phonologically voiced stops were prevoiced while phonologically voiceless stops were voiceless aspirated \textcolor{red}{CITE}p{cite adjarian and double check}. 
	
	Because of Adjarian's foundational work, nearly all subsequent linguistic discussions on Western Armenian treat the language as having a [D-Tʰ] distinction. But more recent work has shown the actual acoustic value of stops is subject to extensive geographic variation. This variation is based on the dominant language of the community in which Western Armenian is spoken. 
	
	For example, for speakers of Western Armenian in Istanbul, these speakers have a [D-Tʰ] distinction, just as previously reported by Adjarian over a century ago. This VOT distinction is likewise found for Turkish, the dominant language of the Istanbul \textcolor{red}{CITE}p{turkish}. But for speakers in Lebanon, these people have a [D-T] distinction where the phonologically voiceless stop is unaspirated \citep{kellyKeshishian-2019-voicingStopsAffricatesArmenianLebanon}. This distinction matches that of Lebanese Arabic. As for speakers in the US, they have a [T-Tʰ] distinction where the phonologically voiced stops are phonetically voiceless unaspirated, while the phonologically voiceless stops are phonetically voiceless aspirated \citep{kellyKeshishian-2021-VoicingWesternArmenian}. This matches the situation for North American English. Similar geographic effects are documented for Armenian communities in Canada \citep{Tahtadjian-2021-PhoneticInterferenceProductionStopsWesternArmenianBilingual}. 
	
	For consistency, all phonologically voiced and voiceless stops in this grammar are transcribed with the /D-Tʰ/ distinction even though this contrast phonetically varies by speaker. For example for HD, he lived in Lebanon up until 2014 at the age of 21, so his voicing system was likely a [D-T] system. But since 2014, he has been in an English-dominant environment for the US so his voicing system is [T-Tʰ] with some occasional prevoicing, as described in \citet{Seyfarth-JIPAArmenian}. 
	
	
	It is an open question how the voicing distinction is acoustically manifested in other geographic areas where Western Armenian is spoken, including France, Armenia, Syria, Latin America, and elsewhere. It is likely that their voicing system would match with that of the dominant language in their society.
	\subsection{Affricates}\label{section:segmentalPhono:cons:affr}
	Western Armenian has affricates in two places of articulation: dental /t͡sʰ, d͡z/ and post-alveolar /t͡ʃʰ, d͡ʒ/. We provide minimal pairs in Table \ref{tab:affr minimal pair}. 
	
	\begin{table}[H]
  \centering
  \caption{Near-minimal pairs for affricates}
  \label{tab:affr minimal pair}
  \resizebox{1\textwidth}{!}{%
  	\begin{tabular}{|l|ll|ll|ll| }
    \hline
    & \multicolumn{2}{l|}{Initial}& \multicolumn{2}{l|}{Intervocalic}& \multicolumn{2}{l|}{Final}
    \\ \hline
    /t͡sʰ/ & ˈ\textbf{t͡sʰ}ɑχ & `left' & kʰɑˈ\textbf{t͡sʰ}ɑχ & `vinegar' & ˈtʰɑ\textbf{t͡sʰ} & `wet'
    \\
    & & \armenian{ձախ} & & \armenian{քացախ} & & \armenian{թաց}
    \\
    & \textbf{t͡sʰ}ɑˈmɑkʰ& `continent' & pʰɑ\textbf{t͡sʰ}ɑˈɡɑ & `absent' & ɡɑˈmɑ\textbf{t͡sʰ} & `slow'
    \\
    & & \armenian{ցամաք} & & \armenian{բացակայ} & & \armenian{կամաց}
    \\
    /d͡z/ & ˈ\textbf{d͡z}ɑpʰ & `clap' & ɑˈ\textbf{d͡z}ɑnt͡sʰ & `suffix' & t͡sʰɑ\textbf{d͡z} & `low'
    \\
    & &\armenian{ծափ} & & \armenian{ածանց} & & \armenian{ցած}
    \\
    & \textbf{d͡z}ɑˈnotʰ & `familiar' & ɑ\textbf{d͡z}ɑˈɡɑn & `adjective' & ɑˈɾɑ\textbf{d͡z} & `proverb'
    \\
    & & \armenian{ծանօթ} & & \armenian{ածական} & & \armenian{առած}
    \\
    /t͡ʃʰ/ & ˈ\textbf{t͡ʃʰ}ɑɾ & `bad' & hɑˈ\textbf{\t{tʃʰ}}ɑd͡z & `barked' & ˈɑ\textbf{t͡ʃʰ} & `right'
    \\
    & &\armenian{չար}& & \armenian{հաչած} & & \armenian{աջ}
    \\
    & \textbf{t͡ʃʰ}ɑˈmit͡ʃʰ & `raisin'& ɑ\textbf{t͡ʃʰ}ɑˈɡit͡sʰ & `assistant' & ɡɑˈɡɑ\textbf{t͡ʃʰ} & `tulip'
    \\
    & & \armenian{չամիչ} & & \armenian{աջակից} & & \armenian{կակաչ}
    \\
    /d͡ʒ/ & ˈ\textbf{d͡ʒ}ɑʃ & `food' & dɑˈ\textbf{d͡ʒ}ɑɾ & `temple' & ˈhɑ\textbf{d͡ʒ} & `satisfied' 
    \\
    & & \armenian{ճաշ} & & \armenian{տաճար} && \armenian{հաճ}
    \\
    & \textbf{d͡ʒ}ɑmˈpʰɑ & `road' & hɑ\textbf{d͡ʒ}ɑˈχel & `to frequent' & ɑnˈhɑ\textbf{d͡ʒ}&`unsatisfied'
    \\
    & & \armenian{ճամբայ} & & \armenian{յաճախել} & & \armenian{անհաճ}
    \\
    \hline
  	\end{tabular}
	}\end{table}
	
	In terms of articulation, the series /t͡sʰ, d͡z/ is usually reported to have dental contact. But alveolar contact is reported for some speakers. \textcolor{red}{CITE}p{cite}
	
	As with the stops, there is widespread geographic variation for the acoustics of affricates. This is summarized in Table \ref{tab:affr acoustic variation} . 
	
	
	\begin{table}[H]
  \centering
  \caption{Acoustic variation of affricates}
  \label{tab:affr acoustic variation}
  \begin{tabular}{|ll|ll|}
  	\hline 
  	\multicolumn{2}{|l|}{Phonological value} & [DS-TSʰ] & [DS-TS] \\
  	\hline 
  	Voiced & /d͡z/ & [d͡z] & [d͡z]
  	\\
  	Voiceless & /t͡sʰ/ & [t͡sʰ] & [t͡s] 
  	\\
  	\hline 
  	Voiced & /d͡ʒ/ & [d͡ʒ] & [d͡ʒ]
  	\\
  	Voiceless & /t͡ʃʰ/ & [t͡ʃʰ] & [t͡ʃ] 
  	\\
  	\hline
  	Region & & Turkey & Lebanon \& USA
  	\\
  	\hline\end{tabular}
  
	\end{table} 
	
	
	
	Traditional reports from Adjarian treat the distinction between the phonologically voiced and voiceless affricates as being between prevoicing vs. voiceless aspirated. For communities in Lebanon and the US, more recent acoustic studies find that the distinction is between prevoiced vs. voiceless unaspirated. As with the stops, the variation is due to language contact. Turkish has aspirated affricates, while Lebanese Arabic and North American English do not. \textcolor{red}{CITE}p{cite, section 1.3.1 of this paper https://scottseyfarth.com/docs/SeyfarthGarellek2020.pdf--}
	
	An open question is whether there are subdialects of Western Armenian which acoustically mark the distinction in terms of voiceless unaspirated vs. voiceless aspirated. We expect to find such a distinction for speakers who live in a society where the dominant language has such a distinction. 
	
	For accuracy, we transcribe all voiceless affricates in this grammar as unaspirated because our main speech samples come from HD's Lebanese dialect. 
	
	\subsection{Fricatives}\label{section:segmentalPhono:cons:fric}
	Western Armenian has the following set of fricatives: /s, z, ʃ, ʒ, χ, ʁ, h/. Each voiceless fricative has a voiced counterpart, except for /h/. Near-minimal pairs are in Table \ref{tab:fric minimal pair}.
	
	
	
	
	\begin{table}
  \centering
  \caption{Near-minimal pairs for fricatives}
  \label{tab:fric minimal pair}
  {%\resizebox{1\textwidth}{!}{%
    \begin{tabular}{|l|ll|ll|ll| }
    	\hline
    	& \multicolumn{2}{l|}{Initial}& \multicolumn{2}{l|}{Intervocalic}& \multicolumn{2}{l|}{Final}
    	\\ \hline
    	/f/ & ˈ\textbf{f}ilm & `film' & ʁɑˈ\textbf{f}ɑ &`head' & ˈu\textbf{f} & interjection
    	\\
    	& & \armenian{ֆիլմ} & &\armenian{ղաֆա} & & \armenian{ուֆ}
    	\\
    	/v/ & ˈ\textbf{v}ɑɾtʰ & `rose' & ɑˈ\textbf{v}ɑkʰ & `senior' & ˈlɑ\textbf{v}& `good'
    	\\
    	& & \armenian{վարդ} & & \armenian{աւագ} & & \armenian{լաւ}
    	\\
    	& \textbf{v}ɑˈzel & `to run' & ɑ\textbf{v}ɑˈzɑn & `pool' & ɑkʰˈɾɑ\textbf{v} & `crow'
    	\\
    	& & \armenian{վազել} & & \armenian{աւազան} & & \armenian{ագռաւ}
    	\\
    	
    	/s/ & ˈ\textbf{s}ɑɾ & `ice' & hɑˈ\textbf{s}ɑɡ & `height' & ˈmɑ\textbf{s} & `part'
    	\\
    	& & \armenian{սառ} & & \armenian{հասակ} & & \armenian{մաս}
    	\\
    	& \textbf{s}ɑˈhun & `smooth' & tʰɑ\textbf{s}ɑˈɡɑn & `classical' & vəˈnɑ\textbf{s} & `damage'
    	\\
    	& &\armenian{սահուն} & & \armenian{դասական} & & \armenian{վնաս}
    	\\
    	/z/ & ˈ\textbf{z}ɑd & `separate' & kʰɑˈ\textbf{z}ɑn & `beast' & ˈmɑ\textbf{z} & `hair'
    	\\
    	& & \armenian{զատ} & & \armenian{գազան} & & \armenian{մազ}
    	\\
    	& \textbf{z}ɑˈdiɡ & `Easter' & ʁɑ\textbf{z}ɑˈɾoz & masc. name & ɑˈvɑ\textbf{z} & `sand'
    	\\
    	& & \armenian{Զատիկ} & & \armenian{Ղազարոս} & & \armenian{աւազ}
    	\\
    	/ʃ/ & ˈ\textbf{ʃ}ɑh &`gain' & d͡ʒɑˈ\textbf{ʃ}ɑɡ & `taste' & ˈd͡ʒɑ\textbf{ʃ} & `food'
    	\\
    	& & \armenian{շահ} & & \armenian{ճաշակ} & & \armenian{ճաշ}
    	\\
    	& \textbf{ʃ}ɑˈbiɡ & `shirt' & ɑ\textbf{ʃ}ɑˈɡeɾd & `student' & lɑˈvɑ\textbf{ʃ} & `lavash'
    	\\
    	& & \armenian{շապիկ} & & \armenian{աշակերտ} & & \armenian{լաւաշ}
    	\\
    	/ʒ/ & ˈ\textbf{ʒ}ɑm & `time' & pʰɑˈ\textbf{ʒ}ɑɡ & `cup' & ˈu\textbf{ʒ} & `strength'
    	\\
    	& & \armenian{ժամ} & & \armenian{բաժակ} & & \armenian{ուժ}
    	\\
    	& \textbf{ʒ}ɑˈɾɑŋkʰ & `heir' & tʰɑ\textbf{ʒ}ɑˈnil& `to tire' & bɑˈdi\textbf{ʒ} & `punishment'
    	\\
    	& & \armenian{ժառանգ} & & \armenian{տաժանիլ} & & \armenian{պատիժ}
    	\\
    	/χ/ & ˈ\textbf{χ}ɑʁ & `game' & d͡zɑˈ\textbf{χ}ɑd͡z & `sold' & ˈt͡sɑ\textbf{χ} & `left'
    	\\
    	& & \armenian{խաղ} & & \armenian{ծախած} & & \armenian{ձախ}
    	\\
    	& \textbf{χ}ɑˈpʰel & `to trick' & nɑ\textbf{χ}ɑˈkʰɑh & `president' & uˈɾɑ\textbf{χ} & `happy'
    	\\
    	& & \armenian{խաբել} & & \armenian{նախագահ} & & \armenian{ուրախ}
    	\\
    	/ʁ/ & ˈ\textbf{ʁ}eɡ & `helm' & ɑˈ\textbf{ʁ}ɑntʰ & `sect' & ˈɑ\textbf{ʁ} & `salt'
    	\\
    	& & \armenian{ղեկ} & & \armenian{աղանդ} & & \armenian{աղ}
    	\\
    	& \textbf{ʁ}ɑˈzɑɾ & masc. name & ɑ\textbf{ʁ}ɑvˈni& `pigeon' & χɑˈʁɑ\textbf{ʁ} & `peaceful'
    	\\
    	& & \armenian{Ղազար} & & \armenian{աղաւնի} & & \armenian{խաղաղ}
    	\\
    	/h/ & ˈ\textbf{h}ɑz & `cough' & bɑˈ\textbf{h}ɑɡ & `guard' & ˈkʰɑ\textbf{h} & `throne'
    	\\
    	& & \armenian{հազ} & & \armenian{պահակ} & & \armenian{գահ}
    	\\
    	& \textbf{h}ɑˈzɑɾ & `thousand' & ɑ\textbf{h}ɑˈkʰin & `numerous' & səˈɾɑ\textbf{h} & `hall'
    	\\
    	& & \armenian{հազար} & & \armenian{ահագին} & & \armenian{սրահ}
    	\\
    	\hline
    \end{tabular}
  	}
  \end{table}
  
  Note that the fricatives are attested in all prosodic positions, but some fricatives are less common. The fricative /ʁ/ is rarely found word-initially. The fricative /f/ is rare throughout Armenian. Most occurrences of /f/ come from loanwords that entered Armenian after the Middle Ages. For example, two out of three words with /f/ in Table \ref{tab:fric minimal pair} are loanwords: [ˈfilm] from `film', and [ʁɑˈfɑ] from Turkish ``kafa''.\footnote{{https://en.wiktionary.org/wiki/kafa\#Turkish}} 
  
  In term of articulation, there is some divergence on the place of articulation for the series /s, z/. Some grammars report a dental articulation and some alveolar. For some individuals that we've asked, some report that the tongue touches the upper teeth, while some report that the tongue touches the lower teeth.
  
  For /χ, ʁ/, the voiced fricative has a typically uvular articulation. But the voiceless fricative can vary between velar and uvular. We suspect that the fricatives /χ, ʁ/ have free variation between velar and uvular place. Part of our suspicion is the fact that the native authors of this grammar cannot easily hear the difference between velar vs. uvular fricatives.
  
  For the fricative [v], some argue that this sound is always a surface pre-vocalic allophone of /u/ , and that /v/ is not phonemic \citep[13]{Vaux-1998-ArmenianPhono}. Evidence for this is that /u/ is sometimes replaced by [v] before vowels, as a type of vowel hiatus repair discussed in Section \S\ref{section:syllable:VowelHiatus}. We treat /v/ as a separate phoneme though. This is because of the following reasons: 
  
  \begin{enumerate}[noitemsep, topsep= 0pt]
  	\item Our native intuitions treat /v/ as a phoneme. 
  	\item There is a dedicated grapheme for /v/ \armenian{վ} that is used in the Reformed orthography.
  	\item There are words where [v] is used even though there's no evidence that this [v] is synchronically related to an [u] sound such as in the examples in Table \ref{tab:fric minimal pair} like [vɑzel] `to run'. If we treat [v] as non-phonemic, then we would have to argue that this word is derived from an underlying /uɑzel/ but there's no evidence for this underlying /u/. 
  	\item There are words which show schwa epenthesis breaking up an orthographic <vC> cluster, such as [vənɑs] `harm' \armenian{վնաս} <vnas>. If we treat [v] as not phonemic, then we would have to argue that such words are either derived from /unɑs/ with un-motivated /u/$\rightarrow$[v] change, or derived from /uənɑs/ where the otherwise epenthetic schwa is causing the /u/ to change to [v]. (\textcolor{red}{cite vuax})
  \end{enumerate}
  
  Thus, although there is a synchronic rule of /u/ becoming [v] before vowels, there is evidence that [v] is also a separate phoneme. 
  
  \subsection{Nasals}\label{section:segmentalPhono:cons:nasal}
  Western Armenian has nasals /m, n/. Near-minimal pairs are in Table \ref{tab:nasal minimal pair}.
  
  
  
  
  \begin{table} 
  	\centering
  	\caption{Near-minimal pairs for nasals}
  	\label{tab:nasal minimal pair}
  	{%\resizebox{1\textwidth}{!}{%
    	\begin{tabular}{|l|ll|ll|ll| }
      \hline
      & \multicolumn{2}{l|}{Initial}& \multicolumn{2}{l|}{Intervocalic}& \multicolumn{2}{l|}{Final}
      \\ \hline
      /n/ & ˈ\textbf{m}ɑh & `death' & ɑˈ\textbf{m}ɑn & `vessel' & ˈhɑ\textbf{m} & `taste'
      \\
      & & \armenian{մահ} & &\armenian{աման} & & \armenian{համ}
      \\
      & ˈ\textbf{m}ɑd͡zun & `yogurt' & ɑ\textbf{m}ɑˈnoɾ & `New Year' & ɑnˈtʰɑ\textbf{m} & `member'
      \\
      & & \armenian{մածուն} & & \armenian{ամանոր} & & \armenian{անդամ}
      \\
      /m/ & ˈ\textbf{n}ɑv & `ship' & tʰɑˈ\textbf{n}ɑɡ & `knife' & ˈtʰɑ\textbf{n} & `ayran'
      \\
      & & \armenian{նաւ} & & \armenian{դանակ} & & \armenian{թան}
      \\
      & \textbf{n}ɑˈmɑɡ & `letter' & ɑ\textbf{n}ɑˈbɑd & `desert' & iʃˈχɑ\textbf{n} & `prince'
      \\
      & & \armenian{նամակ} & & \armenian{անապատ} & & \armenian{իշխան}
      \\
      \hline
    	\end{tabular}
    }
  	\end{table}
  	
  	In terms of articulation, /m/ is bilabial. The nasal /n/ typically has dental articulation [n̪], but we transcribe this segment as [n] for ease. There a velar nasal [ŋ]. But this sound is not a phoneme. It is an allophone of /n/ before velar stops. This is a discussed in Section \S\ref{section:segmentalPhono:nasalPlace}.
  	\subsection{Rhotic}\label{section:segmentalPhono:cons:rhotic}
  	Western Armenian has only one rhotic phoneme /ɾ/ (Table \ref{tab:flap minimal pair}). 
  	
  	
  	
  	
  	\begin{table}[H]
    \centering
    \caption{Near-minimal pairs for rhotic /ɾ/}
    \label{tab:flap minimal pair}
    {%\resizebox{1\textwidth}{!}{%
      \begin{tabular}{|l|ll|ll|ll| }
      	\hline
      	& \multicolumn{2}{l|}{Initial}& \multicolumn{2}{l|}{Intervocalic}& \multicolumn{2}{l|}{Final}
      	\\ \hline
      	/ɾ/ & \textbf{ɾ}oˈbe & `second' & tʰɑˈ\textbf{ɾ}ɑɡ & `shelf' & ˈtʰɑ\textbf{ɾ} & `century'
      	\\
      	& & \armenian{րոպէ} & & \armenian{դարակ} & & \armenian{դար}
      	\\
      	& \textbf{ɾ}ɑˈfi & masc. name & ɑ\textbf{ɾ}ɑˈɾɑd͡z& `creature' & tʰəʒˈvɑ\textbf{ɾ} & `difficult'
      	\\
      	& & \armenian{Րաֆֆի} & & \armenian{արարած} & & \armenian{դժուար}
      	\\
      	
      	\hline
      \end{tabular}
    	}
    \end{table}
    
    The rhotic /ɾ/ is alveolar. Acoustically, the rhotic is often spirantized \citep{Toparlak-2019-MAArmenianPhonetics}. It is relatively rare to find rhotic-initial words. 
    
    Note that Eastern Armenian has a phonemic flap-trill distinction: /ɾ, r/. Each Eastern phoneme is presented by two graphemes \armenian{ր, ռ} <r, \.{r}>. The same trill-flap distinction is reported for Classical Armenian \textcolor{red}{CITE}p{macak}. For Western Armenian, the two types of rhotics have merged into a single flap /ɾ/, causing the letters \armenian{ր, ռ} to be homophonous. This is illustrated in Table \ref{tab:trill tap merge}.
    
    \begin{table}[H]
    	\centering
    	\caption{Trill-flap merger in Western (WA) but not Eastern (EA)}
    	\label{tab:trill tap merge}
    	{%\resizebox{1\textwidth}{!}{%
      	\begin{tabular}{|l|ll|ll|ll| }
   \hline
   & \multicolumn{2}{l|}{Initial}& \multicolumn{2}{l|}{Intervocalic}& \multicolumn{2}{l|}{Final}
   \\ \hline
   \armenian{ր} & \armenian{Րաֆֆի} && \armenian{ծարաւ} & \armenian{հարազատ} & \armenian{քար} & \armenian{տկար}
   \\
   <r> & <raffi> && <d͡zaraw> & <harazad> & <k'ar> & <dgar> 
   \\
   EA /ɾ/ & \textbf{ɾ}ɑ(f)ˈfi && t͡sɑˈ\textbf{ɾ}ɑv & hɑ\textbf{ɾ}ɑˈzɑt & kʰɑ\textbf{ɾ} & təˈkɑ\textbf{ɾ}
   \\
   WA /ɾ/ & \textbf{ɾ}ɑˈfi && d͡zɑˈ\textbf{ɾ}ɑv & hɑ\textbf{ɾ}ɑˈzɑd & kʰɑ\textbf{ɾ}& dəˈɡɑ\textbf{ɾ}
   \\
   &\multicolumn{2}{l|}{masc. name} &`thirsty'& `kindred'& `rock' & `weak'
   \\ 
   \hline
   \armenian{ռ} & \armenian{ռուս} & \armenian{ռազմիկ} & \armenian{առաջ} & \armenian{առարկայ} & \armenian{ծառ}& \armenian{օճառ}
   \\
   <\.{r}>& <\.{r}us> & <\.{r}azmig> & <a\.{r}at͡ʃ> & <a\.{r}argay> & <d͡za\.{r}> & <od͡ʒa\.{r}> 
   \\
   EA /r/ & ˈ\textbf{r}us & \textbf{r}ɑzˈmik & ɑˈ\textbf{r}ɑt͡ʃ & ɑ\textbf{r}ɑɾˈkɑ & ˈt͡sɑ\textbf{r} & oˈt͡ʃɑ\textbf{r}
   \\
   WA /ɾ/ & ˈ\textbf{ɾ}us& \textbf{ɾ}ɑzˈmik & ɑˈ\textbf{ɾ}ɑt͡ʃ & ɑ\textbf{ɾ}ɑɾˈɡɑ& ˈd͡zɑ\textbf{ɾ} & oˈd͡ʒɑ\textbf{ɾ}
   \\
   & `Russian' & `warrior' & `before' & `object' & `tree' & `soap'
   \\ \hline
      	\end{tabular}
      }
    	\end{table}
    	
    	
    	
    	
    	Historically, earlier stages of Western Armenian did have a phonemic flap-trill distinction \textcolor{red}{adjarian}. This is reported in early textbooks of Western Armenian \textcolor{red}{cite}. 
    	
    	For modern communities, there is no longer an active flap-trill distinction. Some textbooks acknowledge this and state that the letters \armenian{ր ռ} <r \.{r}> are homophonous. \textcolor{red}{cholak} 
    	
    	But many schools and textbooks artificially teach a trill. They often qualify the distinction by saying that the pronunciation difference between the letters is ``blurry''. These textbooks teach a distinction for two reasons. One reason is diachronic conservatism -- they want to teach what was spoken a century or more ago. The other reason is to teach students ways to figure out the right spelling of words. For example, a teacher may exaggerate the pronunciation of a word that has the letter \armenian{ռ} <\.{r}> by excessively trilling the letter. But neither students, teachers, nor communities actually use a trill phoneme in real speech. Many teachers in HD's experience don't even bother to speak artificially and acknowledge to students that the letters \armenian{ր ռ} <r, \.{r}> are homophonous.
    	
    	There is no active trill-flap distinction for communities in the US (Samuel Chakmadjian), Canada (Talia Tahtadjian), Turkey (Tabita Toparlak), Lebanon (HD, Avedis Samuelian), or Syria (HS, Setrag Hovsepian).
    	
    	To illustrate this complex set of affairs, consider the Armenian community of Canada. Talia Tahtadjian informs us some older speakers have a trill-flap distinction and schools try to teach this distinction. However, students don't truly acquire this distinction because there is little significant difference in their articulation of the letter \armenian{ր} <r> and \armenian{ռ} <\.{r}>.
    	
    	
    	
    	
    	
    	
    	
    	
    	
    	
    	Note that there are reports that some Western Armenian communities have an allophonic rule of changing /ɾ/ to [r] before nasals \textcolor{red}{CITE}p{sakabedoyan in jipa}. We have not been able to confirm or reproduce such reports. 
    	
    	Some speakers likewise variably trill their flaps in certain environments \citep{Seyfarth-JIPAArmenian}. Tabita Toparlak notes that in her impression, the Lebanese rhotic sounds like a trill more often than the Lebanese rhotic. 
    	
    	\subsection{Lateral}\label{section:segmentalPhono:cons:lateral}
    	Western Armenian has a lateral /l/ (Table \ref{tab:lateral minimal pair}).
    	
    	
    	
    	
    	\begin{table}[H]
      \centering
      \caption{Near-minimal pairs for lateral /l/}
      \label{tab:lateral minimal pair}
      {%\resizebox{1\textwidth}{!}{%
   \begin{tabular}{|l|ll|ll|ll| }
   	\hline
   	& \multicolumn{2}{l|}{Initial}& \multicolumn{2}{l|}{Intervocalic}& \multicolumn{2}{l|}{Final}
   	\\ \hline
   	/l/ & ˈ\textbf{l}ɑjn & `wide' & tʰɑˈ\textbf{l}ɑɾ & `green'& ˈkʰɑl & `to come'
   	\\
   	& & \armenian{լայն}& & \armenian{դալար} & & \armenian{գալ}
   	\\
   	& \textbf{l}ɑˈt͡soʁ & `crier' & hɑ\textbf{l}ɑˈd͡zel & `to persecute' & səˈχɑ\textbf{l} & `wrong'
   	\\
   	& & \armenian{լացող} & & \armenian{հալածել} & & \armenian{սխալ}
   	\\
   	\hline
   \end{tabular}
      	}
      \end{table}
      
      The liquid /l/ is palatal. The lateral is a clear lateral and there is no allophonic lateral darkening or lateral velarization for the Lebanese community. Tabita Toparlak reports that Armenians in Istanbul often velarize the lateral [ɫ] because of influences from Turkish. As of writing, we don't know the conditions for lateral velarization for Istanbul Armenian. 
      
      \subsection{Glides}\label{section:segmentalPhono:cons:glide}
      Western Armenian has a phonemic glide /j/ that is used throughout the language. The sound /w/ also exists as a marginal phoneme that's restricted to a small set of loanwords, mostly recent. This sound is usually nativized with [v]. See Table \ref{tab:glide minimal pair}. 
      
      
      
      
      \begin{table}[H]
      	\centering
      	\caption{Near-minimal pairs for glides }
      	\label{tab:glide minimal pair}
      	{%\resizebox{1\textwidth}{!}{%
   	\begin{tabular}{|l|ll|ll|ll| }
     \hline
     & \multicolumn{2}{l|}{Initial}& \multicolumn{2}{l|}{Intervocalic}& \multicolumn{2}{l|}{Final}
     \\ \hline
     /j/ & ˈ\textbf{j}eɾkʰ & `song' & tʰiˈ\textbf{j}ɑɡ & `corpse' & ˈχo\textbf{j} & `ram'
     \\
     & & \armenian{եօթ} & & \armenian{դիակ} & & \armenian{խոյ}
     \\
     & \textbf{j}ɑvˈriɡ & `dear' & mi\textbf{j}ɑˈnɑl & `to unite' & ˈtʰe\textbf{j} & `tea'
     \\
     & & \armenian{եաւրիկ} & & \armenian{միանալ}& & \armenian{թէյ}
     \\
     \hline
     /w/ & \textbf{w}ikʰipʰedˈjɑ &`Wikipedia' & sɑmˈ\textbf{w}el & `Samuel' & & 
     \\
     & vikʰipʰedˈjɑ &\armenian{Ուիքիփետիա} & sɑmˈvel &\armenian{Սամուէլ} & & 
     \\ \hline
   	\end{tabular}
   }
      	\end{table}
      	
      	
      	
      	
      	
      	
      	
      	
      	The glide [j] has a rather complex distribution. Word-initially, it is mostly found before the vowel [e]. Orthographically, the word-initial [je] is written as just \armenian{ե} <e> . The high rate of word-initial [je] sequences is due to a process of glide-epenthesis, discussed in (\textcolor{red}{cite chapter dipthongization}). Outside of [je] sequences, word-initial [j] is rarely found. Some common natives word with initial [jV] where V is not /e/ include the words for `seven' and `oil', and their derivations. There are some. loanwords with [jɑ]. These were borrowed and adapted from Turkish `yavru' and its related forms\footnote{https://en.wiktionary.org/wiki/yavru}.
      	
      	
      	\begin{table}[H]
   \centering
   \caption{Distribution of the glide [j] word-initially}
   \label{tab: glide j distribution initial}
   \begin{tabular}{|l| ll| ll| llll}
   	\hline 
   	{}[je] & ˈ\textbf{j}ez & `ox' & \textbf{j}eˈɾɑz&`dream' 
   	\\
   	& \armenian{եզ}
   	& <ez>
   	& \armenian{երազ} 
   	& <eraz> 
   	\\
   	{}[jo] & ˈ\textbf{j}otʰ & `seven' & \textbf{j}otʰnɑmˈjɑ & `septennial' 
   	\\
   	& \armenian{եօթ} 
   	& <eōt'> 
   	& \armenian{եօթնամեայ}
   	& <eōtnameay> 
   	
   	\\
   	{}[ju] & ˈ\textbf{j}uʁ & `oil' & \textbf{j}uˈʁod & `oily' 
   	\\
   	& \armenian{իւղ}
   	& <iwʁ>
   	& \armenian{իւղոտ}
   	& <iwʁod> 
   	\\
   	{}[jɑ] & ˈjɑvˈɾəm' & `my dear' & jɑvˈɾiɡ & `dear' 
   	\\
   	& \armenian{եաւրըմ} & {<eawrəm>}& \armenian{եաւրիկ} & <eawrig> 
   	\\
   	\hline
   \end{tabular}
   
      	\end{table}
      	
      	
      	Word-medially, most occurrences of [j] are epenthetic. See Table \ref{tab: glide j distribution medial}. Inserting [j] is a common repair for vowel-vowel sequences, i.e., vowel hiatus (see Section \S\ref{section:syllable:VowelHiatus}). There are some morphemes where the surface [j] is not epenthetic and is adjacent to a consonant. These morphemes can be either roots or suffixes. Their syllabification is complicated, see Section \S\ref{section:syllable:ComplexOnset:medialGlide}. 
      	
      	
      	\begin{table}[H]
   \centering
   \caption{Distribution of the glide [j] word-medially}
   \label{tab: glide j distribution medial}
   \resizebox{1\textwidth}{!}{%
   	\begin{tabular}{| ll|ll|ll| }
     \hline
     \multicolumn{2}{|l|}{Epenthetic}& \multicolumn{2}{l|}{Root-medial }& \multicolumn{2}{l|}{Suffix-initial }
     \\ \hline
     /V-V/$\rightarrow$[VjV] &cf. & /CjV/ & /Vj(C)/ & /C-jV/ & cf. 
     \\
     /kʰodi + e/ & /kʰodi/ & /ɑɾɑkʰjɑl/ &/kʰɑjl/ & /pʰɑjd + jɑ/ & /pʰɑjd/ 
     \\
     kʰodiˈ\textbf{j}e & kʰoˈdi & ɑɾɑkʰˈ\textbf{j}ɑl & ˈkʰɑ\textbf{j}l & pʰɑjdˈ\textbf{j}ɑ & ˈpʰɑjd
     \\
     `belt-{\abl}' & `belt' &`apostle' & `wolf' & `wooden' & `wood'
     \\
     \armenian{գօտիէ} & \armenian{գօտի} & \armenian{առաքեալ} & \armenian{փայտեայ} & \armenian{գայլ} & \armenian{փայտ}
     \\
     <kōdiē> & <kōdi> & <a\.{r}ak'eal> & <p'aydeay> & <kayl> & <p'ayd> 
     \\
     \hline
     /ɡɑdu + ov/ 
     &/ɡɑdu/
     & /mɑɾjɑm/ 
     & /kʰujn/ & 
     /ɡɑdɑɾ + jɑl/ & /ɡɑdɑɾ/ 
     \\
     ɡɑduˈ\textbf{j}ov
     &ɡɑˈdu & 
     mɑɾˈ\textbf{j}ɑm
     & ˈkʰu\textbf{j}n 
     & ɡɑdɑɾˈjɑl & ɡɑˈdɑɾ
     \\ 
     `cat-{\ins}'
     &`cat' & 
     `Mary'
     & `color'&
     `perfect'& `end'
     \\
     \armenian{կատուով} & \armenian{կատու} &
     \armenian{Մարիամ}& \armenian{գոյն}
     & \armenian{կատարեալ} & \armenian{կատար}
     \\
     <gadowov> & <gadow> & <mariam> & <koyn> & <gadareal> & <gadar>
     \\
     \hline
   	\end{tabular}
   }
      	\end{table}
      	
      	In the phonology of Armenian, some have argued that [j] is a separate phoneme /j/ \citep[10]{Fairbanks-1948-PhonologyMorphoWestern}. In contrast, \citet[12-3]{Vaux-1998-ArmenianPhono} argues that there is no phonemic /j/. He argues that non-epenthetic cases of surface [j] are due to allophony from underlying /i/. For example, the bi-morphemic word [ɡɑdɑɾ-jɑl] `perfect' would be analyzed as underlyingly /ɡɑdɑɾ-iɑl/ \citep[28]{Vaux-1998-ArmenianPhono}. And similarly, the name [mɑrjɑm] `Mary' would be derived from underlying /mɑɾiɑm/. He would analyze [j] as derived from /i/ via some rule of vowel-hiatus repair. His evidence is based on diachrony and orthography. The classical orthography represents the word-medial surface [jɑ] via various diagraphs, such as \armenian{եա} <ea> or <ia>. 
      	
      	But in our native intuitions, such cases of non-epenthetic [j] are not due to allophony at all but are from an underlying /j/. The fact that the digraph sequence <ea> or <ia> is pronounced as [jɑ] is just a spelling-pronunciation rule. Furthermore, there are no morpheme-alternations to support treating these non-epenthetic [jɑ] sequences as anything other than /jɑ/. 
      	
      	Word-finally (Table \ref{tab:glide distribition final}), the glide [j] is found only in monosyllabic nouns and in compounds where the second member is a monosyllabic noun. For some polysyllabic words and monosyllabic verbs, the orthography has a final <y> letter but this letter is silent. Such a glide was pronounced in Classical Armenian, but has been lost in the modern language \textcolor{red}{CITE}p{macak}.
      	
      	\begin{table}[H]
   \centering
   \caption{Distribution of the glide [j] word-finally}
   \label{tab:glide distribition final}
   {%\resizebox{1\textwidth}{!}{%
     \begin{tabular}{| ll|l|l| }
     	\hline
     	\multicolumn{2}{|l|}{Monosyllabic noun}& \multicolumn{1}{l|}{Monosyllabic }& \multicolumn{1}{l|}{Polysyllabic }
     	\\
     	\multicolumn{2}{|l| }{and derivatives} &non-noun & 
     	\\ 
     	\hline
     	/pɑj/ & /mɑɡ + pɑj/ & /ɡɑ/ & /ɑɡɾɑ/ 
     	\\
     	ˈpʰɑj & mɑɡˈpɑj & ˈɡɑ & ɑɡˈɾɑ
     	\\
     	`verb' & `adverb' & `exists.{\prs}.3{\sg}' & `witness'
     	\\
     	\armenian{բայ} & \armenian{մակբայ} & \armenian{կայ} & \armenian{ակռայ}
     	\\
     	<pay> & <magpay> & < gay> & <ag\.{r}ay> 
     	\\ \hline 
     	/hɑj/ & /iɾɑn + ɑ + hɑj/ & /la/ & /kʰulbɑ/
     	\\
     	ˈhɑj & iɾɑnɑˈhɑj & ˈlɑ & kʰulˈbɑ
     	\\
     	`Armenian & `Iranian Armenian' & `cries.{\prs}.3{\sg}' & `sock'
     	\\
     	\armenian{հայ} & \armenian{իրանահայ} & \armenian{լայ} & \armenian{գուլպայ}
     	\\
     	<hay> & <iranahay> & <lay> & <kowlbay> 
     	\\
     	\hline
     \end{tabular}
   	}
   \end{table}
   
   
   For the above polysyllabic words, \citet[20]{Vaux-1998-ArmenianPhono} treated the silent letter \armenian{յ} <y> as indicating an underlying glide /j/ that got deleted. We disagree with his analysis. Our native intuitions don't `feel' that there is any such glide. The orthography just has a silent letter. See discussion of the silent glide in Section \S\ref{section:ortho:systems}, in regards to glide epenthesis and rule reversal. 
   
   
   
   \section{Vowels}\label{section:segmentalPhono:vowel}
   
   \subsection{Canonical vowels}\label{section:segmentalPhono:vowel:canonical}
   Armenian uses the following core vowels: /ɑ, e, i, o, u/. These vowels can be used in all phonological and prosodic positions. Table \ref{tab:core vowel minimal pair} lists words which have a core vowel in either a stressed or unstressed position. Stress is final.
   
   \begin{table}[H]
   	\centering
   	\caption{Near-minimal pairs for core vowels /{ɑ, e, i, o, u}/}
   	\label{tab:core vowel minimal pair}
   	\begin{tabular}{|l|ll|ll|ll| }
     \hline 
     & \underline{\'{$\sigma$}} & & \underline{$\sigma$}\'{$\sigma$}& & $\sigma$\underline{\'{$\sigma$}} & 
     \\
     \hline 
     /{ɑ}/ & {ˈtʰ\textbf{ɑ}d} & `cause' 
     & {d͡z\textbf{ɑ}kʰum} & `origin' 
     & {tʰuˈtʰ\textbf{ɑ}ɡ} & `parrot'
     \\
     & & \armenian{դատ} 
     & & \armenian{ծագում} 
     & & \armenian{թութակ}
     \\
     & {ˈb\textbf{ɑ}kʰ} & `fast' 
     & {s\textbf{ɑ}ˈloɾ} & `plum'
     & {jeˈl\textbf{ɑ}ɡ} & `strawberry'
     \\
     & & \armenian{պաք} 
     & & \armenian{սալոր}
     & & \armenian{ելակ}
     \\
     \hline
     /{e}/ & {ˈs\textbf{e}χ} & `melon' 
     & {kʰ\textbf{e}ˈdin} & `ground'
     & {luˈd͡z\textbf{e}l} & `to solve'
     \\
     & & \armenian{սեխ}
     & & \armenian{գետին}
     & & \armenian{լուծել}
     \\
     & {ˈs\textbf{e}v} & `black' 
     & {pʰ\textbf{e}ˈduɾ} & `feather' 
     & {hɑˈm\textbf{e}ʁ} & `tasty'
     \\
     & & \armenian{սեւ} 
     & & \armenian{փետուր}
     & & \armenian{համեղ}
     \\
     \hline 
     /{i}/ & {ˈkʰ\textbf{i}tʰ} & `nose' 
     & {kʰ\textbf{i}ˈdɑɡ} & `adept'
     & {pʰɑˈɾ\textbf{i}kʰ} & `good deed'
     \\
     & & \armenian{քիթ} 
     & & \armenian{գիտակ}
     & & \armenian{բարիք}
     \\
     & {ˈpʰ\textbf{i}ʁ} & `elephant' 
     & {tʰ\textbf{i}ˈdel} & `to watch'
     & {tʰoˈn\textbf{i}ɾ} & `tandoor'
     \\
     & & \armenian{փիղ} 
     & & \armenian{դիտել}
     & & \armenian{թոնիր}
     \\
     \hline
     /{o}/ & {ˈɡ\textbf{o}ʁ} & `rib' 
     & {v\textbf{o}ˈɾɑɡ} & `quality'
     & {liˈm\textbf{o}n} & `lemon'
     \\
     & & \armenian{կող}
     & & \armenian{որակ}
     & & \armenian{լիմոն}
     \\
     & {ˈb\textbf{o}t͡ʃ}& `tail' 
     & {kʰ\textbf{o}ˈmeʃ} & `buffalo' 
     & {heˈɾ\textbf{o}s} & `hero'
     \\
     & & \armenian{պոչ} 
     & & \armenian{գոմէշ}
     & & \armenian{հերոս}
     \\
     \hline 
     /{u}/ & {ˈn\textbf{u}ɾ} & `pomegranate' 
     & {s\textbf{u}ˈlit͡ʃ} & `whistle'
     & {moˈɾ\textbf{u}kʰ} & `beard'
     \\
     & & \armenian{նուռ}
     & & \armenian{սուլիչ}
     & & \armenian{մորուք}
     \\
     & {ˈʃ\textbf{u}kʰ} & `shadow' 
     & {pʰ\textbf{u}ˈʒoʁ} & `healer'
     & {siˈɾ\textbf{u}n} & `lovely'
     \\
     & & \armenian{շուք}
     & & \armenian{բուժող}
     & & \armenian{սիրուն}
     \\ \hline
   	\end{tabular}
   \end{table}
   
   
   The vowels can likewise be used in any position. Table \ref{tab:core vowel minimal pair} listed target vowels in either the first or second (stressed) syllable. Table \ref{tab:core vowel medial } lists words where the vowel is in a word-medial unstressed syllable. 
   
   
   \begin{table}[H]
   	\centering
   	\caption{Core vowels /{ɑ, e, i, o, u}/ in word-medial unstressed position}
   	\label{tab:core vowel medial }
   	\resizebox{1\textwidth}{!}{%
     \begin{tabular}{|l|ll|ll|ll| }
     	\hline 
     	/{ɑ}/
     	& {jun\textbf{ɑ}ˈɾen} & `Greek language'
     	& {jez\textbf{ɑ}ˈɡi} & `singular' 
     	& {hoʁ\textbf{ɑ}ˈtʰɑpʰ} & `slipper' 
     	
     	\\
     	& & \armenian{յունարէն}
     	& & \armenian{եզակի} 
     	& & \armenian{հողաթափ} 
     	
     	
     	\\
     	/{e}/ 
     	& {nɑɾənt͡ʃ\textbf{e}ˈni} & `orange-tree' 
     	& {ɑv\textbf{e}ˈdis} & `good news' 
     	& {ɑhɾ\textbf{e}ˈli} & `horrible'
     	\\
     	& & \armenian{նարնջենի}
     	& & \armenian{աւետիս} 
     	& & \armenian{ահռելի}
     	\\
     	/{i}/
     	& {vost\textbf{i}ˈɡɑn} & `police officer' 
     	& {oɾ\textbf{i}ˈnɑɡ} & `example'
     	& {joɾ\textbf{i}ˈnel} & `to fashion'
     	\\
     	& & \armenian{ոստիկան}
     	& & \armenian{օրինակ}
     	& & \armenian{յօրինել}
     	\\
     	
     	/{o}/ 
     	& {ɑχ\textbf{o}ɾˈʒɑɡ} & `appetite' 
     	& {ɑs\textbf{o}ˈɾi} & `Assyrian'
     	& {ɑŋɡ\textbf{o}ˈʁin} & `bed'
     	\\
     	& & \armenian{ախորժակ} 
     	& & \armenian{ասորի} 
     	& & \armenian{անկողին}
     	\\
     	/{u}/ 
     	& {bɑjtʰ\textbf{u}ˈt͡siɡ} & `firework' 
     	& {tʰit͡s\textbf{u}ˈhi} & `goddess'
     	& {ɑɾkʰ\textbf{u}ˈni} & `royal'
     	\\
     	& & \armenian{պայթուցիկ}
     	& & \armenian{դիցուհի}
     	& & \armenian{արքունի}
     	\\ \hline
     \end{tabular}
   }\end{table}
   
   
   Synchronically, there is no rule of reducing or deleting unstressed word-medial vowels. But diachronically, there have been idiosyncratic cases where a word would lose its medial vowel. We call such a process `syncope'. Syncope is not synchronically active but is part of a fossilized set of morphologically-conditioned alternations. We discuss syncope in (\textcolor{red}{syncope chapter}). 
   
   
   
   The core vowels can be used in virtually any type of phonologically-possible syllable. They can be used in a syllable with or without an onset, and with or without coda (Table \ref{tab:core vowel syll}).
   
   \begin{table}[H]
   	\centering
   	\caption{Core vowels in different types of syllables}
   	\label{tab:core vowel syll}
   	\resizebox{1\textwidth}{!}{%
     \begin{tabular}{|l|ll| ll | ll| ll | }
     	\hline 
     	& VC & & VCC & & CVC & & CVCC & 
     	\\
     	/{ɑ}/ 
     	& {ˈɑpʰ} & `palm'
     	& {ˈɑχt} & `disease'
     	& {ˈbɑb} & `pope'
     	& {ˈpʰɑχt} & `luck'
     	\\
     	& & \armenian{ափ}
     	& & \armenian{ախտ}
     	& & \armenian{պապ}
     	& & \armenian{բաղդ}
     	\\
     	/{e}/ 
     	& {ˈet͡ʃ} & `page' 
     	& {ˈet͡ʃkʰ} & `descent' 
     	& {ˈɡed} & `dot'
     	& {ˈpʰeɾtʰ} & `fortress'
     	\\
     	& & \armenian{էջ}
     	& & \armenian{էջք}
     	& & \armenian{կէտ}
     	& & \armenian{բերդ}
     	\\
     	/{i}/ 
     	& {ˈiʒ} & `viper
     	& {ˈint͡ʃ} & `what'
     	& {ˈbid͡z} & `stain'
     	& {ˈkʰimkʰ} & `palate'
     	\\
     	& & \armenian{իժ}
     	& & \armenian{ինչ}
     	& & \armenian{բիծ}
     	& & \armenian{քիմք}
     	\\
     	/{o}/
     	& {ˈotʰ} & `air'
     	& {oɾtʰ.ˈnel} & `to bless' 
     	& {ˈt͡soɾ} & `valley'
     	& {ˈpʰoɾt͡s} & `attempt'
     	\\
     	& & \armenian{օդ}
     	& & \armenian{օրհնել}
     	& & \armenian{ձոր}
     	& & \armenian{փորձ}
     	\\
     	/{u}/
     	& {ˈuʃ} & `late'
     	& {ˈuʁd} & `camel'
     	& {ˈt͡sul} & `bull'
     	& {ˈd͡zuŋɡ} & `knee'
     	\\
     	& & \armenian{ուշ}
     	& & \armenian{ուղտ}
     	& & \armenian{ցուլ}
     	& & \armenian{ծունկ}
     	\\ \hline
     \end{tabular}
   }\end{table}
   
   
   There are some asymmetries when it comes to word-initial vowels. Based on a word count from a digitized version of \citet{kouyoumdjian-1970-DictionaryArmenianEnglish}'s dictionary, Table \ref{tab: initial vowel number} lists the number of words which start with a core vowel. The most common initial vowel is /ɑ/, while the rarest is /e/. The other core vowels /i, o, u/ occupy an intermediate spot. 
   
   \begin{table}[H]
   	\centering
   	\caption{Number of words with an initial core vowel /{ɑ, e, i, o, u}/}\label{tab: initial vowel number}
   	\begin{tabular}{| l| lllll| l| }
     \hline 
     \textbf{Vowel} & /{ɑ}/ & /{e}/ & /{i}/ & /{o}/ & /{u}/ & Total \\
     \textbf{Count} & \textbf{7050} & \textbf{58} & 736 & 680 & 662 & 9839 
     \\
     \textbf{Percentage} & 
     76.75\% & 0.63\% & 8.01\% & 7.40\% & 7.21\% & 100\%
     
     
     \\ \hline 
   	\end{tabular}
   \end{table}
   
   
   The reason for the relative rarity of word-initial /e/ is due to diachronic sound changes. A series of sound changes from Classical Armenian to Modern Armenian caused the initial [e] sound from Classical Armenian to become [je] in Modern Armenian, while a Classical initial [ē] sound became modern [e]. Such sound changes are reflected in the orthography (\textcolor{red}{cite chapter diphthongization}. The letter \armenian{է} <ē> is used to mark a word-initial [e], while the letter \armenian{ե} <e> is used to mark a word-initial [je]. We discuss the phonological and morphological effects of these pronunciation rules in (\textcolor{red}{diphtonziation chapter}).
   
   
   In terms of acoustics, there have been past studies on the acoustics of Armenian vowels, both Western and Eastern \textcolor{red}{CITEUCKET}. One of the largest studies for Western Armenian is \citet{Toparlak-2019-MAArmenianPhonetics}, which was later interpreted by \citet{Seyfarth-JIPAArmenian}. We showed the vowel space in Figure \ref{fig:vowel inventory}. 
   
   For the mid vowels /e, o/, most previous phonological studies of Armenian transcribe the mid vowels as lax /ɛ, ɔ/. These studies include \textcolor{red}{BUCLET}. However acoustically, these vowels are quite close to [e, o] and we transcribe them as /e, o/. Furthermore, our native intuitions don't hear a clear difference between between [e, o] and [ɛ, ɔ]. This suggests that the Armenian mid vowels can phonetically range from [e, o] and [ɛ, ɔ] as a type of free variation. 
   
   
   
   
   \subsection{Phonology of the schwa}\label{section:segmentalPhono:vowel:schwa}
   The schwa vowel /ə/ has a complicated treatment in Armenian linguistics and philology. These complications involve a) disagreement over its phonemic status, b) orthographic representations of the schwa, and c) origin of the schwa as being derived or underived. 
   
   In terms of the phonological status of the schwa, most occurrences of the schwa are due to morpho-phonological processes. These processes are closely tied to the orthography. All these interactions have caused some to argue that the schwa is not a phoneme. We disagree with this stance and treat the schwa as a phoneme. We first go over asymmetries in the distribution of schwas (\S\ref{section:segmentalPhono:vowel:schwa:minimal}), the derived role of the unwritten schwa (\S\ref{section:segmentalPhono:vowel:schwa:unwritten}), and the phonemic status of underived and written schwas (\S\ref{section:segmentalPhono:vowel:schwa:written}). 
   
   \subsubsection{Minimal pairs and asymmetries}\label{section:segmentalPhono:vowel:schwa:minimal}
   
   In terms of phonemic status, some sources treat the schwa as a phoneme in Armenian \textcolor{red}{cite bucket}. As a phoneme, the schwa can be used to form minimal or near-minimal pairs with other vowels. We provide such pairs in Table \ref{tab:schwa minimal pairs}. 
   
   \begin{table}[H]
   	\centering
   	\caption{Near-minimal pairs for the schwa against core vowels}
   	\label{tab:schwa minimal pairs}
   	\begin{tabular}{|l| ll | ll | }
     \hline 
     /{ə, ɑ}/ 
     & {ɡ\textbf{ə}ˈɾɑɡ} & `fire'
     & {ɡ\textbf{ɑ}ˈɾɑkʰ} & `butter'
     \\
     & & \armenian{կրակ} 
     & & \armenian{կարագ}
     \\
     
     /{ə, e}/ 
     & {h\textbf{ə}ˈɾɑd} & `Mars' 
     & {h\textbf{e}ˈɾu} & `far'
     \\
     & & \armenian{Հրատ} 
     & & \armenian{հեռու}
     \\
     /{ə, i}/ 
     & {m\textbf{ə}ˈnɑl} & `to stay'
     & {m\textbf{i}ˈnɑɡ} & `alone'
     \\
     & & \armenian{մնալ}
     & & \armenian{մինակ}
     \\
     /{ə, o}/ 
     & {tʰ\textbf{ə}ˈkʰɑl} & `spoon'
     & {tʰ\textbf{o}ˈkʰɑχt} & `tuberculosis'
     \\
     & & \armenian{դգալ}
     & & \armenian{թոքախտ}
     \\
     /{ə, u}/ 
     & {ʃ\textbf{ə}ˈʃuɡ} & `whisper'
     & {ʃ\textbf{u}ˈʃɑn} & `lily'
     \\
     & & \armenian{շշուկ}
     & & \armenian{շուշան} 
     \\
     \hline 
   	\end{tabular} 
   \end{table}
   
   A schwa can be found in virtually any type of syllable (Table \ref{tab:schwa vowel syll types}). 
   
   \begin{table}[H]
   	\centering
   	\caption{Schwas in different types of syllables}
   	\label{tab:schwa vowel syll types}
   	{%\resizebox{1\textwidth}{!}{%
     	\begin{tabular}{|l|ll| l | }
  \hline 
  VC & \textbf{əs}.kɑl & `to feel' & \armenian{զգալ}
  \\
  VCC & \textbf{əst}.ˈɾuɡ & `slave' & \armenian{ստրուկ}
  \\
  CV& ˈmɑɾ.\textbf{tʰə} & `man-{\defgloss}' & \armenian{մարդը}
  \\
  CVC & \textbf{pʰəɾ}.ˈtʰel & `to break' & \armenian{փրթել}
  \\
  CVCC & \textbf{ɡəɾɡ}.ˈnel & `to repeat' & \armenian{կրկնել}
  \\ \hline
     	\end{tabular}
   	}\end{table}
   	
   	
   	However, the schwa is subject to more restrictions than other vowels. For example, there are virtually no native words where the only vowel is a schwa. The exceptions are a handful of onomatopoeic words, letter names, prepositions, and borrowings (Table \ref{tab:schwa only words}). One common example is the derivational prefix /əntʰ-/ which is diachronically derived from the archaic preposition of the same form. 
   	
   	\begin{table}[H]
     \centering
     \caption{Words where the only vowel is a schwa}
     \label{tab:schwa only words}
     \begin{tabular}{|l ll | }
     	\hline 
     	{ˈfəɾ} & `rustling sound' & onomatopoeic
     	\\
     	& \armenian{ֆըռ} &
     	\\
     	{ˈətʰ} & `name of letter \armenian{ը} <ə>' & letter name
     	\\
     	& \armenian{ըթ} &
     	\\
     	{ˈəst} & `according to' & preposition
     	\\
     	& \armenian{ըստ} & 
     	\\ 
     	{ˈəntʰ} & `to' & preposition (archaic)
     	\\
     	& \armenian{ընդ} & 
     	\\
     	{əntʰ-hɑɡɑˈrɑɡ} & `on the contrary' & prefixed to /hɑɡɑrɑɡ/ `opposite' 
     	\\
     	& \armenian{ընդհակառակ} & \armenian{հակառակ}
     	\\
     	{fəsˈtəχ} & `pistachio' & borrowed from Turkish ``{fıstık}''
     	\\
     	& \armenian{ֆստըխ} & 
     	\\
     	\hline 
     \end{tabular} 
   	\end{table}
   	
   	
   	
   	\subsubsection{Phonology of unwritten schwas}\label{section:segmentalPhono:vowel:schwa:unwritten}
   	Although there is a schwa grapheme <\armenian{ը}> /ə/, most instances of a spoken schwa are unwritten (Table \ref{tab:schwa unwritten category}). These unwritten schwas fall into different morphophonological categories: epenthetic schwas, reduced vowels, and syncopated vowels. The three categories for these unwritten schwas are summarized below. 
   	
   	\begin{table}[H]
     \centering
     \caption{Categories of unwritten schwa}
     \label{tab:schwa unwritten category}
     \begin{tabular}{|l |l ll|}
     	\hline
     	Type& Inserted & Reduced & Syncopated 
     	\\
     	\hline 
     	Example: & \armenian{գրպան} & \armenian{գծել} & \armenian{հասսկնալ}
     	\\
     	& <krban> & <kd͡zel> & <hasgnal> 
     	\\
     	& [kʰ\textbf{ə}ɾˈbɑn] & [kʰ\textbf{ə}ˈd͡zel] & [hɑsk\textbf{ə}ˈnɑl]
     	\\
     	& `pocket' & `to draw'& `to understand'
     	\\
     	\hline
     	Related: & N/A & \armenian{գիծ} & \armenian{հասկանալ}
     	\\
     	& & <k\textbf{i}d͡z> & <hasganal> 
     	\\
     	& & [ˈkʰ\textbf{i}d͡z] & [hɑsk\textbf{ɑ}ˈnɑl]
     	\\
     	& & `line'& `to understand (archaic)'
     	\\ \hline
     	
     \end{tabular}
   	\end{table}
   	
   	Most instances of the unwritten schwa are categorized as `inserted' or `epenthetic' schwas. These are schwas that surface in words like [kʰəɾbɑn] `pocket'. Such words do not have any morphologically-related word where the schwa is replaced by a non-schwa vowel, e.g., there is no such thing as word like *\textit{kʰiɾbɑn}. 
   	
   	Armenian orthography allows rather long clusters of consonants to be written. These clusters are broken up by a schwa in pronunciation. We discuss the phonology of such inserted schwas in (\textcolor{red}{cite chapter schwa epenthesis}). 
   	
   	The second category of schwas is reduced schwas. These are schwas which are derived from destressed high vowels /i,u/. Such schwas are created when words are derived from other words that have stress high vowels. For example, the word [ˈkʰid͡z] `line' has a stress high vowel /i/. The word [kʰəˈd͡zel] `to draw' is derived from this word by adding the suffix sequence /-el/. Stress shifts from the vowel /i/ to the suffix vowel /e/. This causes the high vowel to be replaced by a schwa in pronunciation. The orthography does not mark this schwa. 
   	
   	The derivation of schwas via high vowel reduction is quite complicated. We discuss the phonology/morphology of high vowel reduction in (\textcolor{red}{cite chapter vowel reduction}). 
   	
   	The last category of unwritten schwas is syncopated schwas. This set of words is rather small. These are words which, in earlier stages of the languages, had a word-medial and unstressed non-high vowel like /ɑ/: [hɑskɑˈnɑl] `to understand (archaic)'. In more contemporary registers of the language, this word-medial vowel is eithe.r deleted or pronounced as a schwa: [hɑskəˈnɑl]. The phonology and morphology of syncope is discussed in (\textcolor{red}{cite syncope chapter}). 
   	
   	Note that of the three categories, schwa insertion/epenthesis and high vowel reduction are synchronically productive and wide-spread in the Armenian lexicon. The third category, syncope of non-high vowels, is unproductive and limited to a small set of words. Syncope is more a sporadic diachronic process than a productive synchronic process. Speakers have to memorize the set of words which have a syncope-derived schwa.
   	
   	\subsubsection{Phonology of written schwas}\label{section:segmentalPhono:vowel:schwa:written}
   	
   	Because of the existence of epenthesis and vowel reduction, some phonological treatments of Armenian treat the schwa as a non-phoneme \textcolor{red}{cite}. Such work argues that the schwa is always epenthetic or derived, and thus not part of the phonemic inventory of the language. We disagree with this stance. Although it is true that most occurrences of the schwa are derived from epenthesis and reduction, there are some words or morphemes where the schwa must be of the underlying form of the morpheme. In such cases, the orthography would represent the underlying schwa with the grapheme \armenian{ը}. 
   	
   	The intuition among speakers is that if a pronounced schwa cannot be predicted from epenthesis and reductions, then it must be written in the orthography. Such cases are few, but they exist. They are found in both functional and non-functional morphemes.
   	
   	Among functional morphemes (Table \ref{tab:schwa functional written}), the most common use of an underlying schwa is the definite suffix. This suffix is /-ə/ after consonants, and /-n/ after vowels. This allomorphy is discussed further in (\textcolor{red}{cite definite alloomprohy}). Another common case is the irregular ablative suffix for words of time: /-vəne/. This suffix is explained further in (\textcolor{red}{cite irregular dative -van}). In both of these cases, the schwa is written in the orthography with the grapheme \armenian{ը} <ə>. If the schwa was unwritten, then the word would be pronounced incorrectly. 
   	
   	\begin{table}[H]
     \centering
     \caption{Functional morphemes with a written schwa}
     \label{tab:schwa functional written}
     \begin{tabular}{|lll | ll l | }
     	\hline 
     	Definite & & & Ablative & & 
     	\\
     	/-{ə}/ & {ˈpʰɑɾ} & {ˈpʰɑɾ-ə} 
     	& /-{vəne}/ & {ɑjˈsoɾ} & {ɑjˈsoɾ-vəne}
     	\\
     	-\armenian{ը} & \armenian{բառ} & \armenian{բառ\textbf{ը}}
     	& \armenian{-ուընէ} & \armenian{այսօր}& \armenian{այսօրու\textbf{ը}նէ} 
     	\\
     	<-{ə}> & <{p'a\.{r}}> & <{p'a\.{r}\textbf{ə}}> 
     	& <{owəne}> & <{aysōr}> & <{aysōro\textbf{w}əne}> 
     	\\
     	& `word' & `word-{\defgloss}' 
     	& & `today' & `today-{\abl}' 
     	\\ 
     	& & `the word' & & & `from today'
     	\\
     	\hline 
     	If unwritten: & & {ˈpʰɑɾ} & & & *{ɑjsorune}
     	\\
     	& & `word' && & nonce word
     	\\
     	\hline 
     \end{tabular} 
   	\end{table}
   	
   	Among non-functional or lexical morphemes (Table \ref{tab:lex written schwa}), a written schwa is used when the word starts with a schwa. Such morphemes include verbs, nouns, and adjectives. If the schwa as unwritten, then a schwa would be epenthesized in the wrong slot due to the rules for schwa epenthesis. There is likewise a derivational prefix /{ənt-}/, described in (\textcolor{red}{cite chapter prefixes}). For these lexical morphemes, the initial schwa usually precedes a nasal sound. 
   	
   	\begin{table}[H]
     \centering
     \caption{Non-functional or lexical morphemes with a written schwa}
     \label{tab:lex written schwa}
     \begin{tabular}{| l| lll| l| }
     	\hline & & & & If unwritten
     	\\
     	Verb 
     	& {əˈsel} & `to say' & \armenian{ըսել}& *{sel}
     	\\
     	& {əmˈbel} & `to drink' & \armenian{ըմպել}& *{məbel}
     	\\ 
     	
     	Noun
     	& {əχt͡sɑnkʰ} & `desire'
     	& \armenian{ըղձանք} & *{χət͡sɑnkʰ}
     	\\
     	& {əŋˈɡeɾ} & `friend' & \armenian{ընկեր}& *{nəɡeɾ} 
     	\\
     	Adjective 
     	& {əndɑˈni} & `familiar' & \armenian{ընտանի}
     	& *{nədɑni}
     	\\
     	
     	& {əmˈpʰost} & `stubborn'
     	& \armenian{ըմբոստ} &*{məpʰost}
     	\\
     	\hline 
     \end{tabular}
   	\end{table}
   	
   	
   	
   	In sum, the schwa is a controversial sound in Armenian. In many cases, a surface schwa is not present in the underlying form of words. Although such derived schwas exists, there are likewise morphemes where the schwa is part of the underlying form of word. Because of the existence of these underived schwas, we treat the schwa as a phoneme.\footnote{As an alternative, Vaux argues that the schwa is non-phonemic and always derived. For cases of written schwas that we argue are underived, like [əŋɡeɾ] `friend' \armenian{ընկեր} <ənger>, Vaux would argue that the underlying form has an empty vocalic slot /Vnɡeɾ/. A rule would then fill these empty slots with epenthetic schwas. We don't entertain this analysis. \textcolor{red}{vaux}}
   	
   	
   	\subsection{Front round vowels}\label{section:segmentalPhono:vowel:frontRound}
   	The basic set of vowels in Western Armenian are the core vowels /ɑ, e, i, o, u/ and the schwa /ə/. However, due to contact with Turkish, Western Armenian has developed a sound /ʏ/ which is found across the Armenian lexicon. It like has developed a marginal phoneme /œ/ which is found in a handful of loanwords. 
   	
   	
   	For /œ/ (Table \ref{tab:front mid round}), this vowel is written with the digraph \armenian{էօ} <ēō> and is optionally nativized with /o/. This vowel developed out of contact with Ottoman Turkish and is found in a handful of loanwords from Ottoman Turkish. Speakers vary in the rate of nativizing such words. For example, HD's intuition is that in Lebanon, it is more common to nativize these words with /o/ than to use the marginal phoneme /œ/. The rate of nativization likely varies by area and age. 
   	
   	\begin{table}[H]
     \centering
     \caption{Words with marginal phoneme /{œ}/}
     \label{tab:front mid round}
     \begin{tabular}{|llll| }
     	\hline 
     	With /{œ}/ & Nativized & & Meaning \& origin
     	\\
     	{t͡ʃœˈɾeɡ} 
     	& 
     	{t͡ʃoˈɾeɡ} 
     	& 
     	\armenian{չէօրէկ}
     	
     	&
     	pastry item from Turkish ``{çörek}''
     	\\
     	{bœˈɾeɡ} 
     	& 
     	{boˈɾeɡ} 
     	& 
     	\armenian{պէօրէկ} 
     	& 
     	pastry item from Turkish ``{börek}''
     	\\
     	{œʒeˈni} & {oʒeˈni} 
     	& \armenian{Էօժէնի}
     	& fem. given name from French Eugénie
     	\\
     	{dœˈʃeɡ} 
     	& {doˈʃeɡ}
     	& \armenian{տէօշէկ}
     	& `mattress' from Turkish ``{döşek}''
     	\\
     	
     	{kʰœˈfte} 
     	& {kʰofˈte}
     	& \armenian{քէօֆթէ}
     	& `kofta' from Turkish ``köfte"
     	\\ \hline
     \end{tabular}
   	\end{table}
   	
   	We transcribe the vowel as /œ/, though we think it's free to vary with  /ø/ without a consistent articulatory target. 
   	
   	More such loanwords are reported in \citet{Adjarian-1902-TUrkishWordsArmenian} study on Turkish borrowings in early modern Istanbul Armenian. We have not been able to extensively analyze this dictionary in order to find more such loanwords that survived into the colloquial Western of non-Istanbul Armenians. 
   	
   	
   	For the vowel /ʏ/, its use is more complicated and is closely tied with the orthography. The Armenian script has the digraph \armenian{իւ} <iw> (Table \ref{section:segmentalPhono:alloEastern:palatalization}). This digraph is pronounced as [iv] word-finally and before vowels in both Western and Eastern Armenian. Word-initially, this digraph is pronounced as [ju]. In all other positions, Eastern Armenian pronounces this digraph as [ju],\footnote{In Eastern Armenian, when the digraph \armenian{իւ} in traditional spelling is pronounced as [ju], it is replaced by \armenian{յու} <{yow>} in the reformed spelling system. } while Western Armenian generally uses [ʏ]. There are some complications with the nominalizing suffix /-utʰjʏn/ \armenian{-ութիւն}; discussed in Section \S\ref{section:segmentalPhono:alloEastern:palatalization}. 
   	
   	
   	\begin{table}[H]
     \centering
     \caption{Pronunciations of the digraph \armenian{իւ} <iw> }
     \label{tab:iw digraph pronunciations}
     \resizebox{1\textwidth}{!}{%
     	\begin{tabular}{| llllll| }
  \hline 
  & Western & Eastern & & 
  \\
  \hline Initial
  & {ˈ\textbf{ju}ʁ} 
  & {ˈ\textbf{ju}ʁ} 
  & `oil' & 
  \armenian{\textbf{իւ}ղ} & {<\textbf{iw}ɣ>} 
  \\
  & {\textbf{ju}ɾɑkʰɑnˈt͡ʃʏɾ} 
  & {\textbf{ju}ɾɑkʰɑnˈt͡ʃjuɾ} 
  & `each' & 
  \armenian{\textbf{իւ}րաքանչիւր}
  & {<\textbf{iw}rak'ant͡ʃ'iwr>} 
  \\
  & {\textbf{ju}ɾɑhɑˈduɡ}
  & {\textbf{ju}ɾɑhɑˈtuk}
  & `specific'
  & 
  \armenian{\textbf{իւ}րայատուկ}
  & {<\textbf{iw}rayadowg>} 
  \\
  \hline 
  Medial 
  & {ˈd͡ʒ\textbf{ʏ}ʁ}
  & {ˈt͡ʃ\textbf{ju}ʁ}
  & `branch'
  & \armenian{ճ\textbf{իւ}ղ} & {<d͡ʒ\textbf{iw}ɣ>}
  \\
  pre-C& {ɑŋˈɡ\textbf{ʏ}n} 
  & {ɑŋˈk\textbf{ju}n} 
  & `corner'
  & \armenian{անկ\textbf{իւ}ն} & {<ang\textbf{iw}n>}
  \\
  & {ɑˈl\textbf{ʏ}ɾ} 
  & {ɑˈl\textbf{ju}ɾ} 
  & `flour ' 
  & \armenian{ալ\textbf{իւ}ր} & {<al\textbf{iw}r>}
  \\
  \hline
  Medial
  & {t\textbf{iˈv}ɑn} 
  & {d\textbf{iˈv}ɑn} 
  & `divan' & 
  \armenian{դ\textbf{իւ}ան} & {<t\textbf{iw}an>}
  \\
  pre-V
  & {h\textbf{iˈv}ɑntʰ}
  & {h\textbf{iˈv}ɑnd}
  & `sick' 
  & \armenian{հ\textbf{իւ}անդ} & {<h\textbf{iw}ant>}
  \\
  & {tʰ\textbf{iv}ɑˈɡɑn} 
  & {d\textbf{iv}ɑˈkɑn} 
  & `diabolic'
  & \armenian{դ\textbf{իւ}ական} & {<t\textbf{iw}agan>}
  \\
  \hline Final
  & {ˈtʰ\textbf{iv}} 
  & {ˈtʰ\textbf{iv}} 
  & `number' 
  & \armenian{թ\textbf{իւ}} 
  & {<t'\textbf{iw}>}
  \\
  & {ɑˈn\textbf{iv}} 
  & {ɑˈn\textbf{iv}} 
  & `wheel'
  & \armenian{ան\textbf{իւ}}
  & {<an\textbf{iw}>}
  \\
  & {ɡəˈɾ\textbf{iv}} 
  & {kəˈr\textbf{iv}} 
  & `fight'
  & \armenian{կռ\textbf{իւ}}
  & {<g\.{r}\textbf{iw}>}
  \\ \hline
     	\end{tabular}
   	}\end{table}
   	
   	
   	Within a morpheme, he segment [ʏ] is restricted to closed CVC(C) syllables. Suffixation can make this segment lose a coda: [hʏ.ɾ-i] `guest-{\gen}'. The closest counter-examples we found were loanwords: [bʏtʰi] `Pythia' \armenian{Պիւթի}. 
   	
   	As for needing an onset, some [ju]-initial words can be optionally pronounced with [ʏ] in Western: [{ʏɾɑkʰɑnˈt͡ʃʏɾ}] `each', [{ʏɾɑhɑˈduɡ}] `specific'.
   	
   	Diachronically, the modern [ʏ] sound may have developed from an earlier [iu̯] sequence \citep{Avetyan-2015-WesternRoundVowel}. This sequence changed to [ʏ], whether via dialect-internal sound changes, contact with other dialects, or via contact with Turkish. 
   	
   	
   	
   	The acoustic quality of this /ʏ/ can range from [ʏ] to [y]. Tabita Toparlak reports that for Istanbul Armenian is more like [y]. For Syrian Armenians, our impression is that their vowel is more often [ʏ]. Of course, in depth acoustic studies are needed to verify or disconfirm these impressions. 
   	
   	
   	
   	Depending on the word and speaker, the vowel [ʏ] can be replaced with [uj], [ʏj], [jʏ], [jʏ], [ju]. Table \ref{tab:round y words} lists a set of common words that are pronounced with [ʏ], along with possible alternative pronunciations from HD's speech. Our impression is that this variation is a type of free variation that is closely tied to the speaker's sociolinguistic origins. For example, HD reports that his family and peers in Lebanon would most often have the [ʏj] or [uj] forms. For HS from Syria, her own idiolect seems to almost always have [ʏ]. 
   	
   	\begin{table}[H]
     \centering
     \caption{Words with {[ʏ] and alternative pronunciations}}
     \label{tab:round y words}
     \resizebox{1\textwidth}{!}{%
     	\begin{tabular}{|lllll|ll| }
  \hline 
  
  {}{[ʏ]} & {[ʏj]} & {[uj]} &{[jʏ]} &{[ju]} & &
  \\
  \hline 
  {ˈt͡s\textbf{ʏ}n} & {ˈt͡s\textbf{ʏj}n} &{ˈt͡s\textbf{uj}n} & & & `snow' & \armenian{ձիւն}
  \\
  {ˈh\textbf{ʏ}ɾ} & {ˈh\textbf{ʏj}ɾ} &{ˈh\textbf{uj}ɾ} & & 
  & `guest' & \armenian{հիւր}
  \\
  {ˈn\textbf{ʏ}tʰ} & {ˈn\textbf{ʏj}tʰ} & {ˈn\textbf{uj}tʰ} & & & 
  `topic' & \armenian{նիւթ}
  \\
  {ˈm\textbf{ʏ}s} & {ˈm\textbf{ʏj}s} & {ˈm\textbf{uj}s} & & &
  `other' & \armenian{միւս}
  \\
  {ˈkʰ\textbf{ʏ}ʁ} & {ˈkʰ\textbf{ʏj}ʁ} & & {ˈkʰ\textbf{jʏ}ʁ} & {ˈkʰ\textbf{ju}ʁ} & 
  `village' & \armenian{գիւղ}
  \\
  {s\textbf{ʏ}ˈnɑɡ} & {s\textbf{ʏj}ˈnɑɡ} & & & & 
  `column' & \armenian{սիւնակ}
  \\
  {h\textbf{ʏ}ˈsis} & & & & & 
  `north' & \armenian{հիւսիս}
  \\
  {ɑˈɾ\textbf{ʏ}n} & {ɑˈɾ\textbf{ʏj}n} & {ɑˈɾ\textbf{uj}n} & & & 
  `blood' & \armenian{արիւն}
  \\
  {ɑˈʁ\textbf{ʏ}s} & {ɑˈʁ\textbf{ʏj}s} & {ɑˈʁ\textbf{uj}s} & & & 
  `brick' & \armenian{աղիւս}
  \\
  {ɑˈɾ\textbf{ʏ}d͡z} & {ɑˈɾ\textbf{ʏj}d͡z} & {ɑˈɾ\textbf{uj}d͡z} & & & 
  `lion' & \armenian{առիւծ}
  \\
  {məɾˈt͡ʃ\textbf{ʏ}n} & {məɾˈt͡ʃ\textbf{ʏj}n} & {məɾˈt͡ʃ\textbf{uj}n} & & & 
  `ant' & \armenian{մրջիւն}
  \\
  {hənˈt͡ʃ\textbf{ʏ}n} & {hənˈt͡ʃ\textbf{ʏj}n} & {hənˈt͡ʃ\textbf{uj}n} & & & 
  `sound' & \armenian{հնչիւն}
  \\
  {hɑˈɾ\textbf{ʏ}ɾ} & {hɑˈɾ\textbf{ʏj}ɾ} & {hɑˈɾ\textbf{uj}ɾ} & & & 
  `hundred' & \armenian{հարիւր}
  \\
  {zeˈpʰ\textbf{ʏ}ɾ} & {zeˈpʰ\textbf{ʏj}ɾ} & {zeˈpʰ\textbf{uj}ɾ} & & & 
  `zephyr' & \armenian{զեփիւռ}
  \\
  {ɑχˈp\textbf{ʏ}ɾ} & {ɑχˈp\textbf{ʏj}ɾ} & {ɑχˈp\textbf{uj}ɾ} & & & 
  `fountain' & \armenian{աղբիւր}
  \\
  {əsˈp\textbf{ʏ}ɾkʰ} & & & & & 
  `diaspora' & \armenian{սփիւռք}
  
  \\ 
  ɑɾˈtʰ\textbf{ʏ}ŋkʰ & ɑɾˈtʰ\textbf{ʏj}ŋkʰ & ɑɾˈtʰ\textbf{uj}ŋkʰ && & `result' & \armenian{արդիւնք}\\
  \hline \end{tabular}
     }
     
   	\end{table}
   	
   	Furthermore, there are reports that colloquial speech can reduce the Western [ʏ] vowel (Eastern [ju]) to a [u] in some high-frequency words \textcolor{red}{cite}. For example, in HD's experience, some attested reduced words are [ˈt͡s\textbf{u}n] `snow' \armenian{ձուն} instead of [ˈt͡s\textbf{ʏ}n] \armenian{ձիւն}. 
   	
   	Another rare pronunciation of [ʏ] is as [əju] or [əjʏ]. In HD's judgments, such a division happens sometimes for word-initial [Cʏ] sequences like [kʰəjuʁ, kʰəjʏʁ] `village' instead of [kʰʏʁ] `village' \armenian{գիւղ}. We suspect that such a divided pronunciation is restricted to emphatic speech. 
   	
   	As a last note, there are little to no phonetic work on the [ʏ] sound. The only one to our knowledge is unpublished work by Hrayr Khanjian. \textcolor{red}{cite and expand}.
   	
   	
   	
   	\subsection{Turkish-induced centralization}\label{section:segmentalPhono:vowel:centralization}
   	In modern Western Armenian, it is unknown if there are any allophonic rules that affect vowels. The closest example that we know is vowel laxing in Turkish-speaking Armenians. 
   	
   	For speakers of Western Armenian who don't know Turkish, such as HD and the Lebanese community, the vowels /i, u, e, o/ are pronounced as just [i, u, e, o]. But it is reported that Western Armenian speakers who are Turkish-speaking apply an allophonic rule of changing /i, u, e, o/ to [ɪ, ʊ, ɛ, ɔ] when either word-final or before a vowel \citep[3-4]{Fairbanks-1948-PhonologyMorphoWestern}. 
   	
   	\textcolor{red}{cite fairbanks data}
   	
   	
   	The main source for this process is Fairbanks who documents extensive allophonic laxing for his informants who were from Istanbul \citep[1]{Fairbanks-1948-PhonologyMorphoWestern}. HS seems to show this allophonic process as well in her own speech. Although she was raised in Syria, her grandparents were Turkish-speaking and HS was raised as a Turkish-Armenian bilingual. 
   	
   	Because this process is specific to Turkish-speaking Armenians, we can't provide acoustic data on this because our main phonological informant (HD) is from Lebanon and doesn't speak Turkish. 
   	
   	
   	It is an open question if this allophonic process is still active in the speech of Turkish-Armenian bilinguals in Istanbul. 
   	
   	\section{Allophony of laryngeal processes}\label{section:segmentalPhono:allphonLaryng}
   	\textcolor{red}{does deaspiration happen in geminates?}
   	
   	
   	In terms of its phonemes, Western Armenian has voiced stops and voiceless aspirated stops. The stops however can change their voicing quality or aspiration quality depending on the phonological (phonotactic) context that they're used. Two such processes are deaspiration and voicing assimilation. Both processes tend to occur simultaneously. 
   	
   	Deaspiration is a process where a voiceless stop loses its aspiration (Rule \ref{tab:rule obstr deaspiration}). If a voiceless stop is part of an obstruent cluster, then it generally loses its aspiration. We found this process to be exceptionless in HD's speech for intervocalic clusters /VCCV/, but variable for word-final clusters /VCC\#/. Deaspiration can apply in different types of obstruent clusters, such as fricative-stop, stop-fricative, affricate-stop, stop-affricate, and stop-stop clusters.\footnote{We predict that affricates also undergo deaspiration after fricatives. However, the authors' subdialects (Syrio-Lebanese/USA) do not aspirate affricates in general. So we cannot test this hypothesis for now. Such a hypothesis can be tested with speakers from Turkey who do have aspirated affricates. 
   	}
   	
   	\begin{ruleblock}
     {Deaspiration in obstruent clusters}
     
     \label{tab:rule obstr deaspiration}
     \begin{tabular}{| llll lll| }
     	\hline \multicolumn{7}{| l| }{\textit{Before or after a voiceless obstruent, voiceless stops are deaspirated. }} 
     	\\
     	& \multicolumn{6}{l|}{\textbf{Post-fricative deaspiration}: After a fricative, deaspirate a voiceless stop }
     	\\
     	& & /hɑ\textbf{stʰ}/ & $\rightarrow$ & [ˈhɑ\textbf{st}] & `thick' & \armenian{հաստ}
     	\\
     	& \multicolumn{6}{l|}{\textbf{Pre-fricative deaspiration}: Before a fricative, deaspirate a voiceless stop }
     	\\
     	& & /ɑ\textbf{pʰs}e/ &$\rightarrow$& [ɑ\textbf{pˈs}e] & `tray' & \armenian{ափսէ}
     	\\
     	& \multicolumn{6}{l|}{\textbf{Post-affricate deaspiration}: After an affricate, deaspirate a voiceless stop }
     	\\
     	& & /lu\textbf{t͡skʰi}/ & $\rightarrow$ & [lu\textbf{t͡sˈk}i] & `match (fire)' & \armenian{լուցկի}
     	\\
     	
     	& \multicolumn{6}{l|}{\textbf{Pre-affricate deaspiration}: Before an affricate, deaspirate a voiceless stop }
     	\\
     	& & /ɑ\textbf{kʰt͡s}ɑn/ & $\rightarrow$ & ɑ\textbf{kˈt͡s}ɑn & `switch' & \armenian{աքցան}
     	\\
     	& \multicolumn{6}{l|}{\textbf{Stop-stop deaspiration}: In a stop-stop cluster, deaspirate a voiceless stop }
     	\\
     	& & /t͡sɑ\textbf{tʰkʰ}el/ & $\rightarrow$& [t͡sɑ\textbf{tˈk}el] & `to jump'& \armenian{ցատքել}
     	\\
     	
     	\hline 
     \end{tabular}
   	\end{ruleblock} 
   	
   	
   	
   	
   	
   	Voicing assimilation is when in a cluster of obstruents, the obstruents have to agree in voicing: either both voiced or both voiceless (Rule \ref{tab:rule vociing assiimlation}). When words are derived or morphemes are combined, we can get sequences of obstruents that underlyingly have different voicing qualities. But when pronounced, voiced obstruents become voiceless when next to a voiceless sound, regardless if the underlyingly voiced obstruent is before or after the voiceless obstruent.
   	
   	When the underlyingly voiced sound is after the voiceless found, we find progressive assimilation. When the underlyingly voiced sound is before the voiceless found, we find regressive assimilation. 
   	
   	\begin{ruleblock}
     {Voicing assimilation in obstruent clusters}
     
     \label{tab:rule vociing assiimlation}
     \begin{tabular}{|llllll| }
     	\hline 
     	\multicolumn{6}{|l|}{\textbf{Progressive assimilation: }}
     	\\
     	\multicolumn{6}{|l|}{\textit{After another voiceless obstruent, a voiced obstruent becomes voiceless. }} 
     	\\
     	. & /tʰɑntʰɑ\textbf{ʁ-ɡ}od/ & $\rightarrow$ & [tʰɑntʰɑ\textbf{ʁ-ˈɡ}od] & `slowish' & \armenian{դանդաղկոտ}
     	\\
     	& 
     	/vɑ\textbf{χ-ɡ}od/ & $\rightarrow$ & [vɑ\textbf{χ-ˈk}od] & `cowardly' & \armenian{վախկոտ}
     	\\
     	\hline 
     	\multicolumn{6}{|l|}{\textbf{Regressive assimilation: }} 
     	\\
     	\multicolumn{6}{|l|}{\textit{Before another voiceless obstruent, a voiced obstruent becomes voiceless. }} 
     	\\
     	& /hɑleb/ & $\rightarrow$ & [hɑleb] & `Aleppo' & \armenian{Հալէպ}
     	\\
     	& /hɑle\textbf{b-t͡s}i/ & $\rightarrow$ & [hɑle\textbf{p-ˈt͡s}i] & `Aleppoite' & \armenian{հալէպցի}
     	\\
     	\hline 
     	
     \end{tabular}
     
   	\end{ruleblock}
   	
   	
   	Voicing assimilation and stop deaspiration often opply in the same words, meaning they interact together. In terms of rule interaction, voicing assimilation feeds deaspiration: /hɑle\textbf{b}-t͡si/ $\rightarrow$ //hɑle\textbf{pʰ}-t͡si// $\rightarrow$ [hɑle\textbf{p}-t͡si]. The end result is transparent application of both rules 
   	
   	The two processes can be found in both underived and derived context. A context is underived when the segments involved are part of the same root or morpheme, and they are always pronounced the way are. A context is derived when it's created via combining morphemes or by applying word-internal changes. Because these processes apply in derived contexts, the same morpheme like `Aleppo' [hɑle\textbf{b}] can change its pronunciation depending on the following morpheme: [hɑle\textbf{p}-t͡si] `Aleppoite'. 
   	
   	The following subsections provide examples for both of these processes, both in derived and underived contexts. Data is organized by the subcategory of the process, i.e., post-fricative deaspiration vs. pre-fricative deaspiration. We distinguish between 2-consonant clusters (VCCV) and 3-consonant clusters (VCCCV). We first focus on intervocalic clusters, and then final clusters. 
   	
   	Note that some possible morphological exception to voicing assimilation are the passive suffix \textit{-v-} and reduplication. The passive is discussed in Section \S\ref{section:segmentalPhono:allphonLaryng:assiimlation:v}, while reduplication are discussed in (\textcolor{red}{cite chapter, include vazvezl types and gab-gabujd types, tepteʁin}). 
   	
   	
   	\subsection{Post-fricative deaspiration (intervocalic)}\label{section:segmentalPhono:allphonLaryng:postFricDeasp}
   	We go over post-fricative deaspiration in underived contexts (\S\ref{section:segmentalPhono:allphonLaryng:postFricDeasp:underived}) and derived contexts (\S\ref{section:segmentalPhono:allphonLaryng:postFricDeasp:derived}). 
   	\subsubsection{Underived contexts}\label{section:segmentalPhono:allphonLaryng:postFricDeasp:underived}
   	
   	
   	Word-initially, the orthography has roots that start with a /s/-stop and /ʃ/-stop cluster. In Western Armenian, this cluster undergoes schwa prothesis: /stʰoɾ/ $\rightarrow$ [əstoɾ] `blind'. The sibilant causes the following voiceless stop to deaspirate (Table \ref{tab:post sibilant deaspiration inital cluster underived}). 
   	
   	\begin{table}[H]
     \centering
     \caption{Post-fricative deaspiration in root-initial sibilant-stop clusters (underived contexts)}
     \label{tab:post sibilant deaspiration inital cluster underived} \resizebox{1\textwidth}{!}{%
     	\begin{tabular}{| c| ll| ll| ll| }
  \hline
  /\#ʃtʰV/ 
  & /\textbf{ʃtʰ}ɑb/ & `haste'& /\textbf{ʃtʰ}emɑɾɑn/ 
  & `storehouse'
  & & 
  \\
  {}[əʃtV] & [ə\textbf{ʃˈt}ɑb] && [ə\textbf{ʃt}emɑˈɾɑn]
  & 
  & & 
  \\
  & <\textbf{ʃd}ab> & \armenian{շտապ} & <\textbf{ʃd}emaran> 
  & \armenian{շտեմարան} & & 
  
  \\\hline
  /\#spʰV/ 
  & /\textbf{spʰ}ɑnɑχ/
  & `spinach'
  & /\textbf{spʰ}ɑsoʁ/ 
  & `expectant'
  & /\textbf{spʰ}idɑɡ/ 
  & `white'
  \\
  {}[əspV]
  & [ə\textbf{sp}ɑˈnɑχ]
  & 
  & [ə\textbf{sp}ɑˈsoʁ]
  & 
  & [ə\textbf{sp}iˈdɑɡ]
  & 
  \\
  & <\textbf{sb}anax> & \armenian{սպանախ}
  & <\textbf{sb}asoɣ> & \armenian{սպասող}
  & <\textbf{sb}idag> & \armenian{սպիտակ}
  \\
  \hline
  /\#stʰV/
  & /\textbf{stʰ}ɑnɑl/ 
  & `to receive'
  & /\textbf{stʰ}eʁd͡zel/ 
  & `to create'
  & /\textbf{stʰ}oɾ/ 
  & `blind (n)'
  \\
  {}[əstV] & [ə\textbf{st}ɑˈnɑl]
  & 
  & [ə\textbf{st}eʁˈd͡zel]
  & 
  & [ə\textbf{sˈt}oɾ]
  & 
  \\
  & <\textbf{sd}anal> & \armenian{ստանալ}
  & <\textbf{sd}eɣd͡zel> & \armenian{ստեղծել}
  & <\textbf{sd}or> & \armenian{ստոր}
  \\
  \hline
  /\#skʰV/ & 
  /\textbf{skʰ}uʃ/ 
  & `careful'
  & /\textbf{skʰ}ɑl/ 
  & `to feel'
  & /\textbf{skʰ}estʰ/ 
  & `clothing'
  \\
  {}[əskV]& [ə\textbf{sˈk}uʃ]
  & 
  & [ə\textbf{sˈk}ɑl]
  &
  & [ə\textbf{sˈk}est]
  & 
  \\
  & <\textbf{zk}oyʃ> & \armenian{զգոյշ}
  & <\textbf{zk}al> & \armenian{զգալ}
  & <\textbf{zk}esd> & \armenian{զգեստ}
  \\
  \hline
     	\end{tabular}
     } 
   	\end{table}
   	
   	There is evidence that the cluster lacks a schwa in the underlying or lexical representation. For discussion of schwa epenthesis and prothesis, see (\textcolor{red}{cite schwa epenthesis chapter}). Furthermore, many of these clusters orthographically have voiced stops like \armenian{տ} <d> in \armenian{ստոր} <sdor> [əstoɾ] `blind'. This voiced stop is just an orthographic residue of diachronic sound changes. There is no synchronic evidence that the post-fricative stops in Table \ref{tab:post sibilant deaspiration inital cluster underived} are underlyingly voiced. For discussion of voicing mismatches with the orthography, see Section \S\ref{section:ortho:mismatch:clusters}. 
   	
   	
   	We likewise find deaspiration in word-medial intervocalic contexts (Table \ref{tab:post sibilant deaspiration intervocalic cluster underived}). 
   	
   	\begin{table}[H]
     \centering
     \caption{Post-fricative deaspiration in word-medial intervocalic sibilant-stop clusters (underived contexts) }
     \label{tab:post sibilant deaspiration intervocalic cluster underived}
     
     \begin{tabular}{| c| ll| ll| }
     	\hline
     	
     	/VspʰV/
     	& /te\textbf{spʰ}ɑrez/ & `ambassador' 
     	& /a\textbf{spʰ}ɑˈrez/ & `career' 
     	\\
     	{}[VspV]
     	& [te\textbf{sˈp}ɑn] &
     	& [ɑ\textbf{sp}ɑˈɾez] & 
     	\\
     	& <te\textbf{sb}an> & \armenian{դեսպան}& <a\textbf{sb}arēz> & \armenian{ասպարէզ} \\
     	\hline
     	/VstʰV/
     	& /ɑ\textbf{stʰ}ɑɾ/ & `lining' 
     	& /ɑ\textbf{stʰ}iʒɑn/ & `degree' 
     	\\
     	{}[VstV]
     	& [ɑ\textbf{sˈt}ɑɾ] & 
     	& [ɑ\textbf{sˈt}iʒɑn] & 
     	\\
     	& <a\textbf{sd}a\.{r}> & \armenian{աստառ} & <a\textbf{sd}id͡ʒan>
     	& \armenian{աստիճան} \\
     	\hline
     	/VskʰV/
     	& /ɑ\textbf{skʰ}-eɾ/ & `nation-{\pl}'
     	& /ɡɑ\textbf{skʰ}ɑd͡z/ & `doubt' 
     	\\
     	{}[VskV] & [ɑ\textbf{sˈk}eɾ] &
     	& [ɡɑ\textbf{sˈk}ɑd͡z] & 
     	\\
     	& <a\textbf{zk}er> & \armenian{ազգեր}
     	& <ga\textbf{sg}ad͡z>& \armenian{կասկած}
     	\\
     	\hline 
     	/VʃtʰV/ 
     	& /hɑ\textbf{ʃtʰ}el/ 
     	& `to reconcile'
     	& /ʃe\textbf{ʃtʰ}oʁ/ 
     	& `stressing'
     	\\
     	{}[VʃtV]
     	& [hɑ\textbf{ʃˈt}el] &
     	& [ʃe\textbf{ʃˈt}oʁ] & 
     	
     	\\
     	& <ha\textbf{ʃd}el> & \armenian{հաշտել}
     	& <ʃe\textbf{ʃd}oɣ> & \armenian{շեշտող}
     	\\
     	\hline 
     	/VχtʰV/ 
     	& /hɑ\textbf{χtʰ}ɑnkʰ/ 
     	& `victory'
     	& /χe\textbf{χtʰ}el/ 
     	& `to strangle'
     	\\
     	{}[VχtV]
     	& [hɑ\textbf{χˈt}ɑŋkʰ] & 
     	& [χe\textbf{χˈt}el] & 
     	\\
     	& <ha\textbf{xd}ank'> & \armenian{յաղթանք}
     	& <xe\textbf{ɣd}el> & \armenian{խեղդել}
     	\\
     	\hline
     \end{tabular}
   	\end{table}
   	
   	\subsubsection{Derived contexts }\label{section:segmentalPhono:allphonLaryng:postFricDeasp:derived}
   	
   	Table \ref{tab:post sibilant deaspiration derived not k suffix} list some of the few examples that we found for intervocalic post-fricative deaspiration. We find deaspiration. Table \ref{tab:post sibilant deaspiration derived not k suffix} lists cases where deaspiration is caused by adding a suffix with a voiced stop /ɡ, b/ after a voiceless fricative. The fricative triggers deaspiration. Because the stop is underlyingly voiced, then it is devoiced and losses prevoicing.
   	
   	\begin{table}[H]
     \centering
     \caption{Post-fricative deaspiration from suffixation in intervocalic (VCCV) derived contexts } \label{tab:post sibilant deaspiration derived not k suffix}
     
     \begin{tabular}{| lll l| }
     	\hline
     	Morpheme 1: & məˈ\textbf{s}-il & `to shiver' & \armenian{մսիլ}
     	\\
     	Morpheme 2: & /-\textbf{ɡ}od/ & derivational suffix& 
     	\\
     	& [tʰɑntʰɑʁ-ˈ\textbf{ɡ}od]& `slowish' & \armenian{դանդաղկոտ}
     	\\
     	$\rightarrow$ & /mə\textbf{s-ɡ}od/ & & 
     	\\
     	& mə\textbf{s-ˈk}od & `chilly' & \armenian{մսկոտ}
     	\\ \hline 
     	Morpheme 1: & pɑmpɑˈ\textbf{s}-el & `to gossip' & \armenian{բամբասել}
     	\\
     	Morpheme 2: & /-\textbf{ɡ}od/ & derivational suffix & 
     	\\
     	$\rightarrow$ & /pɑmpɑ\textbf{s-ɡ}od/ & & 
     	\\
     	& pɑmpɑ\textbf{s-ˈk}od & `gossipy' & \armenian{բամբասկոտ}
     	\\ \hline 
     	Morpheme 1: & [bɑˈ\textbf{h}-el] & `to keep' & \armenian{պահել}
     	\\
     	Morpheme 2:& /-\textbf{b}ɑn/ & \multicolumn{2}{l| }{nominalizing suffix for guarding }
     	\\
     	& [t͡ʃʰəɾ-ˈ\textbf{b}ɑn] & `irrigation official' & \armenian{ջրպան}
     	\\
     	Compare & t͡ʃʰuɾ & `water' & \armenian{ջուր}
     	\\
     	$\rightarrow$ & /bɑ\textbf{h-b}ɑn/ & & 
     	\\
     	& bɑ\textbf{h-ˈp}ɑn & `guardian' & \armenian{պահպան}
     	\\
     	\hline 
     \end{tabular}
   	\end{table}
   	
   	Other derived contexts include vowel reduction (Table \ref{tab:post sibilant deaspiration derived not k vowel reduction}). The root consists of an underlying fricative-vowel-stop sequence. In compounds, the vowel is deleted,\footnote{The deletion is due to destressed high vowel reduction. See (\textcolor{red}{cite chapter reduction}).} and thus causing the fricative to precede the stop. The fricative deaspirates the stop, while the stop devoices the fricative. 
   	
   	
   	\begin{table}[H]
     \centering
     \caption{Post-fricative deaspiration from vowel reduction in intervocalic (VCCV) derived contexts } \label{tab:post sibilant deaspiration derived not k vowel reduction}
     
     \begin{tabular}{| l ll l| }
     	\hline
     	& [kʰɑˈvi\textbf{tʰ}] & `courtyard' & \armenian{գաւիթ}
     	\\
     	$\rightarrow$ & /kʰɑ\textbf{vitʰ}-ɑ-bɑh/ & \multicolumn{2}{l| }{with compound linker /-ɑ-/ }
     	\\
     	& [kʰɑ\textbf{ft}-ɑ-ˈbɑh] & `gatekeeper' & \armenian{գաւթապահ}
     	\\
     	\hline \end{tabular}
   	\end{table}
   	
   	
   	There are some derivational prefixes that end in a voiced fricative: synonymous [tʰəʒ-] or [dəʒ-] (Table \ref{tab:post fric deasp prefix tz}). The fricative causes deaspiration of the following stop, and then gets devoiced. 
   	
   	\begin{table}[H]
     \centering
     \caption{Post-fricative deaspiration after the derivational prefixes in VCCV contexts}
     \label{tab:post fric deasp prefix tz}
     \begin{tabular}{|lllll| }
     	\hline 
     	& & ˈ\textbf{pʰ}ɑχt & `luck' & \armenian{բախտ}
     	\\
     	$\rightarrow$ & /tə\textbf{ʒ-pʰ}ɑχt/ &tʰə\textbf{ʃ-ˈp}ɑχt & `unlucky' & \armenian{դժբախտ}
     	\\
     	$\rightarrow$ & /də\textbf{ʒ-pʰ}ɑχt/ &də\textbf{ʃ-ˈp}ɑχt & `unlucky' & \armenian{տժբախտ}
     	\\ \hline 
     	& & ˈ\textbf{kʰ}oh & `satisfied' & \armenian{գոհ}
     	\\
     	$\rightarrow$ & /tə\textbf{ʒ-kʰ}oh/ &tʰə\textbf{ʃ-ˈk}oh & `dissatisfied' & \armenian{դժգոհ}
     	\\ 
     	$\rightarrow$ & /də\textbf{ʒ-kʰ}oh/ &də\textbf{ʃ-ˈk}oh & `dissatisfied' & \armenian{տժգոհ}
     	\\ \hline 
     	& & ˈ\textbf{kʰ}ujn & `color' & \armenian{գոյն}
     	\\
     	$\rightarrow$ & /tə\textbf{ʒ-kʰ}ujn/ &tʰə\textbf{ʃ-ˈk}ujn & `discolored' & \armenian{դժգոյն}
     	\\
     	$\rightarrow$ & /də\textbf{ʒ-kʰ}ujn/ &də\textbf{ʃ-ˈk}ujn & `discolored' & \armenian{տժգոյն}
     	\\ \hline 
     	
     \end{tabular}
   	\end{table}
   	
   	
   	The above data concerned VCCV clusters where the cluster consists of only two consonants. We can also find a few cases where the cluster has 3 consonants VC\textsubscript{1}C\textsubscript{2}.C\textsubscript{3}V such that the C\textsubscript{1}C\textsubscript{2} form a complex coda (Table \ref{tab:post fric deaspiration vcccv }). Here again, we find deaspiration of C\textsubscript{3}. If C\textsubscript{2} is a voiceless obstruent and C\textsubscript{3} is voiced obstruent, we also find devoicing (loss of prevoicing). 
   	
   	
   	\begin{table}[H]
     \centering
     \caption{Post-fricative deaspiration in intervocalic VCCCV derived contexts} \label{tab:post fric deaspiration vcccv }
     
     \begin{tabular}{| l lll| }
     	\hline 
     	
     	& bɑɾˈ\textbf{siɡ} & `Persian person' & \armenian{պարսիկ}
     	\\
     	& [bɑ\textbf{ɾs.k}-eˈɾen] & `Persian language' & \armenian{պարսկերէն}
     	\\ \hline
     	& ˈɑ\textbf{js} + ˈ\textbf{d}eʁ & `this' + `place' & \armenian{այս, տեղ}
     	\\
     	$\rightarrow$ & ɑ\textbf{js-ˈt}eʁ & `here, this place' & \armenian{այստեղ}
     	\\ \hline
     	& ˈɑ\textbf{js} + ˈ\textbf{kʰ}ɑn & `this' + `much' & \armenian{այս, քան}
     	\\
     	$\rightarrow$ & ɑ\textbf{js-ˈk}ɑn & `this much' & \armenian{այսքան}
     	\\ \hline
     \end{tabular}
   	\end{table}
   	
   	
   	
   	
   	
   	\subsection{Pre-fricative deaspiration (intervocalic)}\label{section:segmentalPhono:allphonLaryng:preFricDeasp}
   	Compared to post-fricative stops, it is relatively rarer to find pre-fricative stops. In the few types of examples that we find, we also find pre-fricative deaspiration. 
   	
   	It is relatively rare to find roots with stop-fricative clusters (Table \ref{tab:pre fric deasp underived}). Such clusters can be voiced or voiceless. When voiceless, the stop is unaspirated. 
   	
   	
   	\begin{table}[H]
     \centering
     \caption{Pre-fricative deaspiration in VCCV clusters in underived contexts }
     \label{tab:pre fric deasp underived}
     \begin{tabular}{|llll| }
     	\hline 
     	/di\textbf{dʁ}os/ & di\textbf{dˈʁ}os & `title' & \armenian{տիտղոս}
     	\\
     	/ɑ\textbf{pʰs}e/ & ɑ\textbf{pˈs}e & `tray' & \armenian{ափսէ}
     	\\
     	/ɑ\textbf{pʰʃ}-um/ & ɑ\textbf{pˈʃ}um & `surprise' & \armenian{ապշում}
     	\\
     	/ɑ\textbf{pʰs}os/ & ɑ\textbf{pˈs}os & `alas' & \armenian{ափսոս}
     	\\
     	/ɑ\textbf{kʰs}or/ & ɑ\textbf{kˈs}os & `exile' & \armenian{աքսոր}
     	\\
     	/tʰɑ\textbf{tʰχ}-el/ & tʰɑ\textbf{tˈχ}el & `to wet' & \armenian{թաթխել}
     	\\
     	
     	\hline 
     \end{tabular}
   	\end{table}
   	
   	
   	
   	
   	
   	
   	
   	
   	
   	
   	Many contexts for pre-fricative deaspiration come from derived contexts, such as come from vowel reduction of destressed /i,u/ (Table \ref{tab:pre fric deasp vc(c)cv reduction}). Some cases come from reduction or syncope of /ɑ/ in some roots. 
   	
   	\begin{table}[H]
     \centering
     \caption{Pre-fricative deaspiration from vowel reduction and syncope in VC(C)CV derived contexts }
     \label{tab:pre fric deasp vc(c)cv reduction}
     \begin{tabular}{|llll| }
     	\hline 
     	& seˈ\textbf{buh} & `gentleman' & \armenian{սեպուհ}
     	\\
     	$\rightarrow$ & se\textbf{ph}-ɑˈɡɑn & `appropriate' & \armenian{սեպհական}
     	\\ \hline 
     	& bɑˈ\textbf{ɡɑs} & `missing' & \armenian{պակաս}
     	\\
     	$\rightarrow$ & bɑ\textbf{kˈs}-il & `to lessen' & \armenian{պակսիլ}
     	\\ \hline 
     	& ʃɑ\textbf{mˈpʰuʃ} & `foolish' & \armenian{շամբուշ}
     	\\
     	$\rightarrow$ & ʃɑ\textbf{mp.ʃ}-ɑˈɡɑn & `foolish' & \armenian{շամբշական}
     	\\ \hline 
     	& ə\textbf{mˈpʰiʃ} & `athlete' & \armenian{ըմբիշ}
     	\\
     	$\rightarrow$ & ə\textbf{mp.ʃ}-ɑˈɡɑn & `athletic' & \armenian{ըմբշական}
     	\\ \hline 
     	& kʰə\textbf{ŋˈkʰuʃ} & `delicate' & \armenian{քնքուշ}
     	\\
     	$\rightarrow$ & kʰə\textbf{ŋk.ˈʃ}-ɑŋkʰ & `delicacy' & \armenian{քնքշանք}
     	\\ \hline 
     	& ɡə\textbf{ŋˈkʰuʁ} & `hood' & \armenian{կնգուղ}
     	\\
     	$\rightarrow$ & ɡə\textbf{ŋk.χ}-ɑˈvoɾ & `Capuchin friar' & \armenian{կնգղաւոր}
     	\\ \hline 
     \end{tabular}
   	\end{table}
   	
   	Another context is the names for the days of the week (Table \ref{tab:pre fric deasp days week}). These names are compounds without the linking vowel /-ɑ-/. Note that the second root in these compounds changes its form, likely a type of allomorphy.
   	
   	
   	\begin{table}[H]
     \centering
     \caption{Pre-fricative deaspiration in compounds for days of the week }
     \label{tab:pre fric deasp days week}
     \begin{tabular}{|llll| }
     	\hline 
     	& jeˈɾe\textbf{kʰ} + \textbf{ʃ}ɑˈ\textbf{pʰɑtʰ} & `two' + `week' & \armenian{երեք, շաբաթ}
     	\\
     	$\rightarrow$ & jeˈɾe\textbf{k-ʃ}ɑ\textbf{pˈt}i & `Tuesday' & \armenian{երեքշաբթի}
     	\\\hline 
     	& /t͡ʃoɾe\textbf{kʰ}-/ + \textbf{ʃ}ɑˈ\textbf{pʰɑtʰ} & `quatro-' + `week' & 
     	\\
     	$\rightarrow$ & t͡ʃoɾe\textbf{k-ʃ}ɑ\textbf{pˈt}i & `Wednesday' & \armenian{չորեքշաբթի}
     	\\\hline 
     	& ˈhi\textbf{ŋkʰ} + \textbf{ʃ}ɑˈ\textbf{pʰɑtʰ} & `five' + `week' & \armenian{հինգ, շաբաթ}
     	\\
     	$\rightarrow$ & hi\textbf{ŋk-ʃ}ɑ\textbf{pˈt}i & `Thursday' & \armenian{հինգշաբթի}
     	\\
     	\hline 
     \end{tabular}
   	\end{table}
   	
   	Some numerals likewise show pre-fricative deaspiration (Table \ref{tab:pre fric deasp number}). These are formed from a root and decade suffix /-sun/.
   	
   	
   	\begin{table}[H]
     \centering
     \caption{Pre-fricative deaspiration in some numerals}
     \label{tab:pre fric deasp number}
     \begin{tabular}{|llll| }
     	\hline 
     	& ˈu\textbf{tʰ} + /-sun/ & `eight' + `-th' & \armenian{ութ}
     	\\
     	$\rightarrow$ & u\textbf{t-ˈs}un & `eighty' & \armenian{ութսուն}
     	\\\hline 
     	Base &/vɑ\textbf{tʰ}/ + /-sun/ & `hexa-' + `-th' & 
     	\\
     	$\rightarrow$ & vɑ\textbf{t-ˈs}un & `sixty' & \armenian{վաթսուն}
     	\\\hline 
     	
     \end{tabular}
   	\end{table}
   	
   	Another context is passivization (Table \ref{tab:pre fric deasp passive }). The passive suffix /-v-/ is devoiced after voiceless stops and triggers deaspiration. Note that the devoicing of passive /-v-/ is complicated. See Section \S\ref{section:segmentalPhono:allphonLaryng:assiimlation:v}. The gloss for active verbs is $\sqrt{}$-{\thgloss}-{\infgloss}, while for passives it's $\sqrt{}$-{\pass}-{\thgloss}-{\infgloss}.
   	
   	
   	
   	
   	\begin{table}[H]
     \centering
     \caption{Pre-fricative deaspiration from passivization in VCCV derived contexts }
     \label{tab:pre fric deasp passive }
     \begin{tabular}{|llll| }
     	\hline 
     	Base &tʰɑˈ\textbf{pʰ}-e-l & `to throw away' & \armenian{թափել}
     	\\
     	$\rightarrow$ & tʰɑ\textbf{p-ˈv̥}-i-l & `to be thrown away' & \armenian{թափուիլ}
     	\\ \hline 
     	Base &χɑˈ\textbf{pʰ}-e-l & `to trick' & \armenian{խաբել}
     	\\
     	$\rightarrow$ & χɑ\textbf{p-ˈv̥}-i-l & `to be tricked' & \armenian{խաբուիլ}
     	\\ \hline 
     	& noɾoˈ\textbf{kʰ}-e-l & `to restore' & \armenian{նորոգել}
     	\\
     	$\rightarrow$ & noɾo\textbf{k-ˈv̥}-i-l & `to be restored' & \armenian{նորոգուիլ}
     	\\ \hline 
     	& hɑvɑˈ\textbf{kʰ}-e-l & `to gather' & \armenian{հաւաքել}
     	\\
     	$\rightarrow$ & hɑvɑ\textbf{k-ˈv̥}-i-l & `to be gathered' & \armenian{հաւաքուիլ}
     	\\ \hline 
     \end{tabular}
   	\end{table}
   	
   	
   	
   	
   	
   	
   	
   	
   	\subsection{Post-affricate deaspiration (intervocalic)}\label{section:segmentalPhono:allphonLaryng:postAffrDeasp}
   	Similar to post-fricative deaspiration, we also have post-affricate aspiration. There are fewer examples of this process though. 
   	
   	So far, we've only found one example of post-affricate deaspiration in a root (Table \ref{tab:post affr deaspiration vccv underived }). 
   	
   	\begin{table}[H]
     \centering
     \caption{Post-affricate deaspiration in intervocalic VCCV underived contexts} \label{tab:post affr deaspiration vccv underived }
     
     \begin{tabular}{| llll| }
     	\hline 
     	/lu\textbf{t͡skʰi}/ & lu\textbf{t͡sˈk}i & `match (fire)' & \armenian{լուցկի}
     	\\
     	\hline
     	
     \end{tabular}
   	\end{table}
   	
   	
   	
   	In derived contexts (Table \ref{tab:post affr deaspiration vccv derived intervocalic}), most but not all attested examples of post-affricate deaspiration simultaneously involve devoicing of the stop because the stop was underlyingly voiced. 
   	
   	
   	
   	\begin{table}[H]
     \centering
     \caption{Post-affricate deaspiration in intervocalic VCCV derived contexts} \label{tab:post affr deaspiration vccv derived intervocalic}
     
     \begin{tabular}{| ll lll| }
     	\hline 
     	Aorist stem: & & moɾɑˈ\textbf{t͡s}-ɑd͡z & `forgotten' & \armenian{մոռացած}
     	\\
     	$\rightarrow$ & /moɾɑ\textbf{t͡s-ɡ}od/ & moɾɑ\textbf{t͡s-ˈk}od & `forgetful' & \armenian{մոռացկոտ}
     	\\
     	\hline
     	& & ˈpɑ\textbf{t͡s} + \textbf{pʰ}eˈɾɑn & `open' + `mouth' & \armenian{բաց, բերան}
     	\\
     	$\rightarrow$ & /pʰɑ\textbf{t͡s-pʰ}eɾɑn] & pʰɑ\textbf{t͡s-p}eˈɾɑn & `babbler' & \armenian{բացբերան}
     	\\ 
     	\hline 
     	& & ˈχɑt͡ʃ + ˈkʰɑɾ & `cross' + `stone' & \armenian{խաչ, քար} 
     	\\
     	$\rightarrow$ & /χɑ\textbf{t͡ʃ-kʰ}ɑɾ/ & χɑ\textbf{t͡ʃ-ˈk}ɑɾ & `cross-stone' & 
     	\armenian{խաչքար}
     	\\ \hline
     	
     	& & jeˈɾet͡s + ˈɡin & `elder' + `woman' & \armenian{երէց, կին}
     	\\
     	$\rightarrow$ & /jeɾe\textbf{t͡s-ɡ}in/ & jeɾe\textbf{t͡s-ˈk}in & `pastor's wife' & \armenian{երէցկին}
     	\\\hline
     	
     	& & ˈme\textbf{t͡ʃ} + ˈ\textbf{d}eʁ & `in' + `place' & \armenian{մէջ, տեղ}
     	\\
     	$\rightarrow$ &/met͡ʃ -deʁ/ & me\textbf{t͡ʃ-ˈt}eʁ & `middle' & \armenian{մէջտեղ}
     	\\
     	
     	
     	
     	\hline 
     	
     	& & miˈt͡ʃuɡ &`nucleus' & \armenian{միջուկ} 
     	\\
     	$\rightarrow$ & /mi\textbf{t͡ʃuɡ}-ɑjin/ & mi\textbf{t͡ʃk}-ɑˈjin & `nuclear' & \armenian{միջկային}
     	\\\hline
     	& & kʰeʁeˈ\textbf{t͡siɡ} &`pretty' & \armenian{գեղեցիկ} 
     	\\
     	$\rightarrow$ & /kʰeʁe\textbf{t͡siɡ}-ɑnɑl/ & kʰeʁe\textbf{t͡sk}-ɑˈnɑl & `to become pretty' & \armenian{գեղեցկանալ}
     	\\
     	
     	\hline 
     	
     \end{tabular}
   	\end{table}
   	
   	
   	The above data concerned VCCV clusters where the cluster consists of only two consonants. We can also find cases where the cluster has 3 consonants VC\textsubscript{1}C\textsubscript{2}.C\textsubscript{3}V such that the C\textsubscript{1}C\textsubscript{2} form a complex coda (Table \ref{tab:post affr deaspiration vcccv }). Here again, we find deaspiration of C\textsubscript{3}. If C\textsubscript{2} is a voiceless obstruent and C\textsubscript{3} is voiced obstruent, we also find devoicing (loss of prevoicing). 
   	
   	\begin{table}[H]
     \centering
     \caption{Post-affricate deaspiration in intervocalic VCCCV derived contexts} \label{tab:post affr deaspiration vcccv }
     
     \begin{tabular}{| ll lll| }
     	\hline 
     	& & ɑχˈt͡ʃi\textbf{ɡ} & `girl' & \armenian{աղջիկ}
     	\\
     	$\rightarrow$ & /ɑ\textbf{χt͡ʃiɡ}-ɑɡɑn/ & ɑ\textbf{χt͡ʃ.k}-ɑˈɡɑn & `feminine' & \armenian{աղջկական}
     	\\ \hline 
     	
     	Base & & tʰɑpʰɑnˈt͡si\textbf{ɡ} & `transparent' & \armenian{թափանցիկ} 
     	\\
     	$\rightarrow$ & /tʰɑpʰɑ\textbf{nt͡siɡ}-ɑnɑl/ & tʰɑpʰɑ\textbf{nt͡sk}-ɑˈnɑl & `to become transparent' & \armenian{թափանցկանալ}
     	\\ \hline 
     	
     	& & hoˈɾɑ\textbf{nt͡ʃ} & `yawn' & \armenian{յօրանչ}
     	\\
     	$\rightarrow$ &/hoɾɑ\textbf{nt͡ʃ-ɡ}od/ & hoɾɑ\textbf{nt͡ʃ-ˈk}od & `yawning' & \armenian{յօրանչկոտ}
     	\\ \hline 
     \end{tabular}
   	\end{table}
   	
   	
   	
   	It should be noted that in these intervocalic VCCCV contexts, HD perceives a smaller degree of deaspiration than in VC.CV clusters. But based on observing the spectrograms, such clusters do show deaspiration and devoicing. It is possible that what HD perceives is incomplete neutralization, or that this is just a perceptual illusion. 
   	\subsection{Pre-affricate deaspiration (intervocalic)}\label{section:segmentalPhono:allphonLaryng:preAffrDeasp}
   	Deaspiration can also apply before affricates. 
   	
   	So far, we've only found very few cases of deaspiration in underived contexts (Table \ref{tab:pre aff deasp underived}). 
   	
   	\begin{table}[H]
     \centering
     \caption{Pre-affricate deaspiration in underived contexts}
     \label{tab:pre aff deasp underived}
     \begin{tabular}{|llll| }
     	\hline 
     	/ɑ\textbf{kʰt͡s}ɑn/ & ɑ\textbf{kˈt͡s}ɑn & `switch' & \armenian{աքցան}
     	\\
     	\hline 
     	
     \end{tabular}
   	\end{table}
   	
   	
   	
   	In derived contexts, stops deaspirate before affricates. A common morphological construction that shows this process is causativization. For the roots in Table \ref{tab:pre aff deasp vc(c)cv tsnel}, the root ends in either voiceless aspirated stop or a voiced stop. Aspiration is clearer when the stop is intervocalic. To form a causative verb, the suffix sequence /-t͡sən-e-l/ is added. The affricate causes devoicing and deaspiration.\footnote{The gloss for /t͡sən-e-l/ is [{\caus}-{\thgloss}-{\infgloss}]. } This affects both VCCV and VCCCV clusters
   	
   	
   	\begin{table}[H]
     \centering
     \caption{Pre-affricate deaspiration before the causative suffix /-t͡sən-/}
     \label{tab:pre aff deasp vc(c)cv tsnel}
     \begin{tabular}{|lllll| }
     	\hline 
     	& & ˈmu\textbf{tʰ} & `dark' & \armenian{մութ}
     	\\
     	&& muˈ\textbf{tʰ}-e-n & `dark ({\abl}-{\defgloss})' & \armenian{մութէն}
     	\\
     	$\rightarrow$ & /mu\textbf{tʰ-t͡s}ən-e-l/ &mu\textbf{t-t͡s}əˈn-e-l & `to darken' & \armenian{մութցնել}
     	\\ \hline 
     	& & kʰi\textbf{d}-e-m & `I know ($\sqrt{~}$-{\thgloss}-1{\sg})' & \armenian{գիտեմ}
     	\\
     	$\rightarrow$ & /kʰi\textbf{d-t͡s}ən-e-l/ &kʰi\textbf{t-t͡s}əˈn-e-l & `to notify' & \armenian{գիտցնել}
     	\\ \hline 
     	& & ˈdɑ\textbf{kʰ} & `hot' & \armenian{տաք}
     	\\
     	&& dɑˈ\textbf{kʰ}-e-n & `hot ({\abl}-{\defgloss})' & \armenian{տաքէն}
     	\\
     	$\rightarrow$ & /dɑ\textbf{kʰ-t͡s}ən-e-l/ &dɑ\textbf{k-t͡s}əˈn-e-l & `to heat' & \armenian{տաքցնել}
     	\\ \hline 
     	& & lo\textbf{kʰ}-ɑŋkʰ & `bath ($\sqrt{~}$-{\nmlz})' & \armenian{լոգանք}
     	\\
     	$\rightarrow$ & /lo\textbf{kʰ-t͡s}ən-e-l/ &lo\textbf{k-t͡s}əˈn-e-l & `to notify' & \armenian{լոգցնել}
     	\\ \hline 
     	& & χoˈɾu\textbf{ŋɡ} & `deep' & \armenian{խորունկ}
     	\\
     	$\rightarrow$ & /χoɾu\textbf{nɡ-t͡s}ən-e-l/ &χoˈɾu\textbf{ŋk-t͡s}əˈn-e-l & `to deepen' & \armenian{խորունկցնել}
     	\\ \hline 
     	
     \end{tabular}
   	\end{table}
   	
   	
   	
   	Another case involves irregular verbs with an affricate infix. The roots in Table \ref{tab:pre aff deasp vccv tSil} have an underlying voiced stop. This voiced stop surfaces in some inflected forms when the stop is intervocalic. In their citation or infinitive form, the root takes a meaningless affricate suffix /-t͡ʃ-/ suffix that triggers devoicing and deaspiration on the root's consonant. 
   	
   	
   	\begin{table}[H]
     \centering
     \caption{Pre-affricate deaspiration before the meaningless infix /-t͡ʃ-/}
     \label{tab:pre aff deasp vccv tSil}
     \begin{tabular}{|llllll| }
     	\hline 
     	& & tʰəˈ\textbf{b}-ɑ-v & `he touched'&$\sqrt{~}$-{\pst}-3{\sg} & \armenian{դպաւ}
     	\\
     	$\rightarrow$ & /tʰə\textbf{b-t͡ʃ}-i-l/ &tʰə\textbf{p-ˈt͡ʃ}-i-l & `to touch'& $\sqrt{~}${\thgloss}-{\infgloss} & \armenian{դպչիլ}
     	\\ \hline 
     	& & pʰɑˈ\textbf{ɡ}-ɑ-v & `he stuck to'&$\sqrt{~}$-{\pst}-3{\sg} & \armenian{փակաւ}
     	\\
     	$\rightarrow$ & /pʰɑ\textbf{ɡ-t͡ʃ}-i-l/ &pʰɑ\textbf{k-ˈt͡ʃ}-i-l & `to stick to'& $\sqrt{~}${\thgloss}-{\infgloss} & \armenian{փակչիլ}
     	\\ \hline 
     	
     \end{tabular}
   	\end{table}
   	
   	
   	Vowel reduction can likewise create contexts for pre-affricate devoicing. All the examples in Table \ref{tab:pre aff deasp vc(c)cv red} involve roots that take a derivational suffix, either /-it͡ʃ/ or /-ɡit͡s/. When another derivational suffix is added, the vowel of /-it͡ʃ/ is deleted, causing the consonants to become adjacent and trigger deaspiration. 
   	
   	
   	\begin{table}[H]
     \centering
     \caption{Pre-affricate deaspiration from vowel reduction in VC(C)CV contexts }
     \label{tab:pre aff deasp vc(c)cv red}
     \begin{tabular}{|llll| }
     	\hline 
     	& noɾoˈ\textbf{kʰ-it͡ʃ} & `reformer' & \armenian{նորոգիչ}
     	\\
     	$\rightarrow$ &noɾo\textbf{k-t͡ʃ}-ɑˈɡɑn & `of reforms (adj.)' & \armenian{նորոգչական}
     	\\ \hline 
     	& nəvɑˈ\textbf{kʰ-it͡ʃ} & `musician' & \armenian{նուագիչ}
     	\\
     	$\rightarrow$ &nəvɑ\textbf{k-t͡ʃ}-uˈhi & `fem. musician' & \armenian{նուագչուհի}
     	\\ \hline 
     	& mijɑ-ˈ\textbf{ɡit͡s} & `united' & \armenian{միակից}
     	\\
     	$\rightarrow$ &mijɑ-\textbf{kt͡s}-uˈtʰjʏn & `junction' & \armenian{միակցութիւն}
     	\\ \hline 
     	& χəntʰɑ-ˈ\textbf{ɡit͡s} & `united' & \armenian{խնդակից}
     	\\
     	$\rightarrow$ &χəntʰɑ-\textbf{kt͡s}-ɑˈɡɑn & `junction' & \armenian{խնդակցական}
     	\\ \hline 
     	& je\textbf{ɾˈk-it͡ʃ} & `singer' & \armenian{երգիչ}
     	\\
     	$\rightarrow$ &je\textbf{ɾk-t͡ʃ}-uˈhi & `fem. singer' & \armenian{երգչուհի}
     	\\ \hline 
     	
     \end{tabular}
   	\end{table}
   	
   	
   	
   	
   	
   	
   	\subsection{Deaspiration in stop-stop clusters (intervocalic)}\label{section:segmentalPhono:allphonLaryng:StopStopDeasp}
   	
   	In stop-stop clusters, voiceless stops are deaspirated in both underived and derived contexts. 
   	
   	In underived contexts (Table \ref{tab: stop stop deaspiration underived}), it is relatively difficult to find intervocalic stop-stop clusters. There are some roots and bound roots which have such clusters. Here, we find that the two stops are either both voiced stops or both voiceless unaspirated stops. We do not find intervocalic clusters where either of the stops is aspirated.  
   	
   	\begin{table}[H]
     \centering
     \caption{Intervocalic stop-stop restrictions in underived contexts }\label{tab: stop stop deaspiration underived}
     {% \resizebox{1\textwidth}{!}{%
  \begin{tabular}{| ll l l| }
  	\hline
  	
  	Voiced + Voiced & jekʰi\textbf{bˈd}os & `Egypt'& \armenian{Եգիպտոս}
  	\\
  	& ɑ\textbf{bˈd}ɑɡ & `slap'& \armenian{ապտակ}
  	\\
  	& pʰe\textbf{ɡˈd}-el & `to break'& \armenian{բեկտել}
  	\\
  	
  	Voiceless + Voiceless & bɑ\textbf{tˈk}ɑm & `message'& \armenian{պատգամ}
  	\\
  	& t͡sɑ\textbf{tˈk}-el & `to jump'& \armenian{ցատքել}
  	\\
  	* Aspirated + Stop & *bɑ\textbf{tʰɡ}ɑm & & 
  	\\
  	& *bɑ\textbf{tʰkʰ}ɑm & & 
  	\\
  	* Stop + Aspirated & *bɑ\textbf{ɡtʰ}ɑm & & 
  	\\
  	\hline 
  \end{tabular}
     	}
     	
     \end{table}
     
     
     More cases are found in derived contexts, when a voiceless aspirated stop becomes adjacent to another stop. For the derived contexts in Table \ref{tab: stop stop deaspiration k-g vccv}, the voiceless stop precedes an underlyingly voiced stop. The voiced stop devoices due to voicing assimilation. Both stops become voiceless, and neither of them is aspirated. 
     
     \begin{table}[H]
     	\centering
     	\caption{Stop-stop deaspiration in derived /VCCV/ contexts where the first is voiceless and second is voiced }\label{tab: stop stop deaspiration k-g vccv}
     	{% \resizebox{1\textwidth}{!}{%
  	\begin{tabular}{| ll l l| }
    \hline
    
    Bases & ˈtʰɑkʰ + ˈɡɑb & `crown' + `link' & \armenian{թագ, կապ}
    \\ 
    $ \rightarrow$ & tʰɑ\textbf{k-k}ɑb-uˈtʰun & `coronation' & \armenian{թագկապութիւն}
    \\ \hline 
    & χɑˈnu\textbf{tʰ} & `store' & \armenian{խանութ}
    \\
    $\rightarrow$ & χɑnu\textbf{t-ˈp}ɑn & `shopkeeper' & \armenian{խանութպան}
    \\ \hline 
    & oˈ\textbf{kʰud} & `benefit' & \armenian{օգուտ}
    \\
    $\rightarrow$ & o\textbf{kt}-ɑˈɡɑɾ & `useful' & \armenian{օգտակար}
    \\ \hline 
    
  	\end{tabular}
  }
  
     	\end{table}
     	
     	This constraint is likewise active in /VCCCV/ contexts. For the words in Table \ref{tab: stop stop deaspiration k-g vcccv}, the voiceless stop is underlyingly aspirated and part of a complex coda. In derivation, the voiceless stop precedes a voiced stop. Both stops become voiceless unaspirated. 
     	
     	\begin{table}[H]
  \centering
  \caption{Stop-stop deaspiration in derived /VCCCV/ contexts where the second is voiceless and third is voiced }\label{tab: stop stop deaspiration k-g vcccv}
  {% \resizebox{1\textwidth}{!}{%
    \begin{tabular}{| ll l l| }
    	\hline
    	& pʰɑˈ\textbf{pʰuɡ} & `delicate' & \armenian{փափուկ}
    	\\
    	$\rightarrow$ & pʰɑ\textbf{p.k}-ɑˈɡɑn & `delicate' & \armenian{փափկական} 
    	\\ \hline 
    	& tʰə\textbf{mˈpʰuɡ} + hɑˈɾ-el & `drum' + `to beat' & \armenian{թմբուկ, հարել}
    	\\
    	$\rightarrow$ & tə\textbf{mp.k}-ɑ-ˈhɑɾ & `drummer' & \armenian{թմբկահար}
    	\\ \hline 
    	& kʰə\textbf{nˈtʰiɡ} + ˈkʰɑɾ & `marble' + `rock' & \armenian{գնդիկ, քար}
    	\\
    	$\rightarrow$ & kʰə\textbf{nt.k}-ɑ-ˈkʰɑɾ & `globulite' & \armenian{գնդկաքար}
    	\\ \hline 
    	& sə\textbf{nˈtʰiɡ} & `mercury' & \armenian{սնդիկ}
    	\\
    	$\rightarrow$ & sə\textbf{nt.k}-ɑˈjin & `mercurial' & \armenian{սնդկային} 
    	\\ \hline 
    	
    	
    \end{tabular}
  	}
  	
  \end{table}
  
  
  
  \subsection{Deaspiration in final clusters}\label{section:segmentalPhono:allphonLaryng:finalDeasp}
  The previous sections focused on deaspiration in word-medial contexts. Word-finally, we think there is deaspiration, in both underived contexts (\S\ref{section:segmentalPhono:allphonLaryng:finalDeasp:underived}) and derived contexts (\S\ref{section:segmentalPhono:allphonLaryng:finalDeasp:derived}).
  
  But there is some variation. The main problem is that it's difficult to accurately measure aspiration in word-final stops. When a word was said in isolation, we noticed an audible release and some degree of noise: [ɑskʰ] `nation'. But it's not always clear if this `noise' is aspiration or just noise. Once these final clusters were placed in the middle of a sentence (before a vowel), then the aspiration was gone [ɑsk əsɑv] `he said nation'. For consistency, we transcribe these final clusters as deaspirated even for words in isolation. 
  
  \subsubsection{Underived contexts}\label{section:segmentalPhono:allphonLaryng:finalDeasp:underived}
  Among underived contexts, deaspiration can apply in word-final clusters (Table \ref{tab:post sibilant deaspiration final cluster underived}).\footnote{To understand why the word `lament' is underlyingly /(v)oχpʰ/ with (v), see (\textcolor{red}{cite chapter diphthongization}).} 
  
  \begin{table}[H]
  	\centering
  	\caption{Post-fricative deaspiration in word-final sibilant-stop clusters (underived contexts) }
  	\label{tab:post sibilant deaspiration final cluster underived}
  	\resizebox{1\textwidth}{!}{%
    \begin{tabular}{| c| ll l|l ll| }
    	\hline
    	/Vspʰ\#/
    	& /bɑɾi\textbf{spʰ}/ & `fortress'& \armenian{պարիսպ}
    	& /zu\textbf{spʰ}/ &`restrained' & \armenian{զուսպ}
    	\\
    	{}[Vsp\#] 
    	& [bɑˈɾi\textbf{sp}] & & <bari\textbf{sb}> 
    	& [ˈzu\textbf{sp}] & & <zow\textbf{sb}>
    	\\ \hline
    	/Vstʰ\#/
    	& /bɑdɾɑ\textbf{stʰ}/ & `ready'& \armenian{պատրաստ}
    	& /χi\textbf{stʰ}/ &`strict' & \armenian{խիստ}
    	\\
    	{}[Vst\#] 
    	& [bɑdˈɾɑ\textbf{st}] & & <badra\textbf{sd}> 
    	& [ˈχi\textbf{st}] & & <xi\textbf{sd}>
    	\\ \hline
    	/Vskʰ\#/
    	& /ɑ\textbf{skʰ}/ & `nation'& \armenian{ազգ}
    	& /i\textbf{skʰ}/ &`but' & \armenian{իսկ}
    	\\
    	{}[Vsk\#] 
    	& [ˈɑ\textbf{sk}] & & <a\textbf{zk}> 
    	& [ˈi\textbf{sk}] & & <xi\textbf{sg}>
    	\\ \hline
    	/Vʃtʰ\#/
    	& /d͡ʒi\textbf{ʃtʰ}/ & `correct'& \armenian{ճիշդ}
    	& /ɡu\textbf{ʃtʰ}/ &`sated' & \armenian{կուշտ}
    	\\
    	{}[Vʃt\#] 
    	& [ˈd͡ʒi\textbf{ʃt}] & & <d͡ʒi\textbf{ʃt}> 
    	& [ˈɡu\textbf{ʃt}] & & <gow\textbf{ʃd}>
    	\\ \hline
    	/Vχpʰ\#/
    	& /ɑ\textbf{χpʰ}/ & `trash'& \armenian{աղբ}
    	& /(v)o\textbf{χpʰ}/ &`lament' & \armenian{ողբ}
    	\\
    	{}[Vχp\#] 
    	& [ˈɑ\textbf{χp}] & & <a\textbf{ɣp}> 
    	& [ˈvo\textbf{χp}] & & <xi\textbf{sg}>
    	\\ \hline
    	/Vχtʰ\#/
    	& /ɑ\textbf{χtʰ}/ & `disease'& \armenian{ախտ}
    	& /kʰɑ\textbf{χtʰ}/ &`emigration' & \armenian{գաղթ}
    	\\
    	{}[Vχt\#] 
    	& [ˈɑ\textbf{χt}] & & <a\textbf{xd}> 
    	& [ˈkɑ\textbf{χt}] & & <ka\textbf{ɣt'}>
    	\\ \hline
    	
    \end{tabular}
  	}
  	
  \end{table}
  
  Note that although we transcribe these stops as deaspirated, it is possible that there is some degree of aspiration. The degree of aspiration for final post-fricative stops seems to be weaker than when the stop is intervocalic. Furthermore, word-finally, the stop is often produced with a perceivable release. It's difficult to tease apart aspiration and just audible releases: [ɑsk(ʰ)] `nation'. If the word is however sentence-medial, then aspiration is fully absent (\ref{ex:phonosegm:deaspSentence}). 
  
  \begin{exe}
  	\ex \gll  ɑsk əs-ɑ-v \\
  	nation say-{\pst}-3{\sg} \\
  	`He said ``nation''.'\\ \label{ex:phonosegm:deaspSentence}
  	\armenian{«Ազգ» ըսաւ։}
  \end{exe}
  
  
  
  
  \subsubsection{Derived contexts}\label{section:segmentalPhono:allphonLaryng:finalDeasp:derived}
  
  Word-finally in derived contexts, a VCC or VCCC cluster is formed by adding the nominalizer suffix /-kʰ/. 
  
  
  The suffix can derive nouns from roots and from other words. Its syllable structure is quite complicated, and it is often analyzed as an extrasyllabic appendix  (\S\ref{section:syllable:ConsonantClusters:Appendix}). When this suffix is word-final after a fricative, it has some degree of aspiration in our recordings when uttered in isolation. But when another inflectional suffix is added, this /-kʰ/ becomes word-medial and fully deaspirates.\footnote{Between a fricative and consonant (/VC-kʰ-CV/), the suffix /-kʰ/ tends to significantly overlap with the preceding fricative. This suggests some type of gestural overlap. } Furthermore, if the word is sentence-medial, then aspiration is again fully lost. We first show data from VC-kʰ contexts (Table \ref{tab:post sibilant deaspiration derived k}). 
  
  
  
  \begin{table}[H]
  	\centering
  	\caption{Variable post-fricative deaspiration for the suffix /-kʰ/ in VC-kʰ contexts }
  	\label{tab:post sibilant deaspiration derived k}
  	\begin{tabular}{| l ll l| }
    \hline
    
    & təˈʒo\textbf{χ} & `hard' & \armenian{դժոխ} 
    \\
    Derived & təˈʒo\textbf{χ-k(ʰ)}& `hell'& \armenian{դժոխք}
    \\
    Inflected & təʒo\textbf{χ-ˈk}-ov& `hell-{\ins}' & \armenian{դժոխքով}
    \\
    & təʒo\textbf{χ-k}-ˈneɾ & `hell-{\pl}' & \armenian{դժոխքներ} 
    \\
    \hline
    Root & /hɾɑʃ-/ & bound root as in... & 
    \\
    & həɾɑ\textbf{ʃ}-ɑˈli&`marvelous' & \armenian{հրաշալի} 
    \\
    Derived & həˈɾɑ\textbf{ʃ-k(ʰ)}& `miracle'& \armenian{հրաշք}
    \\
    Inflected & həɾɑ\textbf{ʃ-ˈk}-i& `miracle-{\gen}' & \armenian{հրաշքի}
    \\
    & həɾɑ\textbf{ʃ-k}-ˈneɾ & `miracle-{\pl}' & \armenian{հրաշքներ} 
    \\
    \hline
    Root & /de\textbf{s}-/ & bound root as in... & 
    \\
    & deˈ\textbf{s}-ɑd͡z&`seen' & \armenian{տեսած} 
    \\
    Derived & ˈde\textbf{s-k(ʰ)}& `sight'& \armenian{հրաշք}
    \\
    Inflected & de\textbf{s-ˈk}-i& `sight-{\abl}' & \armenian{տեսքէ}
    \\
    \hline
    Root & /χo\textbf{s}-/ & bound root as in... & 
    \\
    & χoˈ\textbf{s}-il&`to speak' & \armenian{խօսիլ} 
    \\
    Derived & ˈχo\textbf{s-k(ʰ)}& `speech'& \armenian{խօսք}
    \\
    Inflected & χo\textbf{s-ˈk}-eɾ& `speech-{\pl}' & \armenian{խօսքեր}
    \\
    \hline
  	\end{tabular}
  	
  \end{table}
  
  
  If the fricative is part of a complex coda, we also find that the /-kʰ/ seems to resist deaspiration in isolation (Table \ref{tab:post sibilant deaspiration derived k vcck}). Deaspiration is more visible when this segment becomes word-medial by adding inflectional suffixes, or even sentence-medial. 
  
  
  \begin{table}[H]
  	\centering
  	\caption{Variable post-fricative deaspiration for the suffix /-kʰ/ in VC-kʰ contexts }
  	\begin{tabular}{| l ll l| }
    \hline
    Base &ˈbɑ\textbf{ɾs} & `Persian (archaic) & \armenian{պարս}
    \\
    Derived & ˈbɑ\textbf{ɾs-k(ʰ)} & `Persia & \armenian{Պարսք}
    \\
    Inflected & bɑ\textbf{ɾs-ˈk}-e-n & `Persia-{\abl}-{\defgloss}' & \armenian{Պարսքէն}
    \\
    \hline 
    Base &ˈvɑ\textbf{ɾs} & `hair (archaic) & \armenian{վարս}
    \\
    Derived & ˈvɑ\textbf{ɾs-k(ʰ)} & `hair of head & \armenian{վարսք}
    \\
    Inflected & vɑ\textbf{ɾs-ˈk}-ov & `hair-{\ins}' & \armenian{վարսքով}
    
    \\
    \hline
  	\end{tabular}
  	\label{tab:post sibilant deaspiration derived k vcck}
  \end{table}
  
  
  The same patterns are found when the preceding consonant is a voiceless affricate (Table \ref{tab:post affricate deaspiration derived k vck}). 
  
  \begin{table}[H]
  	\centering
  	\caption{Variable post-affricate deaspiration for the suffix /-kʰ/ in VC-kʰ contexts } \label{tab:post affricate deaspiration derived k vck}
  	\begin{tabular}{| l ll l| }
    \hline
    Base & ˈme\textbf{t͡ʃ} & `in' & \armenian{մէջ} 
    \\
    Derived & ˈme\textbf{t͡ʃ-k(ʰ)} & `waist' & \armenian{մէջք}
    \\
    Inflected & me\textbf{t͡ʃ-ˈk}-e & `waste-{\abl} & \armenian{մէջքէ}
    \\
    \hline
    Base & tʰəˈɾi\textbf{t͡ʃ} & `flight' & \armenian{թռիչ} 
    \\
    Derived & tʰəˈɾi\textbf{t͡ʃ-k(ʰ)} & `flight' & \armenian{թռիչք}
    \\
    Inflected & təɾi\textbf{t͡ʃ-ˈk}-i & `flight-{\gen} & \armenian{թռիչքի}
    \\
    & təɾi\textbf{t͡ʃ-k}-ˈneɾ & `flight-{\gen} & \armenian{թռիչքներ}
    \\
    \hline
    Aorist stem & uɾeˈ\textbf{t͡s}-ɑd͡z & `swollen' & \armenian{ուռեցած} 
    \\
    Derived & uˈɾe\textbf{t͡s-k(ʰ)} & `swelling (n)' & \armenian{ուռեցք}
    \\
    Inflected & uɾe\textbf{t͡s-ˈk}-ov & `swelling-{\ins} & \armenian{ուռեցքով}
    \\
    & uɾe\textbf{t͡s-k}-ˈneɾ & `swelling-{\pl} & \armenian{ուռեցքներ}
    \\
    \hline
    Aorist stem & ləvɑ\textbf{ts}-oʁ & `washer' & \armenian{լուացող} 
    \\
    Derived & ləˈvɑ\textbf{t͡s-k(ʰ)} & `laundry' & \armenian{լուացք}
    \\
    Inflected & ləvɑ\textbf{t͡s-ˈk}-ov & `laundry-{\gen} & \armenian{լուացքի}
    \\
    & ləvɑ\textbf{t͡s-k}-ˈneɾ & `laundry-{\pl} & \armenian{լուացքներ}
    \\
    \hline
  	\end{tabular}
  	
  \end{table}
  
  
  We find the same variation when the affricate is part of a complex coda (Table \ref{tab:post affricate deaspiration derived k vcck}).  
  
  \begin{table}[H]
  	\centering
  	\caption{Variable post-affricate deaspiration for the suffix /-kʰ/ in VCC-kʰ contexts } \label{tab:post affricate deaspiration derived k vcck}
  	\begin{tabular}{| l ll l| }
    \hline
    Base &ˈvɑ\textbf{ɾt͡s} & `reward' & \armenian{վարձ}
    \\
    Derived & ˈvɑ\textbf{ɾt͡s-k(ʰ)} & `wage' & \armenian{վարձք}
    \\
    Inflected & vɑ\textbf{ɾt͡s-ˈk}-er & `wage-{\pl}' & \armenian{վարձքեր}
    \\
    \hline
    Base &ˈve\textbf{ɾt͡ʃ} & `end'& \armenian{վերջ}
    \\
    Derived & ˈve\textbf{ɾt͡ʃ-k(ʰ)} & `edge' & \armenian{վերջք}
    \\
    Inflected & ve\textbf{ɾt͡ʃ-ˈk}-i-n & `edge-{\pl}' & \armenian{վերջքին}
    \\
    \hline
    Base &zəˈɾu\textbf{jt͡s} & `tale'& \armenian{զրոյց}
    \\
    Derived & zəˈɾu\textbf{jt͡s-k(ʰ)} & `conversation' & \armenian{զրոյցք}
    \\
    Inflected & zəɾu\textbf{jt͡ʃ-ˈk}-e & `tale-{\abl}' & \armenian{զրոյցքէ}
    \\
    & zəɾu\textbf{ɾt͡ʃ-k}-ˈneɾ & `tale-{\pl}' & \armenian{զրոյցքներ}
    \\
    \hline
    Base &bɑˈhɑ\textbf{nt͡ʃ} & `demand'& \armenian{պահանջ}
    \\
    Derived & bɑˈhɑ\textbf{nt͡ʃ-k(ʰ)} & `credit' & \armenian{պահանջք}
    \\
    Inflected & bɑhɑ\textbf{nt͡ʃ-ˈk}-ov & `credit-{\ins}' & \armenian{պահանջքով}
    \\
    & bɑhɑ\textbf{nt͡ʃ-k}-ˈneɾ & `credit-{\pl}' & \armenian{պահանջքներ}
    \\
    \hline
  	\end{tabular}
  	
  \end{table}
  
  
  After a stop, the suffix /-kʰ/ causes devoicing and deaspiration of the stop. The suffix has the same variable deaspiration as before. Note that many of the roots in Table \ref{tab: stop stop deaspiration vck} are bound roots. Many also involve vowel reduction. 
  
  \begin{table}[H]
  	\centering
  	\caption{Stop-stop deaspiration in derived /VC-kʰ/ contexts }\label{tab: stop stop deaspiration vck}
  	{% \resizebox{1\textwidth}{!}{%
    	\begin{tabular}{| ll l l| }
      \hline
      & ˈhe\textbf{d} & `with' & \armenian{հետ}
      \\
      $\rightarrow$ & ˈhe\textbf{t-k(ʰ)} & `trace' & \armenian{հետք}
      \\ \hline 
      & ɑʁotʰ- & bound root for praying & 
      \\
      & ɑʁoˈ\textbf{tʰ}-el & `to pray' & \armenian{աղօթել}
      \\
      $\rightarrow$ & ɑˈʁo\textbf{t-k(ʰ)} & `trace' & \armenian{աղօթք}
      \\ \hline 
      & vod- & bound root for feet & 
      \\
      & voˈ\textbf{d}-iɡ & `tiny foot' & \armenian{ոտիկ}
      \\
      $\rightarrow$ & ˈvo\textbf{t-k(ʰ)} & `foot' & \armenian{ոտք}
      \\ \hline 
      & mi\textbf{d}- & bound root for mind & 
      \\
      & mə\textbf{d}-ɑˈjin & `mental' & \armenian{մտային}
      \\
      $\rightarrow$ & ˈmi\textbf{t-k(ʰ)} & `mind' & \armenian{միտք}
      \\ \hline 
      
    	\end{tabular}
    }
    
  	\end{table}
  	
  	We acknowledge though that the lack of deaspiration of /-kʰ/ is confounded with how this segment is word-final and said in isolation. It's possible that this segment is truly deaspirated after all fricatives/affricates even in isolation. But by being word-final in isolation, the stop gets some degree of intonational prominence, noisy enhancement, a stronger release, or breathiness. 
  	
  	
  	As of writing this grammar, we have not been able to do a systematic acoustic study of this suffix's deaspiration across multiple speakers. That seems like a worthwhile future research question. 
  	
  	
  	
  	
  	
  	
  	\subsection{Obstruent voicing assimilation}\label{section:segmentalPhono:allphonLaryng:assiimlation}
  	\textcolor{red}{make a separate subsection for v and then check that any of the refs to assimilation should be redirected to v instead }
  	
  	
  	Voicing assimilation is when in a consonant cluster C\textsubscript{1}C\textsubscript{2}, the two consonants have identical voicing quality. Such a cluster consists of two obstruents (stop, affricate, fricative). Either both consonants are voiced or both are voiceless. 
  	
  	This behavior seems exceptionless within roots. In the previous sections, we listed various underived contexts for deaspiration. All these contexts also had the obstruents match in voicing. Although the orthography may suggest that within a morpheme, the two consonants of a root have different voicing, the consonants are always pronounced with the same voicing quality. In this case, it's more accurate to say that this is a phonology-orthography mismatch (\S\ref{section:ortho:mismatch:clusters}), rather than a synchronic phonological rule that changes the voicing quality of root-internal clusters (Table \ref{tab:voicing assimilation underived}).\footnote{Some speakers (Tabita Toparlak) can pronounce the word `star' as [ɑstəʁ] instead of [ɑstχ]. For such speakers, the final fricative is underlyingly /ɑstʰʁ/ with schwa epenthesis. But for speakers like HD who never have the fricative non-adjacent from the stop, there is never any alternations so the word is underlyingly /ɑstʰχ/. }
  	
  	\begin{table}[H]
    \centering
    \caption{Voicing assimilation in underived contexts}
    \label{tab:voicing assimilation underived}
    \begin{tabular}{|lll ll | }
    	\hline 
    	/jekʰi\textbf{bd}os/ & /bɑ\textbf{tʰkʰ}ɑm/ & /χe\textbf{χtʰ}el/ & /ˈɑ\textbf{stʰχ}/& /ˈɑ\textbf{stʰv}ɑd͡z/
    	\\
    	{} [jekʰi\textbf{bˈd}os] & [bɑ\textbf{tˈk}om]& [χe\textbf{χˈt}el] & [ˈɑ\textbf{stχ}] & [ɑ\textbf{stˈv̥}ɑd͡z]
    	
    	\\
    	<eki\textbf{bd}os> & <ba\textbf{dk}am> & <xe\textbf{ɣd}el> & <a\textbf{sdɣ}> & <asdowad͡z> 
    	\\
    	`Egypt' & `message' & `to strangle' & `star'&`God'
    	\\
    	\armenian{Եգիպտոս} & \armenian{պատգամ} & \armenian{խեղդել} & \armenian{աստղ} & \armenian{Աստուած}
    	
    	\\ \hline 
    	
    	
    \end{tabular}
    
  	\end{table}
  	
  	
  	The issue of ambiguously devoiced /v/ as in `God' is discussed later in Section \S\ref{section:segmentalPhono:allphonLaryng:assiimlation:v}. 
  	
  	Voicing assimilation is likewise a productive phonological rule. When morphology or phonology causes two obstruents to become adjacent C\textsubscript{1}-C\textsubscript{2}, the two obstruent assimilate in voicing. If one of the obstruents is voiceless, both become voiceless. This devoicing happens either from C\textsubscript{1} to C\textsubscript{2} (progressive assimilation), or from C\textsubscript{2} to C\textsubscript{1} (regressive assimilation). We saw many instances of this in the previous sections on deaspiration. The following subsections present more data on progressive assimilation (\S\ref{section:segmentalPhono:allphonLaryng:assiimlation:prog}), variable devoicing for /v/ (\S\ref{section:segmentalPhono:allphonLaryng:assiimlation:v}), regressive assimilation in intervocalic clusters (\S\ref{section:segmentalPhono:allphonLaryng:assiimlation:regIntervoc}), and regressive final clusters due to \textit{-k} (\S\ref{section:segmentalPhono:allphonLaryng:assiimlation:regFinal}). 
  	
  	
  	
  	
  	
  	
  	
  	\subsubsection{Progressive assimilation in intervocalic VC(C)CV contexts}\label{section:segmentalPhono:allphonLaryng:assiimlation:prog}
  	
  	In terms of suffixation, there are few productive suffixes that start with a voiced obstruent. One such suffix is /-bɑn/ which is used to derive nouns that loosely have the meaning of `guardian or keeper of X'. This suffix starts with a voiced stop /-b/ after voiced segments.\footnote{The vowel /-ɑ-/ in these examples is the vowel used to connect stems to form compounds. Some suffixes also use this vowel. } The orthography marks this as a voiced stop \armenian{պ} <b> as well. But after a voiceless obstruent, this stop devoices to /-p/. More cases of devoicing were found in Table \ref{tab:prog asssimilation vc(C)cv ban} after voiceless fricatives and obstruents which cause deaspiration and loss of prevoicing. 
  	
  	\begin{table}[H]
    \centering
    \caption{Progressive voicing assimilation for the derivational suffix /-bɑn/ in VC(C)CV contexts}
    \label{tab:prog asssimilation vc(C)cv ban}
    \begin{tabular}{|lllll| }
    	\hline 
    	& & bɑɾˈde\textbf{z} & `garden' & \armenian{պարտէզ}
    	\\
    	$\rightarrow$ & /bɑɾde\textbf{z-b}ɑn/ & bɑɾdi\textbf{z-ˈb}ɑn & `gardener' & \armenian{պարտիզպան}
    	\\ \hline 
    	& & jeɡeʁe\textbf{ˈt͡s}i & `church' & \armenian{եկեղեցի}
    	\\
    	$\rightarrow$ & /jeɡeʁet͡si-\textbf{ɑ-b}ɑn/ & jeɡeʁet͡s-\textbf{ɑ-ˈb}ɑn & `church warden' & \armenian{եկեղեցապան}
    	\\ \hline 
    	& & ˈhɑ\textbf{t͡s} & `bread' & \armenian{հաց}
    	\\
    	$\rightarrow$ & /hɑ\textbf{t͡s-b}ɑn/ & hɑ\textbf{t͡s-ˈp}ɑn & `baker' & \armenian{հացպան}
    	\\ \hline 
    	& & bɑ\textbf{ʃˈt}-el & `to worship' & \armenian{պաշտել}
    	\\
    	$\rightarrow$ & /bɑ\textbf{ʃt-b}ɑn/ & bɑ\textbf{ʃt-ˈp}ɑn & `protector' & \armenian{պաշտպան}
    	\\ \hline 
    	
    	
    \end{tabular}
  	\end{table}
  	
  	
  	Another suffix with a voiced stop is /-ɡod/. This suffix surfaces with [ɡ] after vowels, voiced obstruents, and sonorants. We saw this derivational suffix throughout the deaspiration sections, where it would devoice into a voiceless unaspirated [k] after voiceless obstruents. We cite additional examples in Table \ref{tab:prog asssimilation vccv god}.\footnote{This suffix is often used after aorists stems. Most aorist stems end with /t͡s/, and this explains why Table \ref{tab:prog asssimilation vccv god} is overpopulated with /t͡s/ examples. } 
  	
  	
  	\begin{table}[H]
    \centering
    \caption{Progressive voicing assimilation for the derivational suffix /-ɡod/ in VCCV contexts}
    \label{tab:prog asssimilation vccv god}
    \begin{tabular}{|lllll| }
    	\hline 
    	Base && d͡zɑˈɾɑ\textbf{v} & `thirsty' & \armenian{ծարաւ}
    	\\
    	$\rightarrow$ & /d͡zɑɾɑ\textbf{v-ɡ}od/ & d͡zɑɾɑ\textbf{v-ˈg}od & `very thirsty' & \armenian{ծարաւկոտ}
    	\\ \hline 
    	Aorist stem && jeɾɑˈ\textbf{t͡s}-ɑd͡z & `bubbled' & \armenian{եռացած}
    	\\
    	$\rightarrow$ & /jeɾɑ\textbf{t͡s-ɡ}od/ & jeɾɑ\textbf{t͡s-ˈk}od & `effervescent' & \armenian{եռացկոտ}
    	\\ \hline 
    	Aorist stem && pʰɑɾɡɑˈ\textbf{t͡s}-ɑd͡z & `angry' & \armenian{բարկացած}
    	\\
    	$\rightarrow$ & /pʰɑɾɡɑ\textbf{t͡s-ɡ}od/ & pʰɑɾɡɑ\textbf{t͡s-ˈk}od & `irritable' & \armenian{բարկացկոտ}
    	\\ \hline 
    	Aorist stem && t͡sɑnt͡sɾɑ\textbf{t͡s}-ɑd͡z & `bored' & \armenian{ձանձրացած}
    	\\
    	$\rightarrow$ & /t͡sɑnt͡sɾɑ\textbf{t͡s-ɡ}od/ & t͡sɑnt͡sɾɑ\textbf{t͡s-ˈk}od & `easily bored' & \armenian{ձանձրացկոտ}
    	\\ \hline 
    	Aorist stem && zɑɾmɑ\textbf{t͡s}-ɑd͡z & `surprised' & \armenian{զարմացած}
    	\\
    	$\rightarrow$ & /zɑɾmɑ\textbf{t͡s-ɡ}od/ & zɑɾmɑ\textbf{t͡s-ˈk}od & `easily surprised' & \armenian{զարմացկոտ}
    	\\ \hline 
    	
    	
    \end{tabular}
  	\end{table}
  	
  	This suffix is likewise found after CC-final roots (Table \ref{tab:prog asssimilation vcccv god}). Here, we see progressive assimilation happening in VCC-CV contexts. The suffix /-ɡod/ is devoiced to [-kod] without any aspiration or prevoicing. 
  	
  	
  	
  	\begin{table}[H]
    \centering
    \caption{Progressive voicing assimilation for the derivational suffix /-ɡod/ in VCCCV contexts}
    \label{tab:prog asssimilation vcccv god}
    \begin{tabular}{|lllll| }
    	\hline 
    	& & bɑˈhɑ\textbf{nt͡ʃ} & `demand' & \armenian{պահանջ}
    	\\
    	$\rightarrow$ & /bɑhɑ\textbf{nt͡ʃ-ɡ}od/ &bɑhɑ\textbf{nt͡ʃ-ˈk}od & `demanding' & \armenian{պահանջկոտ}
    	\\ \hline 
    	& & pʰɑ\textbf{χˈt͡ʃ}-il & `to flee' & \armenian{փախչիլ}
    	\\
    	$\rightarrow$ & /pʰɑ\textbf{χt͡ʃ-ɡ}od/ &pʰɑ\textbf{χt͡ʃ-ˈk}od & `fugitive' & \armenian{փախչկոտ}
    	\\ \hline 
    	& & nɑˈχɑ\textbf{nt͡s} & `jealousy' & \armenian{նախանձ}
    	\\
    	$\rightarrow$ & /nɑχɑ\textbf{nt͡s-ɡ}od/ &nɑχɑ\textbf{nt͡s-ˈk}od & `jealous' & \armenian{նախանձկոտ}
    	\\ \hline 
    	& & dəɾˈdu\textbf{nt͡ʃ} & `grunt' & \armenian{տրտունջ}
    	\\
    	$\rightarrow$ & /dəɾdu\textbf{nt͡ʃ-ɡ}od/ &dəɾdə\textbf{nt͡s-ˈk}od & `grunting' & \armenian{տրտնջկոտ}
    	\\ \hline 
    	
    	
    \end{tabular}
  	\end{table}
  	
  	
  	
  	
  	Progressive assimilation is also found in compounds (Table \ref{tab:prog asssimilation vccv compound}). Most compounds are formed via a linking vowel /-ɑ-/. But there are cases of vowel-less or unlinked compounds where a) there is no such vowel, and b) the lack of a vowel causes the two stems to be adjacent. Here, we can find voicing assimilation as well. 
  	
  	
  	\begin{table}[H]
    \centering
    \caption{Progressive voicing assimilation for the compounds without a linking vowel /VC-CV/ }
    \label{tab:prog asssimilation vccv compound}
    \begin{tabular}{|llll| }
    	\hline 
    	Bases & ˈχɑt͡ʃ + ˈɡɑb & `cross' + `link' & \armenian{խաչ, կապ}
    	\\
    	$ \rightarrow$ & χɑ\textbf{t͡ʃ-ˈk}ɑb & `cross piece' & \armenian{խաչկապ}
    	\\
    	\hline
    	Bases & ˈhɑt͡s + ˈɡeɾ & `cross' + `eat!' & \armenian{հաց, կեր}
    	\\ 
    	$ \rightarrow$ & hɑ\textbf{t͡s-k}eˈɾ-ujtʰ & `banquet' & \armenian{հացկերոյթ}
    	\\ \hline 
    	
    	
    \end{tabular}
  	\end{table}
  	
  	Another context for progressive assimilation is vowel reduction of /i,u/. For the forms in Table \ref{tab:prog asssimilation vccv vowel reduction }, the root has a final (C)CVC sequence where the final onset and coda are separated. When a suffix is added, the root's vowel is deleted, causing the consonants to assimilate. 
  	
  	\begin{table}[H]
    \centering
    \caption{Progressive voicing assimilation from vowel reduction and syncope in /VCCV/ }
    \label{tab:prog asssimilation vccv vowel reduction }
    \begin{tabular}{|llll| }
    	\hline 
    	& tʰɑˈ\textbf{hid͡ʒ} & `executioner' & \armenian{դահիճ}
    	\\
    	$\rightarrow$ & tʰɑ\textbf{ht͡ʃ}-ɑˈbed & `chief executioner' & \armenian{դահճապետ}
    	\\ \hline 
    	& sɑ\textbf{sˈtiɡ} & `intense' & \armenian{սաստիկ}
    	\\
    	$\rightarrow$ & sɑ\textbf{st.k}-ɑˈɡɑn & `intensive' & \armenian{սաստկական} 
    	\\ \hline 
    	
    	& pʰə\textbf{sˈtuʁ} & `pistachio' & \armenian{փստուղ}
    	\\
    	$\rightarrow$ & pʰə\textbf{st.χ}-eˈni & `pistachio tree' & \armenian{փստղենի} 
    	\\ \hline 
    	& hə\textbf{nˈtʰiɡ} & `Indian' & \armenian{հնդիկ}
    	\\
    	$\rightarrow$ & hə\textbf{nt.k}-ɑsˈtɑn & `India' & \armenian{հնդկաստան} 
    	\\ \hline 
    	& mɑ\textbf{ɾˈtʰiɡ} & `mankind' & \armenian{մարդիկ}
    	\\
    	$\rightarrow$ & mɑ\textbf{ɾt.k}-ɑˈjin & `human (adj)' & \armenian{մարդկային} 
    	\\ \hline 
    	& t͡sə\textbf{nˈt͡suʁ} & `bronchus' & \armenian{ցնցուղ}
    	\\
    	$\rightarrow$ & t͡sə\textbf{nt͡s.χ}-ɑˈjin & `bronchial' & \armenian{ցնցղային} 
    	\\ \hline 
    	& t͡ʃə\textbf{χˈt͡ʃiɡ} + nəˈmɑn & `bat' + `like' & \armenian{չղջիկ, նման}
    	\\
    	$\rightarrow$ & t͡ʃə\textbf{χt͡ʃ.k}-ɑ-nəˈmɑn & `bat-like' & \armenian{չղջկանման}
    	\\ \hline 
    	
    	
    \end{tabular}
  	\end{table}
  	
  	
  	
  	
  	\subsubsection{Variable progressive devoicing for /v/ }\label{section:segmentalPhono:allphonLaryng:assiimlation:v}
  	The segment /v/ can devoice after voiceless obstruents. But there is some degree of optionality. 
  	
  	
  	In some morphological contexts, a root-final /u/ becomes [v] before a vowel suffix. This [v] can then devoice after voiceless consonants (Table \ref{tab:prog asssimilation vccv u v frication}). 
  	
  	
  	
  	\begin{table}[H]
    \centering
    \caption{Progressive voicing assimilation from /u/ frication: /VCu-V/ $\rightarrow$ //VCv-V// $\rightarrow$ [VCf-C] }
    \label{tab:prog asssimilation vccv u v frication}
    \begin{tabular}{|llll| }
    	\hline 
    	& gɑˈ\textbf{du} & `cat' & \armenian{կատու}
    	\\
    	$\rightarrow$ & gɑ\textbf{dv}-ɑˈɡɑn & `feline' & \armenian{կատուական}
    	\\ \hline 
    	& diɾɑˈ\textbf{t͡su} & `clerk' & \armenian{տիրացու}
    	\\
    	$\rightarrow$ & diɾɑ\textbf{t͡sf}-uˈtʰun & `clerkship' & \armenian{տիրացուութիւն}
    	\\ \hline 
    	& tʰəˈ\textbf{tʰu} & `sour' & \armenian{թթու}
    	\\
    	$\rightarrow$ & tʰə\textbf{tf}-eˈʁen & `pickles' & \armenian{թթուեղէն}
    	\\ \hline 
    	
    	
    	
    \end{tabular}
  	\end{table}
  	
  	After a voiceless sound, we transcribe /v/ as either fully devoiced [f] or as partially devoiced [v̥]. When the fricative is part of the stressed syllable, we strongly perceive hearing a [v] sound even though the spectrogram shows little voicing on the fricative. We transcribe this situation as [v̥]. But when the fricative is not part of the stressed syllable, the perception of a [f] is more salient to our ears. 
  	
  	We found similar ambiguity from vowel reduction (Table \ref{tab:prog asssimilation vccv vowel reduction v f}). The devoicing of /v/ seems more salient when unstressed.
  	
  	\begin{table}[H]
    \centering
    \caption{Progressive voicing assimilation from vowel reduction for /v/ in /VCCV/ }
    \label{tab:prog asssimilation vccv vowel reduction v f}
    \begin{tabular}{|llll| }
    	\hline 
    	& hɑˈ\textbf{ʃiv} & `account' & \armenian{հաշիւ}
    	\\
    	$\rightarrow$ & hɑ\textbf{ʃˈv᷂}-el & `to count' & \armenian{հաշուել}
    	\\
    	$\rightarrow$ & hɑ\textbf{ʃˈf}-ɑbɑh & `accountant' & \armenian{հաշուապահ}
    	\\ \hline 
    	
    	
    \end{tabular}
  	\end{table}
  	
  	
  	
  	\textcolor{red}{add data from ուան suffix like երբուան}
  	
  	We find similar ambiguity for post-voiceless [v] from the passive suffix \textit{-v-} (Table \ref{tab:prog asssimilation vccv passive}). This suffix surfaces as voiced after voiced obstruents and sonorants. It is likewise written with the digraph \armenian{ու} that is `supposed' to be pronounced as [v] before vowels.\footnote{The glosses for the words in Table \ref{tab:prog asssimilation vccv passive} is as follows. The active verbs consist of a root, theme vowel (-e-), and then an infinitive suffix \textit{-l}. The passives have the passive suffix \textit{-v-}, theme vowel \textit{-i-}, and infinitive \textit{-l}. Infinitives can get further nominal inflection like with case markers; adding case causes the theme vowel to change to \textit{-e-}. See (\textcolor{red}{cite chapter theme i neutralization}).} When stressed and after a voiceless segment, we see little to no voicing on the fricative (on the spectogram), but we strongly hear a [v] sound. When unstressed, the perception of [v] is less strong.\footnote{After CC clusters, the passive /-v-/ triggers schwa epenthesis: [ɑsˈt-e-l] `to influence' vs. [ɑstə-ˈv-i-l] `to be influenced'. Thus voicing assimilation is blocked. See (\textcolor{red}{cite chapter passive epenthesis}). }
  	
  	
  	\begin{table}[H]
    \centering
    \caption{Progressive voicing assimilation for passive /-v-/ }
    \label{tab:prog asssimilation vccv passive}
    \begin{tabular}{|llll| }
    	\hline 
    	& ɡɑˈ\textbf{b}-e-l & `to connect' & \armenian{կապել}
    	\\
    	$\rightarrow$ & hɑ\textbf{nˈ-v}-i-l & `to be removed' & \armenian{կապուիլ}
    	\\ \hline 
    	& hɑˈ\textbf{n}-e-l & `to remove' & \armenian{հանել}
    	\\
    	$\rightarrow$ & hɑ\textbf{nˈ-v}-i-l & `to be removed' & \armenian{հանուիլ}
    	\\ \hline 
    	& t͡ʃɑˈ\textbf{pʰ}-e-l & `to measure' & \armenian{չափել}
    	\\
    	$\rightarrow$ & t͡ʃɑ\textbf{pˈ-v̥}-i-l & `to be measured' & \armenian{չափուիլ}
    	\\
    	$\rightarrow$ & t͡ʃɑ\textbf{p-f}-e-ˈl-ov & `to be removed ({\ins})' & \armenian{չափուելով}
    	\\ \hline 
    	& χɑˈ\textbf{tʰ}-e-l & `to push in' & \armenian{խոթել}
    	\\
    	$\rightarrow$ & χo\textbf{tˈ-v̥}-i-l & `to be pushed in' & \armenian{խոթուիլ}
    	\\
    	$\rightarrow$ & χo\textbf{t-f}-e-ˈl-e & `to be pushed in ({\abl})' & \armenian{խոթուելէ}
    	\\ \hline 
    	& t͡səˈ\textbf{kʰ}-e-l & `to leave s.o.' & \armenian{ձգել}
    	\\
    	$\rightarrow$ & t͡sə\textbf{kˈ-v̥}-i-l & `to be left' & \armenian{ձգուիլ}
    	\\
    	$\rightarrow$ & t͡sə\textbf{k-f}-e-ˈl-u & `to be left ({\gen})' & \armenian{ձգուելէ}
    	\\ \hline 
    	& əˈ\textbf{s}-e-l & `to say' & \armenian{ըսել}
    	\\
    	$\rightarrow$ & ə\textbf{sˈ-v̥}-i-l & `to be said' & \armenian{ըսուիլ}
    	\\
    	$\rightarrow$ & ə\textbf{s-f}-e-ˈl-ov & `to be said ({\ins})' & \armenian{ըսուելով}
    	\\ \hline 
    	& kʰɑˈ\textbf{ʃ}-e-l & `to say' & \armenian{քաշել}
    	\\
    	$\rightarrow$ & kʰɑ\textbf{ʃˈ-v̥}-i-l & `to be said' & \armenian{քաշուիլ}
    	\\
    	$\rightarrow$ & kʰɑ\textbf{ʃ-f}-e-ˈl-e & `to be said ({\abl})' & \armenian{քաշուելէ}
    	\\ \hline 
    	& t͡sɑˈ\textbf{χ}-e-l & `to sell' & \armenian{ծախել}
    	\\
    	$\rightarrow$ & t͡sɑ\textbf{χˈ-v̥}-i-l & `to be sold' & \armenian{ծախուիլ}
    	\\
    	$\rightarrow$ & t͡sɑ\textbf{χ-f}-e-ˈl-u & `to be sold ({\gen})' & \armenian{ծախուելու}
    	\\ \hline 
    	& kʰoˈ\textbf{t͡s}-e-l & `to close' & \armenian{գոցել}
    	\\
    	$\rightarrow$ & kʰo\textbf{t͡sˈ-v̥}-i-l & `to be closed' & \armenian{գոցուիլ}
    	\\
    	$\rightarrow$ & kʰo\textbf{t͡s-f}-e-ˈl-ov & `to be closed ({\ins})' & \armenian{գոցուելով}
    	\\ \hline 
    	& ɡoˈ\textbf{t͡ʃ}-e-l & `to close' & \armenian{կոչել}
    	\\
    	$\rightarrow$ & ɡo\textbf{t͡ʃˈ-v̥}-i-l & `to be closed' & \armenian{կոչուիլ}
    	\\
    	$\rightarrow$ & ɡo\textbf{t͡ʃ-f}-e-ˈl-e & `to be closed ({\abl})' & \armenian{կոչուելէ}
    	\\ \hline 
    	
    	
    	
    \end{tabular}
  	\end{table}
  	
  	
  	
  	
  	
  	
  	
  	
  	
  	
  	It's possible that the the sound /v/ is gesturally pronounced at the same time as the following vowel, thus making it difficult to determine when its voicing (or lack of voicing) starts. Ideally future phonetic research can look into why we have such difficulties in determining the voicing of /v/. 
  	
  	It is possible that the behavior of /v/ is complicated by its articulatory behavior. Cross-linguistically, it is known that some langauges have the segment /v/ undergoing voicing assimilation rules when in a pre-consontal (vC) or word-final position (v\#) The segment however resists undergoing or applying voicing assimilation in post-consontal (Cv) contexts. Such languages include Russian, Czech, Hungarian, among others \citep{Padgett-2002-RussianVoicingAssimilationFinalDevoicingProblemV,Hall-2004-FormalApproachEvidenceCzechSlovak,BarkaïHorvath-1978-VoicingAssimilationSonorityHierarchyEvidenceRussianHebrewHungarian}.\todo{read  Christina Bjorndahl's dissertation}. A surprising similarity is that the Armenian /v/ obligatorily undergoes  regressive devoicing assimilation (as we see in the next two subsections), while progressive devoicing is variable. 
  	
  	\subsubsection{Regressive assimilation in intervocalic VC(C)CV contexts}\label{section:segmentalPhono:allphonLaryng:assiimlation:regIntervoc}
  	
  	In [V(C)C\textsubscript{1}-C\textsubscript{2}V] we likewise find cases of regressive assimilation where a voiceless obstruent C\textsubscript{2} causes C\textsubscript{1} to devoice. 
  	
  	A common morphological construction that causes regressive devoicing is forming aorist stems. The roots in Table \ref{tab:reg devoicing inchoative nal tsav vccv} can form inchoative verbs by adding the suffix sequence /-n-ɑ-l/. Their aorist stems are formed by replacing the suffix /n/ with the suffix /t͡s/. This stem is used to from the simple past (past perfective).\footnote{The full glosses are /-n-ɑ-l/ [{\inch}-{\thgloss}-{\infgloss}], /t͡s-ɑ-v/ [{aor}-{\pst}-3{\sg}. } The /t͡s/ causes the preceding obstruent to devoice.\footnote{The sequence of [Vt͡s-t͡sV] as a single segment, i.e., a geminate [t͡sː]. }
  	
  	\begin{table}[H]
    \centering
    \caption{Regressive voicing assimilation before aorist suffix /-t͡s-/ in VCCV contexts}
    \label{tab:reg devoicing inchoative nal tsav vccv}
    \begin{tabular}{|l lll l| }
    	\hline 
    	Base && ˈʃɑ\textbf{d} & `many' & \armenian{շատ} \\
    	Verb & & ʃɑ\textbf{d}-ˈn-ɑ-l & `to multiply (intr.)' & \armenian{շատնալ}
    	\\
    	Aorist & /ʃɑ\textbf{d-t͡s}-ɑ-v/ & ʃɑ\textbf{t-ˈt͡s}-ɑ-v & `it multiplied' & \armenian{շատցաւ}
    	\\ \hline 
    	
    	Base && bəzˈdi\textbf{ɡ} & `small' & \armenian{պզտիկ} \\
    	Verb & & bəzdi\textbf{ɡ}-ˈn-ɑ-l & `to get small' & \armenian{պզտիկնալ}
    	\\
    	Aorist & /bəzdi\textbf{ɡ-t͡s}-ɑ-v/ & bəzdi\textbf{k-ˈt͡s}-ɑ-v & `it got small' & \armenian{պզտիկցաւ}
    	\\ \hline 
    	Base && ˈme\textbf{d͡z} & `big' & \armenian{մեծ} \\
    	Verb & & me\textbf{d͡z}-ˈn-ɑ-l & `to grow up' & \armenian{մեծնալ}
    	\\
    	Aorist & /me\textbf{d͡z-t͡s}-ɑ-v/ & me\textbf{t͡s-ˈt͡s}-ɑ-v & `he grew up' & \armenian{մեծցաւ}
    	\\ \hline 
    	Base && ˈse\textbf{v} & `black' & \armenian{սեւ} \\
    	Verb & & se\textbf{v}-ˈn-ɑ-l & `to blacken' & \armenian{սեւնալ}
    	\\
    	Aorist & /se\textbf{v-t͡s}-ɑ-v/ & ˈse\textbf{f-ˈt͡s}-ɑ-v & `it blackened' & \armenian{սեւցաւ}
    	\\ \hline 
    	Base && dəˈkʰe\textbf{ʁ} & `ugly' & \armenian{տգեղ} \\
    	Verb & & dəkʰe\textbf{ʁ}-ˈn-ɑ-l & `to get ugly' & \armenian{տգեղնալ}
    	\\
    	Aorist & /dəkʰe\textbf{ʁ-t͡s}-ɑ-v/ & /dəkʰe\textbf{χ-ˈt͡s}-ɑ-v & `it got ugly' & \armenian{տգեղցաւ}
    	\\ \hline 
    	
    	
    \end{tabular}
  	\end{table}
  	
  	We find the same regressive pattern before the derivational suffix /-t͡si/  (Table \ref{tab:reg asssimilation vccv tsi}). This suffix is used to create names for people from cities, countries, and ethnic groups (ethnonyms). This suffix is underlyingly just a voiceless affricate. It causes regressive devoicing on obstruents. 
  	
  	\begin{table}[H]
    \centering
    \caption{Regressive voicing assimilation for the derivational suffix /-t͡si/ in VCCV contexts}
    \label{tab:reg asssimilation vccv tsi}
    \begin{tabular}{|lllll| }
    	\hline 
    	& & pʰɑˈɾi\textbf{z} & `Paris' & \armenian{Պարիզ}
    	\\
    	$\rightarrow$ & /pʰɑɾi\textbf{z-t͡s}i/ &pʰɑɾi\textbf{s-ˈt͡s}i & `Parisian' & \armenian{պարիզցի}
    	\\ \hline 
    	& & ʁɑɾɑˈbɑ\textbf{ʁ} & `Karabakh' & \armenian{Ղարաբաղ}
    	\\
    	$\rightarrow$ & /ʁɑɾɑbɑ\textbf{ʁ-t͡s}i/ &ʁɑɾɑbɑ\textbf{χ-ˈt͡s}i & `Karabaghian' & \armenian{ղարաբաղցի}
    	\\ \hline 
    	& & hɑˈle\textbf{b} & `Aleppo' & \armenian{Հալէպ}
    	\\
    	$\rightarrow$ & /hɑlɑ\textbf{b-t͡s}i/ &hɑle\textbf{p-ˈt͡s}i & `Aleppoite' & \armenian{հալէպցի}
    	\\ \hline 
    	
    	
    \end{tabular}
  	\end{table}
  	
  	We again find regressive assimilation before the causative suffix [-t͡sən-] (Table \ref{tab:reg asssimilation vccv tsn}).\footnote{The gloss for /t͡sən-e-l/ is [{\caus}-{\thgloss}-{\infgloss}]. }
  	
  	
  	\begin{table}[H]
    \centering
    \caption{Regressive voicing assimilation before the causative suffix /-t͡sən-/ in VCCV contexts}
    \label{tab:reg asssimilation vccv tsn}
    \begin{tabular}{|lllll| }
    	\hline 
    	& & ˈɡɑ\textbf{b} & `link' & \armenian{կապ}
    	\\
    	$\rightarrow$ & /ɡɑ\textbf{b-t͡s}ən-e-l/ &ɡɑ\textbf{p-t͡s}əˈn-e-l & `to connect (trns.)' & \armenian{կապցնել}
    	\\ \hline 
    	& & ɑʃχɑˈ\textbf{d}-i-l & `to work' & \armenian{աշխատիլ}
    	\\
    	$\rightarrow$ & /ɑʃχɑ\textbf{d-t͡s}ən-e-l/ &ɑʃχɑ\textbf{t-t͡s}əˈn-e-l & `to employ' & \armenian{աշխատցնել}
    	\\ \hline 
    	& & pʰɑˈɾɑ\textbf{ɡ} & `thin' & \armenian{բարակ}
    	\\
    	$\rightarrow$ & /pʰɑɾɑ\textbf{ɡ-t͡s}ən-e-l/ &pʰɑɾɑ\textbf{k-t͡s}əˈn-e-l & `to make thin' & \armenian{բարակցնել}
    	\\ \hline 
    	& & t͡sɑ\textbf{d͡z} & `low' & \armenian{ցած}
    	\\
    	$\rightarrow$ & /t͡sɑ\textbf{d͡z-t͡s}ən-e-l/ &t͡sɑ\textbf{t͡s-t͡s}əˈn-e-l & `to lower' & \armenian{ցածցնել}
    	\\ \hline 
    	& & tʰeˈtʰə\textbf{v} & `light' & \armenian{թեթեւ}
    	\\
    	$\rightarrow$ & /tʰetʰə\textbf{v-t͡s}ən-e-l/ &tʰetʰə\textbf{f-t͡s}əˈn-e-l & `to lighten' & \armenian{թեթեւցնել}
    	\\ \hline 
    	& & vɑ\textbf{z}-e-l & `to run' & \armenian{վազել}
    	\\
    	$\rightarrow$ & /vɑ\textbf{z-t͡s}ən-e-l/ &vɑ\textbf{s-t͡s}əˈn-e-l & `to hasten' & \armenian{վազցնել}
    	\\ \hline 
    	& & ˈχɑ\textbf{ʁ} & `game' & \armenian{խաղ}
    	\\
    	$\rightarrow$ & /χɑ\textbf{ʁ-t͡s}ən-e-l/ &χɑ\textbf{χ-t͡s}əˈn-e-l & `to make to play' & \armenian{խաղցնել}
    	\\ \hline 
    	
    \end{tabular}
  	\end{table}
  	
  	We've also found another C-initial suffix /-kʰin/ (Table \ref{tab:reg asssimilation vccv kin compo}). This derivational suffix is rare but it can cause regressive assimilation. There are also some cases where compounds create contexts for regressive assimilation. 
  	
  	
  	\begin{table}[H]
    \centering
    \caption{Regressive voicing assimilation before the derivational suffix /-kʰin-/ and in compounds in VCCV contexts}
    \label{tab:reg asssimilation vccv kin compo}
    \begin{tabular}{|lllll| }
    	\hline 
    	& & ˈu\textbf{ʒ} & `strength' & \armenian{ուժ}
    	\\
    	$\rightarrow$ & /u\textbf{ʒ-kʰ}in/ &u\textbf{ʃ-ˈk}in & `strong' & \armenian{ուժգին}
    	\\ \hline 
    	& & ˈpʰuj\textbf{ʒ} + ˈ\textbf{kʰ}ujɾ& `healing' + `sister' & \armenian{բոյժ, քոյր}
    	\\
    	$\rightarrow$ & &pʰu\textbf{ʃ-ˈk}ujr & `strong' & \armenian{բուժքոյր}
    	\\ \hline 
    	
    \end{tabular}
  	\end{table}
  	
  	There are some rare derivational prefixes that end in a voiced obstruent: synonymous [tʰəʒ-] \armenian{դժ} and [dəʒ-] \armenian{տժ} (Table \ref{tab:reg asssimilation vccv tz}). The fricative surfaces as voiced before voiced segments, and devoices before voiceless obstruents. We report on just [tʰəʒ-] becomes its more common in speech.\footnote{Note that the schwa in this prefix is likely epenthetic, but we transcribe the schwa here for illustration.}
  	
  	
  	
  	\begin{table}[H]
    \centering
    \caption{Regressive voicing assimilation after the derivational prefix [tʰəʒ-] in VCCV contexts}
    \label{tab:reg asssimilation vccv tz}
    \begin{tabular}{|lllll| }
    	\hline 
    	& & ˈ\textbf{ɡ}eɾb & `form' & \armenian{կերպ}
    	\\
    	$\rightarrow$ & /tə\textbf{ʒ-ɡ}eɾb/ &tʰə\textbf{ʃ-ˈɡ}eɾb & `ugly' & \armenian{դժկերպ}
    	\\ \hline 
    	& & ˈ\textbf{pʰ}ɑχt & `luck' & \armenian{բախտ}
    	\\
    	$\rightarrow$ & /tə\textbf{ʒ-pʰ}ɑχt/ &tʰə\textbf{ʃ-ˈp}ɑχt & `unlucky' & \armenian{դժբախտ}
    	\\ \hline 
    	& & ˈ\textbf{kʰ}oh & `satisfied' & \armenian{գոհ}
    	\\
    	$\rightarrow$ & /tə\textbf{ʒ-kʰ}oh/ &tʰə\textbf{ʃ-ˈk}oh & `dissatisfied' & \armenian{դժգոհ}
    	\\ \hline 
    	& & ˈ\textbf{kʰ}ujn & `color' & \armenian{գոյն}
    	\\
    	$\rightarrow$ & /tə\textbf{ʒ-kʰ}ujn/ &tʰə\textbf{ʃ-ˈk}ujn & `discolored' & \armenian{դժգոյն}
    	\\ \hline 
    	
    \end{tabular}
  	\end{table}
  	
  	
  	
  	
  	Vowel reduction can likewise create contexts for regressive assimilation. For the word in Table \ref{tab:reg asssimilation vccv vowel reduction v f}, the root has a final CVC syllable. When a derivational suffix is added, the root's vowel is deleted and this causes the consonants to become adjacent. Note that some of these examples include a derivational suffix /-it͡ʃ/. Some of these examples include a devoiced /v/. 
  	
  	\begin{table}[H]
    \centering
    \caption{Regressive voicing assimilation from vowel reduction in /VCCV/ }
    \label{tab:reg asssimilation vccv vowel reduction v f}
    \begin{tabular}{|llll| }
    	\hline 
    	& ɡəˈ\textbf{dut͡s} + ˈt͡sev & `beak' + `shape' & \armenian{կտուց, ձեւ}
    	\\
    	$\rightarrow$ & ɡə\textbf{tt͡s}-ɑ-ˈt͡sev & `beak-shaped' & \armenian{կտցաձեւ}
    	\\ \hline 
    	& ɑt͡ʃɑˈ\textbf{ɡit͡s} & `assistant' & \armenian{աջակից}
    	\\
    	$\rightarrow$ & ɑt͡ʃɑ\textbf{kt͡s}-uˈtʰun & `assistance' & \armenian{աջակցութիւն}
    	\\ \hline 
    	& ɑˈ\textbf{d͡zuχ} & `coal' & \armenian{ածուխ}
    	\\
    	$\rightarrow$ & ɑ\textbf{t͡sχ}-ɑˈjin & `carbonic' & \armenian{ածխային}
    	\\ \hline 
    	
    	
    	& hɑmoˈ\textbf{z-it͡ʃ} & `persuasive' & \armenian{համոզիչ}
    	\\
    	$\rightarrow$ & hɑmo\textbf{s-t͡ʃ}-uˈtʰun & `persuasion' & \armenian{համոզչութիւն}
    	\\ \hline 
    	& kʰəɾɑˈ\textbf{v-it͡ʃ} & `attractive' & \armenian{գրաւիչ}
    	\\
    	$\rightarrow$ & kʰəɾɑ\textbf{f-t͡ʃ}-uˈtʰun & `attractiveness' & \armenian{գրաւչութիւն}
    	\\ \hline 
    	     	& [kʰɑˈvi\textbf{tʰ}] & `courtyard' & \armenian{գաւիթ}
    	\\
    	$\rightarrow$ & /kʰɑ\textbf{vitʰ}-ɑ-bɑh/ & \multicolumn{2}{l| }{with compound linker /-ɑ-/ }
    	\\
    	& [kʰɑ\textbf{ft}-ɑ-ˈbɑh] & `gatekeeper' & \armenian{գաւթապահ}
    	\\
    	\hline
    \end{tabular}
  	\end{table}
  	
  	
  	The above regressive assimilation patterns are also attested for VCCCV clusters with 3 consonants (Table \ref{tab:reg devoicing inchoative caus vcccv}). Such sequences are relatively rare but attested. We find regressive devoicing in this clusters, such as before the aorist suffix /-t͡s-/ o causative /-t͡sən-/. 
  	
  	\begin{table}[H]
    \centering
    \caption{Regressive voicing assimilation in VCCCV contexts before aorist suffix /-t͡s-/ or causative /-t͡sən-/ }
    \label{tab:reg devoicing inchoative caus vcccv}
    \begin{tabular}{|l lll l| }
    	\hline 
    	Base && ˈɡɑ\textbf{ɾd͡ʒ} & `short' & \armenian{կարճ} \\
    	Verb & & ɡɑ\textbf{ɾd͡ʒ}-ˈn-ɑ-l & `to get short' & \armenian{կարճնալ}
    	\\
    	Aorist & /ɡɑ\textbf{ɾd͡ʒ-t͡s}-ɑ-v/ & ɡɑ\textbf{ɾt͡ʃ-ˈt͡s}-ɑ-v & `it got short' & \armenian{կարճցաւ}
    	\\
    	Caus. & /ɡɑ\textbf{ɾd͡ʒ-t͡s}ən-e-l/ & ɡɑ\textbf{ɾt͡ʃ-t͡s}əˈn-e-l/ & `to make short' & \armenian{կարճցնել}
    	\\ \hline 
    	Base && ɡɑˈbu\textbf{jd} & `blue' & \armenian{կապոյտ} \\
    	Verb & & ɡɑbu\textbf{jd}-ˈn-ɑ-l & `to become blue' & \armenian{կապոյտնալ}
    	\\
    	Aorist & /ɡɑbu\textbf{jd-t͡s}-ɑ-v/ & ɡɑbu\textbf{jt-ˈt͡s}-ɑ-v & `it became blue' & \armenian{կապոյտցաւ}
    	\\
    	Caus. & /ɡɑbu\textbf{ɾd-t͡s}ən-e-l/ & ɡɑbu\textbf{jt-t͡s}əˈn-e-l/ & `to make blue' & \armenian{կապոյտցնել}
    	\\ \hline 
    	Base && χoˈɾu\textbf{ŋɡ} & `deep' & \armenian{խորունկ} \\
    	Verb & & χoɾu\textbf{ŋɡ}-ˈn-ɑ-l & `to become deep' & \armenian{խորունկնալ}
    	\\
    	Aorist & /χoɾu\textbf{nɡ-t͡s}-ɑ-v/ & χoɾu\textbf{ŋk-ˈt͡s}-ɑ-v & `it became deep' & \armenian{խորունկցաւ}
    	\\
    	Caus. & /χoɾu\textbf{ɾd-t͡s}ən-e-l/ & χoɾu\textbf{ŋɡ-t͡s}əˈn-e-l/ & `to make deep' & \armenian{խորունկցնել}
    	\\ \hline 
    	
    \end{tabular}
  	\end{table}
  	
  	Vowel reduction can likewise create VCCCV clusters with regressive assimilation (Table \ref  {tab:reg asssimilation vcccv vowel reduction 1}). So far, the only examples that we've found involved the derivational suffix /-it͡ʃ/, whose vowel can delete and thus cause regressive assimilation. 
  	
  	\begin{table}[H]
    \centering
    \caption{Regressive voicing assimilation from vowel reduction in /VCCCV/ }
    \label  {tab:reg asssimilation vcccv vowel reduction 1}
    \begin{tabular}{|llll| }
    	\hline 
    	& ɡɑzmɑɡeɾˈ\textbf{b-it͡ʃ} & `organizer' & \armenian{կազմակերպիչ}
    	\\
    	$\rightarrow$ & ɡɑzmɑɡeɾ\textbf{p-t͡ʃ}-ɑˈɡɑn & `organizational' & \armenian{կազմակերպչական}
    	\\ \hline 
    	& məɡəɾˈ\textbf{d-it͡ʃ} & `baptist' & \armenian{մկրտիչմ}
    	\\
    	$\rightarrow$ & məɡəɾ\textbf{t-t͡ʃ}-ɑˈɡɑn & `Baptist (religion)' & \armenian{մկրտչական}
    	\\ \hline 
    	& neɾˈ\textbf{ɡ-it͡ʃ} & `painter' & \armenian{ներկիչ}
    	\\
    	$\rightarrow$ & neɾ\textbf{k-t͡ʃ}-ɑˈɡɑn & `fem. painter' & \armenian{ներկչուհի}
    	\\ \hline 
    	& pʰəɾˈ\textbf{ɡ-it͡ʃ} & `savior' & \armenian{փրկիչ}
    	\\
    	$\rightarrow$ & pʰəɾ\textbf{k-t͡ʃ}-ɑˈɡɑn & `pertaining to Christ' & \armenian{փրկչական}
    	\\ \hline 
    	& mɑɾˈ\textbf{z-it͡ʃ} & `trainer' & \armenian{մարզիչ}
    	\\
    	$\rightarrow$ & mɑɾ\textbf{s-t͡ʃ}-ɑˈɡɑn & `pertaining to trainers' & \armenian{մարզչական}
    	\\ \hline 
    \end{tabular}
  	\end{table}
  	
  	
  	
  	
  	
  	
  	\subsubsection{Regressive assimilation in appendix contexts: /VC(C)-k/}\label{section:segmentalPhono:allphonLaryng:assiimlation:regFinal}
  	
  	Regressive assimilation happens in a word-final context for the suffix /-kʰ/. As previewed in Section \S\ref{section:segmentalPhono:allphonLaryng:finalDeasp:derived}, this suffix can be added after any consonant cluster and cause devoicing. We go over this devoicing process as a type of regressive assimilation. 
  	
  	
  	The suffix /-kʰ/ can be after virtually any type of consonant, even if the consonant + \textit{kʰ} cannot form a complex coda based on their sonority (Table \ref{tab:re asssimilation vck}). Because of this behavior, this suffix is often phonologically analyzed as an appendix  (\S\ref{section:syllable:ConsonantClusters:Appendix}). In terms of segmental phonology, this voiceless suffix triggers regressive voicing assimilation (devoicing) on any preceding obstruent. Some of these examples include a devoiced /v/. 
  	
  	
  	\begin{table}[H]
    \centering
    \caption{Regressive voicing assimilation from appendix /-kʰ/ reduction in /VC-kʰ/ sequences }
    \label{tab:re asssimilation vck}
    \begin{tabular}{|llll| }
    	\hline 
    	& ˈtʰe\textbf{b} & `towards (archaic)' & \armenian{դէպ}
    	\\
    	$\rightarrow$ & ˈtʰe\textbf{p-k(ʰ)} & `event' & \armenian{դէպք}
    	\\ \hline 
    	& hɑˈvɑ\textbf{d} & `belief (archaic)' & \armenian{հաւատ}
    	\\
    	$\rightarrow$ & hɑˈvɑ\textbf{t-k(ʰ)} & `belief' & \armenian{հաւատք}
    	\\ \hline 
    	& dɑˈɾɑ\textbf{d͡z} & `spread out' & \armenian{տարած}
    	\\
    	$\rightarrow$ & dɑˈɾɑ\textbf{t͡s-k(ʰ)} & `extent' & \armenian{տարածք}
    	\\ \hline 
    	& ˈkʰo\textbf{v} & `praise (archaic)' & \armenian{գով}
    	\\
    	$\rightarrow$ & ˈkʰo\textbf{f-k(ʰ)} & `praise' & \armenian{գովք}
    	\\ \hline 
    	& vɑˈ\textbf{z}-el & `race' & \armenian{վազել}
    	\\
    	$\rightarrow$ & ˈvɑ\textbf{s-k(ʰ)} & `race' & \armenian{վազք}
    	\\ \hline 
    	& ˈɡo\textbf{ʁ} & `side' & \armenian{կող}
    	\\
    	$\rightarrow$ & ˈɡo\textbf{χ-k(ʰ)} & `book cover' & \armenian{կողք}
    	\\ \hline 
    \end{tabular}
  	\end{table}
  	
  	As discussed in Section \S\ref{section:segmentalPhono:allphonLaryng:finalDeasp:derived}, the level of aspiration on the final \textit{kʰ} seems variable. When word-final and sentence-final, the suffix tends to resist deaspiration after fricatives. Thus we transcribe the segment as [-k(ʰ)] for this chapter. However after stops, our recordings suggest that there's a lot less variable aspiration on the /-kʰ/. So a word like [tʰep-kʰ] `event' could alternative be transcribed as [tʰep-k] even in isolation. The problem is that, by being word-final, it's difficult to measure the degree of aspiration. 
  	
  	After a consonant cluster as well (VCC-\textit{kʰ}), the suffix triggers devoicing (Table \ref{tab:prog asssimilation vcck}). 
  	
  	
  	
  	
  	
  	\begin{table}[H]
    \centering
    \caption{Regressive voicing assimilation from appendix /-kʰ/ reduction in /VCC-kʰ/ sequences }
    \label{tab:prog asssimilation vcck}
    \begin{tabular}{|llll| }
    	\hline 
    	& ˈbɑ\textbf{ɾd} & `debt (archaic)' & \armenian{պարտ}
    	\\
    	$\rightarrow$ & ˈbɑ\textbf{rt-k(ʰ)} & `debt' & \armenian{պարտք}
    	\\ \hline 
    	& ˈɡu\textbf{ɾd͡z} & `core of pumpkin' & \armenian{կուրծ}
    	\\
    	$\rightarrow$ & ˈɡu\textbf{ɾt͡s-k(ʰ)} & `chest' & \armenian{կուրծք}
    	\\ \hline 
    	& ˈʃɑ\textbf{ɾʒ} & `motion' & \armenian{շարժ}
    	\\
    	$\rightarrow$ & ˈʃɑ\textbf{rʃ-k(ʰ)} & `motion' & \armenian{շարժք}
    	\\ \hline 
    \end{tabular}
  	\end{table}
  	
  	
  	\section{Place assimilation of nasals}\label{section:segmentalPhono:nasalPlace}
  	
  	
  	Before the velar stops /kʰ, ɡ/, the nasal /n/ becomes velar [ŋ] (Rule \ref{tab:rule nasal place velar}). Orthographically, the nasal is still written as a coronal nasal \armenian{ն} <n>. This process applies both in underived (\S\ref{section:segmentalPhono:nasalPlace:velarUnderived}) and derived contexts (\S\ref{section:segmentalPhono:nasalPlace:velarDerived}). In derived contexts, we can find cases where a surface [n] alternates with [ŋ]. Complications arise though from cases where /n/ has stress and can resist velarization. 
  	
  	
  	\begin{ruleblock}
    {Nasal place assimilation before velars}
    
    \label{tab:rule nasal place velar}
    \begin{tabular}{| llll lll| }
    	\hline \multicolumn{7}{| l| }{\textit{Before a velar stop /kʰ, ɡ/, the nasal /n/ becomes [ŋ]. }} 
    	\\
    	& & /ɑ\textbf{nkʰ}ɑm/ & $\rightarrow$ & [ɑ\textbf{ŋˈkʰ}ɑm] & `time' & \armenian{անգամ}
    	\\
    	& & /t͡sɑ\textbf{nɡ}/ & $\rightarrow$ & [ˈt͡sɑ\textbf{ŋɡ}] & `list' & \armenian{ցանկ}
    	\\
    	\hline 
    \end{tabular}
  	\end{ruleblock} 
  	
  	
  	Nasal place assimilation does not occur before other dorsal sounds, such as before /χ, ʁ/ (\S\ref{section:segmentalPhono:nasalPlace:other}). There is likewise no productive rule of assimilation before labials. However, labial assimilation was a widespread diachronic rule. There is likewise a degree of labial assimilation in connected speech for high-frequency words. 
  	\subsection{Velar assimilation in underived contexts}\label{section:segmentalPhono:nasalPlace:velarUnderived}
  	
  	In VCCV clusters, the nasal /n/ becomes [ŋ] before the velar stops /kʰ, ɡ/. Table \ref{tab:nasal velar vccv underived} shows the application of this rule in roots (underived contexts). 
  	
  	\begin{table}[H]
    \centering
    \caption{Nasal place assimilation before velar stops in VCCV contexts }
    \label{tab:nasal velar vccv underived}
    \begin{tabular}{|ll| ll| ll| }
    	\hline 
    	/kʰɑ\textbf{nkʰ}ɑd/ & `complaint' 
    	& /kʰɑ\textbf{nkʰ}uɾ/ & `curly'
    	& /hɑ\textbf{nkʰ}istʰ/ & `comfortable'
    	\\
    	{}[kʰɑ\textbf{ŋˈkʰ}ɑd] & \armenian{գանգատ} 
    	& [kʰɑ\textbf{ŋˈkʰ}uɾ] & \armenian{գանգուր}
    	& [hɑ\textbf{ŋˈkʰ}ist] & \armenian{հանգիստ}
    	\\
    	\hline 
    	/ə\textbf{nɡ}ujz/ & `walnut'
    	&/hɑ\textbf{nɡ}ɑɾd͡z/ & `suddenly'
    	& /ɑ\textbf{nɡ}oʁin/ & `bed'
    	\\
    	{}[ə\textbf{ŋˈɡ}ujz] & \armenian{ընկոյզ}
    	&[hɑ\textbf{ŋˈɡ}ɑɾd͡z] & \armenian{յանկարծ}
    	& [ɑ\textbf{ŋɡ}oˈʁin] & \armenian{անկողին}
    	\\\hline 
    \end{tabular}
  	\end{table}
  	
  	Velar place assimilation also applies in VCCCV clusters where the first two consonants are the nasal and velar (Table \ref{tab:nasal velar vcccv underived}). Such clusters tend to be found in bound roots. 
  	
  	
  	\begin{table}[H]
    \centering
    \caption{Nasal place assimilation before velar stops in VCCCV (VNCCV) contexts }
    \label{tab:nasal velar vcccv underived}
    \begin{tabular}{|ll| ll| }
    	\hline /ɑ\textbf{nkʰ}l-ijɑ/ & `English' ($\sqrt{}$-{\nmlz})
    	& 
    	/ɡɑ\textbf{nkʰ}n-i-l/ & `to stand' ($\sqrt{}$-{\thgloss}-{\infgloss})
    	\\
    	{}[ɑ\textbf{ŋkʰ}liˈjɑ] & \armenian{Անգլիա}
    	& [ɡɑ\textbf{ŋkʰ}ˈnil] & \armenian{կանգնիլ}
    	\\\hline 
    	/ə\textbf{nɡ}d͡ʒ-i-l/ & `to succumb' ($\sqrt{}$-{\thgloss}-{\infgloss})
    	& /hɑ\textbf{nkʰ}-t͡ʃ-i-l/ & `to rest' ($\sqrt{}$-{\vx}-{\thgloss}-{\infgloss})
    	\\
    	{}[ə\textbf{ŋɡ}ˈd͡ʒil]& \armenian{ընկճիլ}
    	&[hɑ\textbf{ŋk}ˈt͡ʃil] & \armenian{յանգչիլ}
    	\\
    	\hline 
    \end{tabular}
  	\end{table}
  	
  	Place assimilation also applies word-finally in VCC clusters (Table \ref{tab:nasal velar vcc underived}). 
  	
  	
  	\begin{table}[H]
    \centering
    \caption{Nasal place assimilation before velar stops in final VCC contexts }
    \label{tab:nasal velar vcc underived}
    \begin{tabular}{|ll|ll| ll| }
    	\hline 
    	/ʒɑɾɑ\textbf{nkʰ}/ & `heir'
    	& /nɑhɑ\textbf{nkʰ}/ & `state'
    	& /vɑɾu\textbf{nkʰ}/ & `cucumber'
    	\\
    	{}[ʒɑˈɾɑ\textbf{ŋkʰ}]& \armenian{ժառանգ}
    	&[nɑˈhɑ\textbf{ŋkʰ}] & \armenian{նահանգ}
    	&[vɑˈɾu\textbf{ŋkʰ}]& \armenian{վարունգ}
    	\\\hline 
    	/χu\textbf{nɡ}/ & `incense'
    	& /vɑ\textbf{nɡ}/ & `syllable'
    	&/ɑnɑ\textbf{nɡ}/ & `that way'
    	\\
    	{}[ˈχu\textbf{ŋɡ}] & \armenian{խունկ}
    	&[ˈvɑ\textbf{ŋɡ}] & \armenian{վանկ}
    	& [ɑˈnɑ\textbf{ŋɡ}] & \armenian{անանկ}
    	\\
    	\hline 
    \end{tabular}
  	\end{table}
  	
  	Although rare, there is one word which shows velar assimilation in a final VCCC cluster (Table \ref{tab:nasal velar vccc underived}). 
  	
  	
  	
  	\begin{table}[H]
    \centering
    \caption{Nasal place assimilation before velar stops in final VCCC contexts }
    \label{tab:nasal velar vccc underived}
    \begin{tabular}{|ll ll| }
    	\hline 
    	/ɑ\textbf{nkʰ}χ/ & [ˈɑ\textbf{ŋk}χ] & `vulture' & \armenian{անգղ}
    	\\
    	\hline 
    \end{tabular}
  	\end{table}
  	
  	The above examples concerned place assimilation in roots. There are likewise derivational suffixes that contain a sequence of a nasal /n/ and a velar stop: /-ɑnkʰ, -unkʰ/. These suffixes show place assimilation: [-ɑŋkʰ, -uŋkʰ]. Note that many of the suffixed forms in Table \ref{tab:nasal velar vcc suffixes} are derived from bound roots, that are often used in verbs. 
  	
  	
  	\begin{table}[H]
    \centering
    \caption{Nasal place assimilation before velar stops in derivational suffixes /-ɑnkʰ, -unkʰ/}
    \label{tab:nasal velar vcc suffixes}
    \resizebox{1\textwidth}{!}{%
    	\begin{tabular}{|llll|llll| }
      \hline 
      ɑbˈɾ-i-l & `to live' & $\sqrt{}$-{\thgloss}-{\infgloss} & \armenian{ապրիլ}
      & 
      d͡zɑʁˈɾ-e-l & `to mock' & $\sqrt{}$-{\thgloss}-{\infgloss} & \armenian{ծաղրել}
      \\
      ɑbˈɾ-ɑ\textbf{ŋkʰ} & `goods' & $\sqrt{}$-{\nmlz} & \armenian{ապրանք}
      & 
      d͡zɑʁˈɾ-ɑ\textbf{ŋkʰ} & `mockery' & $\sqrt{}$-{\nmlz} & \armenian{ծաղրանք}
      \\
      \hline
      mɑɾˈz-e-l & `to exercise' & $\sqrt{}$-{\thgloss}-{\infgloss} & \armenian{մարզել}
      & 
      hɑɾˈkʰ-e-l & `to respect' & $\sqrt{}$-{\thgloss}-{\infgloss} & \armenian{յարգել}
      \\
      mɑɾˈz-ɑ\textbf{ŋkʰ} & `exercise' & $\sqrt{}$-{\nmlz} & \armenian{մարզանք}
      & 
      hɑɾˈkʰ-ɑ\textbf{ŋkʰ} & `respect' & $\sqrt{}$-{\nmlz} & \armenian{յարգանք}
      \\
      \hline
      pəsˈχ-e-l & `to vomit' & $\sqrt{}$-{\thgloss}-{\infgloss} & \armenian{փսխել}
      & 
      iˈɾɑv & `truly' & $\sqrt{}$ & \armenian{իրաւ}
      \\
      pəsˈχ-u\textbf{ŋkʰ} & `vomit' & $\sqrt{}$-{\nmlz} & \armenian{փսխունք}
      & 
      iɾɑˈv-u\textbf{ŋkʰ} & `right (n)' & $\sqrt{}$-{\nmlz} & \armenian{իրաւունք}
      \\
      
      \hline 
    	\end{tabular}
  	}\end{table}
  	
  	
  	Thus nasal place assimilation is productive before velar stops in underived contexts. 
  	
  	\subsection{Velar assimilation in derived contexts}\label{section:segmentalPhono:nasalPlace:velarDerived}
  	In derived contexts, we can see roots which on their own surface with a coronal nasal [n]. When new words are derived from these roots, the nasal /n/ becomes adjacent to a velar stop /kʰ, ɡ/ and it becomes [ŋ]. 
  	
  	
  	In terms of suffixation, the only relevant suffix is the derivational suffix /-ɡod/ (Table \ref{tab:velar nasal suffix god}). This suffix is relatively rare. We've found roots that end in /n/ and that take this suffix. We see nasal place assimilation. 
  	
  	\begin{table}[H]
    \centering
    
    \caption{Nasal place assimilation before velar stops in suffixation}
    \label{tab:velar nasal suffix god}
    \begin{tabular}{|lllll| }
    	\hline
    	& bɑɾd͡zeˈ\textbf{n}-ɑ-l & `to boast' & $\sqrt{}$-{\thgloss}-{\infgloss} & \armenian{պարծենալ}
    	\\
    	$\rightarrow$ & bɑɾd͡ze\textbf{ŋ-ˈɡ}od & `boastful'& $\sqrt{}$-{\nmlz} & \armenian{պարծենկոտ}
    	\\\hline
    	& ˈkʰu\textbf{n} & `sleep' & $\sqrt{}$ & \armenian{քուն}
    	\\
    	$\rightarrow$ & kʰə\textbf{ŋ-ɡ}od & `sleepy' & $\sqrt{}$-{\nmlz} & \armenian{քնկոտ}
    	\\
    	\hline
    \end{tabular}
  	\end{table}
  	
  	Vowel reduction creates more contexts for nasal place assimilation. For the words in Table \ref{tab:velar nasal suffix vowel reduction}, the base ends in a NVC sequence. Some of these bases are suffixed forms themselves. In the derived forms, the vowel is deleted, the nasal and velar stop become adjacent, and the nasal assimilates. 
  	
  	
  	\begin{table}[H]
    \centering
    
    \caption{Nasal place assimilation before velar stops from vowel reduction}
    \label{tab:velar nasal suffix vowel reduction}
    \begin{tabular}{|lllll| }
    	\hline
    	& ˈpʰu\textbf{n} & `original' & $\sqrt{}$ & \armenian{բուն}
    	\\
    	& pʰəˈ\textbf{n-iɡ} & `native' & $\sqrt{}$-{\adjz} & \armenian{բնիկ}
    	\\
    	$\rightarrow$ & pʰə\textbf{ŋ-ɡ}-ɑˈt͡sʰum & `nativization' & $\sqrt{}$-{\adjz}-{\nmlz}& \armenian{բնկացում} 
    	\\\hline
    	
    	& ˈʃu\textbf{n} & `dog' & $\sqrt{}$ & \armenian{շուն}
    	\\
    	& ʃəˈ\textbf{n-iɡ} + ˈt͡suɡ & `puppy' + 
    	`fish' & $\sqrt{}$-{\dimgloss} & \armenian{շնիկ, ձուկ}
    	\\
    	$\rightarrow$ & ʃə\textbf{ŋ-ɡ}-ɑˈt͡suɡ & `dog-fish' & $\sqrt{}$-{\dimgloss}-{\lvgloss}-$\sqrt{}$& \armenian{շնկաձուկ} 
    	\\\hline
    	& mɑˈ\textbf{nuɡ} & `child' & $\sqrt{}$ & \armenian{մանուկ}
    	\\
    	$\rightarrow$ & mɑ\textbf{ŋɡ}-ɑˈɡɑn & `childish' & $\sqrt{}$-{\adjz} & \armenian{մանկական} 
    	\\
    	\hline
    	& jeɾt͡ʃɑˈ\textbf{niɡ} & `happy' & $\sqrt{}$ & \armenian{երջանիկ}
    	\\
    	$\rightarrow$ & jeɾt͡ʃɑ\textbf{ŋɡ}-ɑˈved & `happy' & $\sqrt{}$-{\adjz} & \armenian{երջանկաւէտ} 
    	\\
    	\hline
    \end{tabular}
  	\end{table}
  	
  	
  	Place assimilation likewise applies word-finally in derived contexts. For the words in Table \ref{tab:velar nasal appendix vnk}, the suffix /-kʰ/ is added after the nasal, causing the nasal to assimilate. 
  	
  	
  	\begin{table}[H]
    \centering
    
    \caption{Nasal place assimilation before velar stops from the suffix /-kʰ/ in VC-C contexts }
    \label{tab:velar nasal appendix vnk}
    \begin{tabular}{|lllll| }
    	\hline
    	& t͡ʃɑˈ\textbf{n}-ɑ-l & `to try' & $\sqrt{}$-{\thgloss}-{\infgloss} & \armenian{ջանալ}
    	\\
    	$\rightarrow$ & ˈt͡ʃɑ\textbf{ŋ-kʰ} & `effort' & $\sqrt{}$-{\nmlz} & \armenian{ջանք}
    	\\
    	\hline
    	& ɡəɾo\textbf{n}-ɑɡɑn & `religious' & $\sqrt{}$-{\adjz} & \armenian{կրօնական}
    	\\
    	$\rightarrow$ & ɡəˈɾo\textbf{ŋ-kʰ} & `religion' & $\sqrt{}$-{\nmlz} & \armenian{կրօնք}
    	\\
    	\hline
    	& məˈɡɑ\textbf{n} & `muscle' & $\sqrt{}$ & \armenian{մկան}
    	\\
    	$\rightarrow$ & məˈɡɑ\textbf{ŋ-kʰ} & `muscles' & $\sqrt{}$-{\nmlz} & \armenian{մկանք}
    	\\
    	\hline
    	& ɑɾˈʒɑ\textbf{n} & `cheap' & $\sqrt{}$ & \armenian{արժան}
    	\\
    	$\rightarrow$ & ɑɾˈʒɑ\textbf{ŋ-kʰ} & `worthiness' & $\sqrt{}$-{\nmlz} & \armenian{արժանք}
    	\\
    	\hline
    	& oˈɾe\textbf{n} & `law (archaic)' & $\sqrt{}$ & \armenian{օրէնք}
    	\\
    	$\rightarrow$ & oˈɾe\textbf{ŋ-kʰ} & `law' & $\sqrt{}$-{\nmlz} & \armenian{օրէնք}
    	\\
    	\hline 
    \end{tabular}
  	\end{table}
  	
  	Assimilation likewise occurs when /-kʰ/ is added after a VCN sequence (Table \ref{tab:velar nasal appendix vcnk}). 
  	
  	\begin{table}[H]
    \centering
    
    \caption{Nasal place assimilation before velar stops from the suffix /-kʰ/ in VCC-C contexts }
    \label{tab:velar nasal appendix vcnk}
    \begin{tabular}{|lllll| }
    	\hline
    	& ˈlɑ\textbf{jn} & `wide' & $\sqrt{}$ & \armenian{լայն}
    	\\
    	$\rightarrow$ & ˈlɑ\textbf{jŋ-kʰ} & `width' & $\sqrt{}$-{\nmlz} & \armenian{լայնք}
    	\\
    	\hline
    	& jeɾˈɡɑ\textbf{jn} & `long' & $\sqrt{}$ & \armenian{երկայն}
    	\\
    	$\rightarrow$ & jeɾˈɡɑ\textbf{jŋ-kʰ} & `length' & $\sqrt{}$-{\nmlz} & \armenian{երկայնք}
    	\\
    	\hline
    	& hɑˈmɑ\textbf{jn} & `whole' & $\sqrt{}$ & \armenian{համայն}
    	\\
    	$\rightarrow$ & hɑˈmɑ\textbf{jŋ-kʰ} & `community' & $\sqrt{}$-{\nmlz} & \armenian{համայնք}
    	\\
    	\hline
    \end{tabular}
  	\end{table}
  	
  	
  	
  	Complications arise when place assimilation interacts with stress (Table \ref{tab:velar assimilation prefix an-}). Armenian has primary stress on the final non-schwa vowel. There is reports of initial secondary stress but such a stress is very weak and not really perceptible. The exception is the negative prefix /ɑn-/. This prefix takes very perceptible secondary stress. In casual speech, the nasal becomes [ŋ] before velar stops. But in careful speech, the secondary stress on /ɑn-/ can cause the nasal to optionally surface as [ɑn-]. 
  	
  	\begin{table}[H]
    \centering
    \caption{Variable nasal place assimilation for the prefix /ɑn-/}
    \label{tab:velar assimilation prefix an-}
    \resizebox{1\textwidth}{!}{%
    	\begin{tabular}{|llll|llll| }
      \hline 
      & ˈhɑm & `taste' & \armenian{համ} 
      & & ɑɾˈtʰɑɾ & `just' & \armenian{արդար} \\
      $\rightarrow$ & ˈɑn-ˈhɑm & `tasteless' & \armenian{անհամ} 
      & $\rightarrow$ & ˈɑn-ɑɑɾˈtʰɑɾ & `unjust' & \armenian{անարդար}
      \\ \hline
      & ˈ\textbf{kʰ}oɾd͡z& `work' & \armenian{գործ} 
      && ˈ\textbf{ɡ}eʁd͡z& `false' & \armenian{կեղծ} 
      \\
      $\rightarrow$ & ˌɑ\textbf{ŋ-ˈkʰ}oɾd͡z & `unemployed' & \armenian{անգործ} 
      & $\rightarrow$ & ˌɑ\textbf{ŋ-ˈɡ}eʁd͡z & `sincere' & \armenian{անկեղծ} 
      \\ 
      $\rightarrow$ & ˌɑ\textbf{n-ˈkʰ}oɾd͡z & & (careful) & 
      $\rightarrow$ & ˌɑ\textbf{n-ˈɡ}eʁd͡z & & (careful) 
      \\ \hline
      & \textbf{kʰ}iˈdɑɡ & `wise' & \armenian{գիտակ} 
      && \textbf{ɡ}ɑsˈkɑd͡z& `doubt' & \armenian{կասկած} \\
      $\rightarrow$ & ˌɑ\textbf{ŋ-kʰ}iˈdɑɡ & `unknowing' & \armenian{անգիտակ} 
      & $\rightarrow$ & ˌɑ\textbf{ŋ-ɡ}ɑsˈkɑd͡z & `doubtless' & \armenian{անկասկած} 
      \\ 
      $\rightarrow$ & ˌɑ\textbf{n-kʰ}iˈdɑɡ & & (careful) & 
      $\rightarrow$ & ˌɑ\textbf{n-ɡ}ɑsˈkɑd͡z & & (careful) 
      \\ \hline
      & \textbf{kʰ}eʁeˈt͡siɡ & `beautiful' & \armenian{գեղեցիկ} 
      && \textbf{ɡ}ɑɾeˈli & `possible' & \armenian{կարելի} \\
      $\rightarrow$ & ˌɑ\textbf{ŋ-kʰ}eʁeˈt͡siɡ & `ugly' & \armenian{անգեղեցիկ} 
      & $\rightarrow$ & ˌɑ\textbf{ŋ-ɡ}ɑɾeˈli & `impossible' & \armenian{անկարելի} 
      \\ 
      $\rightarrow$ & ˌɑ\textbf{n-kʰ}eʁeˈt͡siɡ & & (careful) & 
      $\rightarrow$ & ˌɑ\textbf{n-ɡ}ɑɾeˈli & & (careful) 
      \\ \hline
    	\end{tabular}
    }
  	\end{table}
  	
  	
  	The above is based on HD's perception though. Experimental data is needed to accurately know the degree of velarization (or lack of velarization) of the prefix /ɑn-/ in natural and controlled speech. 
  	
  	The effect of stress is stronger the words in Table \ref{tab:velar assimilation irregular quantifier}. These words are compounds where the first stem ends in a nasal, and the second stem starts with a velar stop. These words are quantifier and they have irregular primary stress on the first stem. In HD's perception, the nasal is preferably [n] instead of [ŋ]. 
  	
  	
  	\begin{table}[H]
    \centering
    \caption{Variable nasal place assimilation words with irregular stress}
    \label{tab:velar assimilation irregular quantifier}
    \begin{tabular}{|llll l| }
    	\hline 
    	& ˈɑ\textbf{jn} + ˈ\textbf{kʰ}ɑn & & `that' + `than' & \armenian{այն, քան} 
    	\\
    	$\rightarrow$ & ˈɑ\textbf{jn}-\textbf{ˈkʰ}ɑn & ˈɑ\textbf{jŋ}-\textbf{ˈkʰ}ɑn & `that much' & \armenian{այնքան} 
    	\\
    	\hline 
    	& ˈnu\textbf{jn} + ˈ\textbf{kʰ}ɑn && `same' + `than' & \armenian{նոյն, քան} \\
    	$\rightarrow$ & ˈnu\textbf{jn}-\textbf{ˈkʰ}ɑn & ˈnu\textbf{jŋ}-\textbf{ˈkʰ}ɑn & `as much' & \armenian{նոյնքան}
    	\\ \hline
    \end{tabular}
  	\end{table}
  	
  	In sum, nasal place assimilation is productive before velar stops in derived contexts. There is some complications from stress. Stressed nasals seem to resist velarization, but more acoustic data is needed. 
  	
  	\subsection{Other forms of nasal place assimilation}\label{section:segmentalPhono:nasalPlace:other}
  	
  	
  	Nasal place assimilation is productive before the velar stops /kʰ, ɡ/. But there is little to no evidence of productive place assimilation before other types of places, i.e., there is no assimilation before uvulars or labials. 
  	
  	The dorsal fricatives /χ, ʁ/ do not trigger any place assimilation (Table \ref{tab:nasal assimilation no dorsal fricative}). The nasal /n/ surfaces as [n] before them. We haven't found relevant examples from underived contexts, but there are some cases from derived contexts. 
  	
  	\begin{table}[H]
    \centering
    \caption{No nasal place assimilation before dorsal uvular fricatives}
    \label{tab:nasal assimilation no dorsal fricative}
    \begin{tabular}{|lllll|}
    	\hline 
    	& ɡɑˈ\textbf{nuχ} & `early' & $\sqrt{}$ & \armenian{կանուխ}
    	\\
    	$\rightarrow$ & ɡɑ\textbf{nˈχ}-e-l & `to anticipate' & $\sqrt{}$-{\thgloss}-{\infgloss} & \armenian{կանխել}
    	\\
    	\hline 
    	& ˈ\textbf{χ}iʁd͡ʒ & `conscience' & $\sqrt{}$ & \armenian{խիղճ}
    	\\
    	$\rightarrow$ & ˌɑ\textbf{n-ˈχ}iʁd͡ʒ & `unscrupulous' & {\negan}-$\sqrt{}$ & \armenian{անխիղճ}
    	\\
    	\hline 
    	& \textbf{ʁ}eɡ-ɑ-ˈvɑɾ & `leader' & $\sqrt{}$-{\lvgloss}-$\sqrt{}$ & \armenian{ղեկավար}
    	\\
    	$\rightarrow$ & ˌɑ\textbf{n-ʁ} eɡ-ɑ-ˈvɑɾ & `undirected' & {\negan}-$\sqrt{}$-{\lvgloss}-$\sqrt{}$ & \armenian{անղեկավար}
    	\\
    	\hline 
    \end{tabular}
  	\end{table}
  	
  	Before labial stops /pʰ, b/, the nasal /n/ surfaces as [n] (Table \ref{tab:nasal place bilabial synchrony no}). However, it is rare to find roots or underived contexts which have an underlying /npʰ/ or /nb/ sequence. In derived contexts, compounding and prefixation can create [npʰ] and [nb] sequences, without any assimilation. 
  	
  	\begin{table}[H]
    \centering
    \caption{No synchronic nasal place assimilation before bilabial stops}
    \label{tab:nasal place bilabial synchrony no}
    \begin{tabular}{|llll|llll| }
    	\hline 
    	& ˈɑj\textbf{n} + ˈ\textbf{b}es & `that' + `way' & \armenian{այն, պէս}
    	
    	\\
    	$\rightarrow$ & ɑj\textbf{n}-ˈ\textbf{b}es & `like that' & \armenian{այնպէս}
    	\\ \hline
    	& ˈpʰɑ\textbf{n} + ˈ\textbf{pʰ}eɾ & `word (archaic)' + 'bring!' & \armenian{բան, բեր}
    	\\ $\rightarrow$ & pʰɑ\textbf{n}-ˈ\textbf{pʰ}eɾ & `messenger' & \armenian{բանբեր}
    	\\\hline
    	& \textbf{b}əˈduχ + & `fruit' & \armenian{պտուղ}
    	\\
    	$\rightarrow$ & ɑ\textbf{n}-\textbf{b}əˈduχ & `unfruitful' & \armenian{անպտուղ}
    	\\ \hline 
    	& \textbf{pʰ}ənɑˈɡɑn + & `natural' & \armenian{բնական}
    	\\
    	$\rightarrow$ & ɑ\textbf{n}-\textbf{pʰ}ənɑˈɡɑn & `unnatural' & \armenian{անբնական}
    	\\ \hline 
    \end{tabular}
  	\end{table}
  	
  	Diachronically however, there is a process of nasal place assimilation before labials (Table \ref{tab:nasal place bilabial diachrony}). For example, there are modern words which have a [mpʰ] or [mb] cluster, but these cluster diachronically derived from an /n(V)pʰ/ or /n(V)b/ sequence. 
  	
  	
  	\begin{table}[H]
    \centering
    \caption{Diachronic nasal place assimilation before bilabial stops}
    \label{tab:nasal place bilabial diachrony}
    \resizebox{1\textwidth}{!}{%
    	\begin{tabular}{|lllll| }
      \hline 
      Modern form: &d͡ʒɑ\textbf{mˈpʰ}ɑ & `road' & \armenian{ճամբայ} & 
      \\
      Historical source: & d͡ʒɑ\textbf{nɑˈpʰ}ɑɾ & `road' & \armenian{ճանապարհ} & \citep[182-3]{Adjarian-1979-Etymology}
      \\ \hline
      Modern form: &ɑ\textbf{mb}ɑˈɾiʃt & `wicked' & \armenian{ամբարիշտ} & 
      \\
      Historical source: & <a\textbf{nb}ariʃd> & reconstructed & \armenian{անպարիշտ} & \citep[149]{Adjarian-1979-Etymology}
      \\ \hline
    	\end{tabular}
    }
  	\end{table}
  	
  	
  	
  	
  	
  	Before labial nasal /m/, the nasal /n/ does not assimilate. We have not found this sequence in any underived contexts, but it is abundant in prefixation. However, there are some high-frequency words where we do find optional assimilation (Table \ref{tab:nasal place bilabial n m before m}). 
  	
  	
  	\begin{table}[H]
    \centering
    \caption{Lack of synchronic nasal place assimilation before nasal /m/, with one exception}
    \label{tab:nasal place bilabial n m before m}
    \resizebox{1\textwidth}{!}{%
    	\begin{tabular}{|llll|llll| }
      \hline 
      & ˈ\textbf{m}ɑh & `death' & \armenian{մահ}
      && ˈ\textbf{m}ɑjɾ & `mother' & \armenian{մայր}
      \\
      $\rightarrow$ & ˌɑ\textbf{n-ˈm}ɑh & `deathless' & \armenian{անմահ}
      &$\rightarrow$ & ˌɑ\textbf{n-ˈm}ɑjɾ & `motherless' & \armenian{անմայր}
      \\\hline
      & \textbf{m}ɑˈkʰuɾ & `clean' & \armenian{մաքուր}
      && \textbf{m}eɾʒeˈli & `rejectable' & \armenian{մերժելի}
      \\
      $\rightarrow$ & ˌɑ\textbf{n-m}ɑˈkʰuɾ & `unclean' & \armenian{անմաքուր}
      &$\rightarrow$ & ˌɑ\textbf{n-m}eɾʒeˈli & `irrecusable' & \armenian{անմերժելի}
      \\\hline
      & pʰɑ\textbf{n-m}ə & `a thing' & \armenian{բան մը} && ˌɑ\textbf{n-b}ɑjˈmɑn & `necessarily' & \armenian{անպայման}
      \\
      $\sim$ & pʰɑ\textbf{m-m}ə &(thing-{\indf}) & (casual speech) & $\sim$ & ˌɑ\textbf{m-b}ɑjˈmɑn &({\negan}-$\sqrt{}$) & (casual speech) 
      \\ \hline 
    	\end{tabular}
    } 
  	\end{table}
  	
  	
  	Within the Armenian lexicon, it is rare to find /n/+labial sequences in roots, but it is quite common to find /m/+labial sequences (Table \ref{tab:root mb mp}). 
  	
  	\begin{table}[H]
    \centering
    \caption{Roots with /mpʰ/ or /mb/}
    \label{tab:root mb mp}
    \begin{tabular}{|lll|lll| }
    	\hline 
    	ɑ\textbf{mˈpʰ}oχt͡ʃ & `entire' & \armenian{ամբողջ} 
    	& ɑ\textbf{mˈpʰ}opʰ &`tight' & \armenian{ամփոփ}
    	\\
    	ˈɑ\textbf{mb} &`cloud' & \armenian{ամպ} 
    	& ˈχu\textbf{mpʰ} & `group' & \armenian{խումբ}
    	\\ \hline
    \end{tabular}
  	\end{table}
  	
  	
  	It is unclear if there is a synchronically active constraint against having /n/+labial sequences in roots in modern Armenian. Such a constraint is likely just diachronic, not synchronic. The only case where do see a synchronic alternation is in high-frequency collocation. 
  	
  	\section{Allophonic differences from Eastern Armenian}\label{section:segmentalPhono:alloEastern}
  	
  	Eastern and Western Armenian are different varieties of Armenian. The two have similar but non-identical phoneme inventories. The dialects however share a large proportion of their allophony in common. In this section, we overview allophonic processes that have been reported in Eastern Armenian and which either exist or don't exist in Western Armenian. 
  	
  	We likewise overview some processes which seem to apply in some varieties of Western Armenian but not others. We also note phonological processes that are optional and restricted to connect speech. 
  	
  	Unless otherwise specified, the Eastern examples are from English Wiktionary. The Wiktionary examples are heavily moderated and are reliable for Eastern Armenian. For transliterations, we adopt ISO 9985 to transliterate the words for Eastern Armenian,\footnote{\url{https://www.translitteration.com/transliteration/en/armenian-eastern-classical/iso-9985/}} while our own transliteration for Western Armenian.
  	
  	
  	\subsection{Palatalization}\label{section:segmentalPhono:alloEastern:palatalization}
  	Because of contact with Russian, Eastern Armenian has been slowly developing a rule of palatalizing dental stops to affricates before /j/: /tʰj/ $\rightarrow$ [t͡sʰj]. This rule is particularly common in the nominalizing suffix /-utʰjun/ \armenian{ություն} which is almost always pronounced as [-ut͡sʰjun] in modern Eastern Armenian. 
  	
  	\textcolor{red}{cite vaux, and cite examples}
  	
  	For Western Armenian, there is no Russian contact so on such palatalization rule exists. The closest analog is affrication of the suffix /-utʰjʏn/ \armenian{-ութիւն}. Whereas this suffix is often pronounced as [-ut͡sʰjun] in Eastern Armenian, this suffix is often pronounced as [-t͡ʃʏn] in Western Armenian. 
  	
  	Note that the suffix /-utʰjʏn/ shows a lot of speaker and register variation (Table \ref{tab:utjun pron}). The most formal pronunciation is [-utʰʏn]. But in casual speech, this suffix can variably be pronounced as [-utʰjun], [-utʰjʏn], [-ut͡ʃjun], [-ut͡ʃʏn], among other options. We do not know the probability or the frequency of the different pronunciations. We do not know what social factors correlate with any of these choices. 
  	
  	
  	\begin{table}[H]
    \centering
    \caption{Variation in the pronunciation of the nominalizer /-utjun/ suffix}
    \label{tab:utjun pron}
    \begin{tabular}{|llll| }
    	\hline 
    	Adjective: & uˈɾɑχ & `happy' & \armenian{ուրախ}
    	\\
    	Nominalized & uɾɑχ-uˈtʰjun & `happiness' & \armenian{ուրախութիւն}
    	\\
    	& uɾɑχ-uˈtʰjʏn& & 
    	\\
    	& uɾɑχ-uˈtʰʏn& & 
    	\\
    	& uɾɑχ-uˈt͡ʃʏn& & 
    	\\
    	& uɾɑχ-uˈt͡ʃjʏn& & 
    	\\
    	& uɾɑχ-uˈt͡ʃjun& & 
    	\\
    	& uɾɑχ-uˈt͡ʃun& & 
    	\\
    	\hline 
    \end{tabular}
  	\end{table}
  	
  	For the vowel, the original vowel sequence /ju/ is often fused into a single round vowel /ʏ/.\footnote{See \citep{Avetyan-2015-WesternRoundVowel} for discussion on the diachronic changes in this suffix's pronunciation. } This is a common process in Armenian (\S\ref{section:segmentalPhono:vowel:frontRound}). The stop /tʰ/ often becomes [t͡ʃ] in this context. The change from /tʰ/ to [t͡ʃ] is unique to this morpheme and is not a language-general rule, i.e., it is a morpheme-specific rule. 
  	
  	
  	
  	
  	\subsection{Deaspiration and voicing assimilation}\label{section:segmentalPhono:alloEastern:deasspAssimi}
  	
  	Western Armenian has deaspiration of stops when adjacent to a fricative, affricate, or another stop. This was surveyed in Section \S\ref{section:segmentalPhono:allphonLaryng}. But to our knowledge, such deaspiration does not exist in Eastern Armenian. 
  	
  	To illustrate, the Table \ref{tab:post fric deaspiration eastern} provides Western and Eastern forms. The Western forms show deaspiration of the stop, while the Eastern form does not. Note that the initial schwa in Eastern is optional. 
  	
  	\begin{table}[H]
    \centering
    \caption{Post-fricative deaspiration in Western but not Eastern Armenian}
    \label{tab:post fric deaspiration eastern}
    \begin{tabular}{|llll|}
    	\hline Spelling& Western & Eastern & Meaning\\
    	\hline \armenian{սփոփել}&
    	ə\textbf{sp}oˈpʰel & (ə)\textbf{spʰ}oˈpʰel & `to comfort'
    	\\
    	\armenian{սթափ}&
    	ə\textbf{sˈt}ɑpʰ & (ə)\textbf{sˈtʰ}ɑpʰ & `sober'
    	\\
    	\armenian{սքանչելի}&
    	ə\textbf{sk}ɑnt͡ʃeˈli & (ə)\textbf{skʰ}ɑnt͡ʃʰeˈli & `wonderful'
    	\\
    	\hline
    \end{tabular}
  	\end{table}
  	
  	One possible reason as to why the dialects differ in this respect is phonemicity. Aspiration is phonemic in Eastern, but it is not in Western. Thus, there is no loss in phonemic contrasts when a Western stop is deaspirated after a fricative.\footnote{We thank Scott Seyfarth for discussion. } 
  	
  	Another apparent area of difference is voicing assimilation (Table \ref{tab:voicing differences assimilation dialectal}). In obstruent clusters, both Western and Eastern Armenian are reported to have regressive assimilation in voiced+voiceless clusters \citep[35,100-107]{Khachatryan-1988-ArmenianPhono}. But for voiceless+voiced clusters, Western Armenian has progressive assimilation (devoicing) while Eastern Armenian can keep the cluster unchanged. 
  	
  	\begin{table}[H]
    \centering
    \caption{Dialectal differences in voicing assimilation}
    \label{tab:voicing differences assimilation dialectal}
    \begin{tabular}{|l|ll|ll|}
    	\hline 
    	& Western & & Eastern& \\
    	\hline 
    	Regressive & ʁɑɾɑˈpʰɑ\textbf{ʁ} & `Karabagh' & ʁɑɾɑˈbɑʁ & `Karabagh' 
    	\\
    	& & \armenian{Ղարաբաղ} & & \armenian{Ղարաբաղ} \\
    	$\rightarrow$ & ʁɑɾɑpʰɑ\textbf{χ-ˈt͡s}i & `Karabaghian' & ʁɑɾɑbɑ\textbf{χ-ˈt͡sʰ}i & `Karabaghian' 
    	\\
    	& & \armenian{ղարաբաղցի} & & \armenian{Ղարաբաղցի} \\
    	\hline 
    	Progressive & tʰɑntʰɑ\textbf{ʁ-ˈɡ}od & `slowish'& pəˈ\textbf{tuʁ} & `fruit'
    	\\
    	& & \armenian{դանդաղկոտ} & & \armenian{պտուղ}
    	\\
    	$\rightarrow$& vɑ\textbf{χ-ˈk}od & `coward' & pə\textbf{tʁ}-ɑ-beɾ & `fruit-bearing'
    	\\
    	& & \armenian{վախկոտ} & & \armenian{պտղաբեր}
    	\\ \hline
    	
    	
    \end{tabular}
  	\end{table}
  	
  	Unfortunately to our knowledge, there isn't a systematic study on productive voicing assimilation processes in Eastern Armenian. So we cannot say if Eastern Armenian truly lacks progressive assimilation. 
  	\subsection{Sonorant devoicing}\label{section:segmentalPhono:alloEastern:sonorantDevoicing}
  	For Eastern Armenian, it is reported that sonorants can devoice when word-final. We have not been able to verify whether this process applies in Western or not. Our impression is that this process is a rather low-level phonetic rule, and thus not perceptible to speakers. 
  	
  	\textcolor{red}{cite}
  	
  	Although there is voicing assimilation of obstruents in an obstruent cluster (\S\ref{section:segmentalPhono:allphonLaryng}), we also don't know if sonorants get devoiced when adjacent to a voiceless obstruent. 
  	
  	\section{Sandhi phenomena or connected speech processes}\label{section:segmentalPhono:sandhi}
  	\subsection{Word-final devoicing}\label{section:segmentalPhono:sandhi:finalDevoice}
  	
  	\textcolor{red}{mention \armenian{օգուտ մէկ տափատ}}
  	
  	In both Western and Eastern Armenian, voicing contrasts can be found word-finally for stops and affricates. We illustrate below with the labial series (Table \ref{tab:final labial phoneme easter}). 
  	
  	\begin{table}[H]
    \centering
    \caption{Phonemic voicing for final labials in Eastern and Western}
    \label{tab:final labial phoneme easter}
    \begin{tabular}{|llll| }
    	\hline 
    	& Eastern & Western & \\
    	\armenian{թագ} 
    	& ˈtʰɑ\textbf{ɡ}& ˈtʰɑ\textbf{kʰ} & `crown'
    	\\
    	\armenian{թակ} 
    	& ˈtʰɑ\textbf{k}& ˈtʰɑ\textbf{ɡ} & `mallet'
    	\\
    	\armenian{թաք} 
    	& ˈtʰɑ\textbf{kʰ}& ˈtʰɑ\textbf{kʰ} & `hiding'
    	\\
    	\hline 
    \end{tabular}
    
  	\end{table}
  	
  	However in both Eastern and Western Armenian, there is evidence that there is some sort of gradient devoicing process. For Eastern Armenian, there is likewise a diachronic devoicing process.
  	
  	In Eastern Armenian, there are some words which are spelled with a final voiced stop/affricate, but this sound is pronounced as voiceless. We provide Western forms for completeness (Table \ref{tab:final devoice stop eastrn}).  
  	
  	\begin{table}[H]
    \centering
    \caption{Words in Eastern Armenian that are spelled with a final voiced stop/affricate but are pronounced as voiceless}
    \label{tab:final devoice stop eastrn}
    \begin{tabular}{|l| l|ll|ll| l| }
    	\hline 
    	Letter & Word & \multicolumn{2}{l|}{Transliteration} & \multicolumn{2}{l|}{Pronunciation} \\
    	& & EA & WA & EA & WA & Meaning 
    	\\
    	\hline 
    	\armenian{բ}& \armenian{Հակոբ} 
    	& <Hako\textbf{b}> & <Hagop> & hɑˈko\textbf{pʰ}& hɑˈɡopʰ & masc. name
    	\\
    	\armenian{գ} & \armenian{ձագ} & 
    	<ja\textbf{g}> & <t͡sak> & ˈd͡zɑ\textbf{kʰ} & ˈt͡sɑkʰ & `cub'
    	\\
    	\armenian{դ} & \armenian{օդ} & 
    	<ò\textbf{d}> & <ōt>
    	& ˈo\textbf{tʰ} & ˈotʰ & `air'
    	\\
    	\armenian{ձ} & \armenian{օձ} & 
    	<ò\textbf{j}> & <ōt͡s>
    	& ˈo\textbf{t͡sʰ} & ˈot͡s & `snake'
    	\\
    	\armenian{ջ} & \armenian{աջ} & 
    	<a\textbf{ǰ}> & <at͡ʃ>
    	& ˈɑ\textbf{t͡ʃʰ} & ˈɑt͡ʃ & `right'
    	\\
    	\hline 
    	
    	
    \end{tabular}
  	\end{table}
  	
  	
  	
  	However, it is unlikely that this devoicing is due to a synchronic phonological rule (Table \ref{tab:final no devoice stop eastern}). For example, voiced stops and affricates can surface word-finally in some words. 
  	
  	\begin{table}[H]
    \centering
    \caption{Words in Eastern Armenian that are spelled with a final voiced stop/affricate and are voiced in pronunciation}
    \label{tab:final no devoice stop eastern}
    \begin{tabular}{|l| l|ll|ll| l| }
    	\hline 
    	Letter & Word & \multicolumn{2}{l|}{Transliteration} & \multicolumn{2}{l|}{Pronunciation} & \\
    	& & EA & WA & EA & WA 
    	&
    	\\\hline 
    	
    	\armenian{բ}& \armenian{արաբ} & 
    	<ara\textbf{b}> & <arap> & ɑˈɾɑ\textbf{b}& ɑˈɾɑpʰ & `Arab' 
    	\\
    	\armenian{գ} & \armenian{արագ} & 
    	<ara\textbf{g}> & <arak> & ɑˈɾɑ\textbf{ɡ} & ɑˈɾɑkʰ & `fast'
    	\\
    	\armenian{դ} & \armenian{բադ} & 
    	<ba\textbf{d}> & <pat> & ˈbɑ\textbf{d} & ˈpʰɑtʰ & `duck'
    	\\
    	\armenian{ձ} & \armenian{նախանձ} & 
    	<naxan\textbf{j}> & <naxant͡s>
    	& nɑˈχɑn\textbf{d͡z} & nɑˈχɑnt͡s & `jealousy'
    	\\
    	\armenian{ջ} & \armenian{քաջ} & 
    	<k’a\textbf{ǰ}> & <k'at͡ʃ>
    	& ˈkʰɑ\textbf{d͡ʒ} & ˈkʰɑt͡ʃ & `brave'
    	\\
    	
    	\hline 
    	
    \end{tabular}
  	\end{table}
  	
  	
  	Furthermore, for those words which have this devoiced stop or affricate in their citation form, the sound is still pronounced as devoiced in other derived or inflected forms (Table \ref{tab:final devoice stop eastrn infected}). 
  	
  	\begin{table}[H]
    \centering
    \caption{Inflected form of words in Eastern Armenian that are spelled with a final voiced stop/affricate but are pronounced as voiceless}
    \label{tab:final devoice stop eastrn infected}
    \begin{tabular}{|l| l|ll|ll| l| }
    	\hline 
    	Letter & Word & \multicolumn{2}{l|}{Transliteration} & \multicolumn{2}{l|}{Pronunciation} & Meaning \\
    	& & EA & WA & EA & WA &
    	
    	\\
    	\hline 
    	\armenian{բ}& \armenian{Հակոբը} 
    	& <Hako\textbf{b}ë> & <Hagopə> & hɑˈko\textbf{pʰ}-ə& hɑˈɡopʰ-ə & masc. name ({\defgloss}) 
    	\\
    	\armenian{գ} & \armenian{ձագեր} & 
    	<ja\textbf{g}er> & <t͡saker> & d͡zɑˈ\textbf{kʰ}-eɾ & t͡sɑˈkʰ-eɾ & `cub-{\pl}'
    	\\
    	\armenian{դ} & \armenian{օդի} & 
    	<ò\textbf{d}i> & <ōti>
    	& oˈ\textbf{tʰ}-i & oˈtʰ-i & `air-{\gen}'
    	\\
    	\armenian{ձ} & \armenian{օձս} & 
    	<ò\textbf{j}s> & <ō\t{tss}>
    	& ˈo\textbf{t͡sʰ-əs} & ˈot͡səs & `snake-{\possFsg}'
    	\\
    	\armenian{ջ} & \armenian{աջով} & 
    	<a\textbf{ǰ}ov> & <at͡ʃov>
    	& ɑˈ\textbf{t͡ʃʰ}-ov & ɑˈt͡ʃ-ov & `right-{\ins}'
    	\\
    	\hline 
    	
    	
    \end{tabular}
  	\end{table}
  	
  	
  	If Eastern Armenian had true final devoicing, we would expect to see morpheme alternations where some morpheme is pronounced with a voiceless stop when said in isolation, but then pronounced with a voiced stop when suffixes are added. This does not happen. 
  	
  	The most likely scenario is that, again, this final devoicing rule is just an orthography-phonology mismatch which applied as a diachronic rule, not an active synchronic rule. 
  	
  	
  	For Western Armenian, we don't see such an orthography-phonology mismatch. Words that are spelled with a final voiced stop are pronounced as such. However, there seems to be a gradient rule of final devoicing that varies by word, speaker, region, and by register. 
  	
  	For example, in HD's ideolect, certain words are prescriptively pronounced with a final voiced stop (Table \ref{tab:western final devoicing}). But in causal speech, the stop is optionally devoiced word-finally. Such devoicing doesn't occur when suffixes are added, making the stop intervocalic. HD self-reports that the ``devoicing'' can also manifest as just un-releasing the final voiced stop. We transcribe this ``devoiced'' or unreleased form as just a voiceless unaspirate. 
  	
  	\begin{table}[H]
    \centering
    \caption{Words with variable final devoicing in HD's Western Armenian pronunciation}
    \label{tab:western final devoicing}
    \begin{tabular}{|lllll |}
    	\hline 
    	& & Final voicing& Final devoicing \\
    	\hline 
    	Root& \armenian{կապ}
    	&ˈɡɑ\textbf{b} & ˈɡɑ\textbf{p} & `connection' \\
    	$\rightarrow$& \armenian{կապեր}
    	&ɡɑˈ\textbf{b}-er& & `connection-{\pl}' \\
    	\hline 
    	Root& \armenian{կապիկ} &ɡɑˈbi\textbf{ɡ} & ɡɑˈbi\textbf{k} & `monkey' \\
    	$\rightarrow$& \armenian{կապիկս} &ɡɑˈbi\textbf{ɡ}-əs & & `monkey-{\possFsg}' \\
    	\hline 
    	Root& \armenian{ազատ} 
    	&ɑˈzɑ\textbf{d} & ɑˈzɑ\textbf{t} & `free' \\
    	$\rightarrow$& \armenian{ազատը} 
    	&ɑˈzɑ\textbf{d}-ə & & `free-{\defgloss}' \\
    	\hline 
    	Root& \armenian{տաբատ} 
    	&dɑˈpʰɑ\textbf{d} & dɑˈpʰɑ\textbf{t} & `pants' \\
    	$\rightarrow$& \armenian{տաբատով} 
    	&dɑpʰɑˈ\textbf{d}-ov & & `pants-{\ins}' \\
    	\hline 
    	Root& \armenian{շատ} 
    	&ˈʃɑ\textbf{d} & ˈʃɑ\textbf{t} & `many' \\
    	$\rightarrow$& \armenian{շատեր} 
    	&ʃɑˈ\textbf{d}-eɾ & & `many-{\pl}' \\
    	\hline 
    	Root& \armenian{ծակ} 
    	&ˈd͡zɑ\textbf{ɡ} & ˈd͡zɑ\textbf{k} & `hole' \\
    	$\rightarrow$& \armenian{ծակի} 
    	&d͡zɑˈ\textbf{ɡ}-i & & `hole-{\gen}' \\
    	\hline 
    	
    \end{tabular}
  	\end{table}
  	
  	For the Lebanese community, this devoicing process is optional and limited to a handful of high-frequency words in connected speech. For Turkish-speaking communities such as in Istanbul, TT reports that devoicing is significantly more common. HS reports significant devoicing as well, and she is a Turkish-Armenian bilingual from Syria. Anaid Donabedian self-reports devoicing in her French community as well. For TT, HS, and Anaid Donabedian, it seems that devoicing is more frequent and more obligatory than for HD and the Lebanese community. 
  	
  	
  	We cannot study in depth the rate of final devoicing. It seems that such a process is highly variable by speaker, geographic region, and by register. An ideal future research question is to examine the rate of devoicing in an oral corpus of natural speech. We speculate that devoice will vary not only by speaker, but may also show signs of incomplete devoicing or incomplete neutralization. 
  	
  	
  	\subsection{/h/ deletion}\label{section:segmentalPhono:sandhi:HDeletion}
  	In modern Armenian, the orthography has a letter \armenian{հ} for the sound /h/. There are many words which are spelled with an <h> either word-initially or word-finally. The /h/ is pronounced in careful speech. But in casual connected speech, this /h/ is optionally deleted in some words (\citealt[162]{Gharagulyan-1974-BookArmenianOrthoepy}; \citealt[64]{Margaryan-1997-ArmenianPhonology}). 
  	
  	We illustrate below with some common words which start with /h/ (\ref{tab:h deletion sandhi yeah}). This /h/ is pronounced in careful speech, but can optionally dropped in casual speech after a consonant or vowel. A frequent target of deletion is the classifier [hɑd]. 
  	
  	\begin{exe}
    \ex Words which show optional /h/ deletion in connected speech \label{tab:h deletion sandhi yeah}
    \begin{xlist}
    	\ex \glll jeɾɡu \textbf{h}ɑzɑɾ (Careful)
    	\\
    	two ɑˈzɑɾ (Casual)
    	\\
    	two thousand 
    	\\
    	\trans `two thousand.' 
    	\\
    	\armenian{երկու հազար}
    	\ex \glll t͡ʃoɾoɾtʰ ˈ\textbf{h}ɑɾɡ-ə (Careful)
    	\\
    	t͡ʃoɾoɾtʰ ˈɑɾɡ-ə (Casual)
    	\\
    	forth floor-{\defgloss} 
    	\\
    	\trans `the fourth floor'
    	\\
    	\armenian{չորրորդ յարկը}
    	\ex \glll meɡ hɑd ˈkʰiɾkʰ (Careful)
    	\\
    	meɡ ɑd kʰiɾkʰ (Casual)
    	\\
    	one {\clf} book 
    	\\
    	\trans `one book.'
    	\\
    	\armenian{մէկ հատ գիրք}
    \end{xlist}
  	\end{exe}
  	
  	A frequent target of /h/ deletion is the classifier [hɑd] (\ref{tab:h deletion sandhi classifier}). It follows numerals and precedes nouns. The /h/ deletes in casual speech after either a vowel or consonant. 
  	
  	
  	\begin{exe}
    \ex /h/ deletion for the classifier /hɑd/ \label{tab:h deletion sandhi classifier}
    \begin{xlist}
    	\ex \glll jeɾɡu \textbf{h}ɑd ˈkʰirk (Careful)
    	\\
    	jeɾɡu ɑd ˈkʰiɾkʰ (Casual)
    	\\
    	two {\clf} book
    	\\
    	\trans `two books.' 
    	\\
    	\armenian{երկու հատ գիրք}
    	\ex \glll jeɾekʰ \textbf{h}ɑd ˈkʰirk (Careful)
    	\\
    	jeɾɡu ɑd ˈkʰirk (Casual)
    	\\
    	three {\clf} book
    	\\
    	\trans `three books.' 
    	\\
    	\armenian{երեք հատ գիրք}
    	
    \end{xlist}
  	\end{exe}
  	
  	
  	For words with an initial <h>, the deletion seems especially common after a /ɾ/-final word, such after some frequent possessive pronouns (\ref{tab:h deletion sandhi yeah rhotic}). It is likewise frequent after the word [ʃɑd] `very' or `much'. 
  	
  	
  	\begin{exe}
    \ex Words which show optional /h/ deletion in connected speech, especially after /ɾ/ or dentals \label{tab:h deletion sandhi yeah rhotic} 
    \begin{xlist}
    	\ex \glll meɾ \textbf{h}ɑkʰust-ˈneɾ-ə (Careful)
    	\\
    	meɾ ɑkʰust-ˈneɾ-ə (Casual)
    	\\
    	our clothing-{\pl}-{\defgloss} 
    	\\
    	\trans `our clothes'
    	\\
    	\armenian{մեր հագուստներ } 
    	\ex \glll ɑsoɾ \textbf{h}ɑˈmɑɾ (Careful)
    	\\
    	ɑsoɾ ɑˈmɑɾ (Casual)
    	\\
    	this.{\gen} reason
    	\\
    	\trans `for this reason'
    	\\
    	\armenian{ասոր համար։} 
    	\ex \glll ʃɑd \textbf{h}ɑˈɾust e (Careful)
    	\\
    	ʃɑd ɑˈruʃt e (Casual)
    	\\
    	very rich is 
    	\\
    	\trans `He is very rich.'
    	\\
    	\armenian{Շատ հարուստ է։}
    \end{xlist}
  	\end{exe}
  	
  	
  	
  	
  	
  	HD feels though that there are some /h/-initial words which resist deletion (\ref{tab:h deletion sandhi nah}). These words seem to all be monosyllabic so there might be some prosodic constraint involved. 
  	
  	
  	\begin{exe}
    \ex Words which resist /h/ deletion in connected speech \label{tab:h deletion sandhi nah}
    \begin{xlist}
    	\ex \gll meɾ ˈ\textbf{h}ɑjɾ-ə (Careful, casual)
    	\\
    	our father-{\defgloss} 
    	\\
    	\trans `our father'
    	\\
    	\armenian{մեր հայրը} 
    	\ex \gll im \textbf{h}ɑˈz-əs (Careful, casual)
    	\\
    	my cough-{\possFsg} 
    	\\
    	\trans `my cough'
    	\\
    	\armenian{իմ հազս} 
    \end{xlist}
  	\end{exe}
  	
  	
  	
  	
  	
  	For verbs with an initial /h/, the deletion seems common after the future particle [bidi], reduced as [bid] (\ref{tab:h deletion sandhi yes inflected}). In the indicative form, these verbs use the prefix [ɡə-] with schwa epenthesis in citation form. In connected speech, the schwa and /h/ can delete together. 
  	
  	
  	\begin{exe}
    \ex Inflected verbs which show /h/ deletion in casual speech \label{tab:h deletion sandhi yes inflected}
    \begin{xlist}
    	\ex \glll bidi \textbf{h}ɑχˈt-e-ŋkʰ (Careful)
    	\\
    	bid ɑχˈt-e-ŋkʰ (Casual)
    	\\
    	will win-{\thgloss}-1{\pl}
    	\\
    	\trans `We will win.'
    	\\
    	\armenian{Պիտի յաղթենք։} 
    	\ex \glll ɡə-\textbf{h}ɑskəˈ-n-ɑ-m ɡoɾ (Careful)
    	\\
    	ɡ-ɑskəˈnɑm ɡoɾ (Casual)
    	\\
    	{\ind}-understand-{\inch}-{\thgloss}-1{\sg} {\prog}
    	\\
    	\trans `I am understanding.'
    	\\
    	\armenian{Կը հասկնամ կոր։} 
    \end{xlist}
  	\end{exe}
  	
  	For words with a final /h/, it is much rarer to find such words (\ref{tab:h deletion sandhi no maher}). In HD's judgments, most of these words don't show deletion in connected speech, whether intervocalically or sentence-finally. 
  	
  	\begin{exe}
    \ex Words that don't delete final /h/ in connected speech or sentence-finally \label{tab:h deletion sandhi no maher}
    \begin{xlist}
    	\ex \gll ˈmɑh. mɑˈh-eɾ. (Careful, casual)
    	\\
    	death death-{\pl} 
    	\\
    	\trans `Death. The deaths.'
    	\\
    	\armenian{Մահ։ Մահեր։}
    	\ex \gll vəsˈtɑh. vəsˈtɑh e-m. (Careful, casual)
    	\\
    	sure Sure is-1{\sg}
    	\\
    	\trans `For sure. I am sure.'
    	\\
    	\armenian{Վստահ։ Վստահ եմ։}
    \end{xlist}
  	\end{exe}
  	
  	For the `president', the final /h/ can delete and its deleted form can affect allomorphy (\ref{tab:h deletion sandhi naxakah}). The definite suffix is /-n/ after vowels, and /-ə/ after consonants. The deletion of the /h/ affects the choice of allomorph. The deletion and subsequent allomorphy is represented in the orthography. 
  	
  	
  	\begin{exe}
    \ex Words where /h/ deletion affects allomorphy \label{tab:h deletion sandhi naxakah}
    \begin{xlist}
    	\ex \glll nɑχɑˈkʰɑh. nɑχɑˈkʰɑh-ə. (Careful)
    	\\
    	nɑχɑˈkʰɑ. nɑχɑˈkʰɑ-n. (Casual)
    	\\
    	president president-{\defgloss} 
    	\\
    	\trans `President. The president'
    	\\
    	\armenian{Նախագահ։ Նախագահը։}
    	\\
    	\armenian{Նախագա։ Նախագան։} 
    \end{xlist}
  	\end{exe}
  	
  	
  	
  	
  	
  	
  	Because of how /h/ deletion applies in only some words and because of how it can feed other morphophonological rules, it is possible that /h/ deletion is actually a type of allomorphy in connected speech \textcolor{red}{find suitable kaisse citation, im thinking of an articel in the inkelas-zec book 1990}. 
  	
  	
  	
  	
  	\subsection{Schwa vowel assimilation}\label{section:segmentalPhono:sandhi:SchwaAssim}
  	
  	The schwa /ə/ is present in Armenian words. Many of its occurrences are epenthetic. As discussed in \textcolor{red}{cite chapter schwa epenth}, consonant clusters in the orthography are broken up by schwas in pronunciation. 
  	
  	In careful speech, a pronounced schwa is pronounced simply as [ə]. But in casual speech, there are some words where the schwa is assimilates to the vowel quality of the following vowel (Table \ref{tab:schwa assimilation words}).\footnote{The word `easy' is prescriptively pronounced as [tʰʏrin] in careful speech, but it's much more common to say [tʰəɾʏn] in careful speech. } 
  	
  	\begin{table}[H]
    \centering
    \caption{Words where the schwa assimilates to the following vowel}
    \label{tab:schwa assimilation words}
    \begin{tabular}{llll}
    	& Careful speech & Casual speech& \\
    	\armenian{գլուխ} & 
    	kʰəˈluχ & kʰuˈluχ & `head' \\
    	\armenian{դիւրին} 
    	& tʰəɾʏn & tʰʏɾʏn & `easy' 
    \end{tabular}
  	\end{table}
  	
  	One common occurrence of schwa assimilation is from the indicative prefix (Table \ref{tab:indc ge hi fuse}). This prefix is /ɡ-/ before vowels, and /ɡə-/ before consonants. The schwa is epenthetic (\textcolor{red}{cite chapter schwa epenthesis}). Before some /hi/-initial words, the prefix is optionally pronounced as [ɡi-]. The /h/ can optionally delete as well, causing the two vowels to then fuse into one vowel. 
  	
  	\begin{table}[H]
    \centering
    \caption{Words where indicative prefix and /hi/-initial words fuse}
    \label{tab:indc ge hi fuse}
    \begin{tabular}{lll}
    	Careful & ɡə-\textbf{hi}ˈʃ-e-n & ɡə-\textbf{h}ivɑntʰ-ɑ-n-ɑ-m 
    	\\
    	Casual & ɡi-\textbf{hi}ˈʃ-e-n & ɡi-\textbf{hi}vɑntʰ-ɑ-n-ɑ-m 
    	\\
    	Casual & ɡ-\textbf{i}ˈʃ-e-n & ɡ-\textbf{i}vɑntʰ-ɑ-n-ɑ-m 
    	\\
    	& {\ind}-remember-{\thgloss}-3{\pl} & {\ind}-sick-{\lvgloss}-{\inch}-{\thgloss}-1{\pl} 
    	\\
    	&`They remember.'&`I become sick.'
    	\\
    	& \armenian{Կը յիշեն։}& \armenian{Կը հիւանդանամ։}
    \end{tabular}
  	\end{table}
  	
  	This process of schwa vowel assimilation (vowel harmony) is limited to a handful of high-frequency words. We suspect that these alternations are just grammaticalized from some type of vowel-vowel coarticulation that occurs in casual speech. Data from oral corpora is needed in order to find out how much schwas can alternate in their vowel quality. For words other than the ones listed above, we suspect that the schwa can have its vowel quality be \textit{gradiently} affected by neighboring vowels, not categorically. 
  	
  	
  	
  	\subsection{Schwa elision}\label{section:segmentalPhono:sandhi:schwaElision}
  	\textcolor{red}{write after done with epenthesis}
  	
  	talk about pokr, dakr, i did a ref from the syllable chapter. Cr meɣr
  	\subsection{Degemination}\label{section:segmentalPhono:sandhi:degemination}
  	\textcolor{red}{write eventually}
