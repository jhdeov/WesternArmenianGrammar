 \chapter{Suprasegmental phonology of word stress}\label{chapter:stress}

Western Armenian is often described as having regular final stress (\S\ref{section:stress:regular}). Suffixation and compounding triggers stress shift.  Based on stress, we can demarcate the right boundary of prosodic words (\S\ref{section:stress:regular:domain}). Regular primary stress is thus final and  quantity-insensitive. 



The main phonological exception is when the word ends in one or two schwas (\S\ref{section:stress:regular:finalSchwa}). In that case, stress retracts to the rightmost non-schwa vowel. Complications arise for words that only have schwas (\S\ref{section:stress:regular:schwaword}). 

Single   clitics are ignored by stress (\S\ref{section:stress:cliticc}). But multiple clitics can trigger stress shift in certain syntactic-morphological contexts. 


The above concerns regular primary stress. There are irregularities in that certain morphemes exceptionally assign non-final stress. This includes  prestressing suffixes (\S\ref{section:stress:prestress}), negation (\S\ref{section:stress:verb}), and other morphemes. The irregular stress of these irregular morphemes can interact with the stress of suffixes, clitics, and each other. 

Besides those morphemes, there are some words which have irregular non-final stress (\S\ref{section:stress:irregularWord}). These are mostly high-frequency functional words. These few words are just exceptions.

Armenian is also often described as having initial secondary stress (\S\ref{section:stress:secondary}). However, sources also disagree on whether initial schwas can avoid secondary stress.  But in general,  initial secondary stress is not really perceptible to speakers thus any generalizations on initial secondary stress aren't truly reliable. In contrast, secondary stress caused by the morphology (such as prefixes) is perceptible. Regardless of the alleged existence of initial secondary stress, stress is thus not iterative. 

For reference, Table \ref{tab:overview stress} provides examples of different morpho-phonological contexts for stress.
For clarity, we use bolding to show stress syllables in this section. 

\begin{table}[H]
	\centering
	\caption{Overview of stress patterns in Western Armenian}
	\label{tab:overview stress}
	\resizebox{\textwidth}{!}{%
		\begin{tabular}{|lllll| }
			\hline      Type&&  Morphemes & Translation &  \\
			\hline 
			\multicolumn{5}{| l| }{\textbf{Regular final stress}}   \\
			Root& kʰɑˈ\textbf{ʁɑkʰ} & city & `city' & \armenian{քաղաք}
			\\
			Derivational suffix
			& kʰɑʁɑkʰ-ɑˈ\textbf{t͡si} &city-{\nmlz} & `citizen' & \armenian{քաղաքացի}
			\\
			Inflectional suffix & kʰɑʁɑkʰ-ɑt͡si-ˈ\textbf{neɾ} & city-{\nmlz}-{\pl} & `citizens' & \armenian{քաղաքացիներ}
			\\
			Clitic avoidance & kʰɑˈ\textbf{ʁɑkʰ}=\underline{ɑl}=\underline{e}  & city=also=is& `is also city' & \armenian{քաղաք ալ է}
			\\
			Compound  & mɑjɾ-ɑ-kʰɑˈ\textbf{ʁɑkʰ}  & mother-{\lvgloss}-city & `capital' & \armenian{մայրաքաղաք}
			\\
			\hline 
			\multicolumn{5}{| l| }{\textbf{Schwa avoidance in primary stress}}  
			\\
			Penult stress& kʰɑˈ\textbf{ʁɑ}kʰ-ə & city-{\defgloss} & `the city' & \armenian{քաղաքը}
			\\
			Antepenult stress& ˈ\textbf{kʰɑ}ɾən-mə & lamb-{\indf} & `a lamb'   & \armenian{գառն մը}
			\\
			Last-resort stress& fəsˈ\textbf{təχ}  & pistachio& `pistachio' & \armenian{ֆստըխ}
			\\
			Root-stress& fəsˈ\textbf{tə}χ-ə  & pistachio-{\defgloss}  & `the pistachio' & \armenian{ֆստըխը}
			\\
			& fəstəˈ\textbf{χ-ov}  & pistachio-{\ins}  & `with pistachio' & \armenian{ֆստըխով}
			\\
			\hline 
			\multicolumn{5}{| l| }{\textbf{Morphemes with irregular stress}}  
			\\
			Initial stress  & ˈ\textbf{t͡ʃ-ok}n-eɾ  &  {\neggloss}-help-{\cn} & `he doesn't help' & \armenian{չ՚օգներ}
			\\
			Prestressing &   ˈ\textbf{vet͡s}-eɾoɾtʰ  &  six-{\ord} & `sixth'  & \armenian{վեցերորդ}
			\\
			Loss of prestressing &   vet͡s-eˈ\textbf{ɾoɾ}tʰ-ə  &  six-{\ord}-{\defgloss} & `the sixth'  & \armenian{վեցերորդը}
			\\
			Irregular word &   ˈ\textbf{nujn}-kʰɑn  &  same-than  & `as much'   & \armenian{նոյնքան}
			\\
			\hline 
			\multicolumn{5}{| l| }{\textbf{Morphemes with irregular stress}}  
			\\
			Alleged initial stress & ˌɡɑɾeˈ\textbf{li} & possible & `possible' & \armenian{կարելի} 
			\\
			Prefix   stress & ˌɑŋ-ɡɑɾeˈ\textbf{li} & {\negan}-possible & `impossible' & \armenian{անկարելի} 
			\\ \hline
		\end{tabular}
	}
\end{table}


\section{Regular lexical stress}\label{section:stress:regular}


\subsection{Regular final stress on non-schwa vowels}\label{section:stress:regular:final}
Regular primary stress is placed on the syllable of the word if that syllable has a non-schwa vowel (Table \ref{tab:stress final non-schwa}). Final stress can be placed on any type of non-schwa vowel.  

\begin{table}[H]
	\centering
	\caption{Regular final stress is blind to type of non-schwa vowel}
	\label{tab:stress final non-schwa}
	\begin{tabular}{|l ll ll ll| }
		\hline 
		/ɑ/ 
		&ɑˈ\textbf{kɑh} & `stingy' & jeˈ\textbf{ɾɑz} & `dream' & iʃˈ\textbf{χɑn} & `prince' 
		\\
		& & \armenian{ագահ} & & \armenian{երազ} & & \armenian{իշխան}
		\\
		/e/ & pʰɑˈ\textbf{ɾev} & `hello' & t͡soˈ\textbf{ɾen} & `wheat' & uˈ\textbf{ʁeʁ} & `brain'
		\\
		& & \armenian{բարեւ} & & \armenian{ցորեն} & & \armenian{ուղեղ}
		\\
		/i/ & ɡɑˈ\textbf{t͡sin} & `axe' & tʰeˈ\textbf{ʁin} & `yellow' & voˈ\textbf{d͡ʒiɾ} & `crime'
		\\
		& & \armenian{կացին} & & \armenian{դեղին} & & \armenian{ոճիր}
		\\
		/o/ &bɑˈ\textbf{ɾon} & `baron' & χeˈ\textbf{lokʰ} & `obedient'  & χoˈ\textbf{ʃoɾ} & `huge'
		\\
		& & \armenian{պարոն} & & \armenian{խելօք} & & \armenian{խոշոր}
		\\
		/u/ & ɑˈ\textbf{ʃun} & `autumn' & heˈ\textbf{ʁuɡ} & `liquid' & ɑˈ\textbf{buɾ} & `soup' 
		\\
		& & \armenian{աշուն} & & \armenian{հեղուկ} & & \armenian{ապուր}
		\\
		/ʏ/ & ɑˈ\textbf{lʏɾ} & `flour' & məɾˈ\textbf{t͡ʃʏn} & `ant' &ɑˈ\textbf{ɾʏd͡z} & `lion'
		\\
		& & \armenian{ալիւր} & & \armenian{մրջիւն} & & \armenian{առիւծ}
		\\ \hline 
	\end{tabular}
\end{table}

Final stress ignores syllable structure. The final syllable can be a closed CVC as in Table \ref{tab:stress final non-schwa}, or open CV or closed CVCC as in Table \ref{tab:stress final syllable}. 

\begin{table}[H]
	\centering
	\caption{Regular final stress is blind to type of syllable structure}
	\label{tab:stress final syllable}
	\begin{tabular}{|l ll   ll| }
		\hline 
		/ɑ/ 
		&pʰeˈ\textbf{sɑ} & `groom' & ɡɑˈ\textbf{ʁɑntʰ} & `Christmas'
		\\
		& & \armenian{փեսայ} & & \armenian{կաղանդ}
		\\  
		/e/ & hɑsˈ\textbf{t͡se} & `address' & hɑˈ\textbf{mest} & `modest'
		\\
		& & \armenian{հասցէ} & & \armenian{համեստ}
		\\
		/i/ & həˈ\textbf{ʁi} & `pregnant'& nɑˈ\textbf{ɾint͡ʃ} & `orange'
		\\
		& & \armenian{յղի} & & \armenian{նարինջ}
		\\
		/o/ & kʰiˈ\textbf{lo} & `kilogram' & ɑˈ\textbf{ɾoχt͡ʃ} & `healthy'
		\\
		& & \armenian{քիլոյ} & & \armenian{առողջ}
		\\
		/u/ & ɑˈ\textbf{ɾu} & `male' & ɑˈ\textbf{ɡumpʰ} & `club'
		\\
		& & \armenian{արու} & & \armenian{ակումբ}
		\\
		/ʏ/ & N/A & & ɑɾˈ\textbf{tʰʏŋkʰ} & `result'
		\\
		& & & & \armenian{արդիւնք}
		\\ \hline
	\end{tabular}
\end{table}


Final stress applies regardless of word-size (Table \ref{tab:stress final 3 syll root}). The above data concerned bisyllabic roots. Although rare, there are some roots that are trisyllabic. We find final stress in larger roots. 


\begin{table}[H]
	\centering
	\caption{Regular final stress   in trisyllabic roots}
	\label{tab:stress final 3 syll root}
	\begin{tabular}{|l ll   | }
		\hline 
		ɑbɑˈ\textbf{kʰɑ} & `future' &   \armenian{ապագայ}  
		\\
		ɑkʰɑˈ\textbf{ɾɑɡ} & `farm' & \armenian{ագարակ}
		\\
		jezɑˈ\textbf{ɡi} & `singular'  & \armenian{եզակի} \\
		ɑɾɑˈ\textbf{kʰil} & `stork' & \armenian{արագիլ}
		\\
		ɑbɑˈ\textbf{hov} & `safe'  & \armenian{ապահով}  
		\\
		pʰɑpʰeˈ\textbf{lon} & `Babylon' & \armenian{Բաբելոն}
		\\ \hline 
	\end{tabular}
\end{table}


Suffixation creates longer words and we find stress shift (Table \ref{tab:stress shift suffixation}). It doesn't matter whether the  suffix is derivational or inflection, monosyllabic or bisyllabic, ending in a consonant or not. Compounding likewise creates larger words. We again see regular final stress in compounds. 

\begin{table}[H]
	\centering
	\caption{Regular final stress in suffixation and compounding}
	\label{tab:stress shift suffixation}
	\resizebox{\textwidth}{!}{%
		\begin{tabular}{|l|l|l|ll| }
			\hline 
			& &  -{\pl}&-{\pl}-{\abl} 
			\\ 
			Root `sugar':& ʃɑˈ\textbf{kʰɑɾ} & ʃɑkʰɑɾ-ˈ\textbf{neɾ}& ʃɑkʰɑɾ-neˈ\textbf{ɾ-e}
			\\
			& \armenian{շաքար}& \armenian{շաքարներ}& \armenian{շաքարներէ}
			\\
			Derivative `sweet'        & ʃɑkʰɑˈ\textbf{ɾ-od}  & ʃɑkʰɑɾ-od-ˈ\textbf{neɾ}  & ʃɑkʰɑɾ-od-neˈ\textbf{ɾ-e} 
			\\
			& \armenian{շաքարոտ} & \armenian{շաքարոտներ}& \armenian{շաքարոտներէ}   \\  
			Derivative `sweets'         & ʃɑkʰɑɾ-eˈ\textbf{ʁen} & ʃɑkʰɑɾ-eʁen-ˈ\textbf{neɾ}& ʃɑkʰɑɾ-eʁen-neˈ\textbf{ɾ-e}
			\\
			& \armenian{շաքարեղէն} & \armenian{շաքարեղէններ}& \armenian{շաքարեղէններէ}  
			\\
			\hline 
			Root `market' & vɑˈ\textbf{d͡ʒɑɾ}  & vɑd͡ʒɑɾ-ˈ\textbf{neɾ}  & vɑd͡ʒɑɾ-neˈ\textbf{ɾ-e} 
			\\
			& \armenian{վաճառ}& \armenian{վաճառներ}& \armenian{վաճառներէ}
			\\
			Derivative `merchant'    & vɑd͡ʒɑɾ-ɑˈ\textbf{ɡɑn}  & vɑd͡ʒɑɾ-ɑɡɑn-ˈ\textbf{neɾ}  & vɑd͡ʒɑɾ-ɑɡɑn-neˈ\textbf{ɾ-e} 
			\\
			& \armenian{վաճառական}& \armenian{վաճառականներ}& \armenian{վաճառականներէ}
			\\
			Derivative `commercial'    & vɑd͡ʒɑɾ-ɑɡɑn-uˈ\textbf{tʰjʏn}  & vɑd͡ʒɑɾ-ɑɡɑn-utʰjʏn-ˈ\textbf{neɾ}  & vɑd͡ʒɑɾ-ɑɡɑn-utʰjʏn-neˈ\textbf{ɾ-e} 
			\\
			& \armenian{վաճառականութիւն}& \armenian{վաճառականութիւններ}& \armenian{վաճառականութիւններէ}
			\\
			Compound `sugar-trader' & ʃɑkɑɾ-ɑ-vɑˈ\textbf{d͡ʒɑɾ}   & ʃɑkɑɾ-ɑ-vɑd͡ʒɑɾ-ˈ\textbf{neɾ}  & ʃɑkɑɾ-ɑ-vɑd͡ʒɑɾ-neˈ\textbf{ɾ-e} 
			\\
			& \armenian{շաքարավաճառ}& \armenian{շաքարավաճառներ}& \armenian{շաքարավաճառներէ}
			\\ \hline
		\end{tabular}
	}
\end{table}

In sum, primary stress is regularly final. Regular final stress is also the norm for   Eastern Armenian. However, there are some non-standard dialects which have penultimate stress as the norm. \citep[134-6,199]{Vaux-1998-ArmenianPhono}. For a study on the diachronic development of these penultimate-stress dialects from the final-stress dialects, see   \citet{DeLisi-2018-ArmenianProsodyDiachrony}.




\subsection{Regular non-final stress for final schwas}\label{section:stress:regular:finalSchwa}
The previous section focused on words where the final syllable had a non-schwa vowel. The main phonological exception to final stress comes from final schwas (Table \ref{tab:final syll schwa stress}). If the final syllable has a schwa, then stress is on the rightmost non-schwa vowel. Throughout this grammar, we use the label `full vowel' to denote non-schwa vowels.  


\begin{table}[H]
	\centering
	\caption{Non-final stress when the final syllable has a schwa}
	\label{tab:final syll schwa stress}
	\begin{tabular}{llll}
		kʰɑˈ\textbf{ʁɑkʰ}  &  city  & {`city'} & \armenian{քաղաք}
		\\
		kʰɑˈ\textbf{ʁɑ}kʰ-ə  &  city-{\defgloss}  &  `the city' & \armenian{քաղաքը}
		\\
		kʰɑˈ\textbf{ʁɑ}kʰ-əs  &  city-{\possFsg}  &  `my city' & \armenian{քաղաքս}
		\\
		kʰɑˈ\textbf{ʁɑ}kʰ-ətʰ  &  city-{\possSsg}  &  `your.{\sg} city' & \armenian{քաղաքդ}
		\\
		kʰɑˈ\textbf{ʁɑ}kʰ-mə  &  city-{\indf}  &  `a city' & \armenian{քաղաք մը}
		
	\end{tabular}
\end{table}


For non-final stress, the most common situation  when the final syllable has a schwa, and the penultimate syllable has a non-schwa. Stress is then on this non-schwa.   This type of configuration is often found when the word has one the four following schwa-headed suffixes: the definite \textit{-ə}, possessive suffixes \textit{-əs, -ətʰ}, and indefinite suffix \textit{-mə}.\footnote{The indefinite morpheme \textit{-mə} is sometimes called a clitic in the literature (\textcolor{red}{cite sigler}). We think this is because this morpheme is spelled with a space before it.  But to our knowledge, there's no non-orthographic evidence for calling it a clitic instead of a suffix.  }

When a word with final stress takes any of the above 4 suffixes, we always see penultimate stress (Table \ref{tab:schwa penult suffix }). Penultimate stress can land on any type of non-schwa vowel as long as that vowel precedes a schwa.  Penultimate stress can be on either a closed or open syllable. 


\begin{table}[H]
	\centering
	\caption{Penultimate stress   when a schwa-headed suffix is added }
	\label{tab:schwa penult suffix }
	\resizebox{\textwidth}{!}{%
		\begin{tabular}{|lll|l|l|l|  l|}
			\hline 
			&  &  &\multicolumn{1}{l|}{Definite -\armenian{ը}}  &\multicolumn{1}{l|}{1{\sg} poss.  -\armenian{ս}}   
			%  &\multicolumn{2}{l|}{2^{nd} person possessive}
			&\multicolumn{1}{l|}{Indefinite \armenian{մը}} &   \\
			/ɑ/ & ˈ\textbf{kʰɑm}& \armenian{գամ} 
			& ˈ\textbf{kʰɑ}.m-ə %& \armenian{գամը}
			& ˈ\textbf{kʰɑ}.m-əs %& \armenian{գամս} 
			%  & ˈ\textbf{kʰɑ}.m-ətʰ & \armenian{գամդ} 
			& ˈ\textbf{kʰɑm}.-mə %& \armenian{գամ մը}
			& `nail'
			\\
			&  ˈ\textbf{pʰɑnd}
			& \armenian{բանտ} 
			& ˈ\textbf{pʰɑn}.d-ə
			%    & \armenian{բանտը}
			& ˈ\textbf{pʰɑn}.d-əs
			%   & \armenian{բանտս} 
			% & ˈ\textbf{pʰɑn}.d-ətʰ
			% & \armenian{բանտդ}
			& ˈ\textbf{pʰɑnd}.-mə
			%& \armenian{բանտ մը}
			& `prison'
			
			\\
			& ɑˈ\textbf{ɾɑd͡z} & \armenian{առած}
			& ɑˈ\textbf{ɾɑ}.-d͡zə% & \armenian{առածը}
			& ɑˈ\textbf{ɾɑ}.-d͡zəs %& \armenian{առածս}
			% & ɑˈ\textbf{ɾɑ}.-d͡zətʰ & \armenian{առածդ}
			& ɑˈ\textbf{ɾɑd͡z}.-mə% & \armenian{առած մը}
			& `proverb'
			\\
			/e/ & ˈ\textbf{tʰev}& \armenian{թեւ } 
			& ˈ\textbf{tʰe}.v-ə %& \armenian{թեւը} 
			& ˈ\textbf{tʰe}.v-əs%&  \armenian{թեւս} 
			% & ˈ\textbf{tʰe}.v-ətʰ&  \armenian{թեւդ} 
			& ˈ\textbf{tʰev}.-mə% & \armenian{թեւ մը} 
			& `arm'
			\\
			& ˈ\textbf{tʰeɾtʰ}& \armenian{թերթ}
			& ˈ\textbf{tʰeɾ}.tʰ-ə%& \armenian{թերթը}
			& ˈ\textbf{tʰeɾ}.tʰ-əs%& \armenian{թերթս}
			% & ˈ\textbf{tʰeɾ}.tʰ-ətʰ& \armenian{թերթդ}
			& ˈ\textbf{tʰeɾtʰ}.-mə%& \armenian{թերթ մը}
			& `newspaper'
			\\
			& ɑɾˈ\textbf{vest}& \armenian{արուեստ}
			& ɑɾˈ\textbf{ves}.t-ə%& \armenian{արուեստը}
			& ɑɾˈ\textbf{ves}.t-əs%& \armenian{արուեստս}
			% & ɑɾˈ\textbf{ves}.t-ətʰ& \armenian{արուեստդ}
			& ɑɾˈ\textbf{vest}.-mə%& \armenian{արուեստ մը}
			& `handicraft'
			\\
			/i/
			&ˈ\textbf{lid͡ʒ} &  \armenian{լիճ}
			&ˈ\textbf{li}.d͡ʒ-ə %&  \armenian{լիճը}
			&ˈ\textbf{li}.d͡ʒ-əs %&  \armenian{լիճս}
			% &ˈ\textbf{li}.d͡ʒ-ətʰ &  \armenian{լիճդ}
			&ˈ\textbf{lid͡ʒ}.-mə %&  \armenian{լիճ մը}
			& `lake'
			\\
			&ˈ\textbf{miɾkʰ} &  \armenian{միրգ}
			&ˈ\textbf{miɾ}.kʰ-ə% &  \armenian{միրգը}
			&ˈ\textbf{miɾ}.kʰ-əs% &  \armenian{միրգս}
			% &ˈ\textbf{miɾ}.kʰ-ətʰ &  \armenian{միրգդ}
			&ˈ\textbf{miɾkʰ}.-mə% &  \armenian{միրգ մը}
			& `fruit'
			\\
			&   ɑɾɑˈ\textbf{kʰil} & \armenian{արագիլ}
			&   ɑɾɑˈ\textbf{kʰi}.l-ə%& \armenian{արագիլը}
			&   ɑɾɑˈ\textbf{kʰi}.l-əs %& \armenian{արագիլս}
			% &   ɑɾɑˈ\textbf{kʰi}.l-ətʰ & \armenian{արագիլդ}
			&   ɑɾɑˈ\textbf{kʰil}.-mə %& \armenian{արագիլ մը}
			& `stork'
			\\
			/o/ 
			& ˈ\textbf{ot͡s} & \armenian{օձ}
			& ˈ\textbf{o}.t͡s-ə% & \armenian{օձը}
			& ˈ\textbf{o}.t͡s-əs %& \armenian{օձս}
			% & ˈ\textbf{o}.t͡s-ətʰ & \armenian{օձդ}
			& ˈ\textbf{ot͡s}.-mə %& \armenian{օձ մը}
			& `snake'
			\\
			& ˈ\textbf{voɾpʰ}& \armenian{որբ}
			& ˈ\textbf{voɾ}.pʰ-ə %& \armenian{որբը}
			& ˈ\textbf{voɾ}.pʰ-əs %& \armenian{որբս}
			% & ˈ\textbf{voɾ}.pʰ-ətʰ & \armenian{որբդ}
			& ˈ\textbf{voɾpʰ}.-mə %& \armenian{որբ մը}
			& `orphan'
			\\
			& sɑʁˈ\textbf{mos} & \armenian{սաղմոս}
			& sɑʁˈ\textbf{mo}.s-ə% & \armenian{սաղմոսը}
			& sɑʁˈ\textbf{mo}.s-əs %& \armenian{սաղմոսս}
			% & sɑʁˈ\textbf{mo}.s-ətʰ & \armenian{սաղմոսդ}
			& sɑʁˈ\textbf{mos}.-mə %& \armenian{սաղմոս մը}
			& `psalm'
			\\
			/u/ 
			& ˈ\textbf{pʰul}& \armenian{փուլ}
			& ˈ\textbf{pʰu}.l-ə%& \armenian{փուլը}
			& ˈ\textbf{pʰu}.l-əs%& \armenian{փուլս}
			% & ˈ\textbf{pʰu}.l-ətʰ& \armenian{փուլդ}
			& ˈ\textbf{pʰul}.-mə %& \armenian{փուլ մը}
			& `phase'
			\\
			& ˈ\textbf{muɾd͡ʒ} & \armenian{մուրճ}
			& ˈ\textbf{muɾ}.d͡ʒ-ə% & \armenian{մուրճը}
			& ˈ\textbf{muɾ}.d͡ʒ-əs% & \armenian{մուրճս}
			% & ˈ\textbf{muɾ}.d͡ʒ-ətʰ & \armenian{մուրճդ}
			& ˈ\textbf{muɾd͡ʒ}.-mə% & \armenian{մուրճ մը}
			& `hammer'
			\\
			& χɑˈ\textbf{nutʰ} & \armenian{խանութ}
			& χɑˈ\textbf{nu}.tʰ-ə %& \armenian{խանութը}
			& χɑˈ\textbf{nu}.tʰ-əs% & \armenian{խանութս}
			% & χɑˈ\textbf{nu}.tʰ-ətʰ & \armenian{խանութդ}
			& χɑˈ\textbf{nutʰ}.-mə %& \armenian{խանութ մը}
			& `store'
			\\
			/ʏ/ 
			& ˈ\textbf{d͡ʒʏʁ} & \armenian{ճիւղ}
			& ˈ\textbf{d͡ʒʏ}.ʁ-ə %& \armenian{ճիւղը}
			& ˈ\textbf{d͡ʒʏ}.ʁ-əs% & \armenian{ճիւղս}
			% & ˈ\textbf{d͡ʒʏ}.ʁ-ətʰ & \armenian{ճիւղդ}
			& ˈ\textbf{d͡ʒʏʁ}.-mə %& \armenian{ճիւղ մը}
			& `branch'
			\\
			& ɑˈ\textbf{ɾʏd͡z} & \armenian{առիւծ}
			& ɑˈ\textbf{ɾʏ}.d͡z-ə% & \armenian{առիւծը}
			& ɑˈ\textbf{ɾʏ}.d͡z-əs% & \armenian{առիւծս}
			% & ɑˈ\textbf{ɾʏ}.d͡z-ətʰ & \armenian{առիւծդ}
			& ɑˈ\textbf{ɾʏd͡z}.-mə %& \armenian{առիւծ մը}
			& `lion'
			\\
			\hline 
		\end{tabular}
	}
	
\end{table}

There is another common construction where we find non-final stress (Table \ref{tab:penult stress cz schwa}). There are roots where the final syllable has a schwa, while the penult has a full vowel. For some of these words, the schwa is optional and variable across speakers. But if this schwa is present, stress is on the penult.\footnote{The word \textit{vɑɾtʰ-ɑ-χɑɾən} `mixed with roses' is a compound of \textit{vɑɾtʰ} `rose' and \textit{χɑɾən} `mixed'. The \textit{-ɑ-} is a linking vowel. }

\begin{table}[H]
	\centering
	\caption{Penultimate  stress when the final root syllable  has  a schwa, while penult has full vowel }
	\label{tab:penult stress cz schwa}
	\resizebox{\textwidth}{!}{%
		\begin{tabular}{|lllll|}
			\hline 
			/meʁɾ/ & ˈ\textbf{me}ʁəɾ  & `honey' & \armenian{մեղր} & <meɣr>     \\
			& ˈ\textbf{meʁɾ}  & &&  \\
			/kʰɑɾn/  & ˈ\textbf{kʰɑ}ɾən & `lamb'& \armenian{գառն} & <ka\.{r}n>    \\    
			& ˈ\textbf{kʰɑɾn} & & &  \\
			/himn/  & ˈ\textbf{hi}mən &     `basis'& \armenian{հիմն} & <himn> \\    
			/pʰɑɾt͡sɾ/  & ˈ\textbf{pʰɑɾ}t͡səɾ      & `high'& \armenian{բարձր} & <part͡sr> \\    
			
			/kʰɑnit͡ss/  & kʰɑˈ\textbf{ni}t͡səs       & `often'& \armenian{քանիցս} & <k'anit͡s's> \\    
			/χɑɾn/  & ˈ\textbf{χɑ}ɾən  & `mixed'& \armenian{խառն} & <xa\.{r}n> \\
			&   ˈ\textbf{χɑɾn}  & & & \\
			/vɑɾtʰ-ɑ-χɑɾn/ & vɑɾtʰ-ɑ-ˈ\textbf{χɑ}ɾən  & `mixed with roses'& \armenian{վարդախառն} & <vartaxa\.{r}n> \\
			&   vɑɾtʰ-ɑ-ˈ\textbf{χɑɾn} & & & \\
			\hline 
		\end{tabular}
	}
\end{table}

In these words, the schwa is arguably epenthetic (\textcolor{red}{cite chapter}). The main evidence for this is that 1) the schwa is absent in the orthography, and 2) the schwa is absent in derived forms in standard speech. The evidence for schwa epenthesis is discussed elsewhere in (\textcolor{red}{cite chapter}). 


For the above words with an epenthetic schwa, stress is on the penultimate syllable. That penult syllable has a full vowel. If a schwa-headed suffix is added like indefinite \textit{-mə},  we  find stress on the same syllable, but now it's further back in the word on the now antepenult syllable.   If we had a suffix with a non-schwa like genitive \textit{-i}, then we see the expected final stress (Table \ref{tab:antepenult stress cz schwa with indef}).  


\begin{table}[H]
	\centering
	\caption{Antepenultimate  stress when the penult root syllable and suffix syllable have schwas, while antepenult has full vowel }
	\label{tab:antepenult stress cz schwa with indef}
	\resizebox{\textwidth}{!}{%
		\begin{tabular}{|l|llll| }
			\hline 
			& + indefinite &  + definite&  + 1\textsuperscript{st} possessive & + genitive 
			\\
			\hline
			`honey' & ˈ\textbf{me}ʁəɾ-mə  &  ˈ\textbf{meʁ}ɾ-ə &  ˈ\textbf{meʁ}ɾ-əs &    ˈ\textbf{me}ʁəɾ-i
			\\
			&   &     ˈ\textbf{me}ʁəɾ-ə  &   ˈ\textbf{me}ʁəɾ-əs&     ˈ\textbf{me}ʁəɾ-i
			\\
			&   \armenian{մեղր մը} & \armenian{մեղր} & \armenian{մեղրս} & \armenian{մեղրի} 
			
			\\
			`lamb'  & ˈ\textbf{kʰɑ}ɾən-mə& ˈ\textbf{kʰɑɾ}n-ə     & ˈ\textbf{kʰɑɾ}n-əs     & kʰɑɾˈ\textbf{n-i}      \\    
			&   \armenian{գառն մը} &   \armenian{գառնը}&   \armenian{գառնս}&   \armenian{գառնի}
			\\
			`basis'  & ˈ\textbf{hi}mən-ə & ˈ\textbf{him}n-ə  & ˈ\textbf{him}n-əs  & himˈ\textbf{n-i}      \\    
			
			&    \armenian{հիմն մը}  &    \armenian{հիմնը}  &     \armenian{հիմնս} &    \armenian{հիմնի}
			\\
			`high' & ˈ\textbf{pʰɑɾ}t͡səɾ-mə   & ˈ\textbf{pʰɑɾt͡s}ɾ-ə    & ˈ\textbf{pʰɑɾt͡s}ɾ-əs   & pʰɑɾt͡sˈ\textbf{ɾ-i}      \\    
			&    &  ˈ\textbf{pʰɑɾ}t͡səɾ-ə     &   ˈ\textbf{pʰɑɾ}t͡səɾ-əs   &  pʰɑɾt͡səˈ\textbf{ɾ-i}      \\    
			&  \armenian{բարձր մը}   &  \armenian{բարձրը}       &  \armenian{բարձրս}       &  \armenian{բարձրի}   
			\\
			`mixed' &  ˈ\textbf{χɑ}ɾən-mə
			&  ˈ\textbf{χɑɾ}n-ə &  ˈ\textbf{χɑɾ}n-əs &  χɑɾˈ\textbf{n-i} 
			\\
			&  &   ˈ\textbf{χɑ}ɾən-ə &     ˈ\textbf{χɑ}ɾən-əs &    χɑɾəˈ\textbf{n-i}
			\\
			&  \armenian{խառն մը}     &  \armenian{խառնը}  &  \armenian{խառնս} &  \armenian{խառնի}
			\\
			`mixed with roses' & vɑɾtʰ-ɑ-ˈ\textbf{χɑ}ɾən-mə& vɑɾtʰ-ɑ-ˈ\textbf{χɑɾ}n-ə&vɑɾtʰ-ɑ-ˈ\textbf{χɑɾ}n-əs& vɑɾtʰ-ɑ-χɑɾˈ\textbf{n-i}   \\
			
			&  \armenian{վարդախառն մը}   &  \armenian{վարդախառնը } &  \armenian{վարդախառնս}  &  \armenian{վարդախառնի}
			\\
			\hline 
		\end{tabular}
	}
\end{table}


When a C-initial suffix like \textit{-mə} is added,  the internal schwa can't be deleted: *\textit{kʰɑɾn-mə} `a lamb'. If a V-initial suffix like   \textit{-ə} or   \textit{-i} is added, the internal schwa is deleted in standard speech. But in colloquial speech, the schwa can be retained in some words. This schwa deletion is discussed in (\textcolor{red}{cite chapter}).  We see the stress still on the rightmost full vowel: \textit{ˈmeʁ(ə)ɾ-ə} `the honey'. 


The generalization thus is that if a word has both schwa and non-schwa vowels, stress is on the rightmost non-schwa vowel. The most typical situation is when the penultimate syllable is a non-schwa while the final is a schwa. Another attested situation is when the antepenult has a non-schwa, while the penult and final have schwas.  To our knowledge, there aren't other logically possible cases such as the non-schwa being on the fourth-to-last syllable in the word. Clitics do present such cases though, discussed in \S\ref{section:stress:cliticc}. 

\subsection{Regular final stress for all-schwa words}\label{section:stress:regular:schwaword} 
The previous subsections concerned the assignment of regular stress in words that include at least one non-schwa vowel. It is relatively rare to find words where all the vowels are schwas. In this situation, there is reported variation on how stress works in these words. Some report final stress, while some report initial stress. HD's judgments though align more with final stress. 


Schwa-only words can be categorized into two types: nativized loanwords and onomatopoeia. 

For the loanword group (Table \ref{tab:ottoman turkish stress}), there are some words that were borrowed from Ottoman Turkish or Lebanese Arabic. For the Turkish-based loanwords, many    of these source Turkish words contain the Turkish vowel /ɯ/ spelled <ı>. The vowel is rendered as a schwa in Armenian. Many of these words likewise end in a velar stop in Turkish, but a uvular fricative in Armenian.


\begin{table}[H]
	\centering
	\caption{Stress in schwa-only words that are from Ottoman Turkish}
	\label{tab:ottoman turkish stress}
	\begin{tabular}{|ll|lll|l|}
		\hline      &&    + indefinite & + definite & + instrumental & Turkish   
		\\
		\hline     `pistachio'   & fəsˈ\textbf{təχ}  & fəsˈ\textbf{təχ}-mə & fəsˈ\textbf{tə}χ-ə & fəstəˈ\textbf{χ-ov}  & `fıstık' \\
		& \armenian{ֆստըխ} & \armenian{ֆստըխ մը}& \armenian{ֆստըխը} & \armenian{ֆստըխի}&
		\\
		`mustache'   & bəjˈ\textbf{jəχ}  & bəjˈ\textbf{jəχ}-mə & bəjˈ\textbf{jə}χ-ə & bəjjəˈ\textbf{χ-ov}  & `bıyık' \\
		& \armenian{պըյըխ} & \armenian{պըյըխ մը}& \armenian{պըյըխը} & \armenian{պըյըխի} & 
		\\
		`ticklish'   & ʁəˈ\textbf{dəχ}  & ʁəˈ\textbf{dəχ}-mə & ʁəˈ\textbf{də}χ-ə & ʁədəˈ\textbf{χ-ov}  & `gıdık' \\
		`hernia'   & fəˈ\textbf{tʰəχ}  & fəˈ\textbf{tʰəχ}-mə & fəˈ\textbf{tʰə}χ-ə & fətʰəˈ\textbf{χ-ov} & `fıtık'   \\
		\hline       
	\end{tabular}
\end{table}



For these words, the root has only schwas and gets final stress: \textit{fəsˈtəχ} `pistachio'. When a non-schwa suffix is added, we see stress shift: \textit{fəstəˈχ-ov} (ins.). But if a schwa suffix is added, we don't see stress shift: \textit{fəsˈtəχ-ə} (def.). 

These words are largely banned from written standard Armenian, but are common in colloquial speech.  We could find the Armenian spelling for some but not all of these words. These words must have entered the language rather early, at least before the 1915 genocide.  More such loanwords are reported in \citet{Adjarian-1902-TUrkishWordsArmenian} study on Turkish borrowings in early modern Istanbul Armenian. 

The other group of schwa-only words are onomatopeic words. For these words, Adjarian reports final stress in what we could be   his native ideolect of Western Armenian. In contrast, Vaux reports initial stress in what is likely the ideolect of his Eastern Armenian informants. 

\textcolor{red}{cite adjarian, vaux data,   \citep[133]{Vaux-1998-ArmenianPhono}   \citep[339]{Adjarian-1971-LiakatarPhono}}

The above stress judgments are taken from Vaux and Adjarian. HD doesn't know any of these words. Thus they're all nonce words for HD. If forced, the most natural pronunciation for HD is to apply final stress. But because these are unknown words for HD, we can't be sure how these words are pronounced by people who do know them. 

We're not sure why Vaux and Adjarian provide conflicting judgments on stress. It is possible that the differences reflect speaker variation, whether by time or region. It is also possible that all perhaps schwa stress is acoustically very weak, thus these differences are due to difficulty in perceiving the exact location of stress. 

\subsection{Morphophonological domain of stress}\label{section:stress:regular:domain}
This section discusses how to define the domain of stress in terms of connecting between what types of morphology are involved in forming final stress.  


The previous sections looked at regular primary stress in words that have diverse morphological structures. In terms of morphological structure, regular primary stress does not distinguish between roots, suffixed roots, and compounds (Table \ref{tab:overview regular stress suffixed compounds}). These structures all get regular  stress on the rightmost non-schwa. For schwa-only words though, stress stays in the root, ignoring schwa-headed suffixes. 

\begin{table}[H]
	\centering
	\caption{Overview of regular primary stress in suffixed roots and compounds}
	\label{tab:overview regular stress suffixed compounds}
	\begin{tabular}{|l| ll   l|}
		\hline
		Root& kʰɑˈ\textbf{ʁɑkʰ} & `city' & \armenian{քաղաք}  
		\\
		& kʰɑʁɑˈ\textbf{kʰ-ov} & `city-{\ins}' & \armenian{քաղաքով}  \\
		&  kʰɑˈ\textbf{ʁɑ}kʰ-ə & `city-{\defgloss}'       & \armenian{քաղաքը}  
		\\
		
		Root& ˈ\textbf{kʰɑ}ɾən & `lamb' & \armenian{գառն} 
		\\
		& kʰɑɾ(ə)ˈ\textbf{n-i} & `lamb-{\gen}' & \armenian{գառնի}  
		\\
		& ˈ\textbf{kʰɑ}ɾən-mə & `lamb-{\indf}' & \armenian{գառն մը}  
		\\
		Root & fəsˈ\textbf{təχ} & `pistachio' &  \armenian{ֆստըխ} 
		\\
		& fəstəχ-ˈ\textbf{neɾ} & `pistachio-{\pl}'&  \armenian{ֆստըխներ}
		\\
		& fəsˈ\textbf{tə}χ-əs & `pistachio-{\possFsg}'  &  \armenian{ֆստըխս}
		\\  \hline
		Compound  & mɑjɾ-ɑ-kʰɑˈ\textbf{ʁɑkʰ}  & `capital'& \armenian{մայրաքաղաք} \\
		& mother-{\lvgloss}-city & & 
		\\
		& mɑjɾ-ɑ-kʰɑʁɑˈ\textbf{kʰ-e}  & `capital-{\abl}'& \armenian{մայրաքաղաքէ}
		\\
		& mɑjɾ-ɑ-kʰɑˈ\textbf{ʁɑ}kʰ-ətʰ  & `capital-{\possSsg}'& \armenian{մայրաքաղաքդ}
		\\
		\hline  
	\end{tabular}
\end{table}


The basic generalization is that the entire morphological word (root, suffixes, compounds) is involved in creating the domain for final stress (the prosodic word). A prosodic word or phonological word is defined as the string of elements (morphemes) that is syllabified together, and the rightmost non-schwa in this string gets regular final stress. This grammar is primarily descriptive so we generally don't provide theoretical trees for words and sentences. But in Representation \ref{rep:stress structure},  we provide a simple morphological and prosodic tree for the inflected compound `capital-{\possSsg}' or   `your capital' from Table \ref{tab:overview regular stress suffixed compounds}. 

\begin{representation}
	Mapping a morphological word (MWord) to a prosodic word (PWord) with final stress for [mɑjɾ-ɑ-kʰɑˈ\textbf{ʁɑ}kʰ-ətʰ] `your capital'. 
	\label{rep:stress structure}
	
	\begin{tabular}{lll}
		\begin{tikzpicture}[scale = 1]
			\Tree  [.MWord [.MWord  mɑjɾ   -ɑ-     kʰɑʁɑkʰ   ]  [.{\possSsg} -ətʰ ] ]
		\end{tikzpicture}
		& 
		$\rightarrow $ & 	
		\begin{tikzpicture}[scale =1]
			\Tree    [.PWord  [.$\sigma$ mɑj ] [.$\sigma$ ɾɑ ] [.$\sigma$ kʰɑ ] [.$\sigma$ ˈ\textbf{ʁɑ} ] [.$\sigma$ kʰətʰ ]
			] 
		\end{tikzpicture}
	\end{tabular}
\end{representation}

\textcolor{red}{make a compound pword}

\section{Stress when clitics are added}  \label{section:stress:cliticc}
Armenian has many derivational and inflectional suffixes. These are included into the domain of stress (the prosodic word) and can get final stress. There are also words that clitics. These clitics are encliticized into the preceding word and don't get stress.

\subsection{Words with one clitic}\label{section:stress:cliticc:one}
Cross-linguistically, a `clitic' is a fuzzy concept \citep{anderson-2005-aspectsTheoryClitic}. A clitic is a morpheme or word which acts `contentful' enough for the morphology and syntax, but they are `weak' for the phonology. For example the English verb `\textit{is}' is a morphological word because it has its own meanings and acts as a verb. In careful speech for a sentence like `\textit{It is here}', the word `\textit{is}' is phonologically heavy enough that it can carry its own stress and it's not syllabified with neighboring words. This means that the word `is' acts as a phonological word. But in casual speech, the verb is often reduced to just \textit{'s} as in `\textit{It's here}'. This reduction means that that the word has been changed from a phonological word to just a clitic that is syllabified with the preceding word. 

For Armenian, there are many particles and function words. A list of such particles is in (\textcolor{red}{cite chapter}). We discuss their clitic status more in \S\ref{section:intonation:clitic}.  Some of them are phonologically clitics (underlined; Table \ref{tab:phono clitic list}) . These are the copula, the word for `also', the conjunction `and' \textit{=ɑl}, the colloquial question particle \textit{=mə}, the progressive marker \textit{=ɡoɾ}, the subjunctive marker \textit{=ne}. The progressive and subjunctive markers can attach to only verbs.\footnote{Note that the progressive marker is prescriptively banned from written formal registers, and registered to just informal spoken speech. This is discussed in (\textcolor{red}{cite chapter}).  } 

\begin{table}[H]
	\centering
	\caption{List of phonological clitics}
	\label{tab:phono clitic list}
	\begin{tabular}{|l| lll|   }
		\hline 
		&Stress & Syllabified &   \\
		Copula    &uˈ\textbf{ɾɑ}χ =\underline{e-n} & u.ˈ\textbf{ɾɑ}.χ\underline{en} &   
		\\
		& happy \underline{is-3{\pl}} & `They are happy.' & \armenian{Ուրախ են։} 
		\\
		`also' \textit{=ɑl}   &bɑˈ\textbf{ni}ɾ =\underline{ɑl} & bɑ.ˈ\textbf{ni}.ɾ\underline{ɑl} &  
		\\
		& cheese =\underline{also} & `Also cheese.' & \armenian{Պանիր ալ։}  
		\\
		`and' \textit{=u}   &ˈ\textbf{hɑ}t͡s =\underline{u} bɑniɾ &  ˈ\textbf{hɑ}.t͡s\underline{u} bɑniɾ &  
		\\
		& bread =\underline{and}  bread & `Bread and cheese.' & \armenian{Հաց ու պանիր։}  
		\\
		Q. \textit{=mə}   &  uˈ\textbf{d-e-m} =\underline{mə} & u.ˈ\textbf{dem}.\underline{mə} &   
		\\
		&    eat-{\thgloss}-1{\sg} \underline{{\q}} & `Do I eat it?' & \armenian{Ուտե՞մ մը։} 
		\\
		Prog. \textit{=ɡoɾ}   &ɡ-uˈ\textbf{d-e-m} =\underline{ɡoɾ} & ɡu.ˈ\textbf{dem}.\underline{ɡoɾ}   & 
		\\
		& {\ind}-eat-{\thgloss}-1{\sg} \underline{{\prog}} & `I am eating.' & \armenian{Կ՚ուտեմ կոր }
		\\
		Subj. \textit{=ne}   &  uˈ\textbf{d-e-m} =\underline{ne} &  u.ˈ\textbf{dem}.\underline{ne}&   
		\\
		&    eat-{\thgloss}-1{\sg} \underline{{\sbjv}} & `If I eat it.' & \armenian{Ուտեմ նէ։ }
		\\\hline 
		
	\end{tabular}
\end{table}

The semantics and uses of  these particles is discussed more in depth elsewhere in (\textcolor{red}{cite chapter}). These section focuses just on their stress properties. Of the above particles, the copula is the only one that can take on different forms for different inflectional features, like present 2PL \textit{=e-kʰ} or past 2SG \textit{=e-ji-ɾ}. The paradigm is discussed in (\textcolor{red}{cite chapter}). 


These clitics are nearly all monosyllabic. The only exception is the copula. This copula is monosyllabic in the present, but it has bisyllabic forms in the past. Note that the past suffix \textit{-i} creates some type of prominence on the preceding  clitic syllable \textit{=e}, but we're not sure if this is just an intonational illusion because of how this form is bisyllabic  (\ref{ex:stress:clitic}). 

\begin{exe}
	\ex \gll uˈ\textbf{ɾɑ}χ =\underline{e-ji-n}
	\\
	happy =\underline{is-{\pst}-3{\pl}}
	\\
	\trans `They were happy' \label{ex:stress:clitic}
	\\
	\armenian{Ուրախ էին։}
\end{exe}

Of the clitics in Table \ref{tab:phono clitic list}, almost all of them are obligatorily syllabified with the preceding word. The exception is the word `and' (\ref{ex:stress:and}) which can syllabify either with or without the preceding word. The lack of syllabification   correlates with having some type of pause before this word. Furthermore, the syllabified version often gives the sense that the two coordinated items are one entity (such as a dish), while the lack of syllabification gives a sense that the items are more separate. But this is not a strict rule because the syllabified form can give the meaning of separate senses. 

\begin{exe}
	\ex \glll V1: ˈ\textbf{hɑ}t͡s =\underline{u} bɑniɾ ɡ-uz-e-m [ˈ\textbf{hɑ}.t͡s\underline{u}]
	\\
	V2: ˈ\textbf{hɑt͡s} u bɑniɾ ɡ-uz-e-m [ˈ\textbf{hɑt͡s} u]
	\\
	~ bread  \underline{and} cheese {\ind}-want-{\thgloss}-1{\sg}
	\\
	\trans V1: `I want a (meal of) bread and cheese.'
	\label{ex:stress:and}
	\\
	V1/V2: `I want bread, and (I want) cheese.'
	\\
\end{exe}

When these clitics are added to a word with final stress, these clitics don't cause any stress shift.\footnote{This generalization is for Western Armenian as spoken by HD and other members of the Lebanese community. In our fieldwork, we've found though that in Eastern Armenian, the `also' clitic [el] can take stress. Clitic behavior in Eastern Armenian is an open question. } This was seen in Table \ref{tab:phono clitic list}. In those words, stress is on the rightmost full vowel of the word. If the word is a schwa-only word  (\ref{ex:stress:clitic more}), stress is on the rightmost schwa of the root. The clitic is ignored. Note that   are no verbs with final schwas,   so we can't attach the progressive or subjunctive marker to them. 

\begin{exe}
	\ex \label{ex:stress:clitic more}
	\begin{xlist}
		
		\ex \gll hɑm-ə fəsˈ\textbf{tə}χ  =\underline{e} [fəs.ˈ\textbf{tə}.χ\underline{e}]
		\\
		taste-{\defgloss} pistachio =\underline{is}
		\\
		\trans `The taste (of this food) is pistachio.' 
		\\
		\armenian{Համը ֆստըխ է։}
		\ex \gll mɑɾtʰ-ə fəˈ\textbf{tʰə}χ  =\underline{ɑl} un-i [fə.ˈ\textbf{tʰə}.χ\underline{ɑl}] 
		\\
		man-{\defgloss} hernia =\underline{also} have-{\thgloss} 
		\\
		\trans `The man also has a hernia.' 
		\\
		\armenian{Մարդը ? ալ ունի։}
		
		\ex \gll mɑɾtʰ-ə bəjˈ\textbf{jə}χ  =\underline{u} moɾukʰ un-i [bəj.ˈ\textbf{jə}.χ\underline{u}] 
		\\
		man-{\defgloss} mustache =\underline{and} beard have-{\thgloss} 
		\\
		\trans `The man   has a mustache and beard.' 
		\\
		\armenian{Մարդը պըյըխ ու մօրուք ունի։}
	\end{xlist}
	
\end{exe}


If a word has penultimate stress because of schwas (\ref{ex:stress:clitic penult}), whether epenthetic root schwas or suffix schwas, then we still don't find any stress shift. Stress stays on the rightmost non-schwa (ignoring the clitic). For schwa-only words, stress stays on the root schwa. Note that epenthetic schwas tend to delete before these V-initial clitics in standard speech, but they can surface in colloquial speech.   

\begin{exe}
	\ex \label{ex:stress:clitic penult} 
	\begin{xlist}
		\ex \glll ɑsiɡɑ ˈ\textbf{me}ʁəɾ =\underline{e} [ˈ\textbf{me}.ʁə.ɾ\underline{e}]   \\
		ɑsiɡɑ ˈ\textbf{meʁ}ɾ =\underline{e} [ˈ\textbf{meʁ}.ɾ\underline{e}]  \\
		this honey =\underline{is} 
		\\
		\trans `This is honey' 
		\\
		\armenian{Ասիկա մեղր է։}
		
		\ex \gll ɑsiɡɑ fəsˈ\textbf{tə}χ-əs =\underline{e} [fəsˈ\textbf{tə}.χʰə.s\underline{e}]  \\
		this pistachio-{\possFsg} =\underline{is} ~ 
		\\
		\trans `This is my pistachio.' 
		\\
		\armenian{Ասիկա ֆստըխս է։}
		
	\end{xlist}
	
\end{exe}

The same judgments are found for other clitics (\ref{ex:stress:clitic penult other}), regardless if the word has an epenthetic schwa or is a schwa-only word. 

\begin{exe}
	\ex \label{ex:stress:clitic penult other}
	\begin{xlist}
		\ex 
		\begin{xlist}
			\ex \glll   ˈ\textbf{me}ʁəɾ =\underline{ɑl} ɡ-uz-e-m [ˈ\textbf{me}.ʁə.ɾ\underline{ɑl}]   \\
			ˈ\textbf{meʁ}ɾ =\underline{ɑl} ɡ-uz-e-m [ˈ\textbf{meʁ}.ɾ\underline{ɑl}]  \\
			honey =\underline{also} {\ind}-want-{\thgloss}-1{\sg} 
			\\
			\trans `I also want honey.'
			\\
			\armenian{Մեղր ալ կ՚ուզեմ։}
			\ex \glll   ˈ\textbf{me}ʁəɾ =\underline{u} ʃɑkʰɑɾ ɡ-uz-e-m [ˈ\textbf{me}.ʁə.ɾ\underline{u}]   \\
			ˈ\textbf{meʁ}ɾ =\underline{u}  ʃɑkʰɑɾ ɡ-uz-e-m [ˈ\textbf{meʁ}.ɾ\underline{u}]  \\
			honey =\underline{and} sugar {\ind}-want-{\thgloss}-1{\sg} 
			\\
			\trans `I   want honey and sugar.'
			\\
			\armenian{Մեղր ու շաքար կ՚ուզեմ։}
			
		\end{xlist}
		\ex 
		\begin{xlist}
			
			\ex \gll   fəsˈ\textbf{tə}χ-əs =\underline{ɑl} ɡ-uz-e-m [fəs.ˈ\textbf{tə}.χʰə.s\underline{ɑl}]  \\
			pistachio-{\possFsg}  =\underline{also} {\ind}-want-{\thgloss}-1{\sg} 
			\\
			\trans `I also want my pistachio.'
			\\
			\armenian{Ֆստըխս  ալ կ՚ուզեմ։}
			
			\ex \gll   fəˈ\textbf{tʰə}χ-əs =\underline{u} tʰɑkʰutʰjʏn-əs  [fə.ˈ\textbf{ʰtə}.χʰə.s\underline{u}]  \\
			hernia-{\possFsg}  =\underline{and} fever-{\possFsg} 
			\\
			\trans `My hernia and fever.'
			\\
			\armenian{Ֆստըխս ու տաքութիւնս։} 
			\ex \gll   fəsˈ\textbf{tə}χ-əs =\underline{mə} ɡ-uz-e-s  [fəs.ˈ\textbf{tə}.χʰəs.\underline{mə}]  \\
			pistachio-{\possFsg}  =\underline{{\q}} {\ind}-want-{\thgloss}-1{\sg} 
			\\
			\trans `Do you want my pistachio (as opposed to something else)?' 
			\\
			\armenian{Ֆստը՞խս մը կ՚ուզես։}\end{xlist}
	\end{xlist}
	
	
\end{exe}

If the word has antepenultimate stress (\ref{ex:stress:clitic antepenult}), then again cliticization does nothing. We end up seeing stress on the fourth-to-last syllable of the word+clitic sequence.  This is the case for words with epenthetic schwas as in below. 

\begin{exe}
	\ex \label{ex:stress:clitic antepenult}
	
	
	\begin{xlist}
		\ex \gll   ɑsiɡɑ ˈ\textbf{me}ʁəɾ-əs =\underline{e} [ˈ\textbf{me}.ʁə.ɾə.s\underline{e}]   \\
		this honey-{\possFsg} =\underline{is}  
		\\
		\trans `This is my honey.'
		\\
		\armenian{Ասիկա մեղրս է։}
		\ex \gll   ˈ\textbf{me}ʁəɾ-əs =\underline{ɑl} ɡ-uz-e-s [ˈ\textbf{me}.ʁə.ɾə.s\underline{ɑl}]   \\
		honey-{\possFsg} =\underline{also}  {\ind}-want-{\thgloss}-2{\sg} 
		\\
		\trans `You also want my honey.' 
		\\
		\armenian{Մեղրս ալ կ՚ուզես։}
		\ex \gll   ˈ\textbf{me}ʁəɾ-əs =\underline{u} ʃɑkʰɑɾ-əs [ˈ\textbf{me}.ʁə.ɾə.s\underline{u}]   \\
		honey-{\possFsg} =\underline{and}  sugar-{\possFsg} 
		\\
		\trans `My honey and (my) sugar.' 
		\\
		\armenian{Մեղրս ու շաքարս։}
		
		\ex \gll   ˈ\textbf{me}ʁəɾ-əs =\underline{mə} ɡ-uz-e-s [ˈ\textbf{me}.ʁə.ɾəs.\underline{mə}]   \\
		honey-{\possFsg} =\underline{{\q}} {\ind}-want-{\thgloss}-2{\sg} 
		\\
		\trans `Do you want my honey (as opposed to something else)?' 
		\\
		\armenian{Մե՞ղրս մը կ՚ուզես։}
		
		
	\end{xlist}
	
\end{exe}


Note that the stress locations are based on HD's perception of prominence. However, in our spectogram recordings, it seems that the pitch rises often continue onto from the stressed syllable all the way to the clitic. It's unknown how the acoustics of stress are affected by cliticization.  

The generalization so far is that if a word has regular primary stress on some syllable, adding a single clitic doesn't cause any changes in the location of stress. Complications arise when either  the word has irregular stress. We discuss these complications later in \S\ref{section:stress:prestress} and \S\ref{section:stress:verb:pastImpf:clitic}. We next turn to clitic clusters.  

\subsection{Words with multiple clitics or clitic clusters}\label{section:stress:cliticc:cluster}
The generalization that clitics are unstressed also applies in clitic clusters. The clitics can come in different combinations. if the verb has regular primary stress on some syllable, clitic clusters generally don't trigger stress shift or take stress. Exceptions arise in clitic clusters with the subjunctive \textit{-ne}.     


As a caveat, the data here is based on just our impressions of stress as native speakers. However, based on inspecting our spectrograms, we  suspect that these clitic clusters cause changes in the pitch contours of the stressed and post-stressed syllables. This pitch changes don't affect the perception of stress, but they do seem to erase the acoustic properties of the syllable that   we think has  stress. 

We first go over clusters that don't trigger stress (\S\ref{section:stress:cliticc:cluster:noShift}), then clusters that do trigger stress (\S\ref{section:stress:cliticc:cluster:Shift}). Three-clitic clusters are rather rare but possible (\S\ref{section:stress:cliticc:cluster:Three}), and their pattern like two-clitic clusters. We summarize in  \S\ref{section:stress:cliticc:cluster:summary}. 

\subsubsection{Clitic clusters that don't trigger stress shift}\label{section:stress:cliticc:cluster:noShift}
For a word with regular stress, it is possible to add the following types of two-clitic clusters without any stress shift:
\begin{enumerate}[noitemsep,topsep = 0pt]
	\item copula + `also' 
	\item copula + Q
	\item `also' + copula
	\item `also' + Q
	\item progressive + `also' 
	\item progressive + Q
	\item subjunctive + `also'
\end{enumerate}

Some orders are more typical than others. Some orders are also more pragmatically special than others. But regardless, stress does not shift for these clusters, regardless if the base word as has final, penultimate, or antepenultimate stress. Glide epenthesis applies between the clitics to repair    vowel hiatus. 


For the copula + `also' sequence, the second word \textit{=ɑl} does not give the meaning of `again' (\ref{example:stress:cliticCluster:CopAlso}). It instead creates a sense of exasperation like `I have already done X.'  We have found it difficult to naturally elicit this cluster except after verbs. Verbs don't have final schwas so we can't see how this cluster behaves with preceding schwas. 

\begin{exe}
	\ex No stress shift in copula + `also' clitic clusters\label{example:stress:cliticCluster:CopAlso}
	\begin{xlist}
		\ex \gll im hed-əs χoˈ\textbf{s-ɑ}d͡z =\underline{e} =\underline{ɑl} [χo.ˈ\textbf{sɑ}.d͡z\underline{e.jɑl}]
		\\
		my.{\gen} with-{\possFsg} speak-{\rptcp} =\underline{is} =\underline{also}
		\\
		\trans `Has spoken to me already.' 
		\\
		\armenian{Իմ հետս խօսած է ալ։} 
	\end{xlist}
\end{exe}

For the copula + Q cluster, again we find no stress shift (\ref{example:stress:cliticCluster:CopQ}). But because the Q particle is semantically loaded, we do find special intonation contours, discussed in Section \S\ref{section:intonation:focus:polar:particle}.  

\begin{exe}
	\ex No stress shift in copula + Q  clitic clusters \label{example:stress:cliticCluster:CopQ}
	\begin{xlist}
		\ex \gll uˈ\textbf{ɾɑ}χ =\underline{e-s} =\underline{mə} [u.ˈ\textbf{ɾɑ}.χ\underline{es.mə}]
		\\
		happy  =\underline{is-2{\sg}}  =\underline{{\q}}
		\\
		\trans `Are you happy?'
		\\
		\armenian{Ուրա՞խ ես մը։}
		\ex \gll ɑsiɡɑ fəsˈ\textbf{tə}χ =\underline{e} =\underline{mə} [fəs.ˈ\textbf{tə}.χ\underline{e.mə}]
		\\
		this pistachio  =\underline{is}  =\underline{{\q}}
		\\
		\trans `Is this pistachio?' 
		\\
		\armenian{Ասիկա ֆստը՞խ է մը։}
		\ex \gll ɑsiɡɑ seˈ\textbf{ʁɑ}n-əs =\underline{e} =\underline{mə} [se.ˈ\textbf{ʁɑ}.nə.s\underline{e.mə}]
		\\
		this table-{\possFsg} =\underline{is}  =\underline{{\q}}
		\\
		\trans `Is this my table?' 
		\\
		\armenian{Ասիկա սեղա՞նս է մը։}
		\ex \gll ɑsiɡɑ fəsˈ\textbf{tə}χ-əs =\underline{e} =\underline{mə} [fəs.ˈ\textbf{tə}.χə.s\underline{e.mə}]
		\\
		this pistachio-{\possFsg}  =\underline{is}  =\underline{{\q}}
		\\
		\trans `Is this my pistachio?' 
		\\
		\armenian{Ասիկա ֆստը՞խս է մը։}
		
		
		\ex \gll ɑsiɡɑ ˈ\textbf{me}ʁəɾ =\underline{e} =\underline{mə} [ˈ\textbf{me}.ʁə.ɾ\underline{e.mə}]
		\\
		this honey  =\underline{is} =\underline{{\q}}
		\\
		\trans `Is this honey?'
		\\
		\armenian{Ասիկա մե՞ղր է մը։}
		
		\ex \gll ɑsiɡɑ ˈ\textbf{vɑ}kʰəɾ-mən =\underline{e} =\underline{mə} [ˈ\textbf{vɑ}.kʰəɾ.mə.n\underline{e.mə}]
		\\
		this tiger-{\indf}  =\underline{is} =\underline{{\q}}
		\\
		\trans `Is this a tiger?' 
		\\
		\armenian{Ասիկա վա՞գր մըն է մը։}
	\end{xlist}
\end{exe}

The cluster of `also' + copula can also be formed (\ref{example:stress:cliticCluster:AlsoCop}). The meaning of this cluster is more straightforward compared to the meaning of   copula + `also'. Stress is stable as expected. 


\begin{exe}
	\ex No stress shift in `also' + copula   clitic clusters\label{example:stress:cliticCluster:AlsoCop}
	\begin{xlist}
		\ex \gll uˈ\textbf{ɾɑ}χ =\underline{ɑl} =\underline{e}   [u.ˈ\textbf{ɾɑ}.χ\underline{ɑ.lə}]
		\\
		happy =\underline{also} =\underline{is}
		\\
		\trans `He is also happy.'
		\\
		\armenian{Ուրախ ալ է։}
		\ex \gll ɑsiɡɑ fəsˈ\textbf{tə}χ =\underline{ɑl} =\underline{e} [fəs.ˈ\textbf{tə}.χ\underline{ɑ.le}]
		\\
		this pistachio  =\underline{also}  =\underline{is}
		\\
		\trans `This is also pistachio.'
		\\
		\armenian{Ասիկա ֆստըխ ալ է։}
		\ex \gll ɑsiɡɑ im seˈ\textbf{ʁɑ}n-əs =\underline{ɑl} =\underline{e} [se.ˈ\textbf{ʁɑ}.nə.s\underline{ɑ.le}]
		\\
		this my.{\gen} table-{\possFsg} =\underline{also}  =\underline{is}
		\\
		\trans `This is also my table.'
		\\
		\armenian{Ասիկա իմ սեղանս ալ   է։}
		\ex \gll ɑsiɡɑ im fəsˈ\textbf{tə}χ-əs =\underline{ɑl} =\underline{e} [fəs.ˈ\textbf{tə}.χə.s\underline{ɑ.le}]
		\\
		this my.{\gen} pistachio-{\possFsg}  =\underline{also}  =\underline{is}
		\\
		\trans `This is also my pistachio.'
		\\
		\armenian{Ասիկա իմ ֆստըխս ալ է։}
		\ex \gll ɑsiɡɑ ˈ\textbf{me}ʁəɾ =\underline{ɑl} =\underline{e} [ˈ\textbf{me}.ʁə.ɾ\underline{ɑ.le}]
		\\
		this honey  =\underline{also} =\underline{is}
		\\
		\trans `This is also honey.'
		\\
		\armenian{Ասիկա մեղր ալ  է։}
		
		\ex \gll ɑsiɡɑ ˈ\textbf{vɑ}kʰəɾ-mən =\underline{ɑl} =\underline{e} [ˈ\textbf{vɑ}.kʰəɾ.mə.n\underline{ɑ.le}]
		\\
		this tiger-{\indf}  =\underline{also} =\underline{is}
		\\
		\trans `This is also a tiger.'
		\\
		\armenian{Ասիկա վագր մըն ալ է։}
	\end{xlist}
\end{exe}

The cluster of `also' + Q can also be formed (\ref{example:stress:cliticCluster:AlsoQ}).  The question particle creates an interrogative   where the preceding word is questioned.  Stress is stable as expected. 


\begin{exe}
	\ex No stress shift in `also' + Q   clitic clusters\label{example:stress:cliticCluster:AlsoQ}
	\begin{xlist}
		\ex \gll bɑˈ\textbf{ni}ɾ =\underline{ɑl} =\underline{mə}  ɡ-uz-e-s [bɑˈ\textbf{ni}.ɾ\underline{ɑl.mə}]
		\\
		cheese =\underline{also} =\underline{{\q}} {\ind}-want-{\thgloss}-2{\sg}
		\\
		\trans `Do you  want also cheese?'
		\\
		\armenian{Պանի՞ր ալ մը կ՚ուզես։}
		\ex \gll  fəsˈ\textbf{tə}χ =\underline{ɑl} =\underline{mə} ɡ-uz-e-s  [fəs.ˈ\textbf{tə}.χ\underline{ɑl.mə}]
		\\
		pistachio  =\underline{also}  =\underline{{\q}} {\ind}-want-{\thgloss}-2{\sg}
		\\
		\trans `Do you  want also pistachio?'
		\\
		\armenian{Ֆստը՞խ ալ մը կ՚ուզես։}
		\ex \gll   im seˈ\textbf{ʁɑ}n-əs =\underline{ɑl} =\underline{mə} ɡ-uz-e-s  [se.ˈ\textbf{ʁɑ}.nə.s\underline{ɑl.mə}]
		\\
		my.{\gen} table-{\possFsg} =\underline{also}  =\underline{{\q}} {\ind}-want-{\thgloss}-2{\sg}
		\\
		\trans `Do you  want also my table?'
		\\
		\armenian{Իմ սեղա՞նս ալ մը կ՚ուզես։}
		\ex \gll   im fəsˈ\textbf{tə}χ-əs =\underline{ɑl} =\underline{mə} ɡ-uz-e-s  [fəs.ˈ\textbf{tə}.χə.s\underline{ɑl.mə}]
		\\
		my.{\gen} pistachio-{\possFsg}  =\underline{also} =\underline{{\q}} {\ind}-want-{\thgloss}-2{\sg}
		\\
		\trans `Do you  want also my pistachio?'
		\\
		\armenian{Իմ ֆստը՞խս ալ մը կ՚ուզես։}
		\ex \gll   ˈ\textbf{me}ʁəɾ =\underline{ɑl}  =\underline{mə} ɡ-uz-e-s  [ˈ\textbf{me}.ʁə.ɾ\underline{ɑl.mə}]
		\\
		honey  =\underline{also}  =\underline{{\q}} {\ind}-want-{\thgloss}-2{\sg}
		\\
		\trans `Do you  want also   honey?'
		\\
		\armenian{Մե՞ղր ալ մը կ՚ուզես։}
		
		\ex \gll   ˈ\textbf{vɑ}kʰəɾ-mən =\underline{ɑl} =\underline{mə} ɡ-uz-e-s   [ˈ\textbf{vɑ}.kʰəɾ.mə.n\underline{ɑl.mə}]
		\\
		tiger-{\indf}  =\underline{also} =\underline{{\q}} {\ind}-want-{\thgloss}-2{\sg}
		\\
		\trans `Do you  want also   a tiger?'
		\\
		\armenian{Վա՞գր մըն ալ մը կ՚ուզես։}
	\end{xlist}
\end{exe}


We also consider three other clusters:  progressive + `also' (\ref{example:stress:cliticCluster:ProgAlso}),  progressive + Q (\ref{example:stress:cliticCluster:ProgQ}), subjunctive + `also' (\ref{example:stress:cliticCluster:SbjvAlso}). These clusters are attached only to verbs. Verbs don't have any schwa suffixes, so we cannot see how this cluster would behave after schwas.  




\begin{exe}
	\ex No stress shift in progressive + `also',  progressive + Q, and subjunctive + `also' clitic clusters 
	\begin{xlist}
		\ex \gll ɡ-uˈ\textbf{d-e-m}  =\underline{ɡoɾ}  =\underline{ɑl}  [ɡu.ˈ\textbf{dem}.\underline{ɡo.ɾɑl}]
		\\
		{\ind}-eat-{\thgloss}-1{\sg} =\underline{{\prog}} =\underline{also}
		\\
		\trans `I am eating already! (Stop telling me to eat)' \label{example:stress:cliticCluster:ProgAlso}
		\\
		\armenian{Կ՚ուտեմ կոր ալ։}
		\ex \gll ɡ-uˈ\textbf{d-e-s}  =\underline{ɡoɾ}  =\underline{mə}  [ɡu.ˈ\textbf{des}.\underline{ɡoɾ.mə}]
		\\
		{\ind}-eat-{\thgloss}-2{\sg} =\underline{{\prog}} =\underline{{\q}}
		\\
		\trans `Are you eating?' \label{example:stress:cliticCluster:ProgQ}
		\\
		\armenian{Կ՚ուտե՞ս կոր մը։}
		\ex \gll jev jetʰe uˈ\textbf{d-e-s}  =\underline{ne}  =\underline{ɑl}  [u.ˈ\textbf{des}.\underline{ne.jɑl}]
		\\
		and if eat-{\thgloss}-2{\sg} =\underline{{\sbjv}} =\underline{{\q}}
		\\
		\trans `And if you eat.' \label{example:stress:cliticCluster:SbjvAlso}
		\\
		\armenian{Եւ եթէ ուտես նէ ալ։}
		
	\end{xlist}
\end{exe}

The meaning of the `also' clitic can vary between being just a simple `I am doing X again', to an exasperated `I am doing X already!'. Context determines which meaning is more dominant. We use the `already!' meaning because that meaning is more automatic without context. 


Thus, the above clitic clusters do not cause any changes in stress when they are attached to a word with regular primary stress. The next section discusses clusters which do cause shifts. 

\subsubsection{Clitic clusters that   trigger stress shift}\label{section:stress:cliticc:cluster:Shift}
Among the possible clusters, the subjunctive clitic \textit{=ne} is special because it can trigger stress shift when part of a cluster. This clitic can only attach to verbs, so we cannot see the effect of this clitic on schwas. 

In the  copula + the subjunctive cluster (\ref{example:stress:cliticCluster:CopSubj}),  the copula acts as an auxiliary verb when attached to a verbal participle. The subjunctive can either trigger stress shift or not. Stress shift is more typical. The lack of stress shift gives a connotation that the verb is focused, or that the subjunctive clitic was an afterthought. 

\begin{exe}
	\ex Stress shift in copula  + subjunctive clitic clusters \label{example:stress:cliticCluster:CopSubj}
	\begin{xlist}
		\ex \gll   deˈ\textbf{s-ɑd͡z} =\underline{e-s}  [de.ˈ\textbf{sɑ}.d͡z\underline{es}]
		\\
		see-{\rptcp} =\underline{is-2sg} 
		\\
		\trans `You have seen.'
		\\
		\armenian{Տեսած ես։}
		\ex \glll  jetʰe deˈ\textbf{s-ɑd͡z} =\underline{e-s} =\underline{ne}  [de.ˈ\textbf{sɑ}.d͡z\underline{es.ne}]
		\\
		jetʰe des-ɑˈ\textbf{d͡z} \textbf{=\underline{e-s}} =\underline{ne}  [de.sɑ.ˈ\textbf{d͡z\underline{es}}.\underline{ne}]
		\\
		if see-{\rptcp} =\underline{is-2sg} =\underline{{\sbjv}}
		\\
		\trans `If you have seen it.'
		\\
		\armenian{Եթէ տեսած ես նէ։}
	\end{xlist}
\end{exe}

In the progressive + subjunctive cluster (\ref{example:stress:cliticCluster:ProgSubj}), stress shift is   possible. Stress shift is typical and feels more typical than not shifting stress \citep[84]{Khanjian-2013-DissNegativeConcord}. 

\begin{exe}
	\ex Stress shift in progressive  + subjunctive clitic clusters \label{example:stress:cliticCluster:ProgSubj}
	\begin{xlist}
		\ex \gll   ɡ-uˈ\textbf{d-e-s} =\underline{ɡoɾ}  [ɡu.ˈ\textbf{des}.\underline{ɡoɾ}]
		\\
		eat-{\thgloss}-2{\sg} =\underline{{\prog}}
		\\
		\trans `You are eating.'
		\\
		\armenian{Կ՚ուտես կոր։}
		\ex \glll  jetʰe  ɡ-uˈ\textbf{d-e-s} =\underline{ɡoɾ}  =\underline{ne}  [ɡu.ˈ\textbf{des}.\underline{ɡoɾ.ne}]
		\\
		jetʰe ɡ-ud-e-s   =ˈ\underline{\textbf{ɡoɾ}} =\underline{ne}  [ɡu.des.ˈ\textbf{\underline{ɡoɾ}}.\underline{ne}]
		\\
		if eat-{\thgloss}-2{\sg} =\underline{{\prog}} =\underline{{\sbjv}}
		\\
		\trans `If you are eating.'\label{example:stress:cliticCluster:gor-ne}
		\\
		\armenian{Եթէ կ՚ուտես կոր    նէ։}
	\end{xlist}
\end{exe}

The clitic \textit{=ne} is special in that it can regular induce stress shift. We suspect this is because this clitic is deeply involved with special intonational contours for subjunctive clauses, discussed in Section \S\ref{section:intonation:other:subjunctive}. 

We also suspect that the original stressed vowel does have some level of prominence in these cliticized forms. So for example in \textit{ɡ-ud-e-s=ɡoɾ=ne} `If you are eating' (\ref{example:stress:cliticCluster:gor-ne}), the most prominent syllable is \textit{ɡoɾ}. But there is some perceived prominence on the originally stressed syllable \textit{des}. It is possible that our perception indicates secondary stress: [ɡu.ˌdes.ˈɡoɾ.ne]. Acoustic data is needed to verify this because there is relatively little phonetic data on secondary stress. 

\subsubsection{Clitic clusters with three members}\label{section:stress:cliticc:cluster:Three}
It is possible to create clitic clusters with three members. These clusters either have 1) the question particle as the last member, or 2) have the subjunctive clitic. These clusters are the following:

\begin{itemize}[noitemsep, topsep= 0pt]
	\item  copula + `also' + Q 
	\item `also' + copula + Q 
	\item  progressive + `also' + Q 
	\item  copula + subjunctive + `also' 
	\item  progressive + subjunctive + `also' 
	
\end{itemize}

These clusters do create a type of prominence on the penultimate clitic. We're not sure if this prominence should be classified as stress, or as special intonation contours. 


For the copula + `also' + Q cluster (\ref{example:stress:cliticCluster:cop also q}), stress stays on the original syllable. However, the Q particle does induce some type of prominence on the preceding clitic `also'. We think this prominence is due to the intonational contours caused by mixing the meanings of `already' from the first clitic, and the meaning of questioning. 

\begin{exe}
	\ex \gll   χoˈ\textbf{s-ɑ}d͡z =\underline{e} =\underline{ɑl} =\underline{mə} [χo.ˈ\textbf{sɑ}.d͡z\underline{e.jɑl.mə}]
	\\
	speak-{\rptcp} =\underline{is} =\underline{also} =\underline{{\q}}
	\\
	\trans `Has he spoken already?'
	\label{example:stress:cliticCluster:cop also q}
	\\
	\armenian{Խօսա՞ծ է ալ մը։} 
\end{exe}

The cluster `also' + copula + Q (\ref{example:stress:cliticCluster:also cop q}) can be attached to words with final stress, penultimate stress, and antepenultimate stress. We perceive stress on the original syllable. But there is a very strong prominence and lengthening on the copula. Again, it is unknown if this is true stress or just intonation. 


\begin{exe}
	\ex Formation of `also' + copula  + Q  clitic clusters \label{example:stress:cliticCluster:also cop q}
	\begin{xlist}
		\ex \gll uˈ\textbf{ɾɑ}χ =\underline{ɑl} =\underline{e}  =\underline{mə} [u.ˈ\textbf{ɾɑ}.χ\underline{ɑ.lə.mə}]
		\\
		happy =\underline{also} =\underline{is} =\underline{{\q}}
		\\
		\trans `Is he also happy?'
		\\
		\armenian{Ուրա՞խ ալ է մը։}
		\ex \gll ɑsiɡɑ fəsˈ\textbf{tə}χ =\underline{ɑl} =\underline{e} =\underline{mə} [fəs.ˈ\textbf{tə}.χ\underline{ɑ.le.mə}]
		\\
		this pistachio  =\underline{also}  =\underline{is} =\underline{{\q}}
		\\
		\trans `Is this  also pistachio?'
		\\
		\armenian{Ասիկա ֆստը՞խ ալ է մը։}
		\ex \gll ɑsiɡɑ im seˈ\textbf{ʁɑ}n-əs =\underline{ɑl}   =\underline{e} =\underline{mə} [se.ˈ\textbf{ʁɑ}.nə.s\underline{ɑ.le.mə}]
		\\
		this my.{\gen} table-{\possFsg} =\underline{also}  =\underline{is} =\underline{{\q}} 
		\\
		\trans `Is this also my table?'
		\\
		\armenian{Ասիկա իմ սեղա՞նս ալ    է մը։}
		\ex \gll ɑsiɡɑ im fəsˈ\textbf{tə}χ-əs =\underline{ɑl} =\underline{e} =\underline{mə} [fəs.ˈ\textbf{tə}.χə.s\underline{ɑ.le.mə}]
		\\
		this my.{\gen} pistachio-{\possFsg}  =\underline{also}  =\underline{is} =\underline{{\q}}
		\\
		\trans `Is this   also my pistachio?'
		\\
		\armenian{Ասիկա իմ ֆստը՞խս ալ է մը։}
		\ex \gll ɑsiɡɑ ˈ\textbf{me}ʁəɾ =\underline{ɑl} =\underline{e}  =\underline{mə} [ˈ\textbf{me}.ʁə.ɾ\underline{ɑ.le.mə}]
		\\
		this honey  =\underline{also} =\underline{is} =\underline{mə}
		\\
		\trans `Is this also honey?'
		\\
		\armenian{Ասիկա մեղր ալ  է մը։}
		
		\ex \gll ɑsiɡɑ ˈ\textbf{vɑ}kʰəɾ-mən =\underline{ɑl} =\underline{e}  =\underline{mə} [ˈ\textbf{vɑ}.kʰəɾ.mə.n\underline{ɑ.le.mə}]
		\\
		this tiger-{\indf}  =\underline{also} =\underline{is} =\underline{{\q}}
		\\
		\trans `Is this also a tiger?'
		\\
		\armenian{Ասիկա վա՞գր մըն ալ է։}
	\end{xlist}
\end{exe}

Similar ambiguities arise for the other clusters: copula + subjunctive + `also', and progressive + subjunctive + `also'. For the first (\ref{example:stress:cliticCluster:cop subj also}), the substring of copula + subjunctive   triggers variable stress shift to the copula. Adding the `also' particle removes this variability, and we find stress on the copula.  

\begin{exe}
	\ex \gll    jetʰe des-ɑˈ\textbf{d͡z} \textbf{=\underline{e-s}} =\underline{ne} =\underline{ɑl}  [de.sɑ.ˈ\textbf{d͡z\underline{es}}.\underline{ne.jɑl}]
	\\
	if see-{\rptcp} =\underline{is-2sg} =\underline{{\sbjv}} =\underline{also}
	\\
	\trans `If you have already seen it.'
	\label{example:stress:cliticCluster:cop subj also}
	\\
	\armenian{Եթէ տեսած ես նէ ալ։}
\end{exe}

Similarly for progressive + subjunctive cluster (\ref{example:stress:cliticCluster:prog subj also}), the subjunctive triggers stress shift to the progressive. Adding the `also' particle doesn't change anything. 

\begin{exe} 
	\ex \gll   
	jetʰe ɡ-ud-e-s   =ˈ\underline{\textbf{ɡoɾ}} =\underline{ne} =\underline{ɑl}  [ɡu.des.ˈ\textbf{\underline{ɡoɾ}}.\underline{ne.jɑl}]
	\\
	if eat-{\thgloss}-2{\sg} =\underline{{\prog}} =\underline{{\sbjv}} =\underline{also}
	\\
	\trans `If you are eating already.'
	\label{example:stress:cliticCluster:prog subj also}
	\\
	\armenian{Եթէ կ՚ուտես կոր    նէ ալ։}
\end{exe}

\subsubsection{Summary of attested clitic clusters}\label{section:stress:cliticc:cluster:summary}

We have surveyed different types of clitic clusters. This is summarized in Table \ref{tab:two clitic cluster summary}.  We examined how these clusters affect the stress of a word that has regular primary stress.  For two-clitic clusters, most of them don't trigger stress shift. Some clusters do trigger stress shift, and these clusters involve the subjunctive. We use asterisk * to mark these clusters that trigger stress shift. 

\begin{table}[H]
	\centering
	\caption{Possible two-clitic clusters and their effects on stress for words with regular stress}
	\label{tab:two clitic cluster summary}
	\begin{tabular}{|l|lll lll|}
		\hline   1\textsuperscript{st} ,  2\textsuperscript{nd} &  copula & `also'  & `and'  & Q  & prog.  & sbjv.   \\
		&  \textit{=e} &  \textit{=ɑl} &   \textit{=u} &   \textit{=mə} &  \textit{=ɡoɾ}&  \textit{=ne}  \\
		\hline 
		copula \textit{=e} & NR & (\ref{example:stress:cliticCluster:CopAlso}) & & (\ref{example:stress:cliticCluster:CopQ}) & & (\ref{example:stress:cliticCluster:CopSubj})*
		\\
		`also' \textit{=ɑl} &(\ref{example:stress:cliticCluster:AlsoCop}) &NR  & & (\ref{example:stress:cliticCluster:AlsoQ}) & & 
		\\
		`and' \textit{-u} & & & NR& & & 
		\\
		Q \textit{=mə} & & &  & NR & & 
		\\
		prog.  \textit{=ɡoɾ}& & (\ref{example:stress:cliticCluster:ProgAlso}) & & (\ref{example:stress:cliticCluster:ProgQ}) &NR & (\ref{example:stress:cliticCluster:ProgSubj})*
		\\
		sbjv.  \textit{=ne}   &  &  (\ref{example:stress:cliticCluster:SbjvAlso}) & &   & & NR
		\\
		\hline
	\end{tabular}
\end{table}


Among the unattested clitic clusters, we cannot have clusters where the same clitic appears twice (NR = no repetition). Some clitics can't combine because of morphosemantics or paradigm reasons. For example, verbs take the progressive marker, but this marker can't be used alongside a copula auxiliary. Some clusters are unattested because of pragmatics: it is difficult to make sense of a clause that is both subjunctive and a question. 

One consistent gap is clusters with the clitic `and' [u] (\ref{example:stress:cliticCluster:gap u}). This morpheme can be used after clitics. But in these situations, the morpheme does not syllabify with the preceding word, but is instead syllabified alone. In this case, the word acts more as a separate phonological word instead of a clitic. 

\begin{exe}
	\ex\gll mɑɾtʰ-ə uˈ\textbf{ɾɑ}χ =\underline{e} u dəχuɾ =e  [u.ˈ\textbf{ɾɑ}.χe u]
	\\
	man-{\defgloss} happy =\underline{is} and sad =is
	\\
	\trans `The man is happy and sad.' 
	\label{example:stress:cliticCluster:gap u}
	\\
	\armenian{Մարդը ուրախ է ու տխուր է։}
	
\end{exe}

From the attested  two-clitic clusters, we can add additional clitics: either Q or `also' (\textcolor{red}{cite chapter section}). Adding the `also' particle doesn't cause any significant stress changes. Adding the question particle does induce special prominence on the second clitic. But we don't know if this prominence should be classified as lexical stress or as just intonational prominence. 



\section{Prestressing derivational suffixes}\label{section:stress:prestress}
Most derivational suffixes are phonologically regular in that they take regular primary stress when they are word-final. However there is a small set of derivational suffixes that irregularly avoid stress when they're word-final (Table \ref{tab:overview irregular stress suffixes}). These suffixes are the ordinal suffixes -eɾoɾtʰ, -ɾoɾtʰ/,  the adverbalizer suffixes /-oɾen, -ɑpʰɑɾ, -ɑbes/, and the hypocoristic suffix /-o/. 

\begin{table}[H]
	\centering
	\caption{Prestressing derivational suffixes which avoid stress when word-final}
	\label{tab:overview irregular stress suffixes}
	\resizebox{\textwidth}{!}{%
		
		\begin{tabular}{|ll|lll|ll|}
			\hline 
			& Uninflected & + Genitive & + Definite & + Clitic `also' or `is' &  &
			\\
			\hline
			Ordinal   & ˈ\textbf{hiŋ}kʰ-eɾoɾtʰ & hiŋkʰ-eɾoɾˈ\textbf{tʰ-i} & hiŋkʰ-eˈ\textbf{ɾoɾ}tʰ-ə & hiŋkʰ-eˈ\textbf{ɾoɾ}tʰ=e & `fifth' & \armenian{հինգերորդ}
			\\
			& ˈ\textbf{jeɡ}-ɾoɾtʰ & jeɡ-ɾoɾˈ\textbf{tʰ-i} & jeɡ-ˈ\textbf{ɾoɾ}tʰ-ə & jeɡ-ˈ\textbf{ɾoɾ}tʰ=e & `second' & \armenian{երկրորդ}
			\\
			\hline
			Adverbalizer & uˈ\textbf{ɾɑχ}-oɾen && & uɾɑχ-oˈ\textbf{ɾen}=ɑl &`happily' & \armenian{ուրախօրէն} 
			\\
			& uˈ\textbf{ɾɑχ}-ɑpʰɑɾ && & uɾɑχ-ɑˈ\textbf{pʰɑɾ}=ɑl &`happily' & \armenian{ուրախաբար} 
			\\
			& pʰɑɾˈ\textbf{t͡səɾ}-ɑbes && & pɑɾt͡səɾ-ɑˈ\textbf{bes}=ɑl &`highly' & \armenian{բարձրապէս} 
			\\ 
			\hline 
			Hypocoristic & ˈ\textbf{mɑ}ɾ-o & mɑɾ-o-ˈ\textbf{ji} & mɑˈ\textbf{ɾ-o-n} &  mɑˈ\textbf{ɾ-o}=je & `Maro' & \armenian{Մարօ}
			\\
			\hline 
		\end{tabular}
	}
\end{table}

These suffixes place stress on the preceding vowel. Most of the time, the stressed preceding vowel is a non-schwa like \textit{uˈ\textbf{ɾɑχ}-oɾen}, but this vowel can be a schwa \textit{pʰɑɾˈt͡səɾ-ɑbes}. 

These irregular suffixes lose their irregularity when another suffix or clitic is added. Thus if we added a full vowel suffix like \textit{-i}, then we get final stress. Similarly, if we added a schwa suffix \textit{-ə}, a non-vocalic suffix \textit{-n}, or a clitic \textit{=ɑl}, then we see stress shift to the irregular suffix. 

The 3 types of derivational suffixes show identical behavior in irregular stress. The following sections list examples of their use. Note that there is significant variation in the irregularity of these suffixes as reported in previous teaching grammars. For the Lebanese community however, their irregularity seems more consistent. 


\subsection{Ordinal suffix}\label{section:stress:prestress:ordinal}
A number can be either a cardinal number like `five' or an ordinal number like `fifth'. In Armenian, ordinals are formed by adding the suffix \textit{-eɾoɾtʰ} or \textit{-ɾoɾtʰ}. The relevant morphology is explained in \textcolor{red}{cite chapter    on ordinal morpho}. This section focuses on the stress patterns of ordinals. 

In isolation form, ordinals have   stress before the suffix, not on the suffix. Table \ref{tab:ordinal stress basic} lists common ordinal numbers. Usually the pre-suffix syllable is also the first syllable of the word. But higher numbers have more syllables. The default ordinal suffix is \textit{-eɾoɾtʰ}, but numbers 2-4 take the  \textit{-ɾoɾtʰ} allomorph. 

\begin{table}[H]
	\centering
	\caption{Stress before the ordinal suffix for ordinals in isolation}
	\label{tab:ordinal stress basic}
	\begin{tabular}{|lll|lll|}
		\hline 
		2nd     & ˈ\textbf{jeɾɡ}-ɾoɾtʰ & \armenian{երկրորդ} 
		& 11th & dɑsnəˈ\textbf{me}ɡ-eɾoɾtʰ & \armenian{տասնըմէկերորդ}    
		\\
		2nd     & ˈ\textbf{jeɡ}-ɾoɾtʰ & \armenian{երկրորդ} 
		& 20th & kʰəˈ\textbf{sɑ}n-eɾoɾtʰ & \armenian{քսաներորդ}    
		\\
		3rd     & ˈ\textbf{jeɾ}-ɾoɾtʰ & \armenian{երրորդ} 
		& 30th & jeɾeˈ\textbf{su}n-eɾoɾtʰ & \armenian{երեսուներորդ}    
		\\
		3rd     & ˈ\textbf{je}-ɾoɾtʰ & \armenian{երրորդ}  
		& 40th & kʰɑɾɑˈ\textbf{su}n-eɾoɾtʰ & \armenian{քառասուներորդ} 
		\\
		4th      & ˈ\textbf{t͡ʃoɾ}-ɾoɾtʰ & \armenian{չորրորդ}   
		& 50th & hiˈ\textbf{su}n-eɾoɾtʰ & \armenian{յիսուներորդ} 
		\\
		4th      & ˈ\textbf{t͡ʃo}-ɾoɾtʰ & \armenian{չորրորդ} 
		& 60th & vɑtˈ\textbf{su}n-eɾoɾtʰ & \armenian{վաթսուներորդ} 
		\\
		
		5th      & ˈ\textbf{hiŋ}ɡ-eɾoɾtʰ & \armenian{հինգերորդ}  
		& 70th & jotʰɑnɑˈ\textbf{su}n-eɾoɾtʰ & \armenian{եօթանասուներորդ} 
		\\
		6th      & ˈ\textbf{ve}t͡s-eɾoɾtʰ & \armenian{վեցերորդ}  
		& 80th & ut\textbf{su}n-eɾoɾtʰ & \armenian{ութսուներորդ} 
		\\
		7th      & ˈ\textbf{jo}tʰ-eɾoɾtʰ & \armenian{եօթերորդ} 
		& 90th & ini\textbf{su}n-eɾoɾtʰ & \armenian{ինիսուներորդ} 
		\\
		8th      & ˈ\textbf{u}tʰ-eɾoɾtʰ & \armenian{ութերորդ}
		& 100th & hɑ\textbf{ɾʏ}ɾ-eɾoɾtʰ & \armenian{հարիւրերորդ} 
		\\
		9th      & ˈ\textbf{i}n-eɾoɾtʰ & \armenian{իներորդ} 
		& 1000th & hɑ\textbf{zɑ}ɾ-eɾoɾtʰ & \armenian{հազարերորդ} 
		\\
		10th      & ˈ\textbf{dɑ}s-eɾoɾtʰ & \armenian{տասերորդ} & & &   \\
		\hline 
	\end{tabular}
\end{table}

These ordinal suffixes place stress on the preceding syllable. This irregularity is lost if an inflectional suffix is added. For example, if we add a suffix with a non-schwa like \textit{-i}, then stress shifts to the inflectional suffix. If we add a suffix with schwa or a clitic (Table \ref{tab:ordinal suffix stress shift}), then stress shifts to the rightmost non-schwa in the word, which will be the ordinal suffix itself. 

\begin{table}[H]
	\centering
	\caption{Loss of irregular stress in suffixed ordinals}
	\label{tab:ordinal suffix stress shift}
	\begin{tabular}{ |l|lll|}
		\hline
		&2nd  \armenian{երկրորդ}&6th \armenian{վեցերորդ} & 1000th \armenian{հազարերորդ}   \\
		Isolation         & ˈ\textbf{jeɡ}-ɾoɾtʰ & ˈ\textbf{ve}t͡s-eɾoɾtʰ & hɑˈ\textbf{zɑ}ɾ-eɾoɾtʰ
		\\
		\hline
		Genitive \textit{-i} \armenian{-ի} & jeɡ-ɾoɾˈ\textbf{tʰ-i} & vet͡s-eɾoɾˈ\textbf{tʰ-i} & hɑzɑɾ-eɾoɾˈ\textbf{tʰ-i}
		\\
		Ablative \textit{-e} \armenian{-է}  & jeɡ-ɾoɾˈ\textbf{tʰ-e} & vet͡s-eɾoɾˈ\textbf{tʰ-e} & hɑzɑɾ-eɾoɾˈ\textbf{tʰ-e}
		\\
		Instrumental \textit{-ov} \armenian{-ով}  & jeɡ-ɾoɾˈ\textbf{tʰ-ov} & vet͡s-eɾoɾˈ\textbf{tʰ-ov} & hɑzɑɾ-eɾoɾˈ\textbf{tʰ-ov}
		\\
		Plural \textit{-neɾ} \armenian{-ներ}
		& jeɡ-ɾoɾtʰ-ˈ\textbf{neɾ} & vet͡s-eɾoɾtʰ-ˈ\textbf{neɾ} & hɑzɑɾ-eɾoɾtʰ-ˈ\textbf{neɾ}
		\\
		\hline 
		Definite \textit{-ə} \armenian{-ը} 
		& jeɡ-ˈ\textbf{ɾoɾ}tʰ-ə & vet͡s-eˈ\textbf{ɾoɾ}tʰ-ə & hɑzɑɾ-eˈ\textbf{ɾoɾ}tʰ-ə
		\\
		1st possessive \textit{-əs} \armenian{-ս}
		& jeɡ-ˈ\textbf{ɾoɾ}tʰ-əs & vet͡s-eˈ\textbf{ɾoɾ}tʰ-əs & hɑzɑɾ-eˈ\textbf{ɾoɾ}tʰ-əs
		\\
		2nd possessive \textit{-ətʰ} \armenian{-դ}
		& jeɡ-ˈ\textbf{ɾoɾ}tʰ-ətʰ & vet͡s-eˈ\textbf{ɾoɾ}tʰ-ətʰ & hɑzɑɾ-eˈ\textbf{ɾoɾ}tʰ-ətʰ
		\\
		Indefinite \textit{-mə} \armenian{մը}
		& jeɡ-ˈ\textbf{ɾoɾtʰ}-mə & vet͡s-eˈ\textbf{ɾoɾtʰ}-mə & hɑzɑɾ-eˈ\textbf{ɾoɾtʰ}-mə
		\\
		\hline
		Clitic `=is'  \armenian{է} 
		& jeɡ-ˈ\textbf{ɾoɾ}tʰ=e  & vet͡s-eˈ\textbf{ɾoɾ}tʰ=e & hɑzɑɾ-eˈ\textbf{ɾoɾ}tʰ=e
		\\
		Clitic `=also'  \armenian{ալ}
		& jeɡ-ˈ\textbf{ɾoɾ}tʰ=ɑl  & vet͡s-eˈ\textbf{ɾoɾ}tʰ=ɑl & hɑzɑɾ-eˈ\textbf{ɾoɾ}tʰ=ɑl
		\\ \hline
		Derived \textit{-ɑɡɑn}   \armenian{-ական}
		& jeɡ-ɾoɾtʰ-ɑˈ\textbf{ɡɑn}  & vet͡s-eɾoɾtʰ-ɑˈ\textbf{ɡɑn}& hɑzɑɾ-eɾoɾtʰ-ɑˈ\textbf{ɡɑn}
		\\ \hline
	\end{tabular}
\end{table}

Some ordinals can also take a derivational suffix like \textit{-ɑɡɑn} to form adjectives. Regular stress is found. The adjective is used to denote meanings like  `secondary', `tertiary', and higher numbers.


The sentences below illustrate how these ordinals can be used in natural speech (\ref{ex:stress:prestress:ord sentece}). The unsuffixed form can be said in isolation, or as a modifier in a noun phrase. 

\begin{exe}
	\ex \gll ɑsiɡɑ im hɑˈ\textbf{zɑ}ɾ-eɾoɾtʰ kʰiɾkʰ-əs e
	\\
	this my.{\gen} thousand-{\ord} book-{\possFsg} is
	\\
	\trans `This is my 1000th book.'
	\label{ex:stress:prestress:ord sentece}
	\\
	\armenian{Ասիկա իմ հազարերորդ գիրքս է։}
\end{exe}

Ordinals can be inflected or cliticized when they're used without a noun (\ref{ex:stress:prestress:ord sentece more}). 

\begin{exe}
	\ex \label{ex:stress:prestress:ord sentece more}
	\begin{xlist}
		\ex \gll  hɑzɑɾ-eɾoɾˈ\textbf{tʰ-e-n} zəz-v-e-t͡s-ɑ-n  
		\\
		thousand-{\ord}-{\abl}{\defgloss} sick.of-{\pass}-{\thgloss}-{\aorperf}-{\pst}-3{\pl}
		\\
		\trans `They got sick of the 1000th one.'
		\\
		\armenian{Հազարերորդէն զզուեցան։}
		\ex \gll ɑsiɡɑ im hɑzɑɾ-eˈ\textbf{ɾoɾ}tʰ-əs e
		\\
		this my.{\gen} thousand-{\ord}-{\possFsg} is
		\\
		\trans `This is my 1000th (item).'
		\\
		\armenian{Ասիկա իմ հազարերորդ  է։}
		\ex \gll hɑzɑɾ-eˈ\textbf{ɾoɾtʰ}-n=ɑl ɡ-uz-e-m
		\\
		thousand-{\ord}-{\defgloss}=also {\ind}-want-{\thgloss}-1{\sg}
		\\
		\trans `I also want the 1000th one.'
		\\
		\armenian{Հազարերորդն ալ կ՚ուզեմ։}
		
	\end{xlist} 
\end{exe}

Note that the clitic `also' is rather unnatural to add to the ordinal directly; it is more natural to add the definite suffix between the ordinal and clitic. 

For the cliticized forms (\ref{ex:stress:prestress:ord sentece more school}), a common scenario is when the ordinal designates the school grade of a student. 

\begin{exe}
	\ex \label{ex:stress:prestress:ord sentece more school}
	\begin{xlist}
		\ex \gll ɑχt͡ʃiɡ-ətʰ voɾ meɡ tʰɑsɑɾɑn-n=e
		\\
		girl-{\possSsg} what one class-{\defgloss}=is
		\\
		\trans `Which grade is your daughter in?' (Lit. What class is your girl?)
		\\
		\armenian{Աղջիկդ ո՞ր մէկ դասարանն է։}
		\ex \gll ɑχt͡ʃiɡ-əs jotʰ-eˈ\textbf{ɾoɾ}tʰ=e
		\\
		girl-{\possFsg} seven-{\ord}=is
		\\
		\trans `My daughter is in the seventh grade.'  (Lit. My girl is seventh.)
		\\
		\armenian{Աղջիկս եօթերորդ է։}
	\end{xlist}
	
	
\end{exe}



The ordinal suffix \textit{-eɾoɾtʰ} can likewise be interrogative pronouns (wh-words) to mean something like `which one?' or `which grade' (Table \ref{tab:wh word ordinal}). Here, the suffix is prestressing if uninflected. The orthography conventionally places the question symbol \armenian{՞} on the stressed vowel of the  root.  When this word is inflected, HD's judgment  is that there is no stress shift. There might be secondary stress on the rightmost non-schwa. For these wh-words, we suspect that there's high variability in the application of stress shift because of interaction between the lexical pre-stressing nature of the ordinal suffix \textit{and} the prominence given to the interrogative root as an inherently focused element.  

\begin{table}[H]
	\centering
	\caption{Interrogative pronouns with the ordinal suffix}
	\label{tab:wh word ordinal}
	
	\begin{tabular}{|l|lll| }
		\hline       `how many'  &{kʰɑˈ\textbf{ni}} & \armenian{քանի՞}& `how many?'   \\
		+ Ord &{kʰɑˈ\textbf{ni}-jeɾoɾtʰ} & \armenian{քանի՞երորդ}  & `which one (from some numbered set)?'
		\\
		+ Ord + Ins&{kʰɑˈ\textbf{ni}-jeɾoɾˌtʰ-ov} & \armenian{քանի՞երորդով}  
		&  `from which one' \\ 
		+ Ord + Def&{kʰɑˈ\textbf{ni}-jeɾˌoɾtʰ-ə} & \armenian{քանի՞երորդը}  
		&  `which one (definite)' \\ 
		\hline
		`which'  &{ˈ\textbf{voɾ}} & \armenian{ո՞ր}& `which?'      \\
		+ Ord &{\textbf{vo}ɾ-eɾoɾtʰ} & \armenian{ո՞րերորդ}  & `which one (from some numbered set)?'
		\\
		+ Ord + Ins&{ˈ\textbf{vo}ɾ-eɾoɾˌtʰ-ov} & \armenian{ո՞րերորդով}  
		&  `from which one' \\ 
		+ Ord + Def&{ˈ\textbf{vo}ɾ-eɾˌoɾtʰ-ə} & \armenian{ո՞րերորդը}  
		&  `which one (definite)' \\ 
		\hline 
		`what'  &{ˈ\textbf{int͡ʃ}} & \armenian{ի՞նչ}& `what?'      \\
		+ Ord &{ˈ\textbf{in}t͡ʃ-eɾoɾtʰ} & \armenian{ի՞նչերորդ}  & `what   (from some numbered set)?'
		\\
		+ Ord + Ins&{ˈ\textbf{in}t͡ʃ-eɾoɾˌtʰ-ov} &  \armenian{ի՞նչերորդով}
		&  `from what' \\ 
		+ Ord + Def&{ˈ\textbf{in}t͡ʃ-eɾˌoɾtʰ-ə} &  \armenian{ի՞նչերորդը}
		&  `what (definite)' \\ 
		\hline 
		
		
	\end{tabular}
\end{table}

The use of these interrogative ordinals is common when asking for the school grade of a person (\ref{ex:stress:grade}). In such contexts, the ordinal often takes a clitic.  


\begin{exe}
	\ex \gll {tʰəbɾot͡s-i-n} {met͡ʃ} {mɑɾjɑm-ə}  {\textbf{ˈin}t͡ʃ-eˌɾoɾtʰ=e} \\
	school-{\gen}-{\defgloss} in Mariam-{\defgloss} what-{\ord}=is \\
	\trans `What grade is Mariam in at school?
	\\
	\label{ex:stress:grade}
	\armenian{Դպրոցին մէջ, Մարիամը ի՞նչերորդ է։}
	
\end{exe}



The above stress data is from HD's judgments, as a person from the Beirut community. Previous grammars and documents report variation in whether \textit{-ɾoɾtʰ} suffix, \textit{-eɾoɾtʰ} suffix, or both suffixes place stress on the preceding syllable. We summarize the reported variation in \textcolor{red}{summarize all the sources on ordinal stress}

We don't know why there is such reported variation. It could indicate that different communities have changed the rules for stressing these ordinals across time.  



\subsection{Adverbalizer suffix}\label{section:stress:prestress:adverb}
There are three derivational suffixes that turn words into adverbs: \textit{-oɾen, -ɑpʰɑɾ, -ɑbes}. Of the three suffixes, we feel that the \textit{-oɾen} is the most productive, but all three suffixes are attested.   For the Lebanese community, these suffixes place stress on the preceding syllable. 

Table \ref{tab:adv suffix irregular stress} lists    adverbs that use these suffixes. As is seen, the stressed vowel is before the suffix. This vowel can have any vowel quality, including a schwa. 

\begin{table}[H]
	\centering
	\caption{Irregular stress before the adverbalizer suffixes}
	\label{tab:adv suffix irregular stress}
	\resizebox{\textwidth}{!}{%
		\begin{tabular}{|l|ll|ll|ll|}
			\hline 
			&     \textit{-oɾen} && \textit{-ɑpʰɑɾ} && \textit{-ɑbes} 
			\\\hline
			/ɑ/ & ɑˈ\textbf{zɑd} & `free' 
			& ˈ\textbf{kʰɑt͡ʃ} & `brave'
			& nəˈ\textbf{mʰɑn} & `similar'
			\\
			& & \armenian{ազատ}
			& & \armenian{քաջ}
			& & \armenian{նման}
			\\
			& ɑˈ\textbf{zɑ}d-oɾen & `freely'
			&  ˈ\textbf{kʰɑ}t͡ʃ-ɑpʰɑɾ & `bravely'
			&  nəˈ\textbf{mɑ}n-ɑbes & `similarly'
			\\
			& & \armenian{ազատօրէն}
			& & \armenian{քաջաբար}
			& & \armenian{նմանապէս}
			\\
			\hline 
			/e/ & tʰeˈ\textbf{tʰev} & `light'
			& əŋˈ\textbf{ɡeɾ} & `friend'
			& ˈ\textbf{veɾt͡ʃ} & `end'
			\\
			& & \armenian{թեթեւ}
			& & \armenian{ընկեր}
			& & \armenian{վերջ}
			\\
			& tʰeˈ\textbf{tʰe}v-oɾen & `lightly'
			& əŋˈ\textbf{ɡe}ɾ-ɑpʰɑɾ & `friendly'
			& ˈ\textbf{veɾ}t͡ʃ-ɑbes & `finally'
			\\
			& & \armenian{թեթեւօրէն}
			& & \armenian{ընկերաբար}
			& & \armenian{վերջապէս}
			\\ \hline
			/i/ & nɑzeˈ\textbf{li} & `gracious'
			& t͡səˈ\textbf{ɾi} & `free'
			& ˈ\textbf{isk} & `real'
			\\
			& & \armenian{նազելի}
			& & \armenian{ձրի}
			& & \armenian{իսկ}
			\\
			& nɑzeˈ\textbf{li}j-oɾen & `graciously'
			& t͡səˈ\textbf{ɾi}j-ɑpʰɑɾ & `for free'
			&  ˈ\textbf{is}k-ɑbes & `really'
			\\
			& & \armenian{նազելիօրէն}
			& & \armenian{ձրիաբար}
			& & \armenian{իսկապէս}
			\\ \hline
			/o/ & hetʰɑˈ\textbf{nos} & `heathen'
			& soˈ\textbf{voɾ} & `accustomed'
			& modɑˈ\textbf{voɾ} & `near'
			\\
			& & \armenian{հեթանոս}
			& & \armenian{սովոր}
			& & \armenian{մօտաւոր}
			\\
			&   hetʰɑˈ\textbf{no}s-oɾen & `heathenishly'
			&    soˈ\textbf{vo}ɾ-ɑpʰɑɾ & `usually'
			&    modɑˈ\textbf{vo}ɾ-ɑbes & `approximately'
			\\
			& & \armenian{հեթանոսօրէն}
			& & \armenian{սովորաբար}
			& & \armenian{մօտաւորապէս}
			\\ \hline
			/u/ & ˈ\textbf{d͡zujl} & `lazy'
			& ɑmuˈ\textbf{sin} & `husband'  
			& əsˈ\textbf{kuʃ} & `careful'  
			\\
			& & \armenian{ծոյլ}
			& & \armenian{ամուսին}
			& & \armenian{զգոյշ}
			\\
			& ˈ\textbf{d͡zu}l-oɾen & `lazily'
			& ɑˈ\textbf{mus}n-ɑpʰɑɾ & `maritably'
			& əsˈ\textbf{ku}ʃ-ɑbes & `carefully'
			\\
			& & \armenian{ծուլօրէն}
			& & \armenian{ամուսնաբար}
			& & \armenian{զգուշապէս}
			\\ 
			\hline 
			/ʏ/&  ɑˈ\textbf{ɾʏd͡z} & `lion'
			& ɑχˈ\textbf{pʏɾ} & `fountain' 
			& ˈ\textbf{tʰʏɾ} & `easy' 
			\\
			&   & \armenian{առիւծ} && \armenian{աղբիւր} & & \armenian{դիւր}
			\\
			&  ɑˈ\textbf{ɾʏ}d͡z-oɾen & `lion-like' 
			& ɑχˈ\textbf{pʏ}ɾ-ɑpʰɑɾ & `abundantly' 
			& ˈ\textbf{tʰʏ}ɾ-ɑbes & `easily'\\
			&    & \armenian{առիւծօրէն} && \armenian{աղբիւրաբար} & &  \armenian{դիւրապէս}
			\\ \hline 
			/ə/ & ˈ\textbf{luɾt͡ʃ} & `serious' 
			& tʰəˈ\textbf{tʰu} & `sour' 
			& ɑzˈ\textbf{niv} & `sincere' 
			\\
			& & \armenian{լուրջ}
			& & \armenian{թթու}
			& & \armenian{ազնիւ}
			\\
			& ˈ\textbf{ləɾ}t͡ʃ-oɾen & `seriously'
			& ˈ\textbf{tʰət}f-ɑpʰɑɾ & `sourly'
			& ɑzˈ\textbf{nə}v-ɑbes & `sincerely'
			\\
			& & \armenian{լրջօրէն}
			& & \armenian{թթուաբար}
			& & \armenian{ազնուապէս}
			\\ \hline
			
		\end{tabular}
}\end{table}




Interestingly, there are some adjectives that have an epenthetic schwa in the last syllable like \textit{ˈ\textbf{pʰɑɾ}t͡səɾ} `high' (Table \ref{tab:adverbalizer stress schwa}). The schwa is usually deleted before vowel-initial suffixes  \textit{ˈ\textbf{pʰɑɾt͡s}ɾ-oɾen} but colloquial pronunciations can allow the schwa to stay. When the schwa is present, the schwa takes stress because it is right before the adverb suffix: \textit{pʰɑɾˈ\textbf{t͡sə}ɾ-oɾen}.

\begin{table}[H]
	\centering
	\caption{Stress on schwas before adverbalizer suffixes}
	\label{tab:adverbalizer stress schwa}
	\begin{tabular}{|lll|lll|}
		\hline 
		ˈ\textbf{tʰɑ}ɾən & `bitter' & \armenian{դառն}   
		& ˈ\textbf{pʰɑɾ}t͡səɾ & `high' & \armenian{բարձր}
		\\
		ˈ\textbf{tʰɑɾ}n-ɑpʰɑɾ & `bitterly' & \armenian{դառնապէս}   
		& ˈ\textbf{pʰɑɾt͡s}ɾ-oɾen & `highly' & \armenian{բարձրօրէն}
		\\
		tʰɑˈ\textbf{ɾə}n-ɑpʰɑɾ & &  
		& pʰɑɾˈ\textbf{t͡sə}ɾ-oɾen &   & 
		\\ \hline
		ˈ\textbf{pʰo}kʰəɾ & `small' & \armenian{փոքր}   
		& ˈ\textbf{kʰɑχ}t͡səɾ & `sweet' & \armenian{փոքր}   
		\\
		ˈ\textbf{pʰokʰ}ɾ-ɑpʰɑɾ & `small-ly' & \armenian{փոքրաբար}   
		& ˈ\textbf{kʰɑχt͡s}ɾ-oɾen & `sweetly' & \armenian{քաղցրօրէն}   
		\\
		pʰoˈ\textbf{kʰə}ɾ-ɑpʰɑɾ & &  
		& kʰɑχˈ\textbf{t͡sə}ɾ-oɾen &   &  
		\\ \hline 
		ˈ\textbf{d͡zɑ}nəɾ & `heavy' & \armenian{ծանր}   
		& ˈ\textbf{tʰɑn}t͡səɾ & `dense' & \armenian{թանձր}   
		\\
		ˈ\textbf{d͡zɑn}ɾ-ɑbes & `heavily' & \armenian{ծանրապէս}   
		& ˈ\textbf{tʰɑnt͡s}ɾ-ɑpʰɑɾ & `densely' & \armenian{թանձրաբար}   
		\\
		d͡zɑˈ\textbf{nə}ɾ-ɑbes & &  
		& tʰɑnˈ\textbf{t͡sə}ɾ-ɑpʰɑɾ &  &  
		\\ \hline 
	\end{tabular}
\end{table}

These adverbalizer suffixes interact  paradoxically with destressed vowel reduction (Table \ref{tab:adv stress high mid vowel reduction}). For a root with a final high vowel like \textit{χist} `serious', the  high vowel is stressed when unsuffixed. As detailed in \textcolor{red}{cite chapter on reduction}, when a derivational suffix is added to such roots, we see stress shift and reduction. The high vowel is either reduced to  a schwa or deleted. For these adverbalizers, they don't trigger stress shift but they do trigger reduction: \textit{ˈχəst-oɾen}. Stress is then on the pre-suffix vowel. Some closed set of roots also show reduction of destressed \textit{e} to \textit{i}. This reduction overapplies before adverbial suffixes. 

\begin{table}[H]
	\centering
	\caption{Paradoxical application of destressed high vowel reduction and destressed midvowel reduction}
	\label{tab:adv stress high mid vowel reduction}
	\resizebox{\textwidth}{!}{%
		
		\begin{tabular}{|ll|ll|ll| }
			\hline
			ˈ\textbf{χist} & `rigorous'& hɑŋˈ\textbf{kʰist}  & `comfortable' & jeɾɑˈ\textbf{ʒiʃt}  & `musician' 
			\\
			& \armenian{խիստ} & & \armenian{հանգիստ}& & \armenian{երաժիշտ}
			\\
			ˈ\textbf{χəs}t-oɾen & `rigorously' & hɑŋˈ\textbf{kʰəs}t-ɑpʰɑɾ  & `comfortable'  & jeɾɑˈ\textbf{ʒəʃ}t-ɑbes  & `musically' 
			\\
			& \armenian{խստօրէն} && \armenian{հանգստաբար}&& \armenian{երաժշտապէս}
			\\ \hline
			hɑˈ\textbf{duɡ} & `particular' & jeɾt͡ʃɑˈ\textbf{niɡ} & `fortunate' & kʰeʁeˈ\textbf{t͡siɡ} & `beautiful'
			
			\\
			&  \armenian{յատուկ}& & \armenian{երջանիկ}& & \armenian{գեղեցիկ}
			\\
			ˈ\textbf{hɑd}ɡ-oɾen & `in particular' & jeɾ\textbf{t͡ʃɑŋ}ɡ-ɑpʰɑɾ & `fortunately'& kʰeˈ\textbf{ʁet͡s}k-ɑbes & `beautifully'
			\\
			& \armenian{յատկօրէն}   & & \armenian{երջանկաբար} & & \armenian{գեղեցկապէս}
			\\ \hline
			də-ˈ\textbf{kʰed} & `stupid' & ɑŋ-ˈ\textbf{kʰed} & `ignorant' & oˈ\textbf{ɾen} & `law (archaic)
			\\
			& \armenian{տգէտ}  && \armenian{անգէտ}  & & \armenian{օրէն}
			\\
			də-ˈ\textbf{kʰi}d-oɾen & `stupid-ly' & ɑŋ-ˈ\textbf{kʰi}d-ɑpʰɑɾ & `ignorantly' &  oˈ\textbf{ɾi}n-ɑbes & `legally' 
			\\
			& \armenian{տգիտօրէն} & & \armenian{անգիտաբար}& &\armenian{օրինապէս}
			\\ \hline
		\end{tabular}
}\end{table}


Synchronically, it is a paradox why these roots change or delete their vowel in these adverbs. It is a paradox because the adverbalizer suffixes do not trigger stress shift in the modern language.  But diachronically or historically, this reduction is  because of how the adverbalizers   likely did trigger stress shift: [{ˈχist}] $\rightarrow$ //{χəst-oˈɾen}//. Reduction applied at this earlier stage. When the modern language turned these suffixes into unstressed suffixes, the reduction still applied: //{χəst-oˈɾen}// $\rightarrow$ [ˈχəst-oɾen]

\textcolor{red}{hisory of these suffixes, and variation}


The suffixes \textit{-oɾen, -ɑpʰɑɾ, -ɑbes} are usually used to form adverbs (Table \ref{tab:adverbalizer not}). But there are a handful of words where these suffixes are used to form nouns. For \textit{-ɑpʰɑɾ}, the suffix developed the function to designate the name of Armenian dialects, possibly due to close link between adverbs and speech, e.g., to speak in the modern way vs. the classical way. For the suffix     \textit{oɾen}, this morpheme can act as a noun root for `law'. Here, the noun-version of these suffixes does not cause irregular stress. These nouns take regular final stress. 

\begin{table}[H]
	\centering
	\caption{Words with where the adverbalizer suffix acts as a non-adverbalizer}
	\label{tab:adverbalizer not}
	\begin{tabular}{|llll| }
		\hline      /oɾen/ as noun root:  & oˈ\textbf{ɾen} & `law (archaic)' & \armenian{օրէն} \\
		& ɑn-oˈ\textbf{ɾen} & `illegal' & \armenian{անօրէն} \\
		& ˈ\textbf{dun} & `house' & \armenian{տուն} \\
		& dən-oˈ\textbf{ɾen} & `director' & \armenian{տնօրէն} \\
		\hline 
		/oɾen/ as nominalizer:  & ɾɑˈ\textbf{miɡ} & `vulgar'   & \armenian{ռամիկ} \\
		& ɾɑmɡ-oˈ\textbf{ɾen} & `Middle Armenian' & \armenian{ռամկօրէն }
		\\         
		& kʰəɾ-ɑˈ\textbf{pʰɑɾ} & `Classical Armenian' & \armenian{գրաբար}            
		\\         
		& ɑʃˈ\textbf{χɑɾ} & `world' & \armenian{աշխարհ}     \\
		& ɑʃχɑɾ-ɑˈ\textbf{pʰɑɾ} & `Modern Armenian' & \armenian{աշխարհաբար}
		\\         
		\hline
	\end{tabular}
\end{table}


For the adverbalizer suffixes (\ref{ex:stress:prestress:adv sentce}), it is very difficult to naturally add any inflection after them. The most typical use of these adverbs is to simply add these words to a sentence, to modify a verb. The clitic `also' can also be added. When the suffix is cliticized, we get regular final stress on the word, before the clitic. 

\begin{exe}
	\ex \label{ex:stress:prestress:adv sentce}
	\begin{xlist}
		\ex \gll uˈ\textbf{ɾɑ}χ-oɾen dun jeɡ-ɑ-n
		\\
		happy-{\advz} house come.{\aorperf}-{\pst}-3{\pl}
		\\
		\trans `We happily came home.' 
		\\
		\armenian{Ուրախօրէն տուն եկան։}
		
		\ex \gll jev ɑɾɑkʰ-oˈ\textbf{ɾen}=ɑl əɾ-i-n
		\\
		and quick-{\advz}=also do.{\aorperf}-{\pst}-3{\pl}
		\\
		\trans `And they did it quickly too.'
		\\
		\armenian{Արագօրէն ալ ըրին։}
	\end{xlist}
	
\end{exe}
\subsection{Hypocoristic suffix}\label{section:stress:prestress:hypocoristic}
Given a name like [mɑɾjɑm] `Mariam', a hypocoristic or nickname is formed by taking the first syllable of the word and adding the suffix \textit{-o}: [ˈ\textbf{mɑ}ɾ-o].   The suffix \textit{-o} irregularly assigns stress before it on the first syllable. 

For morphological and phonological rules on how to form nicknames, see \textcolor{red}{cite chapter nickname}. This section focuses on the stress pattern of these words. Table \ref{tab:hypocoristic stress} provides examples of some common nicknames, adapted and augmented from   \citep[247]{Vaux-1998-ArmenianPhono}.

\begin{table}[H]
	\centering
	\caption{Sample of common nicknames with \textit{-o} and irregular stress}
	\label{tab:hypocoristic stress}
	\begin{tabular}{|ll ll ll|}
		\hline 
		
		{mɑɾˈ\textbf{jɑm}} & \armenian{Մարիամ} 
		& 
		{ˈ\textbf{hɑjɡ}} & \armenian{Հայկ}
		& {ˈ\textbf{vɑɾtʰ}}  & \armenian{Վարդ}
		\\
		{ˈ\textbf{mɑ}ɾ-o} & \armenian{Մարօ}
		& {ˈ\textbf{hɑj}ɡ-o}& \armenian{Հայկօ}
		&{ˈ\textbf{vɑɾ}tʰ-o}& \armenian{Վարդօ}
		\\
		\hline 
		
		{nɑzɑˈ\textbf{ɾetʰ}} & \armenian{Նազարէթ}
		&
		{ɑntʰɾɑˈ\textbf{niɡ}} & \armenian{Անդրանիկ}
		&
		{ɑɾʃɑˈ\textbf{lujs}} & \armenian{Արշալոյս}
		\\
		{ˈ\textbf{nɑ}z-o}& \armenian{Նազօ}
		& {ˈ\textbf{ɑn}tʰ-o} & \armenian{Անդօ}
		& {ˈ\textbf{ɑɾ}ʃ-o}& \armenian{Արշօ}
		\\ \hline 
		
		{jeʁisɑˈ\textbf{pʰetʰ}}  &  \armenian{Եղիսաբէթ}
		&
		{mɑnˈ\textbf{vel}} & \armenian{Մանուէլ}
		&
		{sɑmˈ\textbf{vel}} & \armenian{Սամուէլ}
		\\
		{ˈ\textbf{jeχ}s-o} & \armenian{Եղսօ}
		& {ˈ\textbf{mɑ}n-o} & \armenian{Մանօ}
		& {ˈ\textbf{sɑ}m-o} & \armenian{Սամօ}
		\\ \hline 
		
		{bedˈ\textbf{ɾos}} & \armenian{Պետրոս}
		&
		{setʰˈ\textbf{ɾɑɡ}}  & \armenian{Սեդրակ}
		&
		{dikʰ\textbf{ɾɑn}} & \armenian{Տիգրան}
		\\
		{ˈ\textbf{be}d-o}& \armenian{Պետօ}
		& {ˈ\textbf{se}tʰ-o} & \armenian{Սեդօ}
		& {ˈ\textbf{di}kʰ-o} & \armenian{Տիգո}
		\\ \hline 
		
		{ɑpʰɾɑhɑm} & \armenian{Աբրահամ}
		& 
		{ɑɾˈ\textbf{men}} & \armenian{Արմէն}
		&
		{sɑɾˈ\textbf{kʰis}} & \armenian{Սարգիս}
		\\
		{ˈ\textbf{ɑp}ɾ-o}  & \armenian{Աբրօ}
		& {ˈ\textbf{ɑɾ}m-o} & \armenian{Արմօ}
		&{ˈ\textbf{sɑ}kʰ-o} & \armenian{Սագօ}
		\\
		\hline 
	\end{tabular}
	
\end{table}

It is rather rare to find a hypocoristic where the initial syllable has a schwa. One common name is the nickname of  \armenian{Մկրտիչ}: [məɡəɾˈdit͡ʃ] in Western, [məkəɾˈtit͡ʃʰ] in Eastern. The nickname form uses a schwa. Vaux reports that this nickname stresses the schwa in Eastern: [ˈmək-o]. But in HD's Western judgments, stress is on the suffix: [məˈɡ-o]. Due to limited data, it is unknown if schwas in nicknames generally resist stress in Western Armenian. 




Hypocoristics take initial stress when used in isolation (citation form) and as vocatives in direct address (\ref{ex:stress:prestress:hypo sentce}). Throughout this section, we gloss the hypocoristic {ˈmɑɾ-o} as `Maro-{\hcr}'. See Section \S\ref{section:intonation:other:vocative} for more on vocatives. 







\begin{exe}
	\ex 
	\begin{xlist}
		\ex \gll {pɑˈɾev}, {\textbf{ˈmɑ}ɾ-o} \\
		hello  Maro-{\hcr} \\
		\trans `Hello, Maro.'
		\label{ex:stress:prestress:hypo sentce}
		\\ \armenian{Բարեւ, Մարօ։}
	\end{xlist}
\end{exe}


Like ordinals, we treat the suffix \textit{-o} as irregularly prestressing.   Also like ordinals, the prestressing behavior is lost when the hypocoristic is further inflected or encliticized (\ref{ex:stress:prestress:hypo sentce: add}).   Note how a glide [j] is epenthesized between the \textit{-o} and a vowel. 

\begin{exe}
	\ex \label{ex:stress:prestress:hypo sentce: add}
	\begin{xlist}
		\ex \gll  {mɑɾ-o-ˈ\textbf{ji-n}} {kʰiɾkʰ} {{dəv-i}}  \\
		Maro-{\hcr}-{\dat}-{\defgloss} book give.{\aorperf}-1{\sg} \\
		\trans `I gave books to Maro.'
		\\ \armenian{Մարոյին գիրք տուի։}
		\ex \gll  {mɑɾ-o-ˈ\textbf{je-n}} {kʰiɾkʰ} {{ɑɾ-i}}  \\
		Maro-{\hcr}-{\dat}-{\defgloss} book take.{\aorperf}-1{\sg} \\
		\trans `I got books from Maro.'
		\\ \armenian{Մարոյէն  գիրք առի։}
		\ex \gll  {ɑnun-əs} {mɑˈ\textbf{ɾ-o}=je}  \\
		name-{\possFsg} Maro-{\hcr}=is  \\
		\trans `My name is Maro.'
		\\ \armenian{Անունս Մարօ է։}
	\end{xlist}
\end{exe}


Hypocoristics differ from ordinals however in that hypocoristics end in a vowel (\ref{ex:stress:prestress:hypo sentce: add more}).  This has significance for some types of phonologically-conditioned allomorphy. The definite suffix and the possessives surface with a schwa after consonant-final bases like ordinals: definite \textit{-ə}, 1st possessive \textit{-əs}, 2nd possessive \textit{-ətʰ}. They surface as a consonant after vowel-final bases like hypocoristics: \textit{-n, -s, -tʰ}. When these consonantal suffixes are added to the hypocoristic suffix \textit{-o}, we see stress shift to the suffix \textit{-o}.\footnote{Proper names take the definite suffix when used as the subject of the sentence.}

\begin{exe}
	\ex\label{ex:stress:prestress:hypo sentce: add more}
	\begin{xlist}
		\ex \gll {mɑ\textbf{ɾ-\'o-n}} {jeɡ-ɑ-v}.\\
		Maro-{\hcr}-{\defgloss} come.{\aorperf}-{\pst}-3{\sg}\\
		\trans `Maro came.'
		\\ \armenian{Մարօն եկաւ}
		\ex \gll {im} {mɑ\textbf{ɾ-\'o-s}}    {jeɡ-ɑ-v} \\
		my.{\gen} Maro-{\hcr}-{\possFsg}    come.{\aorperf}-{\pst}-3{\sg}\\
		\trans `My Maro came (as opposed to the Maro who you know).'
		\\ \armenian{Իմ Մարօս եկաւ։}
		\ex \gll   kʰu mɑˈ\textbf{ɾ-o-tʰ}  {jeɡ-ɑ-v} \\
		your.{\gen}.{\sg}   Maro-{\hcr}-{\possSsg} come.{\aorperf}-{\pst}-3{\sg}\\
		\trans `Your Maro came (as opposed to the Maro who I know).'
		\\ \armenian{Քու Մարօդ եկաւ։}
	\end{xlist}
\end{exe}




\section{Irregular stress in verb inflection}\label{section:stress:verb}
In their inflectional paradigm, verbs show regular final stress in almost all possible inflected forms. There are however some cases of exception: negated finite forms, negated periphrastic forms, prohibitives (negative imperatives), and the past imperfective. For the negation-related forms, stress is on the vowel which is either in or close to the relevant negation morpheme.  For the past imperfective,    stress is on the (non-final) theme vowel. The interaction of irregular stress of negation, the past imperfective, and clitics is also quite complicated. 


\subsection{Negative finite forms}\label{section:stress:verb:negFinite}
Verbs show the following basic synthetic forms: present, past imperfective, and past perfective. The past imperfective has irregular stress on the theme vowel; we discuss that later in \S\ref{section:stress:verb:pastImpf:negation}. The other two forms take regular final stress and we focus on them. In their negated forms, the prefix \textit{t͡ʃ(ə)-} is added (Tbale \ref{tab:stress verb negation basic}). The schwa is used if the verb starts with a consonant. Primary stress is on the first syllable, on the vowel that is next to the negation prefix. There is some level of secondary stress on the rightmost non-schwa. 

\begin{table}[H]
	\centering
	\caption{Negation stress in negated finite forms of verbs } 
	\label{tab:stress verb negation basic}
	
	\begin{tabular}{|l|lll|l| }
		\hline 
		&& C-initial   & V-initial  &   \\
		&&   `to measure' &   `to hate' &   \\
		\hline 
		Infinitive &        & t͡ʃɑˈ\textbf{pʰ-e-l}& ɑˈ\textbf{d-e-l}&        \armenian{չափել, ատել}
		\\
		\hline 
		
		Present 3Pl & Pos.   &  t͡ʃɑˈ\textbf{pʰ-e-n}    &  ɑˈ\textbf{d-e-n}  & \armenian{չափեն, ատեն}
		\\ 
		& Neg.&  ˈ\textbf{t͡ʃə}-t͡ʃɑˌpʰ-e-n &  ˈ\textbf{t͡ʃ-ɑ}ˌd-e-n   & \armenian{չչափեն, չատեն}
		\\
		&& \multicolumn{2}{l|}{({\neggloss}-)$\sqrt{}$-{\thgloss}-3{\pl}} &
		\\
		\hline 
		Past Perf.  3PL &Pos.  &   t͡ʃɑpʰ-e-ˈ\textbf{t͡s-i-n}&  ɑd-e-ˈ\textbf{t͡s-i-n}    & \armenian{չափեցին, ատեցին} 
		\\
		& Neg. Std.  &  ˈ\textbf{t͡ʃə}-t͡ʃɑpʰ-e-ˌt͡s-i-n &  ˈ\textbf{t͡ʃ-ɑ}d-e-ˌt͡s-i-n   & \armenian{չչափեցին, չատեցին}
		\\
		& Neg. Coll.  &  ˈ\textbf{t͡ʃi}-t͡ʃɑpʰ-e-ˌt͡s-i-n &      & \armenian{չի չափեցին } 
		\\
		& & \multicolumn{2}{l|}{({\neggloss}-)$\sqrt{}$-{\thgloss}-{\aorperf}-{\pst}-3{\pl}} &
		\\
		\hline 
		
	\end{tabular}
	
\end{table}

Note how the C-initial verb is a near-minimal pair against the negated V-initial verb: [t͡ʃɑˈpʰ-e-n] vs. [ˈt͡ʃ-ɑd-e-n]. Full paradigms for these negated forms are founded in the relevant morphology sections \textcolor{red}{cite chapter on verbal inflection paradigm}

For the past perfective, the standard pronunciation of the negation prefix is just \textit{t͡ʃə-} before a consonant. But colloquial speech allows the form \textit{t͡ʃi} instead. Stress is still on the prefix. This colloquial form is homophonous with the present 3SG negated auxiliary \textit{t͡ʃ-i}; see paradigms in \textcolor{red}{cite auxiliary form chapter with chi}.     It's possible that this colloquial form developed in order to   avoid having a stress schwa.

For the words in Table \ref{tab:stress verb negation basic}, the root's first vowel was /ɑ/ and the root was monosyllabic. But negation takes stress regardless of the type of vowel or word size (Table \ref{tab:negatin stress shift regardless vowel}).  We illustrate below with just negative forms. For space we don't gloss the following examples, but the segmentation is the same as in Table \ref{tab:stress verb negation basic}. 

\begin{table}[H]
	\centering
	\caption{Initial stress in negated finite forms regardless of root vowel quality}
	\label{tab:negatin stress shift regardless vowel}
	\begin{tabular}{|lll|ll| }
		\hline 
		&Present 3PL &  Past perfective 3PL & &  \\
		\hline 
		/ɑ/    & ˈ\textbf{t͡ʃə}-nɑˌj-i-n & ˈ\textbf{t͡ʃə}-nɑj-e-ˌt͡s-ɑ-n & `to look' & \armenian{նայիլ} \\
		& ˈ\textbf{t͡ʃ-ɑj}ˌɾ-i-n & ˈ\textbf{t͡ʃ-ɑj}ɾ-e-ˌt͡s-ɑ-n & `to burn' & \armenian{այրիլ} \\
		& ˈ\textbf{t͡ʃ-ɑɾ}tʰuˌɡ-e-n & ˈ\textbf{t͡ʃ-ɑɾ}tʰuɡ-e-ˌt͡s-i-n & `to iron' & \armenian{արդուկել} \\
		
		\hline 
		/e/    & ˈ\textbf{t͡ʃə}-deˌv-e-n & ˈ\textbf{t͡ʃə}-dev-e-ˌt͡s-i-n & `to last' & \armenian{տեւել} \\
		& ˈ\textbf{t͡ʃ-e}ˌpʰ-e-n & ˈ\textbf{t͡ʃ-e}pʰ-e-ˌt͡s-i-n & `to cook' & \armenian{եփել}
		\\
		& ˈ\textbf{t͡ʃ-e}ɾɑˌz-e-n & ˈ\textbf{t͡ʃ-e}ɾɑz-e-ˌt͡s-i-n & `to dream' & \armenian{երազել}
		\\\hline 
		
		/i/         & ˈ\textbf{t͡ʃə}-siˌɾ-e-n & ˈ\textbf{t͡ʃə}-siɾ-e-ˌt͡s-i-n & `to like' & \armenian{սիրել} \\
		& ˈ\textbf{t͡ʃ-iʃ}ˌχ-e-n & ˈ\textbf{t͡ʃ-iʃ}χ-e-ˌt͡s-i-n & `to rule' & \armenian{իշխել} 
		\\
		& ˈ\textbf{t͡ʃ-i}ɾɑkʰoɾˌd͡z-e-n & ˈ\textbf{t͡ʃ-i}ɾɑkʰoɾd͡z-e-ˌt͡s-i-n & `to effect' & \armenian{իրագործել} 
		\\
		
		\hline 
		/o/     & ˈ\textbf{t͡ʃə}-pʰoˌχ-e-n & ˈ\textbf{t͡ʃə}-pʰoχ-e-ˌt͡s-i-n & `to change' & \armenian{փոխել} \\
		& ˈ\textbf{t͡ʃ-okʰ}ˌn-e-n & ˈ\textbf{t͡ʃ-okʰ}n-e-ˌt͡s-i-n & `to help' & \armenian{օգնել} \\
		& ˈ\textbf{t͡ʃ-o}d͡zɑˌn-e-n & ˈ\textbf{t͡ʃ-o}d͡zɑn-e-ˌt͡s-i-n & `to anoint' & \armenian{օծանել} \\ 
		\hline 
		/u/     & ˈ\textbf{t͡ʃə}-pʰuˌʒ-e-n & ˈ\textbf{t͡ʃə}-pʰuʒ-e-ˌt͡s-i-n & `to heal' & \armenian{բուժել} 
		\\
		& ˈ\textbf{t͡ʃ-u}ˌz-e-n & ˈ\textbf{t͡ʃ-u}z-e-ˌt͡s-i-n & `to want' & \armenian{ուզել} \\
		& ˈ\textbf{t͡ʃ-u}sɑˌn-i-n & ˈ\textbf{t͡ʃ-u}sɑn-e-ˌt͡s-ɑ-n & `to learn' & \armenian{ուսանիլ} \\ 
		\hline 
		/ʏ/  & ˈ\textbf{t͡ʃə}-hʏˌs-e-n & ˈ\textbf{t͡ʃə}-hʏs-e-ˌt͡s-i-n & `to weave' & \armenian{հիւսել}  
		\\ 
		\hline 
		/ə/      & ˈ\textbf{t͡ʃə}-ləˌz-e-n & ˈ\textbf{t͡ʃə}-ləz-e-ˌt͡s-i-n & `to lick' & \armenian{լզել} 
		
		\\
		& ˈ\textbf{t͡ʃ-ənd}ˌɾ-e-n & ˈ\textbf{t͡ʃ-ənd}ɾ-e-ˌt͡s-i-n & `to choose' & \armenian{ընտրել}
		\\
		& ˈ\textbf{t͡ʃ-ən}d͡zɑˌj-e-n & ˈ\textbf{t͡ʃ-ən}d͡zɑj-e-ˌt͡s-i-n & `to offer' & \armenian{ընծայել}
		
		\\
		\hline 
		
	\end{tabular}
\end{table}


For the past perfective form (\ref{ex:stress:neg:pastperf}), these forms can be elicited in isolation without any special sentence structure. 

\begin{exe}
	\ex \gll  ˈ\textbf{t͡ʃə}-t͡ʃɑpʰ-e-t͡s-i-n jev ˈ\textbf{t͡ʃ}-ɑd-e-t͡s-i-n
	\\
	{\neggloss}-measure-{\thgloss}-{\aorperf}-{\pst}-3{\pl} and  {\neggloss}-hate-{\thgloss}-{\aorperf}-{\pst}-3{\pl}
	\\
	\trans `They didn't measure it, and they didn't hate it.'  
	\label{ex:stress:neg:pastperf}
	\\ \armenian{Չչափեցին եւ չատեցին։}
\end{exe}

But for the present form (\ref{ex:stress:neg:present}), the prefixed negative form is restricted to relatively few contexts, such as the subjunctive present contexts. We give some examples below.  


\begin{exe}
	\ex \label{ex:stress:neg:present}
	\begin{xlist}
		\ex \gll ɡ-uz-e-m voɾ ˈ\textbf{t͡ʃə}-t͡ʃɑpʰ-e-n/ˈ\textbf{t͡ʃ}-ɑd-e-n
		\\
		{\ind}-want-{\thgloss}-1{\sg} that {\neggloss}-measure-{\thgloss}-3{\pl}/{\neggloss}-hate-{\thgloss}-3{\pl}
		\\
		\trans `I want them to not measure/hate.' \label{stress neg sub pres 1}
		\\ \armenian{Կ՚ուզեմ որ չչափեն/չատեն։}
		\ex \gll tʰoʁ ˈ\textbf{t͡ʃə}-t͡ʃɑpʰ-e-n/ˈ\textbf{t͡ʃ}-ɑd-e-n
		\\
		let {\neggloss}-measure-{\thgloss}-3{\pl}/{\neggloss}-hate-{\thgloss}-3{\pl}
		\\
		\trans `Let them   not measure/hate.' \label{stress neg sub pres 2}
		\\ \armenian{Թող    չչափեն/չատեն։}
	\end{xlist}
\end{exe}

Another common use of these negative present forms is in the future (Table \ref{tab:stress verb negation future}). The future is made up of the particle \textit{bidi} and then the verb. In standard speech, the negation prefix is   added to the verb.  In colloquial speech, the negation prefix can be added to the future particle \textit{bidi}. In both cases, the negation prefix attracts stress \citep[339]{Adjarian-1971-LiakatarPhono}. 

\begin{table}[H]
	\centering
	\caption{Negation stress in the negated future}
	\label{tab:stress verb negation future}
	
	\resizebox{\textwidth}{!}{%
		\begin{tabular}{|ll|ll | ll| l| }
			\hline          &&  \multicolumn{2}{c|}{C-initial}     &  \multicolumn{2}{c|}{V-initial} &    \\
			&   &   \multicolumn{2}{c|}{`to measure'}   &\multicolumn{2}{c|}{  `to hate'} & \\
			\hline      
			Infinitive    &  && t͡ʃɑˈ\textbf{pʰ-e-l}  & & ɑˈ\textbf{d-e-l} & \armenian{չափել, ատել}
			\\
			\hline   
			Fut 3PL &Pos.       &bidi& t͡ʃɑˈ\textbf{pʰ-e-n}  & bidi& ɑˈ\textbf{d-e-n} & \armenian{պիտի  չափեն/ատեն}
			\\
			Fut 3PL &Neg. Std. &  bidi&  ˈ\textbf{t͡ʃə}-t͡ʃɑˌpʰ-e-n   & bidi &ˈ\textbf{t͡ʃ-ɑ}ˌd-e-n & \armenian{պիտի չչափեն/չատեն}
			\\
			Fut 3PL &Neg. Coll. &  ˈ\textbf{t͡ʃə}-bidi&  t͡ʃɑˌpʰ-e-n   & ˈ\textbf{t͡ʃə}-bidi & ɑˌd-e-n & \armenian{չպիտի չափեն/ատեն}
			\\
			&   & \multicolumn{4}{l}{{({\neggloss})-{\fut}   $\sqrt{}$-{\thgloss}-3{\pl}}}  & 
			\\
			\hline      \end{tabular}
	}
	
	
\end{table}





When present negatives are used in certain types of sentences (\ref{tab:stress verb negation subj}), they can either optionally get final stress (\ref{neg pres ex cond}, \ref{neg pres ex wish}) or obligatorily get final stress (\ref{neg pres ex command}) (\citealt[67]{Avetisyan-2011-ComparativePhonoEastWest},\citealt[338]{Adjarian-1971-LiakatarPhono}).\footnote{Adjarian's grammar of old Istanbul Armenian also reports the contrast on page 148 \textcolor{red}{double check}}. For example, if the negative verb is part of a conditional sentence or a wish-making sentence  (\ref{neg pres ex cond}, \ref{neg pres ex wish}), then it is possible to have final stress. If the verb is used to form a direct command, then final stress is obligatory (\ref{neg pres ex command}). We think that in these contexts, the negation suffix is supposed to assign initial stress, but final stress is assigned because of the  special intonational structure of the sentence. 




\begin{exe}
	\ex \label{tab:stress verb negation subj}
	\begin{xlist}
		
		\ex \glll {jetʰe} {t͡ʃ-ɑˈ\textbf{d-e-n}}, {ɡə-d͡zɑχ-e-m} \\
		{jetʰe} {ˈ\textbf{t͡ʃ-ɑ}{d-e-n}}, {ɡə-d͡zɑχ-e-m}  \\
		if {\neggloss}-hate-{\thgloss}-3{\pl}, {\ind}-sell-{\thgloss}-1{\sg}
		\\
		\trans `If they don't hate it, I'll sell it.'\label{neg pres ex cond}
		\\ \armenian{Եթէ չատեն, կը ծախեմ։}
		\ex \glll {jeɾɑni} {t͡ʃ-ɑˈ\textbf{d-e-n}}  \\
		{jeɾɑni} {ˈ\textbf{t͡ʃ-ɑ}{d-e-n}}  \\
		I.wish {\neggloss}-hate-{\thgloss}-3{\pl}
		\\
		\trans `I wish they don't hate it.'\label{neg pres ex wish}
		\\ \armenian{Երանի չատեն։}
		\ex \gll  {t͡ʃ-ɑˈ\textbf{d-e-s}}, \\
		{\neggloss}-hate-{\thgloss}-2{\sg}
		\\
		\trans `Don't you dare hate it!'\label{neg pres ex command}
		\\ \armenian{Չատես։}
	\end{xlist}
	
\end{exe}



Cross-linguistically, it is common to find negation morphemes triggering special stress patterns. For example, Persian uses a stressed negation prefix  \citep{Kahnemuyipour-2003-syntacticCategoriesStress}, and Turkish \textcolor{red}{stress turkish negation}.  

These negative finite forms can undergo cliticization (\ref{ex:stress:neg clic}). Adding the clitic `also' doesn't shift stress, but the secondary stress of the final vowel feels stronger. Adding the subjunctive clitic however causes stress shift. The clitic \textit{=ne} is generally special in being able to cause stress shifts (\S\ref{section:stress:cliticc:cluster:Shift}, \S\ref{section:intonation:other:subjunctive}). 


\begin{exe}
	\ex \label{ex:stress:neg clic}
	\begin{xlist}
		\ex \gll ˈ\textbf{t͡ʃ-ɑ}d-e-ˌt͡s-i-n=ɑl 
		\\
		{\neggloss}-hate{\thgloss}-{\aorperf}-{\pst}-3{\pl}=also 
		\\ \trans `They also didn't hate it.'
		\\ \armenian{Չատեցին ալ։}
		\ex \gll   bidi ˈ\textbf{t͡ʃ-ɑɾ}tʰuˌɡ-e-n=ɑl. 
		\\
		{\fut} {\neggloss}-iron{\thgloss}-3{\pl}=also
		\\ \trans `They also won't iron it.' 
		\\ \armenian{Պիտի չարդուկեն ալ։}
		\ex \gll   jetʰe t͡ʃ-ɑˈ\textbf{d-e-n}=ne / t͡ʃ-ɑɾtʰuˈ\textbf{ɡ-e-n}=ne
		\\
		if {\neggloss}-hate{\thgloss}-3{\pl}={\sbjv} / {\neggloss}-iron{\thgloss}-3{\pl}={\sbjv}
		\\ \trans `If they  won't hate/iron it.' 
		\\ \armenian{Եթէ  չատեն/չարդուկեն նէ։}
	\end{xlist}
\end{exe}



The above is for negative finite forms. For non-finite forms, adding the negation prefix doesn't cause any irregular stress (Table \ref{tab:stress no shift negative nonfinite verb}). Such non-finite forms include infinitives and participles. Even if  non-finite form is further inflected, we still see regular stress on the rightmost non-schwa.  

\begin{table}[H]
	\centering
	\caption{No stress shift in negated non-finite forms}
	\label{tab:stress no shift negative nonfinite verb}
	\begin{tabular}{|l|lll|l| }
		\hline 
		&& C-initial   & V-initial  &   \\
		&&   `to look' & `to burn' &   \\
		\hline 
		Infinitive &    Pos.     & nɑˈ\textbf{j-i-l}& ɑjˈ\textbf{ɾ-i-l}&        \armenian{նայիլ, այրիլ}
		\\
		&    Neg.     & t͡ʃə-nɑˈ\textbf{j-i-l}&  t͡ʃ-ɑjˈ\textbf{ɾ-i-l}&        \armenian{չնայիլ, չայրիլ}
		\\
		&    Neg.+Def.     & t͡ʃə-nɑˈ\textbf{j-i}-l-ə&  t͡ʃ-ɑjˈ\textbf{ɾ-i}-l-ə&        \armenian{չնայիլը, չայրիլը}
		\\
		&& \multicolumn{2}{l|}{({\neggloss}-)$\sqrt{}$-{\thgloss}-{\infgloss}{(-{\defgloss})}} &
		\\
		\hline 
		
		Subject ptcp.  & Pos.   &  nɑˈ\textbf{j-oʁ}    &  ɑjˈ\textbf{ɾ-oʁ}  & \armenian{նայող, այրող}
		\\ 
		& Neg.&     t͡ʃə-nɑˈ\textbf{j-oʁ}    &  t͡ʃə-ɑjˈ\textbf{ɾ-oʁ}  & \armenian{չնայող, չայրող}
		\\
		&    Neg.+Def.     & t͡ʃə-nɑˈ\textbf{j-o}ʁ-ə&  t͡ʃ-ɑjˈ\textbf{ɾ-o}ʁ-ə&        \armenian{չնայողը, չայրողը}
		\\
		&& \multicolumn{2}{l|}{({\neggloss}-)$\sqrt{}$-{\sptcp}(-{\defgloss})} &
		\\
		\hline 
		Resultative  ptcp.  & Pos.   &  nɑˈ\textbf{j-ɑd͡z}    &  ɑjˈ\textbf{ɾ-ɑd͡z}  & \armenian{նայած, այրած}
		\\ 
		& Neg.&     t͡ʃə-nɑˈ\textbf{j-ɑd͡z}    &  t͡ʃə-ɑjˈ\textbf{ɾ-ɑd͡z}  & \armenian{չնայած, չայրած}
		\\
		&    Neg.+Def.     & t͡ʃə-nɑˈ\textbf{j-ɑ}d͡z-ə&  t͡ʃ-ɑjˈ\textbf{ɾ-ɑ}d͡z-ə&        \armenian{չնայածը, չայրածը}
		\\
		&& \multicolumn{2}{l|}{({\neggloss}-)$\sqrt{}$-{\rptcp}{(-{\defgloss})}} &
		\\
		\hline 
		
	\end{tabular}
\end{table}



Thus the generalization is that in Western Armenian, the negation prefix takes stress when the verb is inflected for tense-agreement. Non-finite forms like participles don't trigger any special negation stress. The attraction of stress towards the negative in Western Armenian has been reported in piecemeal fashion in few published sources. Most sources emphasize the fact that the initial schwa is stressed in the negative (\citealt[322]{Adjarian-1971-LiakatarPhono}, \citealt[cf.67][]{Avetisyan-2011-ComparativePhonoEastWest}.   But for Eastern Armenian, it seems that the negation prefix does not trigger irregular stress. \textcolor{red}{cite eastern data from grammars}


\subsection{Negative periphrastic forms}\label{section:stress:verb:negPeriph}

For verbal inflection, some paradigm cells utilize periphrasis, meaning that the `word' is made up an auxiliary that carries tense-agreement, while the verb is in a non-finite form. In the positive form, stress is on the verb, not the cliticized auxiliary. When these periphrastic forms are negated, the negation prefix is added to the auxiliary. This auxiliary then takes primary stress, while the verb takes secondary stress. We describe the following negative periphrastic tenses: negative indicative present, negative present perfect, and negative present perfect evidential. 

Consider first the negative indicative present (Table \ref{tab:stress neg connegative}). This consists of a negated auxiliary and the connegative. The negated auxiliary is made up of the negation prefix \textit{t͡ʃ-} and the auxiliary \textit{-e-}. The auxiliary carries tense-agreement. The connegative is a special form of the verb, made up of the verb stem and the suffix \textit{-ɾ}. Stress is on the auxiliary. 

\begin{table}[H]
	\centering
	\caption{Stress on the negative auxiliary in the negative indicative present}
	\label{tab:stress neg connegative}
	
	\begin{tabular}{|l|l| l|ll|}
		\hline   & Infinitive &  Neg. Indc. Pres. 3SG & \multicolumn{2}{l|}{Neg. Indc. Pres. 3PL}
		\\\hline 
		{`to measure'}   & t͡ʃɑˈ\textbf{pʰ-e-l}  & ˈ\textbf{t͡ʃ-i}   t͡ʃɑˌpʰ-e-ɾ  & ˈ\textbf{t͡ʃ-e-n}  &t͡ʃɑˌpʰ-e-ɾ  
		\\
		& \armenian{չափել}&  {\armenian{չի չափեր}}     &  \multicolumn{2}{l|}{\armenian{չեն չափեր}}
		
		
		\\
		`to hate'   & ɑˈ\textbf{d-e-l} & ˈ\textbf{t͡ʃ-ɑ}ˌd-e-ɾ & ˈ\textbf{t͡ʃ-e-n} &  ɑˌd-e-ɾ  \\
		& \armenian{ատել}& \armenian{չ՚ատեր}     & \multicolumn{2}{l|}{\armenian{չեն  ատեր}}
		\\
		&$\sqrt{}$-{\thgloss}-{\infgloss} &   {\neggloss}(-is)$\sqrt{}$-{\thgloss}-{\cn}      & {\neggloss}-is-3{\pl}  & $\sqrt{}$-{\thgloss}-{\cn} 
		
		\\\hline 
		
	\end{tabular}
	
\end{table}


For both C-initial and V-initial verbs, the paradigm of negative indicative present is largely the same. But in the 3SG (Table \ref{tab:neg pres indc stress shift regardless vowel}), the negative auxiliary is a stressed word \textit{t͡ʃ-i} before C-initial verbs, while it is a stressed prefix \textit{t͡ʃ-} before V-initial verbs. This stress behavior applies regardless of the type of root-initial vowel. We illustrate below with the 3SG forms. 


\begin{table}[H]
	\centering
	\caption{Initial stress in negative present indicative 3SG  regardless of root vowel quality}
	\label{tab:neg pres indc stress shift regardless vowel}
	\begin{tabular}{|ll ll| }
		\hline 
		/ɑ/    & ˈ\textbf{t͡ʃ-i} nɑˌj-i-ɾ &   `he doesn't look'   & \armenian{չի նայիր} \\
		& ˈ\textbf{t͡ʃ-ɑj}ˌɾ-i-ɾ &     `it doesn't burn' & \armenian{չ՚այրիր} \\
		& ˈ\textbf{t͡ʃ-ɑɾ}tʰuˌɡ-e-ɾ &     `he doesn't iron' & \armenian{չ՚արդուկեր} \\
		\hline 
		/e/    & ˈ\textbf{t͡ʃi} deˌv-e-ɾ &  `it doesn't last' & \armenian{չի տեւեր} \\
		& ˈ\textbf{t͡ʃ-e}ˌpʰ-e-ɾ & `he doesn't cook' & \armenian{չ՚եփեր}
		\\         & ˈ\textbf{t͡ʃ-e}ɾɑˌz-e-ɾ &     `he doesn't dream' & \armenian{չ՚երազեր} \\
		\hline 
		/i/         & ˈ\textbf{t͡ʃi} siˌɾ-e-ɾ &  `he doesn't   like' & \armenian{չի սիրեր} \\
		& ˈ\textbf{t͡ʃ-iʃ}ˌχ-e-ɾ & `he doesn't   rule' & \armenian{չ՚իշխեր} 
		\\
		& ˈ\textbf{t͡ʃ-i}ɾɑkʰoɾˌd͡z-e-ɾ &     `he doesn't practice' & \armenian{չ՚իրագործեր} \\
		\hline 
		/o/     & ˈ\textbf{t͡ʃi} pʰoˌχ-e-ɾ &  `he doesn't change'  & \armenian{չի փոխեր} \\
		& ˈ\textbf{t͡ʃ-okʰ}ˌn-e-ɾ &  `he doesn't help' & \armenian{չ՚օգներ} \\
		& ˈ\textbf{t͡ʃ-o}d͡zɑˌn-e-ɾ &     `he doesn't anoint' & \armenian{չ՚օծաներ} \\
		\hline 
		/u/     & ˈ\textbf{t͡ʃi} pʰuˌʒ-e-ɾ &   `he doesn't    heal' & \armenian{չի բուժեր} 
		\\
		& ˈ\textbf{t͡ʃ-u}ˌz-e-ɾ &   `he doesn't want' & \armenian{չ՚ուզեր} \\
		& ˈ\textbf{t͡ʃ-u}sɑˌn-i-ɾ &     `he doesn't learn' & \armenian{չ՚ուսանիր } 
		\\
		\hline 
		/ʏ/  & ˈ\textbf{t͡ʃi} hʏˌs-e-ɾ &  `he doesn't weave' & \armenian{չի հիւսել}  
		\\ 
		\hline 
		/ə/      & ˈ\textbf{t͡ʃi} ləˌz-e-ɾ &  `he doesn't lick' & \armenian{չի լզեր} 
		
		\\
		& ˈ\textbf{t͡ʃ-ənd}ˌɾ-e-ɾ &   `he doesn't choose' & \armenian{չ՚ընտրեր}
		\\
		& ˈ\textbf{t͡ʃ-ən}d͡zɑˌj-e-ɾ &     `he doesn't offer' & \armenian{չ՚ընծայեր} \\
		\hline 
		
	\end{tabular}
\end{table}



Another periphrastic construction is the present perfect (Table \ref{tab:stress neg present perfect}), made up of the resultative participle and the auxiliary. The participle can use either the suffix \textit{-ɑd͡z} or the evidential suffix \textit{-eɾ}. In the positive form, the auxiliary follows the verb, is unstressed, and is a clitic. In the negative form, the auxiliary takes the negation prefix, precedes the verb, and is stressed. 


\begin{table}[H]
	\centering
	\caption{Stress on the negative auxiliary in the negative present perfect}
	\label{tab:stress neg present perfect}
	\begin{tabular}{|l |ll l| }
		\hline            &  C-initial     &  V-initial &    \\
		&    `to measure'   &  `to hate' & \\
		\hline      
		Infinitive       & t͡ʃɑˈ\textbf{pʰ-e-l}  & ɑˈ\textbf{d-e-l} & \armenian{չափել, ատել}
		\\
		\hline          Pres. Perf.  3PL&    t͡ʃɑˈ\textbf{pʰ-ɑd͡z} e-n  &    ɑˈ\textbf{d-ɑd͡z} e-n  & \armenian{չափած/ատած են }
		\\
		~     with evidential form &    t͡ʃɑˈ\textbf{pʰ-eɾ} e-n  &    ɑˈ\textbf{d-eɾ} e-n  & \armenian{չափեր/ատեր են }
		\\
		& \multicolumn{2}{l}{$\sqrt{}$-{\rptcp}/{\eptcp} is-3{\pl}} & 
		\\
		\hline 
		Neg. Pres. Perf. 3PL& ˈ\textbf{t͡ʃ-e-n} t͡ʃɑˌpʰ-ɑd͡z   & ˈ\textbf{t͡ʃ-e-n} ɑˌd-ɑd͡z & \armenian{չեն չափած, չեն  ատած}
		\\
		~ with evidential form & ˈ\textbf{t͡ʃ-e-n} t͡ʃɑˌpʰ-eɾ   & ˈ\textbf{t͡ʃ-e-n} ɑˌd-eɾ & \armenian{չեն չափեր, չեն  ատեր}
		\\
		& \multicolumn{2}{l}{{\neggloss}-is-3{\pl} $\sqrt{}$-{\rptcp}/{\eptcp}} & 
		\\
		\hline      \end{tabular}
\end{table}

For the negative present and negative present perfect, the auxiliary is stressed and monosyllabic. There are also some negative periphrastic constructions where the negative auxiliary is bisyllabic because it includes the past suffix /-i-/ (Table \ref{tab:stress neg past aux}).\footnote{The negative past auxiliary is bisyllabic for all but the 3SG: [ˈ\textbf{t͡ʃ-e-ɾ}] \armenian{չէր}.  } Stress is on the first vowel of the auxiliary. Such constructions include the negative indicative past imperfective and the negative past perfect. 

\begin{table}[H]
	\centering
	\caption{Initial stress on the negative auxiliary in the negative indicative past imperfective and negative past perfect}
	\label{tab:stress neg past aux}
	\begin{tabular}{|l |ll l| }
		\hline            &  C-initial     &  V-initial &    \\
		&    `to measure'   &  `to hate' & \\
		\hline      
		Infinitive       & t͡ʃɑˈ\textbf{pʰ-e-l}  & ɑˈ\textbf{d-e-l} & \armenian{չափել, ատել}
		\\
		\hline         Neg. Indc. Past Impf.   3PL
		& ˈ\textbf{t͡ʃ-e}-ji-n t͡ʃɑˌpʰ-e-ɾ   & ˈ\textbf{t͡ʃ-e}-ji-n ɑˌd-e-ɾ & \armenian{չէին չափեր/ատեր}
		\\
		& \multicolumn{2}{l}{{\neggloss}-is-{\pst}-3{\pl} $\sqrt{}$-{\thgloss}-{\cn}} & \\
		\hline          Past Perf.  3PL&    t͡ʃɑˈ\textbf{pʰ-ɑd͡z} e-ji-n  &    ɑˈ\textbf{d-ɑd͡z} e-ji-n  & \armenian{չափած/ատած էին }
		\\
		~     with evidential form &    t͡ʃɑˈ\textbf{pʰ-eɾ} e-ji-n  &    ɑˈ\textbf{d-eɾ} e-ji-n  & \armenian{չափեր/ատեր էին }
		\\
		& \multicolumn{2}{l}{$\sqrt{}$-{\rptcp}/{\eptcp} is-{\pst}-3{\pl}} & 
		\\
		\hline 
		Neg. Pres. Perf. 3PL& ˈ\textbf{t͡ʃ-e}-ji-n t͡ʃɑˌpʰ-ɑd͡z   & ˈ\textbf{t͡ʃ-e}-ji-n ɑˌd-ɑd͡z & \armenian{չէին չափած/ատած}
		\\
		~ with evidential form & ˈ\textbf{t͡ʃ-e}-ji-n t͡ʃɑˌpʰ-eɾ   & ˈ\textbf{t͡ʃ-e}-ji-n ɑˌd-eɾ & \armenian{չէին չափեր/ատեր}
		\\
		& \multicolumn{2}{l}{{\neggloss}-is-{\pst}-3{\pl} $\sqrt{}$-{\rptcp}/{\eptcp}} & 
		\\
		\hline      \end{tabular}
\end{table}

These periphrastic forms can be cliticized (\ref{ex:stress neg aux clitic}). Some clitics and clitic clusters can be added directly to the auxiliary. We don't see stress shift. These clitics include the clitic `also' and the question particle.  

\begin{exe}
	\ex \label{ex:stress neg aux clitic}
	\begin{xlist}
		\ex \gll ˈ\textbf{t͡ʃ-e-n}=ɑl / \textbf{t͡ʃ-e}-ji-n=ɑl t͡ʃɑˌpʰ-e-ɾ 
		\\
		{\neggloss}-is-3{\pl}=also / {\neggloss}-is-{\pst}-3{\pl}=also measure-{\thgloss}-{\cn}
		\\
		\trans `They also won't/wouldn't measure it.'
		\\ \armenian{Չեն/Չէին ալ չափեր։}
		\ex \gll ˈ\textbf{t͡ʃ-e-n}=mə / \textbf{t͡ʃ-e}-ji-n=mə t͡ʃɑˌpʰ-e-ɾ 
		\\
		{\neggloss}-is-3{\pl}={\q} / {\neggloss}-is-{\pst}-3{\pl}={\q}  measure-{\thgloss}-{\cn}
		\\
		\trans `Won't/Wouldn't they measure it?'
		\\ \armenian{Չե՞ն/Չէ՞ին մը չափեր։}
		\ex \gll ˈ\textbf{t͡ʃ-e-n}=ɑl=mə / \textbf{t͡ʃ-e}-ji-n=ɑl=mə t͡ʃɑˌpʰ-e-ɾ 
		\\
		{\neggloss}-is-3{\pl}=also={\q} / {\neggloss}-is-{\pst}-3{\pl}=also={\q}  measure-{\thgloss}-{\cn}
		\\
		\trans `Won't/Wouldn't they also measure it?'
		\\ \armenian{Չե՞ն/Չէ՞ին ալ  մը չափեր։}
	\end{xlist}
\end{exe}


Some clitics can be added after the verb directly (\ref{ex:stress neg aux clitic direct}). The clitic `also' (with the meaning of `anymore'),  question particle, and progressive don't trigger stress shift. HD perceives that secondary stress is however stronger on the verb with the clitic, than without the clitic. 

\begin{exe}
	\ex \label{ex:stress neg aux clitic direct}
	\begin{xlist}
		\ex \gll ˈ\textbf{t͡ʃ-e-n}/\textbf{t͡ʃ-e}-ji-n  t͡ʃɑˌpʰ-e-ɾ=ɑl
		\\
		{\neggloss}-is-3{\pl} / {\neggloss}-is-{\pst}-3{\pl} measure-{\thgloss}-{\cn}=also
		\\
		\trans `They won't/wouldn't measure it anymore.'
		\\ \armenian{Չեն/Չէին  չափեր ալ։}
		\ex \gll ˈ\textbf{t͡ʃ-e-n}/ \textbf{t͡ʃ-e}-ji-n t͡ʃɑˌpʰ-e-ɾ=mə 
		\\
		{\neggloss}-is-3{\pl} / {\neggloss}-is-{\pst}-3{\pl}  measure-{\thgloss}-{\cn}={\q}
		\\
		\trans `Won't/Wouldn't they measure it?'
		\\ \armenian{Չե՞ն/Չէ՞ին  չափեր մը։}
		\ex \gll ˈ\textbf{t͡ʃ-e-n}  / \textbf{t͡ʃ-e}-ji-n  t͡ʃɑˌpʰ-e-ɾ=ɡoɾ
		\\
		{\neggloss}-is-3{\pl}  / {\neggloss}-is-{\pst}-3{\pl}  measure-{\thgloss}-{\cn}={\prog}
		\\
		\trans `They aren't/weren't      measuring it.'
		\\ \armenian{Չեն/Չէին  չափեր կոր։}
		
	\end{xlist}
\end{exe}

The subjunctive clitic however does trigger a strong stress on the verb, and HD perceives that this stress is as strong as the auxiliary (\ref{ex:stress neg aux clitic ne}).  

\begin{exe}
	\ex \label{ex:stress neg aux clitic ne}
	\begin{xlist}
		\ex \gll jetʰe ˈ\textbf{t͡ʃ-e-n}  / \textbf{t͡ʃ-e}-ji-n  t͡ʃɑˈ\textbf{pʰ-e-ɾ}=nə
		\\
		if {\neggloss}-is-3{\pl}  / {\neggloss}-is-{\pst}-3{\pl}  measure-{\thgloss}-{\cn}={\sbjv}
		\\
		\trans `If they won't/wouldn't      measure it.'
		\\ \armenian{Եթէ չեն/չէին  չափեր նէ։}
		\ex \gll jetʰe ˈ\textbf{t͡ʃ-e-n}  / \textbf{t͡ʃ-e}-ji-n  t͡ʃɑpʰ-e-ɾ=ˈ\textbf{ɡoɾ}=nə
		\\
		if {\neggloss}-is-3{\pl}  / {\neggloss}-is-{\pst}-3{\pl}  measure-{\thgloss}-{\cn}={\prog}={\sbjv}
		\\
		\trans `If they aren't/weren't        measuring it.'
		\\ \armenian{Եթէ չեն/չէին  չափեր կոր  նէ։}
	\end{xlist}
\end{exe}

Eastern Armenian uses periphrasis too (Table \ref{tab:periphrastic stress neg east west}). Western and Eastern however use periphrasis for different types of tenses, so we cannot directly compare the two varieties. For  example, for the negative present indicative, whereas Western uses connegatives with \textit{-ɾ}, Eastern uses imperfective converbs with \textit{-um}. Here too, we find primary stress on the monosyllabic negative auxiliary, and secondary stress on the verb's final syllable. When the negative auxiliary is bisyllabic like in the past, stress is initial in Western but final in Eastern   (\citealt[338]{Adjarian-1971-LiakatarPhono}, \citealt[77]{Margaryan-1997-ArmenianPhonology}). 


\begin{table}[H]
	\centering
	\caption{Stress in negative periphrastic forms in Western vs. Eastern for the verb `to measure' \armenian{չափել}}
	\label{tab:periphrastic stress neg east west}
	\resizebox{\textwidth}{!}{%
		\begin{tabular}{ | l| ll|ll| }
			\hline 
			& \multicolumn{2}{c|}{Western} &  \multicolumn{2}{c|}{Eastern }
			\\
			\hline       Infinitive   &&t͡ʃɑˈ\textbf{pʰ-e-l} &&t͡ʃʰɑˈ\textbf{pʰ-e-l}   \\
			\hline 
			Neg. Indc. Pres. 3PL       &ˈ\textbf{t͡ʃ-e-n}& t͡ʃɑˌpʰ-e-ɾ & ˈ\textbf{t͡ʃʰ-e-n} &t͡ʃʰɑˌpʰ-um   \\ 
			& {\neggloss}-is-3{\pl}& $\sqrt{}$-{\thgloss}-{\cn}& {\neggloss}-is-3{\pl} &$\sqrt{}$-{\impfcvb}
			\\
			&   \multicolumn{2}{l|}{\armenian{չեն չափեր}}&  \multicolumn{2}{l|}{ \armenian{չեն  չափում}}
			\\
			\hline 
			Neg. Indc. Past Impf. 3PL       &ˈ\textbf{t͡ʃ-e}-ji-n& t͡ʃɑˌpʰ-e-ɾ & t͡ʃʰ-e-ˈ\textbf{ji-n} &t͡ʃʰɑˌpʰ-um   \\ 
			& {\neggloss}-is-{\pst}-3{\pl}& $\sqrt{}$-{\thgloss}-{\cn}& {\neggloss}-is-{\pst}-3{\pl} &$\sqrt{}$-{\impfcvb}
			\\
			&   \multicolumn{2}{l|}{\armenian{չէին չափեր}}&  \multicolumn{2}{l|}{ \armenian{չէին  չափում}}
			\\
			\hline \end{tabular}
}\end{table}



\subsection{Prohibitive}\label{section:stress:verb:proh}

Prohibitives or negative imperatives are made up of two elements: the particle \textit{mi} and the verb (Table \ref{tab:stress verb proh}).  The particle \textit{mi} takes primary stress. The verb is inflected for either 2SG or 2PL, and it carries secondary stress. Similar facts are reported for Eastern Armenian (\citealt[20]{Abeghyan-1933-Meter}, \citealt[77]{Margaryan-1997-ArmenianPhonology}). 

\begin{table}[H]
	\centering
	\caption{Negation stress in the prohibitive (negative imperative)} 
	\label{tab:stress verb proh}
	
	\begin{tabular}{|l|ll | ll| l| }
		\hline          &   \multicolumn{2}{c|}{C-initial}     &  \multicolumn{2}{c|}{V-initial} &    \\
		&       \multicolumn{2}{c|}{`to measure'}   &\multicolumn{2}{c|}{  `to hate'} & \\
		\hline      
		Infinitive       && t͡ʃɑˈ\textbf{pʰ-e-l}  & & ɑˈ\textbf{d-e-l} & \armenian{չափել, ատել}
		\\
		\hline   
		Prohibitive 2SG     &ˈ\textbf{mi}& t͡ʃɑˌpʰ-e-ɾ  & ˈ\textbf{mi}& ɑˌ{d-e-ɾ} & \armenian{մի  չափեր/ատեր}
		\\
		Prohibitive 2PL     &ˈ\textbf{mi}& t͡ʃɑˌpʰ-e-kʰ  & ˈ\textbf{mi}& ɑˌ{d-e-kʰ} & \armenian{մի  չափէք/ատէք}
		\\
		\hline      \end{tabular}
	
	
	
\end{table}

As for cliticization (\ref{ex:stress verb proh cl}), no clitics can be added between the prohibitive particle and the verb. After the verb, some clitics like \textit{=ɑl} `also' can be added with the meaning of `anymore'. There is no stress shift, but the secondary stress on the verb gets stronger. 

\begin{exe}
	\ex \gll ˈ\textbf{mi} t͡ʃɑˌpʰ-e-ɾ=ɑl
	\\
	{\proh} measure-{\thgloss}-2{\sg}=also
	\\
	\trans `Don't measure anymore.'
	\label{ex:stress verb proh cl}
	\\ \armenian{Մի չափեր ալ։}
\end{exe}

\subsection{Irregular theme vowel stress in past imperfectives} \label{section:stress:verb:pastImpf}
In the past imperfective inflection, irregular stress is on the non-final theme vowel of the verb. We go over the general behavior of irregular stress in \S\ref{section:stress:verb:pastImpf:general}. In \S\ref{section:stress:verb:pastImpf:negation}, we document variation and complications when the past imperfective is in the negative. Here, the negation and theme vowel morphemes compete for irregular stress.  Complications also arise in cliticization (\S\ref{section:stress:verb:pastImpf:clitic}). 

\subsubsection{General rule of irregular stress in the past imperfective}\label{section:stress:verb:pastImpf:general}
In the present and past perfective, stress is regularly on the rightmost non-schwa vowel. This is either the theme vowel or the past suffix /-i-/. But in the past imperfective (Table \ref{tab:past imperf theme vowel stress}), stress is on irregularly on the theme vowel even though this vowel is not final. We show zero morphs for illustration.  

\begin{table}[H]
	\centering
	\caption{Irregular stress on theme vowels in the past imperfective for the verb [ɑd-e-l] `to hate' \armenian{ատել}}
	\label{tab:past imperf theme vowel stress}
	\begin{tabular}{|l|lll|}
		\hline 
		& Present & Past Perf. & Past Impf.   \\
		\hline 
		1SG   & ɑˈ\textbf{d-e-m}& ɑd-e-ˈ\textbf{t͡s-i-$\emptyset$}& ɑˈ\textbf{d-e}-ji-$\emptyset$
		\\
		2SG   & ɑˈ\textbf{d-e-s}& ɑd-e-ˈ\textbf{t͡s-i-ɾ}& ɑˈ\textbf{d-e}-ji-ɾ
		\\
		3SG     & ɑˈ\textbf{d-e-$\emptyset$}& ɑˈ\textbf{d-e-t͡s}& ɑˈ\textbf{d-e-$\emptyset$-ɾ}
		\\
		1PL   & ɑˈ\textbf{d-e-ŋkʰ}& ɑd-e-ˈ\textbf{t͡s-i-ŋkʰ}& ɑˈ\textbf{d-e}-ji-ŋkʰ
		\\
		2PL   & ɑˈ\textbf{d-e-kʰ}& ɑd-e-ˈ\textbf{t͡s-i-kʰ}& ɑˈ\textbf{d-e}-ji-kʰ
		\\
		3PL   & ɑˈ\textbf{d-e-n}& ɑd-e-ˈ\textbf{t͡s-i-n}& ɑˈ\textbf{d-e}-ji-n
		\\
		\hline 
		& $\sqrt{}$-{\thgloss}-{\agr}& $\sqrt{}$-{\thgloss}-{\aorperf}-{\pst}-{\agr}& $\sqrt{}$-{\thgloss}-{\pst}-{\agr}
		\\ \hline 
	\end{tabular}
	
\end{table}

For the past imperfective, stress is irregularly on the theme vowel. The past suffix \textit{-i-} is present after the theme vowel for all but the 3SG. The theme vowel is the final vowel for the 3SG \textit{ɑˈd-e-ɾ}, but is penultimate for all other person-numbers like 3PL \textit{ɑˈd-e-ji-n}.  

This rule applies for both the /e/ and /ɑ/ theme vowels (Table \ref{tab:past imperf stress distribution}).\footnote{ The theme vowel /i/ cannot be used in the past imperfective; see \textcolor{red}{cite chapter on i theme vowel change}.} The root or stem can be monosyllabic or polysyllabic. 

\begin{table}[H]
	\centering
	\caption{Irregular theme vowel stress in past imperfective is agnostic to word size and vowel quality}
	\label{tab:past imperf stress distribution}
	\begin{tabular}{|l|lll| ll|}
		\hline 
		&     Infinitive   &      Past Impf. 3SG &     Past Impf. 3PL & 
		\\
		/e/ & siˈ\textbf{ɾ-e-l} & siˈ\textbf{ɾ-e-$\emptyset$-ɾ} & siˈ\textbf{ɾ-e}-ji-n  &  `to like' &  \armenian{սիրել}
		\\
		& ɑɾtʰuˈ\textbf{ɡ-e-l} & ɑɾtʰuˈ\textbf{ɡ-e-$\emptyset$-ɾ} &  ɑɾtʰuˈ\textbf{ɡ-e}-ji-n  &  `to iron' &  \armenian{արդուկել}
		\\
		& ɑd͡ʒɑbɑˈ\textbf{ɾ-e-l} & ɑd͡ʒɑbɑˈ\textbf{ɾ-e-$\emptyset$-ɾ} &ɑd͡ʒɑbɑˈ\textbf{ɾ-e}-ji-n
		& `to hurry' & \armenian{աճապարել}
		\\
		& ɑvedɑɾɑˈ\textbf{n-e-l} & ɑvedɑɾɑˈ\textbf{n-e-$\emptyset$-ɾ}& ɑvedɑɾɑˈ\textbf{n-e}-ji-n & `to evangelize' & \armenian{աւետարանել}
		\\
		\hline 
		/ɑ/ & ɡɑɾˈ\textbf{tʰ-ɑ-l} & ɡɑɾˈ\textbf{tʰ-ɑ-$\emptyset$-ɾ} & ɡɑɾˈ\textbf{tʰ-ɑ}-ji-n  &  `to read' &  \armenian{կարդալ}
		\\
		& ɑpsoˈ\textbf{s-ɑ-l} & ɑpsoˈ\textbf{s-ɑ-$\emptyset$-ɾ} & ɑpsoˈ\textbf{s-ɑ}-ji-n  &  `to pity' &  \armenian{ափսոսալ}
		\\
		& ɑɾɑkʰɑ\textbf{n-ɑ-l} & ɑɾɑkʰɑ\textbf{n-ɑ-$\emptyset$-ɾ} & ɑɾɑkʰɑ\textbf{n-ɑ}-ji-n  &  `to get fast' &  \armenian{արագանալ}
		\\
		& ɑɾɑvodɑ\textbf{n-ɑ-l} & ɑɾɑvodɑ\textbf{n-ɑ-$\emptyset$-ɾ} & ɑɾɑvodɑ\textbf{n-ɑ}-ji-n  &  `to dawn' &  \armenian{առաւօտանալ}
		\\ \hline
	\end{tabular}
\end{table}

This rule applies for past imperfective as spoken for the modern Lebanese community, which HD is a member of. We've asked other Armenians and this rule applies for the Western Armenian communities in Syria (HS), Turkey (TT), and contemporary US (SC). This rule however doesn't apply in France (AD). Furthermore, earlier sources of Western Armenian report regular final stress for these forms \textcolor{red}{add survey of bib for imperfective}. 

Diachronically, it is likely that the past imperfectives once had final regular stress for all Western Armenian communities. Some communities then shifted to apply irregular stress on the theme vowel for some unknown reason. Evidence for this is that in Eastern Armenian, past imperfectives do not have irregular stress, but have regular final stress  \citep[77]{Margaryan-1997-ArmenianPhonology}, e.g., Western [ɑˈ\textbf{d-e}-ji-n]  but Eastern [ɑ{d-e}-ˈ\textbf{ji-n}] for past impf. 3PL of `to hate'.  

\subsubsection{Interaction of past imperfective stress and negation stress}\label{section:stress:verb:pastImpf:negation}
The past imperfective can be used in two moods: indicative and subjunctive. The indicative takes the prefix \textit{ɡ(ə)-}. The subjunctive lacks this prefix. In both forms, stress is on the theme vowel.  The indicative is negated periphrastically (discussed in \S\ref{section:stress:verb:negPeriph}), while the subjunctive is negated with the prefix \textit{t͡ʃ(ə)-}. In this negative form (Table \ref{tab:impf stress forms neg indc subj}), HD feels that stress can variably jump between the negation prefix and the theme vowel.  

\begin{table}[H]
	\centering
	\caption{Irregular theme vowel stress in the past imperfective regardless of mood and polarity}
	\label{tab:impf stress forms neg indc subj} 
	\begin{tabular}{|l|ll|l|}
		\hline      &C-initial   & V-initial &  \\
		& `to measure' &  `to hate' & \\
		Infinitive     & t͡ʃɑˈ\textbf{pʰ-e-l}  & ɑˈ\textbf{d-e-l}& \armenian{չափել, ատել}
		\\
		\hline 
		Subj.  Past. Impf. 3SG   & t͡ʃɑˈ\textbf{pʰ-e-$\emptyset$-ɾ}  & ɑˈ\textbf{d-e-$\emptyset$-ɾ} & \armenian{չաթէր, ատէր}
		\\
		Ind.  Past. Impf. 3SG   & ɡə-t͡ʃɑˈ\textbf{pʰ-e-$\emptyset$-ɾ}  & ɡ-ɑˈ\textbf{d-e-$\emptyset$-ɾ}  & \armenian{կը չաթէր, կ՚ատէր}
		\\
		Neg. Subj.   Past. Impf. 3SG   & t͡ʃə-t͡ʃɑˈ\textbf{pʰ-e-$\emptyset$-ɾ}  & t͡ʃ-ɑˈ\textbf{d-e-$\emptyset$-ɾ}  & \armenian{չչաթէր, չատէր}
		\\
		& ˈ\textbf{t͡ʃə}-t͡ʃɑpʰ-e-$\emptyset$-ɾ   & ˈ\textbf{t͡ʃ-ɑ}{d-e-$\emptyset$-ɾ}  & 
		\\
		& \multicolumn{2}{l|}{{\ind}/{\neggloss}-$\sqrt{}$-{\thgloss}-{\pst}-3{\sg}} &  
		\\
		\hline 
		Subj.  Past. Impf. 3PL   & t͡ʃɑˈ\textbf{pʰ-e}-ji-n  & ɑˈ\textbf{d-e}-ji-n 
		& \armenian{չաթէին, ատէին}
		\\
		Ind.  Past. Impf. 3PL   & ɡə-t͡ʃɑˈ\textbf{pʰ-e}-ji-n  & ɡ-ɑˈ\textbf{d-e}-ji-n & \armenian{կը չաթէին,  կ՚ատէին}
		\\
		Neg. Subj.  Past  Impf. 3PL   & t͡ʃə-t͡ʃɑˈ\textbf{pʰ-e}-ji-n  & t͡ʃ-ɑˈ\textbf{d-e}-ji-n & \armenian{չչաթէին,  չատէին}
		\\
		& ˈ\textbf{t͡ʃə}-t͡ʃɑpʰ-e-ji-n  & ˈ\textbf{t͡ʃ-ɑ}{d-e}-ji-n & 
		
		\\ 
		& \multicolumn{2}{l|}{{\ind}/{\neggloss}-$\sqrt{}$-{\thgloss}-{\pst}-3{\sg}}& 
		\\
		\hline  
	\end{tabular}
\end{table}

As discussed in \S\ref{section:stress:verb:negFinite}, the negation prefix tends to attract stress is finite verbal forms such as the negative present or negative past perfective. But in the past imperfective, either the negation prefix and theme vowel can get stress. 


To illustrate the stress variability (\ref{ex:impf stress forms neg indc subj ex}), consider the two sentences below. HD feels that if the negative subjunctive form is used in a future context, then stress is preferably next to the negation prefix. In contrast, if the verb is part of a conditional sentence, then stress is on the theme vowel. 

\begin{exe}
	\ex \label{ex:impf stress forms neg indc subj ex} 
	\begin{xlist}
		
		\ex \gll bidi ˈ\textbf{t͡ʃ-ɑ}d-e-$\emptyset$-ɾ  / ˈ\textbf{t͡ʃ-ɑ}{d-e}-ji-n 
		\\
		{\fut} {\neggloss}-hate-{\thgloss}-{\pst}-3{\sg} /  {\neggloss}-hate-{\thgloss}-{\pst}-3{\pl}
		\\
		\trans `He was/They were going to not hate (it).' 
		\\ \armenian{Պիտի չատէր/չատէին։}
		\ex \gll jetʰe t͡ʃ-ɑˈ\textbf{d-e-$\emptyset$-ɾ} / t͡ʃ-ɑˈ\textbf{d-e}-ji-n 
		\\
		if {\neggloss}-hate-{\thgloss}-{\pst}-3{\sg}  / {\neggloss}-hate-{\thgloss}-{\pst}-3{\pl} 
		\\
		\trans `If he/they didn't hate (it).' 
		\\ \armenian{Եթէ չատէր/չատէին։}
		
	\end{xlist}
\end{exe}

It is possible that speakers choose which vowel to stress based on choosing which of the two meanings (negative vs. subjunctive) they want to stress.  This type of variability seems appropriate for a larger-scale experiment. 

For V-initial verbs like `to hate', the negative subj. past impf. 3SG  [t͡ʃ-ɑˈ\textbf{d-e-$\emptyset$-ɾ}]  is segmentally homophonous with the negative indicative homophonous with the negative ind. present 3SG [ˈ\textbf{t͡ʃ-ɑ}{d-e-ɾ}]. The two forms differ however in stress (Table \ref{tab:neg subj indc difference})  \citep[cf.][67]{Avetisyan-2011-ComparativePhonoEastWest}. The subjunctive imperfective can variably place  stress on the theme vowel, while the indicative negative  places stress on the first root vowel. 

\begin{table}[H]
	\centering
	\caption{Segmental homophony but prosodic difference for negative subjunctive and negative indicative}
	\label{tab:neg subj indc difference}
	\begin{tabular}{|lll| }
		\hline & `to hate' & \\ \hline
		Infinitive         & ɑˈ\textbf{d-e-l} & $\sqrt{}$-{\thgloss}-{\infgloss}
		\\
		& <adel> & \armenian{ատել} 
		\\ \hline
		Neg. Indc. Pres. 3SG & ˈ\textbf{t͡ʃ-ɑ}d-e-ɾ& {\neggloss}-$\sqrt{}$-{\thgloss}-{\cn} 
		\\ 
		&<t͡ʃ''ader>  & \armenian{չ՚ատեր}
		\\ \hline
		Neg. Subj. Past Impf.   3SG & ˈ\textbf{t͡ʃ-ɑ}d-e-$\emptyset$-ɾ& {\neggloss}-$\sqrt{}$-{\thgloss}-{\pst}-3{\sg} 
		\\
		& t͡ʃ-ˈ\textbf{d-e-ɾ} &
		\\ 
		&<t͡ʃ'ader>  & \armenian{չատեր}
		\\ \hline
	\end{tabular}
\end{table}

These forms are likewise distinguished orthographically. In the indicative 3SG, the negation prefix precedes an apostrophe. In the negative subjunctive, there is no apostrophe. 



The indicative form was discussed in \S\ref{section:stress:verb:negFinite}.  The two types of verbs can be elicited in separate types of sentences (\ref{ex:neg subj indc difference sentence}). The indicative version is quite easy to elicit; essentially any sentence like `He doesn't X'. The subjunctive is more subtle to elicit. We illustrate below For illustration, we also provide forms with the 3PL where there is no homophony. 

\begin{exe}
	\ex
	\label{ex:neg subj indc difference sentence}
	\begin{xlist}
		\ex  Negative indicative
		\begin{xlist}
			\ex 
			\gll mezi ˈ\textbf{t͡ʃ-ɑ}{d-e-ɾ} /  ˈ\textbf{t͡ʃ-e-n} ɑ{d-e-ɾ}\\
			us.{\dat}  {\neggloss}-hate-{\thgloss}-{\cn} /  {\neggloss}-is-3{\pl} hate-{\thgloss}-{\cn}
			\\
			\trans `He/They don't hate us.' 
			\\ 
			\armenian{Մեզի չ՚ատեր/չեն ատեր։}
		\end{xlist}
		\ex Contexts for negative subjunctive
		\begin{xlist}
			\ex \glll ɡ-uz-e-ji voɾ t͡ʃ-ɑˈ\textbf{d-e-$\emptyset$-ɾ} / t͡ʃ-ɑˈ\textbf{d-e}-ji-n 
			\\
			ɡ-uz-e-ji voɾ ˈ\textbf{t͡ʃ-ɑ}d-e-$\emptyset$-ɾ  / ˈ\textbf{t͡ʃ-ɑ}{d-e}-ji-n 
			\\
			{\ind}-want-{\thgloss}-{\pst} that {\neggloss}-hate-{\thgloss}-{\pst}-3{\sg}  / {\neggloss}-hate-{\thgloss}-{\pst}-3{\pl} 
			\\
			\trans `I wanted him/they to not hate (it).' 
			\\ \armenian{Կ՚ուզէի որ չատէր/չատէին։}
			\ex \glll tʰoʁ t͡ʃ-ɑˈ\textbf{d-e-$\emptyset$-ɾ} / t͡ʃ-ɑˈ\textbf{d-e}-ji-n 
			\\
			tʰoʁ ˈ\textbf{t͡ʃ-ɑ}{d-e-ɾ}  / ˈ\textbf{t͡ʃ-ɑ}{d-e}-ji-n 
			\\
			let {\neggloss}-hate-{\thgloss}-{\pst}-3{\sg}  / {\neggloss}-hate-{\thgloss}-{\pst}-3{\pl}  
			\\
			\trans `Let him/them not hate (it).' 
			\\ \armenian{Թող չատէր/չատէին։}
		\end{xlist}
	\end{xlist}
	
\end{exe}

\subsubsection{Theme vowel stress and cliticization}\label{section:stress:verb:pastImpf:clitic}

The past imperfective has a rule of assigning irregular stress to the theme vowel. When these imperfective forms are cliticized, we can see subtle variations in stress placement. 

Various clitics can be added after the past imperfective form (\ref{ex:past impf cl}). These clitics don't trigger stress shift, meaning that stress stays on the irregular theme vowel. As surveyed in \S\ref{section:stress:cliticc}, these clitics include the word `also' \textit{=ɑl}, the conjunction \textit{=u}, the question particle \textit{=mə},  the  progressive    \textit{=ɡoɾ}, and  the subjunctive clitic \textit{=ne}
. 

\begin{exe}
	\ex \label{ex:past impf cl}
	\begin{xlist}
		\ex \gll ɡ-ɑˈ\textbf{d-e-$\emptyset$-ɾ}=ɑl  / ɡ-ɑˈ\textbf{d-e}-ji-n=ɑl 
		\\
		{\ind}-hate-{\thgloss}-{\pst}-3{\sg}=also / {\ind}-hate-{\thgloss}-{\pst}-3{\pl}=also
		\\ \trans `He was/They would also hate it.' 
		\\ \armenian{Կ՚ատէր ալ։ Կ՚ատէին ալ։}
		\ex \gll ɡ-ɑˈ\textbf{d-e-$\emptyset$-ɾ}=u  ɡu-ˈ\textbf{l-ɑ-$\emptyset$-ɾ} 
		\\
		{\ind}-hate-{\thgloss}-{\pst}-3{\sg}=and {\ind}-cry-{\thgloss}-{\pst}-3{\sg}  
		\\ \trans `He would hate it and cry.'  
		\\ \armenian{Կ՚ատէր ու կու լար։}
		\ex \gll ɡ-ɑˈ\textbf{d-e-$\emptyset$-ɾ}=mə  / ɡ-ɑˈ\textbf{d-e}-ji-n=mə 
		\\
		{\ind}-hate-{\thgloss}-{\pst}-3{\sg}={\q} / {\ind}-hate-{\thgloss}-{\pst}-3{\pl}={\q}
		\\ \trans `Would he/they hate it?'
		\\ \armenian{Կ՚ատէ՞ր մը։ Կ՚ատէ՞ին մը։}
		\ex \gll ɡ-ɑˈ\textbf{d-e-$\emptyset$-ɾ}=ɡoɾ  / ɡ-ɑˈ\textbf{d-e}-ji-n=ɡoɾ 
		\\
		{\ind}-hate-{\thgloss}-{\pst}-3{\sg}={\prog} / {\ind}-hate-{\thgloss}-{\pst}-3{\pl}={\prog}
		\\ \trans `He was/They were hating it.' 
		\\ \armenian{Կ՚ատէր կոր։ Կ՚ատէին կոր։}
		\ex \gll jetʰe ɑˈ\textbf{d-e-$\emptyset$-ɾ}=ne  /  ɑˈ\textbf{d-e}-ji-n=ne 
		\\
		if hate-{\thgloss}-{\pst}-3{\sg}={\sbjv} /  hate-{\thgloss}-{\pst}-3{\pl}={\sbjv}
		\\ \trans `If he/they would hate it.' 
		\\ \armenian{Եթէ  ատէր նէ։ Եթէ  ատէին նէ։}
	\end{xlist}
\end{exe}


\textcolor{red}{double check if khanjian had ne trigger stress shift}

With two clitic clusters (\ref{ex:past impf cluster}), we don't see stress shift for `also' + Q, progressive + `also', progressive + Q, and subjunctive + `also'. The lack of stress shift is found both for the past imperfective and for elsewhere in the language \S\ref{section:stress:cliticc:cluster}


\begin{exe}
	\ex \label{ex:past impf cluster}
	\begin{xlist}
		\ex \gll ɡ-ɑˈ\textbf{d-e-$\emptyset$-ɾ}=ɑl=mə / ɡ-ɑˈ\textbf{d-e}-ji-n=ɑl=mə 
		\\
		{\ind}-hate-{\thgloss}-{\pst}-3{\sg}=also={\q} / {\ind}-hate-{\thgloss}-{\pst}-3{\pl}=also={\q}
		\\ \trans `Would he/they also hate it?' 
		\\ \armenian{Կ՚ատէ՞ր ալ մը։ Կ՚ատէ՞ին ալ մը։}
		\ex \gll ɡ-ɑˈ\textbf{d-e-$\emptyset$-ɾ}=ɡoɾ=ɑl  / ɡ-ɑˈ\textbf{d-e}-ji-n=ɡoɾ=ɑl
		\\
		{\ind}-hate-{\thgloss}-{\pst}-3{\sg}={\prog}=also / {\ind}-hate-{\thgloss}-{\pst}-3{\pl}={\prog}=also
		\\ \trans `He was/They were also hating it.' 
		\\ \armenian{Կ՚ատէր կոր ալ։ Կ՚ատէին կոր ալ։}
		\ex \gll ɡ-ɑˈ\textbf{d-e-$\emptyset$-ɾ}=ɡoɾ=mə  / ɡ-ɑˈ\textbf{d-e}-ji-n=ɡoɾ=mə
		\\
		{\ind}-hate-{\thgloss}-{\pst}-3{\sg}={\prog}={\q} / {\ind}-hate-{\thgloss}-{\pst}-3{\pl}={\prog}={\q}
		\\ \trans `Was he/Were they also hating it?'
		\\ \armenian{Կ՚ատէ՞ր կոր մը։ Կ՚ատէ՞ին կոր մը։}
		\ex \gll jetʰe ɑˈ\textbf{d-e-$\emptyset$-ɾ}=ne=ɑl  /  ɑˈ\textbf{d-e}-ji-n=ne=ɑl
		\\
		if hate-{\thgloss}-{\pst}-3{\sg}={\sbjv}=also /  hate-{\thgloss}-{\pst}-3{\pl}={\sbjv}=also
		\\ \trans `If he/they would also hate it.' 
		\\ \armenian{Եթէ  ատէր նէ ալ։ Եթէ  ատէին նէ ալ։}
	\end{xlist}
\end{exe}

For the progressive + subjunctive cluster (\ref{ex:past impf gor}), stress shifts to the progressive. Secondary stress is perceived on the theme vowel. The subjunctive particle is special in being able to shift stress. Note the presence of the indicative prefix, whose meaning is overridden by the subjunctive clitic. 

\begin{exe}
	\ex \gll jetʰe ɡ-ɑd-e-$\emptyset$-ɾ=ˈ\textbf{ɡoɾ}=ne  /  ɡ-ɑd-e-ji-n=ˈ\textbf{ɡoɾ}=ne
	\\
	if {\ind}-hate-{\thgloss}-{\pst}-3{\sg}={\prog}={\sbjv}   /  {\ind}-hate-{\thgloss}-{\pst}-3{\pl}={\prog}={\sbjv} 
	\\ \trans `If he/they were hating it.' 
	\label{ex:past impf gor}
	\\ \armenian{Եթէ  կ՚ատէր/կ՚ատէին կոր նէ։}
\end{exe}





















\section{Words with irregular stress}\label{section:stress:irregularWord}
The previous sections looked at words which had irregular stress because of specific morphological factors. Specifically, these words had irregular stress because they contained a special type of derivational suffix, inflectional feature, or clitic. Their irregularity was thus systematic and consistent. 

Alongside the above systematic types of irregular stress, there's also a handful of words which have irregular stress for purely arbitrary reasons, such as the monomorphemic word [ˈ\textbf{mɑ}nɑvɑntʰ] `especially'.  These are irregular words include interrogative pronouns, common particles, and some common adverbs. There is no synchronic reason or rule behind why these words have irregular stress. Speakers simply have to memorize that these words have irregular stress. 

We provide a list of such words in Table \textcolor{red}{fill}. These words are taken from diverse sources. 

\textcolor{red}{get words from sources:
	\begin{itemize}
		\item from wiktionary, just use wikipron to find stress and then check the sources
		\item Gharagulyan 1974:221: irregular stressed words list
		\item Irregular stress-ed function words (Vaux 1998:133,Syoukyasyan 2004:29, Margaryan 1997:75) with some variation (Margaryan 1997:76) 
		\item \textbf{Emphatic adverbs}: 
		Certain words always get or are adjacent to sentential stress \citep[23]{Abeghyan-1933-Meter}. They sound as if they get narrow focus.
		\item  \armenian{\textcolor{red}{(Abeghyan 1933:23 - Prosody)
				Միայն     մինչեւ անգամ անգամ     նոյնիսկ
				Իսկ մանավանդ}}
\end{itemize}}

For some of these words, there is likely a diachronic reason as to why the word has irregular stress. For example, some of the words above are morphologically compounds: [ˈ\textbf{nujn}-bes] `in the same way'. In an earlier stage of the language, they might've been two separate words that were often said together as part of some phrase or collocation: [ˈ\textbf{nujn} +  bes] `same + way'. In their modern form, they act as one word but the irregular stress is a residue of this older syntactic structure. 

Note also that there are some words that are recent borrowings from Romance, English, or Russian. These words   generally keep their original stress location \citep[223]{Gharagulyan-1974-BookArmenianOrthoepy}. \textcolor{red}{get examples from book}

\section{Secondary stress}\label{section:stress:secondary}
Whereas Armenian has relatively clear rules for primary stress, secondary stress is quite unclear. We catalog the following types of secondary stress: alleged initial secondary stress, prefix-based secondary stress, length-induced secondary stress, and demoted secondary stress. 

Initial secondary stress is often reported in grammars, but there's little to no phonological or phonetic evidence for it; it is likewise difficult to perceive at all. Prefix-based secondary stress is when stress is on special prefixes. Length-induced secondary stress is perceived prominences in substantially long works, such as long compounds.  Demoted secondary stress is when secondary stress is on the final syllable or some other syllable which \textit{should} have gotten primary stress, but then some other morpheme (like negation) got primary stress. 

\subsection{Alleged initial secondary stress}\label{section:stress:secondary:allegedInitial}
Most published grammars report that words have initial secondary stress on the first syllable (Table \ref{tab:allege initial secondary stress}).  This creates a hammock pattern because primary stress is on the final syllable \citep{Gordon-2002-FactorialTypologyStress}.  However, some of these grammars also report that initial secondary stress is very difficult to perceive   \citep[20]{Abeghyan-1933-Meter}. 



\begin{table}[H]
	\centering
	\caption{Alleged initial secondary stress}
	\label{tab:allege initial secondary stress}
	\begin{tabular}{|lll|}
		\hline    
		ˌkʰɑˈ\textbf{ʁɑk}ʰ &  `city' & \armenian{քաղաք}
		\\
		ˌkʰɑhɑˈ\textbf{nɑ} & `priest' & \armenian{քահանայ}
		\\
		ˌbɑdɑsˈ\textbf{χɑn} & `answer' & \armenian{պատասխան}
		\\
		ˌbɑdɑsχɑn-ɑˈ\textbf{voɾ} & `responsible' & \armenian{պատասխանաւոր}
		\\
		ˌbɑdɑsχɑn-ɑvoɾ-ˈ\textbf{neɾ} & `responsible-{\pl}' & \armenian{պատասխանաւորներ}
		\\
		ˌbɑdɑsχɑn-ɑvoɾ-neˈ\textbf{ɾ-ov} & `responsible-{\pl}-{\ins}' & \armenian{պատասխանաւորներով}
		\\ \hline
	\end{tabular}
\end{table}


As a native-speaking phonologist of Armenian, HD  could never hear initial secondary stress on non-prefixed words. HD suspects that reports of initial secondary stress are actually just epiphenomenal. Cross-linguistically, the  word-initial syllable is important for psychological processing of words \citep{BeckerVeinsLevine-2012-AsymmetriesInitialSyllableGeneralization}. The initial syllable can likewise affect the phonological form of later syllables, such as in terms of vowel harmony \citep{Beckman-1997-PositionalFaithfulness}. For Armenian, we suspect that various grammarians impressionistically argue that there is initial secondary stress because the first syllable is psycholinguistically important. 

For un-prefixed words, there is evidence that initial secondary stress is just an illusion, perhaps an illusion from phrasal prosody.  The evidence is that there is widespread disagreement among grammarians on diverse issues \citep{Fairbanks-1948-PhonologyMorphoWestern,Johnson-1954-EastArmGrammar,Gharagulyan-1974-BookArmenianOrthoepy,Margaryan-1997-ArmenianPhonology,Soukyasyan-2004-ArmenianPhonology,Tokhmakhyan-1971-ArmenianWordStress,Tokhmakhyan-1975-StressEssence,Tokhmakhyan-1983-ArmenianPhono}. For example, there's disagreement on whether  i)  secondary stress can only appear in large words, ii) it can appear on the second syllable instead of the first, and iii) it can appear on schwas.    However, these reports are largely impressionistic, with little known acoustic support. 



In terms of word size and schwas (Table \ref{tab:fairbanks initia lsecondary stress}), \citet[12]{Fairbanks-1948-PhonologyMorphoWestern} reports secondary stress only appear on the first syllable in 3-syllable and 5-syllable words, even if the syllable has a schwa. For 4-syllable words, he documents secondary stress on the first syllable if it has a full vowel, otherwise on the second syllable if the first has a schwa. \citet{Kassabian-1971-ContrastStressWesttArmEnglish} agrees with his reports. We convert his data to IPA and to our   segmentation for inflection.\footnote{Fairbanks translated the word [{kʰəɾɑse{ʁɑn}}] as `typewriter', but a more accurate translation is `writing desk'. } 

\begin{table}[H]
	\centering
	\caption{Alleged secondary stress in \citet{Fairbanks-1948-PhonologyMorphoWestern}'s Western data}
	\label{tab:fairbanks initia lsecondary stress}
	\begin{tabular}{|ll | ll|}
		\hline 
		\multicolumn{2}{|l|}{3 and 5 syllable words} &\multicolumn{2}{l|}{4 syllable words} \\ \hline
		{{ˌəndɑˈ\textbf{nikʰ}}} &        `family' 
		&{{ənˌdɑniˈ\textbf{kʰ-i-n}}}   & `family-{\gen}-{\defgloss}'  
		\\
		& \armenian{ընտանիք}&& \armenian{ընտանիքին}
		\\
		{{ˌdesɑˈ\textbf{ɾɑn}}} &`view' 
		& {{ˌsɑɡɑɾɡuˈ\textbf{tʰjʏn}}}   &  `bargaining'
		\\ 
		& \armenian{տեսարան}&& \armenian{սակարկութիւն}  
		\\
		{{ˌɑɾeˈ\textbf{vod}}} &`sunny' 
		& {{ˌjeɡeʁeˈ\textbf{t͡si}}}         & `church
		\\
		& \armenian{արեւոտ}'& & \armenian{եկեղեցի}
		\\
		{{ˌhedɑkʰəɾkʰɾɑˈ\textbf{ɡɑn}}} &`interesting' 
		& {kʰəˌɾɑseˈ\textbf{ʁɑn}}  &`writing desk'
		\\
		& \armenian{հետաքրքրական}  & & \armenian{գրասեղան}
		\\
		{{ˌmed͡zɑmɑsnuˈ\textbf{tʰjʏn}}} &`majority'
		& {{χəˌmoɾeˈ\textbf{ʁen}}}   &  `pastry' \\
		&   \armenian{մեծամասնութիւն} 
		& & \armenian{խմորեղէն} \\
		&   & {{ˌzovɑt͡suˈ\textbf{t͡sit͡ʃ}}} &`refreshing' \\
		&  & & \armenian{զովացուցիչ}
		
		\\
		\hline 
		\textcolor{red}{make sure had all the words}
	\end{tabular}
	
\end{table}

But in contrast, \citet[11,18]{Johnson-1954-EastArmGrammar} documents secondary only on words with at least 4 syllables in her Eastern Armenian consultants (Table \ref{tab:johnson initia lsecondary stress}).    She reports that secondary stress falls on the first syllable, even if it has a schwa.  She reports that schwas with secondary stress are slightly lengthened and backed \citep[18]{Johnson-1954-EastArmGrammar}.\footnote{Johnson transcribes underlying /nn/ sequences as [n] via a rule of degemination in her consultant. We don't transcribe this degemination. } 

\begin{table}[H]
	\centering
	\caption{Alleged secondary stress in \citet{Johnson-1954-EastArmGrammar}'s Eastern data}
	\label{tab:johnson initia lsecondary stress}
	\begin{tabular}{|l ll| }
		\hline 
		{{ˈ\textbf{mɑjɾ}}} & `mother & \armenian{մայր} \\
		{{ɑmuˈ\textbf{sin}}} & `husband' & \armenian{ամուսին} \\
		\hline 
		{{ˌɑmusin-ˈ\textbf{neɾ}}} &`husband-{\pl}'& \armenian{ամուսիններ} \\
		{{ˌɑmusin-neˈ\textbf{ɾ-i}}} & `husband-{\pl}-{\gen}' & \armenian{ամուսինների} \\
		{{ˌusumnɑsiɾɑˈ\textbf{kɑn}}} & `lovers of education' \textcolor{red}{check}& \armenian{ուսումնասիրական} \\
		{{ˌusumnɑsiɾɑkɑn-neˈ\textbf{ɾ-i}}} & `lover of education (-{\pl}-{\gen})'  & \armenian{ուսումնասիրականների} \\
		\hline 
		ˌtʰəʃnɑmi-ˈ\textbf{neɾ} & `enemy-{\pl}' & \armenian{թշնամիներ}
		\\
		\textcolor{red}{make sure had all the words}
		
	\end{tabular}
	
	
\end{table}


\textcolor{red}{get stress data from other grammars}

Throughout this grammar we do not annotate alleged initial secondary stress simply because we can't even hear it. There is no phonological process that references alleged initial secondary stress. The closest example  of such a phonological process is how syncope is blocked in word-initial secondary stress, but as  discussed in \textcolor{red}{cite chapter on syncope},   syncope can just reference  word-medial syllables. The second closest example is how stressed prefixes can block some allophonic rules. But these cases belong to prefix-based secondary stress, not alleged initial secondary stress. 


\subsection{Prefix-based secondary stress }\label{section:stress:secondary:prefix}
Prefix-based secondary stress is when there is secondary stress on some special morpheme. These special morphemes are the negative prefix \textit{ɑn-} (\S\ref{section:stress:secondary:prefix:neg})  and reduplication prefixes (\S\ref{section:stress:secondary:prefix:red}). Secondary stress on these morphemes is significantly more perceivable than alleged secondary stress. For example, HD can hear prefix-based secondary stress, while he cannot hear alleged initial secondary stress. 

\subsubsection{Negative prefix}\label{section:stress:secondary:prefix:neg}
For the negative prefix \textit{ɑn-}, this prefix is added to nouns and adjectives (X) to create a new word that can mean `not X' or `without X' (Table \ref{tab:negative prefix an secondary stress}). This prefix is analogous to the English prefix \textit{un-} and the English suffix \textit{-less}. Secondary stress is on this prefix \cite[179]{Tokhmakhyan-1975-StressEssence}. Secondary stress on this prefix can affect other segmental rules like nasal place assimilation (\S\ref{section:segmentalPhono:nasalPlace}). 

\begin{table}[H]
	\centering
	\caption{Secondary stress on the negative prefix \textit{ɑn-}}
	\label{tab:negative prefix an secondary stress}
	\begin{tabular}{|lll|lll|}
		\hline 
		ˈ\textbf{χid} & `dense' & \armenian{խիտ} 
		& ˌɑn-\textbf{χid} & `scarce' & \armenian{անխիտ} \\
		ˈ\textbf{d͡zuχ} & `smoke' & \armenian{ծուխ} 
		& ˌɑn-\textbf{d͡zuχ} & `smokeless' & \armenian{անծուխ} \\
		\hline
		tʰeˈ\textbf{tʰev} & `light' & \armenian{դեդեւ} 
		& ˌɑn-tʰe\textbf{tʰev} & `firm' & \armenian{անդեդեւ} \\
		leˈ\textbf{zu} & `tongue' & \armenian{լեզու} 
		& ˌɑn-leˈ\textbf{zu} & `tongue-less' & \armenian{անլեզու} \\
		\hline 
		hɑd͡ʒeˈ\textbf{li} & `pleasant' & \armenian{հաճելի} 
		& ˌɑn-hɑd͡ʒeˈ\textbf{li} & `unpleasant' & \armenian{անհաճելի} \\
		ʒɑmɑˈ\textbf{nɑɡ} & `time' & \armenian{ժամանակ} 
		& ˌɑn-ʒɑmɑˈ\textbf{zu} & `inopportune' & \armenian{անժամանակ} \\
		\hline
	\end{tabular}    
\end{table}


\citet[133]{Gharagulyan-1974-BookArmenianOrthoepy} 
reports that the secondary stress of this prefix is not weak. In my intuition, the perceptibility of this prefix's stress is clear and robust. To illustrate, the following bisyllabic words act as near-minimal pairs  (Table \ref{tab:negative prefix minimal pairs}). The initial syllable sounds more stressed when that syllable is the negative prefix. 

\begin{table}[H]
	\centering
	\caption{Near-minimal pairs for secondary stress of the negative prefix \textit{ɑn-}}
	\label{tab:negative prefix minimal pairs}
	\begin{tabular}{|lll|lll|}
		\hline 
		ˌɑn-ˈ\textbf{tʰɑs} & `irregular' & \armenian{անդաս} 
		& 
		ˈ\textbf{tʰɑs}& `class' & \armenian{դաս}
		\\
		ɑnˈ\textbf{tʰɑm} & `member' & \armenian{անդամ}  & & &
		\\
		\hline 
		ˌɑn-ˈ\textbf{dɑʃ} & `rough' & \armenian{անտաշ}
		&
		dɑˈ\textbf{ʃ-e-l} & `to chip' & \armenian{տաշել} 
		\\
		ɑnˈ\textbf{dɑɾ} & `forest' & \armenian{անտառ} & & &
		\\
		\hline
		ˌɑŋ-ˈ\textbf{ɡɑχ} & `independent' & \armenian{անկախ}
		& 
		ˈ\textbf{ɡɑχ} & `hung up' & \armenian{կախ}
		\\
		ɑŋˈ\textbf{kʰɑm} & `time' & \armenian{անգամ} & & &
		\\
		\hline 
	\end{tabular}
\end{table}

\subsubsection{Secondary stress in reduplication} \label{section:stress:secondary:prefix:red}
Reduplication is when part of a word is repeated to add an additional meaning. Reduplication creates secondary stress. The relevant reduplicative processes include root reduplication, emphatic reduplication, and word reduplication. 

For root reduplication (Table \ref{tab:reduplication prefix secondary stress nonschwa}),   but there are words which are made up of a reduplicated or repeated root. The root is monosyllabic.  Many of these words are verbs. In such words, primary stress is on the final syllable. The first syllable (the first copy of the root) has a perceptible secondary stress. The second syllable can either have the original  non-schwa vowel, or a reduced schwa. 

\begin{table}[H]
	\centering
	\caption{Initial secondary stress in root reduplication with the gloss `$\sqrt{}$-$\sqrt{}$-{\thgloss}-{\infgloss}' }
	\label{tab:reduplication prefix secondary stress nonschwa}
	\begin{tabular}{|lll|}
		\hline 
		ˌkʰɑj-kʰɑˈ\textbf{j-e-l}  & `to dissolve' & 
		\armenian{քայքայել}
		\\
		ˌvɑʁ-vɑˈ\textbf{ʁ-e-l}   & `to hurry' & \armenian{վաղվաղել} 
		\\
		ˌt͡ʃɑɾ-t͡ʃɑˈ\textbf{ɾ-e-l} &  `to torture' & \armenian{չարչարել}
		\\
		\hline 
		ˌvɑz-vəˈ\textbf{z-e-l} & `to run around' & \armenian{վազվզել}
		\\
		ˌd͡zɑm-d͡zəˈ\textbf{m-e-l} &  `to chew' & \armenian{ծամծմել}
		\\
		ˌpʰɑl-pʰəˈ\textbf{l-i-l} & `to sparkle' & \armenian{փալփլիլ}
		\\
		\hline 
	\end{tabular}
\end{table}

Such reduplication can likewise place secondary stress on an initial schwa (Table \ref{tab:reduplication prefix secondary stress schwa}).


\begin{table}[H]
	\centering
	\caption{Initial secondary stress on schwas in root reduplication with the gloss `$\sqrt{}$-$\sqrt{}$-{\thgloss}-{\infgloss}' }
	\label{tab:reduplication prefix secondary stress schwa}
	\begin{tabular}{|lll|}
		\hline 
		ˌvəz-vəˈ\textbf{z-ɑ-l} & `to buzz' & \armenian{վզվզալ}
		\\
		ˌχəl-χəˈ\textbf{l-e-l} & `to neglect' & \armenian{խլխլել}
		\\
		ˌɡəm-ɡəˈ\textbf{m-ɑ-l} & `to lisp' & \armenian{կմկմալ}
		\\
		ˌd͡ʒəɾ-d͡ʒəˈ\textbf{ɾ-ɑ-l} & `to creak' & \armenian{ճռճռալ}
		\\
		ˌtʰəɾ-tʰəˈ\textbf{ɾ-e-l} & `to fly about' & \armenian{թռթռել}
		\\
		ˌfəɾ-fəˈ\textbf{ɾ-e-l} & `to rustle'   & \armenian{ֆրֆրել}
		\\
		\hline 
	\end{tabular}
\end{table}

The morphology of this reduplication process is discussed more in \textcolor{red}{reduplication prefix chapter}. It is a type of derivational morphology that is used to derive new words, often with an intensity meaning. 

For emphatic reduplication (Table \ref{tab:secondary stress redup emphatic}), a small number of adjectives have a derived intensive form. In this form, the first CV-sequence of the root is repeated, and either \textit{pʰ} or \textit{s} is added as a coda. Secondary stress is perceivable on this repeated syllable. Note how voicing assimilation can apply across the prefix-root boundary. 

\begin{table}[H]
	\centering
	\caption{Secondary stress in emphatic reduplication}
	\label{tab:secondary stress redup emphatic}
	\begin{tabular}{|lll| lll| }
		\hline 
		mɑˈ\textbf{kʰuɾ} & `clean' & \armenian{մաքուր}     
		& ˌmɑs-mɑˈ\textbf{kʰuɾ} & `very clean' & \armenian{մաս-մաքուր}     \\
		ˈ\textbf{sev} & `black' & \armenian{սեւ}     
		& ˌsep-ˈ\textbf{sev} & `very black' & \armenian{սեփ-սեւ}     \\
		tʰeˈ\textbf{ʁin} & `yellow' & \armenian{դեղին}     
		& ˌtʰep-ˈte\textbf{ʁin} & `very black' & \armenian{դեփ-դեղին}     \\
		ɡɑɾ\textbf{miɾ} & `red' & \armenian{կարմիր}     
		& ˌɡɑs-ˈkɑɾ\textbf{miɾ} & `very red' & \armenian{կաս-կարմիր}     \\
		\hline 
	\end{tabular}
\end{table}

For word reduplication, monosyllabic or bisyllabic adjectives/nouns can be repeated to create an adverbial meaning. When these words are reduplicated, both the first and second word have final  stress. For Eastern Armenian, \citet[76]{Margaryan-1997-ArmenianPhonology} reports that the second word takes primary stress, while the first takes secondary stress. If each word is monosyllabic, he reports that the first word lacks any stress. 

In contrast to those Eastern judgments (Table \ref{tab:secondary stress redup word word adj}), HD reports that in his Western Armenian, the first word takes primary stress, while the second word takes secondary stress. The Eastern judgments below are from \citep[76]{Margaryan-1997-ArmenianPhonology}. 

\begin{table}[H]
	\centering
	\caption{Secondary stress in adverb-forming word reduplication}
	\label{tab:secondary stress redup word word adj}
	\begin{tabular}{|ll|lll|}
		\hline 
		\multicolumn{2}{|l|}{Unreduplicated}& \multicolumn{2}{l}{Reduplicated adverb} & 
		\\
		& &  Eastern & Western &   \\
		\hline 
		ɑˈ\textbf{ɾɑkʰ} & `fast'  & ɑˌɾɑɡ-ɑˈ\textbf{ɾɑɡ} &  ɑˈ\textbf{ɾɑkʰ}-ɑˌɾɑkʰ &  \armenian{արագ արագ}
		\\
		heˈ\textbf{ɾu} & `far'  & heˌru-heˈ\textbf{ru} &  heˈ\textbf{ɾu}-heˌɾu  &  \armenian{հեռու հեռու}
		\\
		siˈ\textbf{ɾun} & `pretty'  & siˌɾun-siˈ\textbf{ɾun} &  siˈ\textbf{ɾun}-siˌɾun  & \armenian{սիրուն սիրուն}
		\\
		ɡɑˈ\textbf{mɑt͡s} & `slow'  & kɑˌmɑt͡sʰ-kɑˈ\textbf{mɑt͡sʰ} &  ɡɑˈ\textbf{mɑt͡s}-ɡɑˌ{mɑt͡s}     & \armenian{կամաց կամաց}
		\\
		jeɾˈ\textbf{ɡu} & `two'  & jeɾˌku-jeɾ\textbf{ku} &  jeɾˈ\textbf{ɡu}-jeɾˌ{ɡu}     & \armenian{երկու երկու}
		\\
		ɡɑˈ\textbf{bujd} & `blue'  & kɑˌ\textbf{pujt}-kɑˈ\textbf{pujt} &  ɡɑˈ\textbf{bujd}-ɡɑˌ{bujd}    & \armenian{կապոյտ կապոյտ}
		\\
		\hline 
		ˈ\textbf{noɾ} & `new'  & noɾ-ˈ\textbf{noɾ} &  ˈ\textbf{noɾ}-ˌ{noɾ}    & \armenian{նոր նոր}
		\\
		ˈ\textbf{sud} & `lie/false'  & sut-ˈ\textbf{sut} &  ˈ\textbf{sud}-ˌ{sud}    & \armenian{սուտ սուտ}
		\\
		ˈ\textbf{meɡ} & `one'  & mek-ˈ\textbf{mek} &  ˈ\textbf{meɡ}-ˌ{meɡ}    & \armenian{մէկ մէկ}
		\\
		ˈ\textbf{kʰit͡ʃ} & `few'  & kʰit͡ʃʰ-ˈ\textbf{kʰit͡ʃʰ} &  ˈ\textbf{kʰit͡ʃ}-ˌ{kʰit͡ʃ}    & \armenian{քիչ քիչ}
		\\
		
		\hline 
	\end{tabular}
\end{table}

Prosodically, each copy of the word is likely its own prosodic word. HD reports that allophonic processes like voicing assimilation don't need to apply across the reduplication boundary. Although we're not sure, it is possible that some assimilation happens in fast speech. 



\subsection{Demoted primary stress}\label{section:stress:secondary:demoted}
The last type of perceivable secondary stress is from demoted primary stress. As a general rule, the rightmost non-schwa vowel in the word gets primary stress. But in certain morphological constructions, primary stress is irregularly on some other element. When this irregularity happens, the syllable that previously had final primary stress now has final secondary stress. 

One such example (Table \ref{tab:demoted stress}) involves imperatives and negative imperatives (prohibitives). For all verbs, the imperative 2SG takes regular final stress. The prohibitive is formed by placing the particle \textit{mi} before the verb. This particle takes primary stress. The verb's final syllable takes secondary stress. 

\begin{table}[H]
	\centering
	\caption{Demoted primary stress as secondary stress in prohibitives}
	\label{tab:demoted stress}
	\begin{tabular}{lllll}
		Infinitive& ɑˈ\textbf{d-e-l} & $\sqrt{}$-{\thgloss}-{\infgloss} & `to hate' & \armenian{ատել} \\
		Imperative 2SG & ɑˈ\textbf{d-e} & $\sqrt{}$-{\thgloss}  & `hate!' & \armenian{ատէ} \\
		Prohibitive 2SG & ˈ\textbf{mi} ɑˌ{d-e-ɾ} & $\sqrt{}$-{\thgloss}-2{\sg} & `Don't hate!' & \armenian{մի ատեր} \\
	\end{tabular}
\end{table}

When forming the prohibitive, the final primary stress of the verb is demoted to secondary stress. As surveyed in \S\ref{section:stress:verb}, stress-attracting morphemes take primary stress, while the phonology places secondary stress on the final syllable. 

\subsection{Length-induced secondary stress in compounds and prefixoids}\label{section:stress:secondary:long}

Because Armenian has agglutinative morphology, it is quite easy to coin words with five or more syllables. This is quite common when creating compounds (\S\ref{section:stress:secondary:long:compound}) or words with prefixoids (\S\ref{section:stress:secondary:long:prefixoid}). Because of how long these words are, some previous grammars report secondary stress on these words. We're not sure if what they report as secondary stress is genuine phonological prominence, vs. some type of rhythm-effect of putting small pauses in large words. 

\subsubsection{Long compounds}\label{section:stress:secondary:long:compound}

Compounds are formed by concatenating  stems with the linking vowel \textit{-ɑ-}. Some compounds idiosyncratically lack this vowel.  Primary stress falls on the rightmost full vowel of the compound (\ref{ex:long stress}). 

\begin{exe}
	\ex \label{ex:long stress}
	\begin{xlist}
		\ex \makebox[3cm][l]{{ˈdon + ˈ{d͡zɑɾ}}}\makebox[4cm][l]{`holiday + tree'}\armenian{տօն, ծառ}
		
		\makebox[3cm][l]{{don-ɑ-ˈ\textbf{d͡zɑɾ}}}\makebox[4cm][l]{`Christmas tree'}\armenian{տօնածառ}
		\ex \makebox[3cm][l]{{ˈχɑt͡ʃ + ˈ{kʰɑɾ}}}\makebox[4cm][l]{`cross + stone'}\armenian{խաչ, քար}
		
		\makebox[3cm][l]{{χɑt͡ʃ-ˈ\textbf{kɑɾ}}}\makebox[4cm][l]{`cross-stone'}\armenian{խաչքար}
	\end{xlist}
\end{exe}


Compounds form a single prosodic word, as signalled by how they take final stress. However, it's unclear if there is any secondary stress inside a compound. Based on past grammars and our own impression, there is some degree of secondary stress for substantially long compounds. 


For substantially long compounds (Table \ref{tab:secondary stress large compound}), \citet[76]{Margaryan-1997-ArmenianPhonology}  reports that there is  secondary stress on the linking vowel.   \citet[30]{Soukyasyan-2004-ArmenianPhonology} reports secondary stress on the first syllable and \textit{only} for compounds with at least 3 syllables. More indeterminancy is documented elsewhere (\citealt[63]{Tokhmakhyan-1971-ArmenianWordStress}; \citealt[22]{Gharagulyan-1974-BookArmenianOrthoepy}; \citealt[74]{Tokhmakhyan-1983-ArmenianPhono}).  All of this work focuses on Eastern Armenian. 


\begin{table}[H]
	\centering
	\caption{Secondary  stress in large compounds}
	\label{tab:secondary stress large compound}
	
	\resizebox{\textwidth}{!}{%
		\begin{tabular}{|llll|}
			\hline 
			& & Morphemes &  Source\\
			\hline 
			E & {ɡiˌt-ɑ-hetɑzot-ɑˈ\textbf{kɑn}}& `research' & 
			\citealt[76]{Margaryan-1997-ArmenianPhonology}
			\\
			W &{kʰiˌd-ɑ-hedɑzod-ɑˈ\textbf{ɡɑn}} & $\sqrt{}$-{\lvgloss}-$\sqrt{}$-{\nmlz}  & \armenian{գիտահետազօտական} 
			\\
			\hline 
			E & oˌtʰ-ɑ-nɑf-kɑˈ\textbf{jɑn} &  `airplane station' & \citealt[76]{Margaryan-1997-ArmenianPhonology}
			\\
			W & oˌtʰ-ɑ-nɑv-ɡɑˈ\textbf{jɑn} & $\sqrt{}$-{\lvgloss}-$\sqrt{}$-$\sqrt{}$& \armenian{օդանաւկայան}
			\\ \hline 
			E & ɡəˌɾ-ɑ-kʰənn-ɑ-ˈ\textbf{dɑt} &  `literary critic' & \citealt[76]{Margaryan-1997-ArmenianPhonology}
			\\
			W & kʰəˌɾ-ɑ-kʰənn-ɑ-ˈ\textbf{tɑd} & $\sqrt{}$-{\lvgloss}-$\sqrt{}$-{\lvgloss}-$\sqrt{}$ & \armenian{օդաքննադատ}
			\\\hline
			E & ɑjˌs-u-heˈ\textbf{tev} & `henceforth' & \citealt[76]{Margaryan-1997-ArmenianPhonology}
			\\ 
			W & ɑjˌs-u-heˈ\textbf{dev} & $\sqrt{}$-{\lvgloss}-$\sqrt{}$ & \armenian{այսուհետեւ}
			\\ \hline
			E &  ɑjn-\`u-ɑmenɑjˈ\textbf{niv}& `nevertheless' & \citealt[76]{Margaryan-1997-ArmenianPhonology}
			\\
			W &  ɑjn-\`u-ɑmenɑjˈ\textbf{niv} & $\sqrt{}$-{\lvgloss}-$\sqrt{}$ & \armenian{այնուամենայնիւ}
			\\ \hline 
			E & mɑnˌɾ-ɑ-mɑsn-oˈ\textbf{ɾen} & `detailed' & \citealt[76]{Margaryan-1997-ArmenianPhonology}
			\\
			W &  mɑnˌɾ-ɑ-ˈ\textbf{mɑs}n-oɾen  & $\sqrt{}$-{\lvgloss}-$\sqrt{}$ &  \armenian{մանրամասնօրէն}
			\\ \hline 
			E & usumˌn-ɑ-dɑstijaɾak-t͡sʰ-ɑˈ\textbf{kɑn}   &`educational'&  \citealt[14]{Margaryan-1993-ArmenianLexicology} 
			\\
			W &  usumˌn-ɑ-tʰɑstijaɾak-t͡s-ɑˈ\textbf{ɡɑn} &$\sqrt{}$-{\lvgloss}-$\sqrt{}$-{\nmlz}-{\adjz} &  \armenian{ուսոմնադաստիարակցական}
			\\ \hline 
			E &  ˌd͡ʒeɾm-ɑ-elektɾ-ɑ-kentˈ\textbf{ɾon} & `thermal power center'& \citealt[29]{Soukyasyan-2004-ArmenianPhonology}
			\\
			W &  ˌt͡ʃeɾm-ɑ-elektɾ-ɑ-ɡendˈ\textbf{ɾon} & $\sqrt{}$-{\lvgloss}-$\sqrt{}$-{\lvgloss}-$\sqrt{}$ & \armenian{ջերմաէլեկտրակենտրոն}
			\\ \hline 
			E & ˌultɾ-ɑ-mɑnuʃɑk-ɑ-ˈ\textbf{ɡujn} & `ultra-violet-colored' &   \citealt[29]{Soukyasyan-2004-ArmenianPhonology}
			\\
			W &  ˌultɾ-ɑ-mɑnuʃɑɡ-ɑ-ˈ\textbf{kʰujn} &  $\sqrt{}$-{\lvgloss}-$\sqrt{}$-{\lvgloss}-$\sqrt{}$ & \armenian{ուլտրամանուշակագոյն}
			\\ \hline 
			E &  ˌhet͡s-ɑnv-ɑ-vɑs-kʰ & `bicycle-ride'& \citealt[29]{Soukyasyan-2004-ArmenianPhonology}
			\\
			W & ˌhed͡z-ɑnv-ɑ-vɑs-kʰ& $\sqrt{}$-$\sqrt{}$-{\lvgloss}-$\sqrt{}$-{\nmlz} & \armenian{հեծանուավազք}
			\\ \hline
			E &    ˌhuʃ-ɑ-kəɾt͡sʰ-kʰ-ɑ-nəˈ\textbf{ʃɑn}   & `commemorative badge' & \citealt[29]{Soukyasyan-2004-ArmenianPhonology}
			\\ 
			W &    ˌhuʃ-ɑ-ɡəɾt͡s-k-ɑ-nəˈ\textbf{ʃɑn}   &$\sqrt{}$-{\lvgloss}-$\sqrt{}$-{\nmlz}-{\lvgloss}-$\sqrt{}$ & \armenian{յուշակրծքանշան}  
			\\ \hline
		\end{tabular}
}\end{table}

For the above words, the reported secondary stresses are from the Eastern Armenian. The Western Armenian forms are from HD. We suspect that the above secondary stresses aren't genuine types of prosodic prominences. Instead, we highly suspect that these `stresses' are caused by optional small pauses between the roots in these very large compounds. 


\subsubsection{Prefixoids or   compound-like prefixes}\label{section:stress:secondary:long:prefixoid}

Armenian does have some other derivational prefixes. These prefixes are considered `learned' prefixes because they're most often used to create words that are high-register, technical, or calques. These prefixes have very similar structure to compounds. They   surface with the same linking vowel \textit{-ɑ-} that's used in compounds. These prefixes occupy a grey area between simple prefixes vs. compounds. So we call them prefixoids. 

\citet[29]{Soukyasyan-2004-ArmenianPhonology} reports secondary stress on the initial syllable of these prefixoids in Eastern Armenian (Table \ref{tab:secondaey stress learned prefixes}).  We think Western Armenian likewise has secondary stress on these prefixes.   We adapt his data to Western Armenian.  \citet[222]{Gharagulyan-1974-BookArmenianOrthoepy} reports more cases of  prefixoids   taking secondary stress.



\begin{table}[H]
	\centering
	\caption{Secondary stress on learned prefixes (adapted from Eastern from \citealt[29]{Soukyasyan-2004-ArmenianPhonology}) }
	\label{tab:secondaey stress learned prefixes}
	\resizebox{\textwidth}{!}{%
		\begin{tabular}{|llll| }
			\hline 
			ˌveɾ-jeɾɡəɾ-ˈ\textbf{jɑ} & \textit{up}-$\sqrt{\text{world}}$-{\adjz} &   `above ground'  & \armenian{վերերկրյա}
			\\
			ˌveɾ-ɑmpɑɾt͡s & \textit{up}-$\sqrt{\text{lift}}$-{\adjz} &   `lift'  & \armenian{վերամբարձ}
			\\
			ˌhɑɡ-ɑ-ɡɑɾɑvɑɾ-ɑˈ\textbf{ɡɑn} & \textit{anti}-{\lvgloss}-$\sqrt{\text{govern}}$-{\adjz} &  `anti-governmental' & \armenian{հակակառավարական}
			\\
			ˌkʰeɾ-hokʰn-ɑd͡z-uˈ\textbf{tʰjʏn} &  \textit{supra}-{\lvgloss}-$\sqrt{\text{tire}}$-{\rptcp}-{\nmlz} &  `over-tiredness'   & \armenian{գերյօգնածութիւն}
			\\
			ˌmɑɡ-əntʰɑt͡s-uˈ\textbf{tʲun} & `tide' & \textit{sub}-$\sqrt{\text{course}}$-{\nmlz} & `tide' 
			\\ \hline
		\end{tabular}
	}
\end{table}






\section{Orthographic encoding of stress}\label{section:stress:ortho}
This chapter went over the location of primary and secondary stress in words. The location was determined based just on the perception of stress by native and non-native speakers of Armenian. Interestingly, the Armenian orthography provides independent evidence for the location of stress. This comes from infixal punctuation symbols. 

As surveyed in \S\ref{section:ortho:punctuation}, some punctuation symbols are placed inside the word on the stressed syllable. For example in questions, the question marker \armenian{՞} <ˀ> is placed on the vowel   which carries the strongest stress in the sentence. In a simple yes-no question or polar question, this stress is on the verb (\ref{ex:question stress mrker}). If some other constituent is the strongest word (=is questioned), then the symbol is on that word (\ref{ex:question stress mrker other}). In this section, we place the  <ˀ> symbol on the stressed vowel in both the transcription and transliteration lines 

\begin{exe}
	\ex 
	\begin{xlist}
		\ex \gll iɾen deˈ\textbf{s-ɑˀ-ɾ} \\
		him.{\dat} see-{\pst}-2{\sg} \\
		\trans `Did you see him?' \label{ex:question stress mrker}\\
		\armenian{Իրեն տե\textbf{սա՞ր}։}
		\\ <iren de\textbf{saˀr}.>
		\ex \gll iˈ\textbf{ɾeˀn} des-ɑ-ɾ \\
		him.{\dat} see-{\pst}-2{\sg} \\
		\trans `Did you see HIM? (and not  someone else) \label{ex:question stress mrker other}\\
		\armenian{Ի\textbf{րե՞ն} տեսար։}
		\\ <i\textbf{reˀn} desar.>
		
	\end{xlist}
\end{exe}

As we shall see, native speaker-authors of Armenian can detect the location of  primary stress in words, and then encode this knowledge in the orthography. This  primary stress can be regularly or irregularly derived. For the rest of this section, we illustrate with example sentences from the Western Armenian translation of the Bible.\footnote{\url{https://hycatholic.ru/biblia/}\\\url{https://wol.jw.org/hyw/wol/binav/r487/lp-r/sbi1}} We're not sure when is the exact age of this translation, but some online sources suggest it's from the 19th century.\footnote{\url{http://armenianbible.org/}} English translations are taken from the New International Version. 


First, as stated, regular primary stress is on the rightmost non-schwa vowel of the word. In the examples below, stress is the plural suffix \textit{-neɾ}. The orthography places the question symbol on it. In (\ref{bible:ner}), this suffix is final and takes stress. In (\ref{bible:ner schwa}), this suffix is before a schwa and takes stress. 

\begin{exe}
	\ex 
	\begin{xlist}
		\ex   \gll ɑmeŋkʰ-ətʰ ɑl hɑzɑɾɑbed-neɾ u hɑɾʏɾɑbed-ˈ\textbf{neˀɾ} bidi ən-e-$\emptyset$ \\
		all-{\possSsg} also chiliarch-{\pl} and centurion-{\pl} {\fut} do-{\thgloss}-3{\sg} \\
		\trans `... Will he make all of you commanders of thousands and commanders of hundreds?' \hfill (1 Samuel 22:7) \label{bible:ner}
		\\
		Literally: `Will he make all of you chiliarchs and centurions?'
		\\ \armenian{... ամէնքդ ալ հազարապետներ ու հարիւրապետ\textbf{նե՞ր} պիտի ընէ,}
		\\ <amēnk't al hazarabedner ow hariwrabed\textbf{neˀɾ} bidi ənē,> 
		
		\ex \gll ɡɑm tʰe jeɾɡiŋkʰ-ˈ\textbf{neˀ}ɾ-ə ɡu-d-ɑ-n ɑnt͡sɾev-ə \\
		or that sky-{\pl}-{\defgloss} {\ind}-give-{\thgloss}-3{\pl} rain-{\defgloss} \\
		\trans `... Do the skies themselves send down showers?' \hfill (Jeremiah 14:22) \label{bible:ner schwa}
		\\
		Literally: `Or that the skies give rain?'  
		\\ \armenian{Կամ թէ երկինքնե՞րը կու տան անձրեւը։}
		\\ <Gam t'ē ergink'\textbf{neˀ}rə gow dan ant͡srewə.> 
		
	\end{xlist}
\end{exe}


Thus phonological primary stress is easily reflected in the orthography. For negation-induced irregular primary stress, the orthography likewise marks this. Recall from \S\ref{section:stress:verb:negFinite} that in the negative indicative, the non-finite verb takes secondary stress while a negated auxiliary takes primary stress. This rule is reflected in the orthography. In \ref{bible:neg indc clitic}, the negated auxiliary is spelled as a separate word, takes stress, and takes the question marker. In \ref{ex:bible neg 3sg indic}, the negated auxiliary is procliticized into the vowel-initial verb. Stress is on the initial vowel. 

\begin{exe}
	\ex 
	\begin{xlist}
		\ex \gll ɡə-ɡɑɾd͡z-e-s tʰe ˈ\textbf{t͡ʃ-eˀ-m} ɡəɾn-ɑ-ɾ himɑ im hoɾ-əs ɑʁɑt͡ʃ-e-l \\
		{\ind}-think-{\thgloss}-2{\sg} that {\neggloss}-is-1{\sg} can-{\thgloss}-{\cn} now my.{\gen} father.{\obl}-{\possFsg} beseech-{\thgloss}-{\infgloss} \\
		\trans ` Do you think I cannot call on my Father,' \hfill (Matthew 26:53) \label{bible:neg indc clitic}
		\\ \armenian{Կը կարծես թէ չե՞մ կրնար հիմա իմ Հօրս աղաչել,}
		Literally: `Do you think that I cannot now beseech my father?' 
		\\ <Gə gard͡zes t'ē \textbf{t͡ʃ'eˀm} grnar hima im hōrs aɣat͡ʃ'el,> 
		\ex \gll u ɑnoɾ ɑh-ə t͡seɾ vəɾɑ ˈ\textbf{t͡ʃ-iˀj}n-ɑ-ɾ \\
		and his.{\gen} dread-{\defgloss} your.{\gen}.{\pl} on {\neggloss}-fall-{\thgloss}-{\cn} \\
		\trans `Would not the dread of him fall on you?' \hfill (Job 13:11)  \label{ex:bible neg 3sg indic}
		\\
		Literally: `And doesn't his dread fall on you?' 
		\\ \armenian{Ու անոր ահը ձեր վրայ չ’ի՞յնար։}
		\\ <Ow anor ahə t͡ser vraj \textbf{t͡ʃ''iˀj}nar.> 
		
	\end{xlist}
\end{exe}

Although the orthography does encode irregular stress in the above cases, there are exceptions. For example in the past perfective, the negation prefix is added to the verb and attracts stress. If the root is vowel-initial, then the first vowel takes stress (\ref{bible:neg past perf vowel}). If the root is consonant-initial, then a schwa is epenthesized and this schwa takes stress (\ref{bible:neg past perf epen}). In both cases, the verb-final syllable takes secondary stress. But in the orthography, the question marker is added to the verb-final syllable, and not the initial  syllable. 

\begin{exe}
	\ex 
	\begin{xlist}
		\ex \gll jes t͡sesz dɑsnəjeɾɡukʰ-ətʰ ˈ\textbf{t͡ʃ-ə}ndɾ-e-t͡s-iˀ-$\emptyset$, \\
		I.{\nom} you.{\pl}.{\acc} twelve-{\possSsg} {\neggloss}-choose-{\thgloss}-{\aorperf}-{\pst}-1{\sg}\\
		\trans `Have I not chosen you, the Twelve?' \hfill (John 6:70)  \label{bible:neg past perf vowel}
		\\
		\armenian{Ես ձեզ տասներկուքդ չընտրեցի՞,} \\
		<es t͡sez dasnergowk't' t͡ʃ'əndre\textbf{t͡s'iˀ},>
		\ex \gll jes  kezi ɾɑkʰel-i-n hɑmɑɾ ˈ\textbf{t͡ʃə}-d͡zɑɾɑj-e-t͡s-iˀ-$\emptyset$ \\
		I.{\nom} you.{\sg}.{\dat} Rachel-{\dat}-{\defgloss} for {\neggloss}-serve-{\thgloss}-{\aorperf}-{\pst}-1{\sg} \\
		\trans `I served you for Rachel, didn’t I?' \hfill (Genesis 29:25) \label{bible:neg past perf epen}
		\\
		Literally: `Didn't I serveg you for Rachael?' 
		\armenian {ես քեզի Ռաքէլին համար չծառայեցի՞.}
		\\ <es kezi \.{R}ak'ēlin hamar t͡ʃ'd͡za\.{r}aye\textbf{t͡s'iˀ}.>  
		
	\end{xlist}
\end{exe}

We don't know why the orthography doesn't place the question marker on the first vowel of the past perfective. It's possible that perhaps that when these orthographic rules were established for Armenian, primary stress was not on the first syllable in the past perfective.  

Despite the exceptionality of punctuation with the past perfective, question punctuation is useful to find variation in the placement of irregular stress. For example, for the interrogative pronoun `how much', stress is irregularly on the first syllable in general: [ˈ\textbf{voɾ}t͡ʃɑpʰ]. In the Bible, there were 6 instances of this word with a question marker on the first syllable (\ref{bible: function initial}). However, there were 2 instances where the question marker was on the second syllable (\ref{bible: function second}), indicating that this word should be read with final stress in these sentences. 


\begin{exe}
	\ex 
	\begin{xlist}
		\ex \gll ˈ\textbf{voˀɾ}t͡ʃɑpʰ e-n kʰu d͡zɑɾɑj-i-tʰ oɾ-eɾ-ə,  \\
		how.much is-3{\pl} your.{\sg}.{\gen} servant-{\gen}-{\possSsg} day-{\pl}-{\defgloss},  \\
		\trans `How long must your servant wait?' \hfill (Psalm 119:84) \label{bible: function initial} \\
		Literally: `How many are the days of your servant?'\\
		\armenian{Ո՞րչափ են քու ծառայիդ օրերը,} \\
		<\textbf{Oˀr}t͡ʃ'ap' en k'ow d͡za\.{r}ayit  ōr-er-ə,>
		\ex \gll voɾˈ\textbf{t͡ʃɑˀpʰ} e-n im ɑnoɾenutʰjʏn-neɾ-əs u meχk-eɾ-əs \\
		how.much is-3{\pl} my.{\gen} iniquity-{\pl}-{\possFsg} and sin-{\pl}-{\possFsg} \\
		\trans `How many wrongs and sins have I committed?' \hfill  (Job 13:23) \label{bible: function second}\\
		Literally: `How many are my iniquities and sins?'\\
		\armenian{Որչա՞փ են իմ անօրէնութիւններս ու մեղքերս։}
		\\ <Orˈ\textbf{t͡ʃ'aˀp'} en im anōrēnowt'iwnners ow meɣk'ers.> 
	\end{xlist}
\end{exe}

We don't know why there is the above variation. It could be that the use of final stress was judged as more stylistically or rhythmically `nicer' for the translator who was translating the verse in (\ref{bible: function second}). To HD's ears, the use of final stress in (\ref{bible: function second}) sounds very emphatic and poetic. 

Besides stylistic variation, the question marker can likewise indicate possible language change. For the   past imperfective, early grammars imply that this inflection had regular final stress. But the modern Lebanese community has  irregular theme vowel stress. In the Bible, we find the question mark on the final syllable, indicating that the final syllable was stressed for the translator. 

\begin{exe}
	\ex \gll ɑɾtʰjokʰ meŋkʰ ɡəɾˈ\textbf{n-ɑ}-jiˀ-ŋkʰ kʰid-n-ɑ-l tʰe ɑniɡɑ bidi əs-e-$\emptyset$ \\
	perhaps we.{\nom} can-{\thgloss}-{\pst}-1{\pl} know-{\inch}-{\thgloss}-{\infgloss} that that {\fut} say-{\thgloss}-3{\sg} \\
	\trans `How were we to know he would say... ?' \hfill (Genesis 43:7)\\
	Literally: `Perhaps we could have known that he will say...?' 
	\\ \armenian{արդեօք մենք կրնայի՞նք գիտնալ թէ անիկա պիտի ըսէ... }
	\\ <arteōk' menk' grnay\textbf{iˀnk'} kidnal t'e aniga bidi əse...> 
	
\end{exe}

Thus, written corpora can be quite useful for finding diachronic and synchronic variation in stress. 

\section{Phonetics of stress and feet}\label{section:stress:phoneticsFeet}
This chapter focuses on the phonology of stress assignment. For the phonetics, there's very little information on the acoustic cues or effects of either primary or secondary stress. For an overview of the latest work on Armenian stress, see \citet{Seyfarth-JIPAArmenian}. 


For Eastern Armenian, there are some phonetic studies of stress from Soviet Armenia \citep{Khachatryan-1988-ArmenianPhono,Tokhmakhyan-1983-ArmenianPhono}. But these studies have various methodological issues that makes it difficult to accurately interpret their stress results. Some modern studies exist   \citep{Haghverdi-2016-ArmenianSchwa}.  

For Western Armenian, their are some   studies \citep{gordon2012sonority,AthanasopoulouVogelMe-2017-AcousticPropertiesCanonicalNonCanonical}. These studies suggest that main acoustic cue for stress is just pitch or f0. Other factors like duration don't significantly mark stress. 





Besides phonetics, there is little to no phonological evidence for the metrical foot in Armenian  (\citealp[42ff]{DeLisi-2015-EpenthesisProtoArmenian}, \citeyear[115]{DeLisi-2018-ArmenianProsodyDiachrony}). It's possible that because pitch is the main cue for stress, that stress does not utilize feet  \citep{Ozccelik-2017-FootObligatoryConstTurkishFrench}.   

\textcolor{red}{add phonetics}


